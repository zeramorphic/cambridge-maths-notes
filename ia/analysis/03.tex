\subsection{Geometric Series}
Let \(a_n = x^{n-1}\), where \(n \geq 1\).
Then
\[
	s_n = \sum_{j=1}^n a_j = 1 + x + x^2 + \dots + x^{n-1}
\]
Then
\[
	s_n = \begin{cases}
		\frac{1 - x^n}{1 - x} & \text{if } x \neq 1 \\
		n                     & \text{if } x = 1
	\end{cases}
\]
This can be shown by observing that
\[
	x s_n = x + x^2 + \dots + x^n = s_n - 1 + x^n \implies s_n(1-x) = 1-x^n
\]
If \(\abs{x} < 1\), then \(x^n \to 0\) as \(x \to \infty\).
So \(s_n \to \frac{1}{1-x}\).
If \(x > 1\), then \(x^n \to \infty\) and so \(s_n \to \infty\).
If \(x < -1\), \(s_n\) oscillates.
For completeness, if \(x=-1\), \(s_n\) oscillates between 0 and 1.

Note that the statement \(s_n \to \infty\) means that given \(a \in \mathbb R\), \(\exists N\) such that \(s_n > a\) for all \(n \geq N\), and a similar statement holds for negative infinity (swapping the inequality).
If \(s_n\) does not converge or tend to \(\pm \infty\), we say that \(s_n\) oscillates.

Thus the geometric series converges if and only if \(\abs{x} < 1\).
Note that to prove that \(x^n \to 0\) if \(\abs{x} < 1\), we can consider the caase \(0 < x < 1\) and write \(1/x = 1 + \delta\) for some positive \(\delta\).
Then \(x^n = \frac{1}{(1 + \delta)^n} \leq \frac{1}{1 + \delta n}\) from the binomial expansion, and this tends to zero as required.

\begin{lemma}
	If \(\sum_{j=1}^\infty a_j\) converges, then \(\lim_{j \to \infty} a_j = 0\).
\end{lemma}
\begin{proof}
	Given \(s_n = \sum_{j=1}^n a_j\), we have \(a_n = s_n - s_{n-1}\).
	If \(s_n \to a\), then \(a_n \to 0\) since \(s_{n-1}\) also tends to \(a\).
\end{proof}
\begin{remark}
	The converse is not true.
	For example, the harmonic series diverges, but the terms approach zero.
	Consider
	\begin{align*}
		s_{2n} & = s_n + \frac{1}{n+1} + \frac{1}{n+2} + \dots + \frac{1}{2n} \\
		       & > s_n + \frac{1}{2n} + \frac{1}{2n} + \dots + \frac{1}{2n}   \\
		       & = s_n + \frac{1}{2}
	\end{align*}
	So as \(n \to \infty\), if the sequence is convergent then the sequences \(s_n\) and \(s_{2n}\) tend to the same limit, but they clearly do not.
\end{remark}

\subsection{Comparison Test}
In this section, we will let \(a_n \in \mathbb R, a_n \geq 0\).
In other words, all series contain only non-negative real terms.
\begin{theorem}
	Suppose \(0 \leq b_n \leq a_n\) for all \(n\).
	If \(\sum_{j=1}^\infty a_j\) converges, then \(\sum_{j=1}^\infty b_j\) converges.
\end{theorem}
\begin{proof}
	Let \(s_N\) be the \(N\)th partial sum over the \(a_n\), and let \(d_N\) be the \(N\)th partial sum over the \(d_N\).
	Since \(b_n \leq a_n, d_N \leq s_N\).
	But \(s_N \to s\), so \(d_N \leq s_N \leq s\).
	So \(d_N\) is an increasing sequence that is bounded above by \(s\), so it converges.
\end{proof}
\noindent For example, let us analyse the behaviour of the sum of the sequence \(\frac{1}{n^2}\).
Note that
\[
	\frac{1}{n^2} < \frac{1}{n(n-1)} = \frac{1}{n-1} - \frac{1}{n}
\]
for \(n \geq 2\).
By the comparison test, it is sufficient to show that the series on the right hand side converges, in order to show that the original series converges.
\[
	\sum_{j=2}^N a_j = 1 - \frac{1}{N} \to 1
\]
as required.
So the original series tends to some value less than or equal to 2.

\subsection{Root Test or Cauchy's Test}
\begin{theorem}
	Suppose we have a sequence of non-negative terms \(a_n\).
	Suppose that \(a_n^{1/n} \to a\) as \(n \to \infty\).
	Then if \(a < 1\), the series \(\sum a_n\) converges.
	If \(a > 1\), the series \(\sum a_n\) diverges.
\end{theorem}
\begin{remark}
	Nothing can be said if \(a=1\).
	There is an example later of this fact.
\end{remark}
\begin{proof}
	If \(a < 1\), let us choose an \(r\) such that \(a < r < 1\).
	By the definition of the limit, \(\exists N\) such that \(\forall n \geq N\), \(a_n^{1/n} < r\).
	This implies that \(a_n < r^n\).
	The geometric series \(\sum r^n\) converges.
	By comparison, the series \(a_n\) converges.

	If \(a > 1\), for all \(n \geq N\), \(a_n^{1/n} > 1\) which implies \(a_n > 1\), thus \(\sum a_n\) diverges, since \(a_n\) does not tend to zero.
\end{proof}

\subsection{Ratio Test or d'Alembert's Test}
\begin{theorem}
	Suppose \(a_n > 0\), and \(\frac{a_{n+1}}{a_n} \to \ell\).
	If \(\ell < 1\), then the series \(\sum a_n\) converges.
	If \(\ell > 1\), then the series \(\sum a_n\) diverges.
\end{theorem}
\begin{remark}
	Like before, no conclusion can be drawn if \(\ell = 1\).
\end{remark}
\begin{proof}
	Suppose \(\ell < 1\).
	We can choose \(\ell < r < 1\), \(\exists N\) such that \(\forall n \geq N\), \(\frac{a_{n+1}}{a_n} < r\).
	Therefore \(a_n < r^{n-N} a_N\).
	Hence, \(a_n < k r^n\) where \(k\) is independent of \(n\).
	Applying the comparison test, the series \(\sum a_n\) must converge.

	If \(\ell > 1\), we can choose \(\ell > r > 1\).
	Then \(\exists N\) such that \(\forall n \geq N\), \(\frac{a_{n+1}}{a_n} > r\).
	As before, \(a_n > r^{n-N} a_N\).
	But the \(r^{n-N}\) diverges, so the original series diverges.
\end{proof}
