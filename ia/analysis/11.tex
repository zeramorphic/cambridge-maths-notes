\subsection{Properties of a Function from its Derivative}
We can deduce certain facts about a function by observing the properties its derivative exhibits.
These results are mostly trivial corollaries to the mean value theorem, proven in the last lecture.
\begin{corollary}
	Let \(f \colon [a,b] \to \mathbb R\) be continuous, and differentiable on \((a, b)\).
	Then we have
	\begin{enumerate}[(i)]
		\item If \(f'(x) > 0\) for all \(x \in (a, b)\), then \(f\) is strictly increasing on \([a, b]\);
		\item If \(f'(x) \geq 0\) for all \(x \in (a, b)\), then \(f\) is increasing on \([a, b]\);
		\item If \(f'(x) = 0\) for all \(x \in (a, b)\), then \(f\) is constant on \([a, b]\).
	\end{enumerate}
\end{corollary}
\noindent Part (iii) of this corollary is essentially solving the most simple differential equation; we are showing that the only possible solutions to this equation are the constant functions.
Note that similar statements about decreasing functions hold.
\begin{proof}
	\begin{enumerate}[(i)]
		\item We have \(f(y) - f(x) = f'(c)(y-x)\) for some \(c \in (x, y)\).
		      If \(f'(c) > 0\), then \(f(y) - f(x) > 0\).
		\item Analogously to before, \(f(y) - f(x) = f'(c)(y-x)\) for some \(c \in (x, y)\).
		      If \(f'(c) \geq 0\), then  \(f(y) - f(x) \geq 0\).
		\item By the mean value theorem on \([a, x]\), if \(f'(c) = 0\), then \(f(x) - f(a) = 0\).
	\end{enumerate}
\end{proof}

\subsection{Inverse Function Theorem}
\begin{theorem}
	Let \(f \colon [a, b] \to \mathbb R\) be a continuous function on \([a, b]\) and differentiable on \((a, b)\), with \(f'(x) > 0\) everywhere on \((a, b)\).
	Let \(f(a) = c, f(b) = d\).
	Then the function \(f \colon [a,b] \to [c,d]\) is bijective, and \(f^{-1} \colon [c,d] \to [a,b]\) is differentiable on \((c, d)\) with
	\[
		\left( f^{-1} \right)' (x) = \frac{1}{f'\left(f^{-1}(x)\right)}
	\]
\end{theorem}
\noindent Note, in lecture 8 it was proven that a continuous strictly increasing function has a continous inverse.
This strengthens that claim to include the differentiability property if the original function was differentiable.
\begin{proof}
	We know from lecture 8 that there exists \(g \colon [c,d] \to [a,b]\) which is a strictly increasing continuous function, which is the inverse of \(f\).
	We must now show that \(g\) is differentiable and that its derivative has the required form as stated in the claim.
	Now, let \(y = f(x)\).
	Given \(k \neq 0\), let \(h\) be given by
	\[
		y + k = f(x+h)
	\]
	Alternatively, written in terms of \(g\),
	\[
		x + h = g(y + k)
	\]
	So clearly \(h \neq 0\).
	Since \(g\) is continuous, if \(k \to 0\) then \(h \to 0\).
	Then
	\begin{align*}
		\frac{g(y + k) - g(y)}{k}                           & = \frac{x + h - x}{f(x+h) - y}           \\
		                                                    & = \frac{h}{f(x+h) - f(x)}                \\
		\therefore \lim_{k \to 0} \frac{g(y + k) - g(y)}{k} & = \lim_{h \to 0} \frac{h}{f(x+h) - f(x)} \\
		                                                    & = \frac{1}{f'(x)}
	\end{align*}
	as required.
\end{proof}

\subsection{Derivative of Rational Powers}
First, let \(g(x) = x^{1/q}\) for some positive integer \(q\).
We can find that \(f(x) = x^q\) has the derivative \(f'(x) = qx^{q-1}\).
By the inverse function theorem, \(g'(x) = \frac{1}{q}x^{1/q - 1}\).
Now, if \(g(x) = x^{p/q}\), where \(p\) is an integer and \(q\) is a positive integer, then by the chain rule \(g'(x) = \frac{p}{q}x^{p/q - 1}\) which matches the expected result.

\subsection{Mean Value Theorem Applied to Limits}
Suppose \(f, g \colon [a,b] \to \mathbb R\) are continuous, and differentiable on \((a, b)\).
Suppose further that \(g(a) \neq g(b)\).
The mean value theorem can be applied to both functions, and will give two points \(s, t \in (a, b)\) such that
\[
	\frac{f(b) - f(a)}{g(b) - g(a)} = \frac{(b-a)f'(s)}{(b-a)g'(t)} = \frac{f'(s)}{g'(t)}
\]
This gives us a way to simplify a limit of the form of the left hand side (as \(b \to a\)) by instead considering the right hand side.
We can apply Cauchy's mean value theorem, seen in the next lecture.
