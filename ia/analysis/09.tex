\subsection{Definitions}
Let \(f \colon E \subseteq \mathbb C \to \mathbb C\). Mostly we will take \(E\) to be an interval in the real numbers, or a disc in the complex plane.
\begin{definition}
	Let \(x \in E\) be a point such that there exists a sequence \(x_n \in E\) with \(x_n \neq x\), but \(x_n \to x\), i.e. \(x\) is a limit point. \(f\) is said to be differentiable at \(x\) with derivative \(f'(x)\) if
	\[ \lim_{y \to x} \frac{f(y) - f(x)}{y - x} = f'(x) \]
\end{definition}
\noindent If \(f\) is differentiable at each point in \(E\), we say that \(f\) is differentiable on \(E\).
\begin{remark}
	One interpretation of the definition is to write it in the form
	\[ \varepsilon(h) := f(x+h) - f(x) - hf'(x);\quad \lim_{h \to 0} \frac{\varepsilon(h)}{h} = 0 \]
	so \(\varepsilon\) is \(o(h)\). Hence,
	\[ f(x+h) = f(x) + hf'(x) + \varepsilon(h) \]
	We could have made an alternative definition for differentiability. \(f\) is differentiable at \(x\) if there exists \(A\) and \(\varepsilon\) such that
	\[ f(x+h) = f(x) + hA + \varepsilon(h) \text{ where } \lim_{h \to 0} \frac{\varepsilon(h)}{h} = 0 \]
	If such an \(A\) exists, then it is unique, since \(A\) is the limit
	\[ A = \lim_{h \to 0} \frac{f(x + h) - f(x)}{h} \]
	We could have alternatively written the definition as
	\[ f(x+h) = f(x) + hf'(x) + h\varepsilon_f(h) \text{ where } \lim_{h \to 0} \varepsilon_f(h) = 0 \]
	or perhaps
	\[ f(x) = f(a) + (x-a)f'(a) + (x-a)\varepsilon_f(x) \text{ where } \lim_{x \to a} \varepsilon_f(x) = 0 \]
	Note further that if \(f\) is differentiable at \(x\), \(f\) is certainly continuous at \(x\). This follows from the fact that \(\varepsilon(h) \to 0\), and hence \(f(x+h) \to f(x)\) as \(h \to 0\).
\end{remark}
\noindent As an example, let us consider \(f(x) = \abs{x}\) for \(f \colon \mathbb R \to \mathbb R\). Is the function at the point \(x=0\) differentiable? If \(x > 0\), we have \(f'(x) = 1\), but if \(x < 0\), we have \(f'(x) = -1\). These results can be checked directly using the definitions above. But we have produced two sequences for \(h \to 0\) which give different values, so the derivative is not defined here.

\subsection{Differentiation of Sums and Products}
\begin{proposition}
	\begin{enumerate}[(i)]
		\item If \(f(x) = c\) for all \(x \in E\), then \(f\) is differentiable with \(f'(x) = 0\).
		\item If \(f\) and \(g\) are differentiable at \(x\), then so is \(f+g\), where \((f+g)'(x) = f'(x) + g'(x)\).
		\item If \(f\) and \(g\) are differentiable at \(x\), then so is \(fg\), where \((fg)'(x) = f'(x)g(x) + g'(x)f(x)\).
		\item If \(f\) is differentiable at \(x\) and \(f(x) \neq 0\), then so is \(\frac{1}{f}\), where \((\frac{1}{f})'(x) = \frac{-f'(x)}{(f(x))^2}\).
	\end{enumerate}
\end{proposition}
\begin{proof}
	\begin{enumerate}[(i)]
		\item \(\lim_{h \to 0} \frac{c-c}{h} = 0\) as required.
		\item Since all relevant limits are well-defined,
		      \[ \lim_{h \to 0} \frac{f(x+h) + g(x+h) - f(x) - g(x)}{h} = \lim_{h \to 0} \frac{f(x+h) - f(x)}{h} + \lim_{h \to 0} \frac{g(x+h) - g(x)}{h} = f'(x) + g'(x) \]
		\item Let \(\phi(x) = f(x)g(x)\). Then, since \(f\) is continuous at \(x\),
		      \begin{align*}
			      \lim_{h \to 0} \frac{\phi(x+h) - \phi(x)}{h} & = \lim_{h \to 0} \frac{f(x+h)g(x+h) - f(x)g(x)}{h}                            \\
			                                                   & = \lim_{h \to 0} f(x+h) \frac{g(x+h) - g(x)}{h} + g(x)\frac{f(x+h) - f(x)}{h} \\
			                                                   & = \lim_{h \to 0} f(x) \frac{g(x+h) - g(x)}{h} + g(x)\frac{f(x+h) - f(x)}{h}   \\
			                                                   & = f(x)g'(x) + g(x)f'(x)
		      \end{align*}
		\item Let \(\phi(x) = \frac{1}{f(x)}\). Then,
		      \begin{align*}
			      \lim_{h \to 0} \frac{\phi(x+h) - \phi(x)}{h} & = \lim_{h \to 0} \frac{\frac{1}{f(x+h)} - \frac{1}{f(x)}}{h} \\
			                                                   & = \lim_{h \to 0} \frac{f(x) - f(x+h)}{hf(x)f(x+h)}           \\
			                                                   & = \frac{-f'(x)}{f(x)f(x)}                                    \\
		      \end{align*}
	\end{enumerate}
\end{proof}
\begin{remark}
	From (iii) and (iv), we can immediately find the quotient rule,
	\[ \left( \frac{f(x)}{g(x)} \right)' = \frac{g(x)f'(x) - f(x)g'(x)}{(g(x))^2} \]
\end{remark}
