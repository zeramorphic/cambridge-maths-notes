\documentclass{article}

\usepackage[UKenglish]{babel}
\usepackage[T1]{fontenc}
\usepackage[utf8]{inputenc}
\usepackage[a4paper]{geometry} % , margin=20mm
\usepackage{textcomp} % makes the "not defining \perthousand"/"\micro" errors go away by including this first
\usepackage{amsmath}
\usepackage{amssymb}
\usepackage{amsthm}
\usepackage{amsfonts}
\usepackage{bbm}
\usepackage{wrapfig}
\usepackage{physics}
\usepackage{bm}
\usepackage{tgpagella}
\DeclareDocumentCommand\mathbf{m}{\bm{\mathrm{#1}}} % make bold work for greek symbols
\DeclareDocumentCommand\vnabla{}{\nabla} % use non-bold nabla for \grad, \curl etc.
% Enabled to unify laplacian symbol between vector and scalar forms
\DeclareDocumentCommand\dotproduct{}{\cdot} % use non-bold dot for scalar product to unify notation
\DeclareDocumentCommand\crossproduct{}{\times} % use non-bold dot for scalar product to unify notation
\usepackage{gensymb}
\usepackage{enumerate}
\usepackage{mathtools}
\usepackage{centernot}
\usepackage{relsize}
\usepackage{mathrsfs}
\usepackage{siunitx}
\usepackage{booktabs}
\usepackage[ruled,vlined]{algorithm2e}
\usepackage{array}
\usepackage{multirow}
\usepackage{pgfplots}
\pgfplotsset{width=10cm,compat=1.9}
\usepgfplotslibrary{external}
\tikzexternalize[prefix=tikz/]
\usepackage[pdfa]{hyperref}
\hypersetup{
	colorlinks=true,
	linktoc=all,
	linkcolor=black,
}
\usepackage{minitoc}

\numberwithin{equation}{section} % make equations be numbered 1.1 not 1

\newcommand{\tableofcontentsnewpage}{\tableofcontents\newpage}

% create the theorem environments
\theoremstyle{definition}
\newtheorem*{definition}{Definition}

\newtheorem*{claim}{Claim}
\newtheorem*{theorem}{Theorem}
\newtheorem*{proposition}{Proposition}
\newtheorem*{lemma}{Lemma}
\newtheorem*{corollary}{Corollary}
\newtheorem*{example}{Example} % todo: convert `as an example...' to the example environment

\theoremstyle{remark}
\newtheorem*{note}{Note}
\newtheorem*{remark}{Remark}

\newcommand{\ddempty}{\mathrm{d}}
\newcommand{\dn}[2]{\mathrm{d}^#1#2}
\newcommand{\st}{\text{ s.t.
	}}
\newcommand{\contradiction}{\(\#\)}
\newcommand{\genset}[1]{\langle{} #1 \rangle}
\newcommand{\nhat}{\vu{n}}
\newcommand{\rdot}{\dot{\vb{r}}}
\newcommand{\rddot}{\ddot{\vb{r}}}
\newcommand{\transpose}{\intercal}
\newcommand{\acts}{\curvearrowright}
\newcommand{\adjugate}[1]{\widetilde{#1}}
\newcommand{\mathhuge}[1]{\mathlarger{\mathlarger{\mathlarger{#1}}}}
\newcommand{\stcomp}[1]{{#1}^c} % consider \complement?
% Personally I think this looks better, and it's what Wikipedia uses
\newcommand{\prob}[1]{\mathbb{P}\left({#1}\right)}
\newcommand{\psub}[2]{\mathbb{P}_{#1}\left({#2}\right)}
\newcommand{\psubx}[1]{\psub{x}{#1}}
\newcommand{\expect}[1]{\mathbb{E}\left[{#1}\right]}
\newcommand{\esub}[2]{\mathbb{E}_{#1}\left[{#2}\right]}
\newcommand{\esubx}[1]{\esub{x}{#1}}
\newcommand{\Var}[1]{\Varop\left({#1}\right)}
\newcommand{\Cov}[1]{\Covop\left({#1}\right)}
\newcommand{\Corr}[1]{\Corrop\left({#1}\right)}
\newcommand{\convdist}{\xrightarrow{d}}
\newcommand{\convprob}{\xrightarrow{\mathbb{P}}}
\newcommand{\wildcard}{{}\cdot{}}
\newcommand{\inner}[1]{\left\langle{#1}\right\rangle}
\newcommand{\Markov}[1]{\Markovop\left({#1}\right)}

\DeclareMathOperator{\vecspan}{span}
\DeclareMathOperator{\HCF}{HCF}
\DeclareMathOperator{\LCM}{LCM}
\DeclareMathOperator{\ord}{ord}
\DeclareMathOperator{\Sym}{Sym}
\DeclareMathOperator{\nullity}{null}
\DeclareMathOperator{\Orb}{Orb}
\DeclareMathOperator{\Stab}{Stab}
\DeclareMathOperator{\ccl}{ccl}
\DeclareMathOperator{\Varop}{Var}
\DeclareMathOperator{\Covop}{Cov}
\DeclareMathOperator{\Corrop}{Corr}
\DeclareMathOperator{\Markovop}{Markov}

\DeclarePairedDelimiter\ceil{\lceil}{\rceil}
\DeclarePairedDelimiter\floor{\lfloor}{\rfloor}

% for arrows in the middle of the line
\usetikzlibrary{decorations.markings}
\tikzset{->-/.style={decoration={
		markings,
		mark=at position #1 with {\arrow{>}}},postaction={decorate}}}


\title{Analysis}
\author{Cambridge University Mathematical Tripos: Part IA}

\begin{document}
\maketitle

\tableofcontentsnewpage{}

\section{Limits and convergence: reviewing Numbers and Sets}
\subsection{???}
Independence results are found across mathematical disciplines.
\begin{enumerate}
    \item The \emph{parallel postulate} is independent from the other four postulates of Euclidean geometry.
    It states that for any given point not on a line, there is a unique line passing through that point that does not intersect the given line.
    In the 19th century, it was shown that the other four postulates are satisfied by hyperbolic geometry, but this postulate is not satisfied.
    This shows that the other four axioms are insufficient to prove the parallel postulate.
    \item Let \( \varphi \) be the statement in the language of fields describing the existence of a square root of 2.
    We know that \( \mathbb Q \) is a field satisfying \( \neg\varphi \), and \( \mathbb Q[\sqrt{2}] \) satisfies \( \varphi \).
    The fields \( \mathbb Q \) and \( \mathbb Q[\sqrt{2}] \) are models of the theory of fields, one of which satisfies \( \varphi \), and one of which satisfies \( \neg\varphi \).
    This shows that the theory of fields does not prove \( \varphi \) or \( \neg\varphi \).
    A similar result holds for the statement \( \varphi \) that says that there are no roots of \( x^4 = -1 \).
    \item G\"odel's incompleteness theorem implies that there must always be an independence result in a sufficiently powerful consistent set theory.
\end{enumerate}
We will show that there are other independence results in set theory that are not self-referential like the G\"odel incompleteness theorems.
\begin{theorem}[Cantor]
    \( \abs{\mathbb N} < \abs{\mathcal P(\mathbb N)} \).
\end{theorem}
The continuum hypothesis is that there are no sets of cardinality strictly between \( \abs{\mathbb N} \) and \( \abs{\mathcal P(N)} = \abs{\mathbb R} \).
\begin{definition}
    The \emph{continuum hypothesis} \( \mathsf{CH} \) states that if \( X \subseteq \mathbb P(\mathbb N) \) is infinite, then either \( \abs{X} = \abs{\mathbb N} \) or \( \abs{X} = \abs{\mathcal P(\mathbb N)} \), or equivalently,
    \[ 2^{\aleph_0} = \aleph_1 \]
\end{definition}
Progress was made on the continuum hypothesis in the 19th and 20th centuries.
\begin{enumerate}
    \item In 1883, Cantor showed that any closed subset of \( \mathbb R \) satisfies \( \mathsf{CH} \).
    \item In 1916, Alexandrov and Hausdorff showed that any Borel set of \( \mathbb R \) satisfies \( \mathsf{CH} \).
    \item In 1930, Suslin strengthened this result to analytic subsets of \( \mathbb R \).
    \item In 1938, G\"odel showed that if \( \mathsf{ZF} \) is consistent, then so is \( \mathsf{ZFC} + \mathsf{CH} \).
    \item However, in 1963, Cohen showed that if \( \mathsf{ZF} \) is consistent, then so is \( \mathsf{ZFC} + \neg\mathsf{CH} \).
\end{enumerate}
In this course, we will prove results (iv) and (v), thus establishing the independence of the continuum hypothesis from \( \mathsf{ZFC} \).

\subsection{Systems of set theory}
The language of set theory \( \mathcal L = \mathcal L_\in \) is a first-order predicate logic with equality and membership as primitive relations.
We assume the existence of infinitely many variables \( v_1, v_2, \dots \) denoting sets.
We will only use the logical connectives \( \vee \) and \( \neg \) as well as the existential quantifier \( \exists \).
Conjunction, implication, and universal quantification can be defined in terms of disjunction, negation, and existential quantification.

We say that an occurrence of a variable \( x \) is \emph{bound} in a formula \( \varphi \) if is in a quantifier \( \exists x \) or lies in the scope of such a quantifier.
An occurrence is called \emph{free} if it is not bound.
We write \( FV(\varphi) \) for the set of free variables of \( \varphi \).
We will write \( \varphi(u_1, \dots, u_n) \) to emphasise the dependence of \( \varphi \) on its free variables \( u_1, \dots, u_n \).
By doing so, we will allow ourselves to freely change the names of the free variables, and assume that substituted variables are free.
The syntax \( \varphi(u_0, \dots, u_n) \) does not imply that \( u_i \) occurs freely, or even at all.

The axioms of set theory are as follows.
% TODO: Add them!

Some common set theories are as follows.
\begin{itemize}
    \item \emph{Zermelo set theory} \( \mathsf{Z} \) consists of axioms (i) to (viii).
    Axioms (ix) and (ix') are equivalent relative to \( \mathsf{Z} \).
    \item \emph{Zermelo--Fraenkel set theory} \( \mathsf{ZF} \) consists of axioms (i) to (ix).
    Axioms (x) and (x') are equivalent relative to \( \mathsf{ZF} \).
    \item \emph{Zermelo--Fraenkel set theory with choice} \( \mathsf{ZFC} \) consists of axioms (i) to (x).
    \item \emph{Zermelo--Fraenkel set theory without power set} \( \mathsf{ZF}^- \) consists of axioms (i) to (vii), with the axiom of collection (ix') instead of replacement (ix); it has been shown that (ix) is weaker than (ix') in the presence of axioms (i) to (vii).
    \item \emph{Zermelo--Fraenkel set theory with choice and without power set} \( \mathsf{ZFC}^- \) consists of axioms (i) to (vii), with the axiom of collection (ix') and the well-ordering principle (x').
\end{itemize}
In this course, our main metatheory will be \( \mathsf{ZF} \), and we will be explicit about the use of choice.

We say that a class \( X \) is \emph{definable} over \( M \) if there exists a formula \( \varphi \) and sets \( a_1, \dots, a_n \in M \) such that for all \( z \in M \), we have \( z \in X \) if and only if \( \varphi(z, a_1, \dots, a_n) \).
A class is \emph{proper} over \( M \) if it is not a set in \( M \).
In this course, we will assume that all mentioned classes are definable.
For example, the universe class \( V = \qty{x \mid x = x} \), the Russell class \( R = \qty{x \mid x \notin x} \), and the class of ordinals are all definable.
Any set is a definable class.
Classes are heavily dependent on the underlying model: if \( M = 2 \) then \( \mathrm{Ord} = 2 = M \), and if \( M = 3 \cup \qty{1} \) then \( \mathrm{Ord} = 3 \neq M \).

\subsection{Adding defined functions}
Often in set theory, we use symbols such as \( 0, 1, \subseteq, \cap, \wedge, \forall \); they do not exist in our language.
\begin{definition}
    Suppose that \( \mathcal L \subseteq \mathcal L' \) and \( T \) is a set of sentences in \( \mathcal L \).
    We say that \( P \) is a \emph{defined \( n \)-ary predicate} symbol over \( T \) if there is a formula \( \varphi \) in \( \mathcal L \) such that
    \[ T \vdash \forall x_1, \dots, x_n.\, (P(x_1, \dots, x_n) \iff \varphi(x_1, \dots, x_n)) \]
    Similarly, we say that \( f \) is a \emph{defined \( n \)-ary function} symbol over \( T \) if there is a formula \( \varphi \) in \( \mathcal L \) such that
    \[ f(x_1, \dots, x_n) = y \text{ if and only if } T \vdash \varphi(x_1, \dots, x_n, y) \]
    and
    \[ T \vdash \forall x_1, \dots, x_n.\, \exists! y.\, \varphi(x_1, \dots, x_n, y) \]
    We say that a set of sentences \( T' \) of \( \mathcal L' \) is an \emph{extension by definitions} of \( T \) over \( \mathcal L \) when \( T' = T \cup S \) and \( S = \qty{\varphi_s \mid s \in \mathcal L' \setminus \mathcal L'} \) and each \( \varphi_s \) is a definition of \( s \) in the language \( \mathcal L \) over \( T \).
\end{definition}
The following, among other things, are defined over \( \mathsf{ZF} \).
\[ 0 \quad 1 \quad \subseteq \quad \cap \quad \mathcal P \quad \bigcup \]
\begin{theorem}
    Suppose that \( \mathcal L \subseteq \mathcal L' \), and that \( T \) is a set of \( \mathcal L \)-sentences and \( T' \) is an extension by definitions of \( T \) to \( \mathcal L' \).
    Then
    \begin{enumerate}
        \item (conservativity) If \( \varphi \) is a sentence of \( \mathcal L \), then \( T \vdash \varphi \iff T' \vdash \varphi \).
        \item (abbreviations) If \( \varphi \) is a formula of \( \mathcal L' \), then there exists a formula \( \hat\varphi \) of \( \mathcal L \) whose free variables are exactly those of \( \varphi \), such that \( T' \vdash \forall x.\, (\varphi \iff \hat\varphi) \).
    \end{enumerate}
\end{theorem}
\begin{example}
    The intersection \( a \cap b \) can be defined as the unique set \( c \) such that
    \[ \forall x\. (x \in c \iff x \in a \wedge x \in b) \]
    This definition makes sense only if there is a unique \( c \) satisfying this formula \( \varphi(c) \).
    If
    \[ M = \qty{a, c, d, \qty{a}, \qty{a, b}, \qty{a, b, c}, \qty{a, b, d}} \]
    then it is easy to check that both \( \qty{a} \) and \( \qty{a, d} \) satisfy \( \varphi \), so intersection cannot be defined.
\end{example}

\section{More on convergence}
\subsection{Simple Markov property}
\begin{theorem}
	Suppose \( X \) is \( \Markov{\lambda, P} \).
	Let \( m \in \mathbb N \) and \( i \in I \).
	Given that \( X_m = i \), we have that the process after time \( m \), written \( (X_{m+n})_{n \geq 0} \), is \( \Markov{\delta_i, P} \), and it is independent of \( X_0, \dots, X_n \).
\end{theorem}
\noindent Informally, the past and the future are independent given the present.
\begin{proof}
	We must show that
	\[
		\prob{X_m = x_0, \dots, X_{m+n} = x_n \mid X_m = i} = \delta_{i x_0} P(x_0, x_1) \dots P(x_{n-1}, x_n)
	\]
	We have
	\[
		\prob{X_{m+n} = x_{m+n}, \dots, X_m = x_m \mid X_m = i}
		= \frac{\prob{X_{m+n} = x_{m+n}, \dots, X_m = x_m} \delta_{i x_m}}{\prob{X_m = i}}
	\]
	The numerator is
	\begin{align*}
		 & \prob{X_{m+n}, \dots, X_m = x_m}                                                                                         \\ & = \sum_{x_0, \dots, x_{m-1} \in I} \prob{X_{m+n} = x_{m+n}, \dots, X_m = x_m, X_{m-1} = x_{m-1}, \dots, X_0 = x_0}       \\
		 & = \sum_{x_0, \dots, x_{m-1}} \lambda_{x_0} P(x_0, x_1) \dots P(x_{m-1}, x_m) P(x_m, x_{m+1}) \dots P(x_{m+n-1}, x_{m+n}) \\
		 & = P(x_m, x_{m+1}) \dots P(x_{m+n-1}, x_{m+n}) \sum_{x_0, \dots, x_{m-1}} \lambda_{x_0} P(x_0, x_1) \dots P(x_{m-1}, x_m) \\
		 & = P(x_m, x_{m+1}) \dots P(x_{m+n-1}, x_{m+n}) \prob{X_m = x_m}                                                           \\
	\end{align*}
	Thus we have
	\[
		\prob{X_{m+n} = x_{m+n}, \dots, X_m = x_m \mid X_m = i}
		= P(x_m, x_{m+1}) \dots P(x_{m+n-1}, x_{m+n}) \delta_{i x_m}
	\]
	Hence \( (X_{m+n})_{n \geq 0} \sim \Markov{\delta_i, P} \) conditional on \( X_m = i \).
	Now it suffices to show independence between the past and future variables.
	In particular, we need to show \( m \leq i_1 < \dots < i_k \) for some \( k \in \mathbb N \) implies that
	\begin{align*}
		 & \prob{X_{i_1} = x_{m+1}, \dots, X_{i_k} = x_{m+k}, X_0 = x_0, \dots, X_m = x_m \mid X_m = i}                                               \\
		 & = \prob{X_{i_1} = x_{m+1}, \dots, X_{i_k} = x_{m+k} \mid X_m = i} \prob{X_0 = x_0, \dots, X_m = x_m \mid X_m = i}                          \\
		\intertext{So let \( i = x_m \), and then}
		 & = \frac{\prob{X_{i_1} = x_{m+1}, \dots, X_{i_k} = x_{m+k}, X_0 = x_0, \dots, X_m = x_m}}{\prob{X_m = i}}                                   \\
		 & = \frac{\lambda_{x_0} P(x_0, x_1) \dots P(x_{m-1}, x_m) \prob{X_{i_1} = x_{m+1}, \dots, X_{i_k} = x_{m+k} \mid X_m = x_m}}{\prob{x_m = i}} \\
		 & = \frac{\prob{X_0 = x_0, \dots, X_m = x_m}}{\prob{X_m = x_m}} \prob{X_{i_1} = x_{m+1}, \dots, X_{i_k} = x_{m+k} \mid X_m = x_m}
	\end{align*}
	which gives the result as required.
\end{proof}

\subsection{Powers of the transition matrix}
Suppose \( X \sim \Markov{\lambda, P} \) with values in \( I \).
If \( I \) is finite, then \( P \) is an \( \abs{I} \times \abs{I} \) square matrix.
In this case, we can label the states as \( 1, \dots, \abs{I} \).
If \( I \) is infinite, then we label the states using the natural numbers \( \mathbb N \).
Let \( x \in I \) and \( n \in \mathbb N \).
Then,
\begin{align*}
	\prob{X_n = x} & = \sum_{x_0, \dots, x_{n-1} \in I} \prob{X_n = x, X_{n-1} = x_{n-1}, \dots, X_0 = x_0} \\
	               & = \sum_{x_0, \dots, x_{n-1} \in I} \lambda_{x_0} P(x_0, x_1) \dots P(x_{n-1}, x)       \\
	\intertext{We can think of \( \lambda \) as a row vector.
		So we can write this as}
	               & = (\lambda P^n)_x
\end{align*}
By convention, we take \( P^0 = I \), the identity matrix.
Now, suppose \( m, n \in \mathbb N \).
By the simple Markov property,
\[
	\prob{X_{m+n} = y \mid X_m = x} = \prob{X_n = y \mid X_0 = x} = ( \delta_x P^n )_y
\]
We will write \( \psubx{A} := \prob{A \mid X_0 = x} \) as an abbreviation.
Further, we write \( p_{ij}(n) \) for the \( (i,j) \) element of \( P^n \).
We have therefore proven the following theorem.
\begin{theorem}
	\[
		\prob{X_n = x} = (\lambda P^n)_x;
	\]
	\[
		\prob{X_{n+m} = y \mid X_m = x} = \psubx{X_n = y} = p_{xy}(n)
	\]
\end{theorem}

\subsection{Calculating powers}
\begin{example}
	Consider
	\[
		P = \begin{pmatrix}
			1-\alpha & \alpha \\ \beta & 1-\beta
		\end{pmatrix};\quad \alpha, \beta \in [0,1]
	\]
	Note that for any stochastic matrix \( P \), \( P^n \) is a stochastic matrix.
	First, we have \( P^{n+1} = P^n P \).
	Let us begin by finding \( p_{11}(n+1) \).
	\[
		p_{11}(n+1) = p_{11}(n)(1-\alpha) + p_{12}(n)\beta
	\]
	Note that \( p_{11}(n) + p_{12}(n) = 1 \) since \( P^n \) is stochastic.
	Therefore,
	\[
		p_{11}(n+1) = p_{11}(n)(1-\alpha-\beta) + \beta
	\]
	We can solve this recursion relation to find
	\[
		p_{11}(n) = \begin{cases}
			\frac{\alpha}{\alpha + \beta} + \frac{\alpha}{\alpha + \beta}(1-\alpha-\beta)^n & \alpha + \beta > 0 \\
			1                                                                               & \alpha + \beta = 0\end{cases}
	\]
\end{example}
\noindent The general procedure for finding \( P^n \) is as follows.
Suppose that \( P \) is a \( k \times k \) matrix.
Then let \( \lambda_1, \dots, \lambda_k \) be its eigenvalues (which may not be all distinct).
\begin{enumerate}[(1)]
	\item All \( \lambda_i \) distinct.
	      In this case, \( P \) is diagonalisable, and hence we can write \( P = U D U^{-1} \) where \( U \) is a diagonal matrix, whose diagonal entries are the \( \lambda_i \).
	      Then, \( P^n = U D^n U^{-1} \).
	      Calculating \( D^n \) may be done termwise since \( D \) is diagonal.
	      In this case, we have terms such as
	      \[
		      p_{11}(n) = a_1 \lambda_1^n + \dots + a_k \lambda_k^n; \quad a_i \in \mathbb R
	      \]
	      First, note \( P^0 = I \) hence \( p_{11}(0) = 1 \).
	      We can substitute small values of \( n \) and then solve the system of equations.
	      Now, suppose \( \lambda_k \) is complex for some \( k \).
	      In this case, \( \overline{\lambda_k} \) is also an eigenvalue.
	      Then, up to reordering,
	      \[
		      \lambda_k = re^{i\theta} = r(\cos \theta + i \sin \theta); \lambda_{k-1} = \overline{\lambda_k} = re^{i\theta} = r(\cos \theta - i \sin \theta)
	      \]
	      We can instead write \( p_{11}(n) \) as
	      \[
		      p_{11}(n) = a_1 \lambda_1^n + \dots + a_{k-1} r^n \cos (n\theta) + a_k r^n \sin (n\theta)
	      \]
	      Since \( p_{11}(n) \) is real, all the imaginary parts disappear, so we can simply ignore them.
	\item Not all \( \lambda_i \) distinct.
	      In this case, \( \lambda \) appears with multiplicity 2, then we include also the term \( (an + b) \lambda^n \) as well as \( b \lambda^n \).
	      This can be shown by considering the Jordan normal form of \( P \).
\end{enumerate}

\section{Convergence tests}
\subsection{Definition of Integration}
We use a Riemann sum to approximate the area under a sufficiently well-behaved function \(f(x)\) on the real numbers.
\begin{equation}\label{riemannsum}
	\sum_{n=0}^{N-1} f(x_n) \Delta x
\end{equation}
where \(\Delta x = (b-a)/N\), and \(x_n = a + n\Delta x\).
How close is \eqref{riemannsum} to the area under \(f(x)\) for large \(N\)? Consider a specific rectangle in the Riemann sum by fixing \(n\).
The area under the curve in the \(n\)th rectangle and the area of the rectangle itself differ by a value we denote here as \(\epsilon\).
By computing \(\epsilon\)'s order of magnitude, we can show how much the total error deviates by.

\begin{theorem}[Mean Value Theorem]
	For a continuous function \(f(x)\),
	\begin{equation}\label{meanvaluetheorem}
		\int_{x_n}^{x_{n+1}} f(x) \dd{x} = f(x_c) \cdot (x_{n+1} - x_n)
	\end{equation}
	for some \(x_c\in (x_n, n_{n+1})\).
\end{theorem}
\noindent We use the Taylor Series of \(f(x)\) at \(x_n\) to compute a value for \(x_c\).
\[
	f(x_c) = f(x_n) + O(x_c - x_n)
\]
as \(x_c - x_n \to 0\).
Since \(x_n < x_c < x_{n+1}\), which implies \(\abs{x_{n+1} - x_n} > \abs{x_c - x_n}\), we can make the statement that
\[
	f(x_c) = f(x_n) + O(x_{n+1} - x_n)
\]
as \(x_{n+1} - x_n \to 0\).
Thus, by \eqref{meanvaluetheorem}
\[
	\int_{x_n}^{x_{n+1}} f(x) \dd{x} = \left[ f(x_n) + O(x_{n+1} - x_n) \right] (x_{n+1} - x_n)
\]
By defining \(\Delta x = x_{n+1} - x_n\), we have
\begin{equation}
	\int_{x_n}^{x_{n+1}} f(x) \dd{x} = \Delta x f(x_n) + O(\Delta x ^ 2)
\end{equation}
By rearranging, we can compute \(\epsilon\):
\[
	\epsilon = \abs{\int_{x_n}^{x_{n+1}} f(x) \dd{x} - \Delta x f(x_n)} = O(\Delta x ^ 2)
\]
Therefore it follows that
\[
	\int_{a}^{b} f(x) \dd{x} = \lim_{\Delta x \to 0} \left[ \left( \sum_{n=0}^{N-1} f(x_n) \Delta x \right) + O(N\Delta x^2) \right]
\]
Note that \(O(N\Delta x^2) = O((\frac{b-a}{N})^2 \cdot N) = O(1/N)\), so
\[
	\int_{a}^{b} f(x) \dd{x} = \lim_{N \to \infty} \left[ \left( \sum_{n=0}^{N-1} f(x_n) \Delta x \right) + O(1/N) \right]
\]
Which gives our final result of
\begin{equation}\label{definiteintegral}
	\int_{a}^{b} f(x) \dd{x} = \lim_{N \to \infty} \sum_{n=0}^{N-1} f(x_n) \Delta x`
\end{equation}

\subsection{Fundamental Theorem of Calculus}
Let \(F(x) = \int_{a}^{x} f(t) \dd{t}\).
From the definition of the derivative, we have
\begin{align*}
	\frac{\dd{F}}{\dd{x}} & = \lim_{h \to 0} \frac{1}{h} \left[ F(x+h) - F(x) \right]                                         \\
	                      & = \lim_{h \to 0} \frac{1}{h} \left[ \int_{a}^{x+h} f(t) \dd{t} - \int_{a}^{x} f(t) \dd{t} \right] \\
	                      & = \lim_{h \to 0} \frac{1}{h} \int_{x}^{x+h} f(t) \dd{t}                                           \\
	\intertext{Using \eqref{definiteintegral}:}
	                      & = \lim_{h \to 0} \frac{1}{h} \left[ hf(x) + O(h^2) \right]                                        \\
	                      & = \lim_{h \to 0} \left[ f(x) + O(h) \right]                                                       \\
	                      & = f(x)
\end{align*}
Therefore:
\begin{equation}\label{ftc}
	\frac{\ddempty}{\dd{x}} \left[ \int_{a}^{x} f(t) \dd{t} \right] = f(x)
\end{equation}

\subsection{Integration Techniques}
Three particularly important methods of integration are:
\begin{itemize}
	\item \(u\)-substitution,
	\item trigonometric substitutions, and
	\item integration by parts.
\end{itemize}
Of particular note is the trigonometric substitution method, since it can be difficult to work out exactly which substitution will yield the result.
A table is provided below.

\medskip\noindent\begin{tabular}{c c c}
	Identity                              & Term in Integrand  & Substitution        \\\midrule
	\(\cos^2 \theta + \sin^2 \theta = 1\) & \(\sqrt{1 - x^2}\) & \(x = \sin \theta\) \\
	\(1 + \tan^2 \theta = \sec^2 \theta\) & \(1 + x^2\)        & \(x = \tan \theta\) \\
	\(\cosh^2 u - \sinh^2 u = 1\)         & \(\sqrt{x^2 - 1}\) & \(x = \cosh u\)     \\
	\(\cosh^2 u - \sinh^2 u = 1\)         & \(\sqrt{x^2 + 1}\) & \(x = \sinh u\)     \\
	\(1 - \tanh^2 u = \sech^2 u\)         & \(1 - x^2\)        & \(x = \tanh u\)
\end{tabular}

\section{More convergence tests}
\subsection{Partial Derivatives}
We define the partial derivative of a two-valued function \(f(x, y)\) with respect to \(x\) (for example) by:
\begin{equation}
	\eval{\frac{\partial f}{\partial x}}_{y} = \lim_{\delta x\to 0} \frac{f(x + \delta x, y) - f(x, y)}{\delta x}
\end{equation}
For example, if \(f(x,y) = x^2 + y^3 + e^{xy^2}\), we have
\begin{align*}
	\eval{\frac{\partial f}{\partial x}}_{y}     & = 2x + y^2 e^{xy^2} \\
	\eval{\frac{\partial^2 f}{\partial x^2}}_{y} & = 2 + y^4 e^{xy^2}
\end{align*}
We can also define `cross-derivatives' by differentiating successively with respect to different variables, for example
\[
	\eval{\frac{\partial}{\partial y} \left( \eval{\frac{\partial f}{\partial x}}_{y} \right)}_{x} = 2ye^{xy^2} + 2xy^3e^{xy^2}
\]
The order of computation of cross-derivatives is irrelevant, provided that the required derivatives all exist.
\begin{equation}
	\frac{\partial^2 f}{\partial x \partial y} = \frac{\partial}{\partial x}\frac{\partial f}{\partial y} = \frac{\partial}{\partial y}\frac{\partial f}{\partial x} = \frac{\partial^2 f}{\partial y \partial x}
\end{equation}
We use a subscript shorthand to denote partial differentiation.
Where the point of evaluation of the derivative is not stated, it is implied to be fixed.
For example:
\[
	\eval{\frac{\partial f}{\partial x}}_{y} = \frac{\partial f}{\partial x} = f_x
\]
However, with a function \(f(x, y, z)\):
\[
	\eval{\frac{\partial f}{\partial x}}_{yz} \neq \eval{\frac{\partial f}{\partial x}}_{y}
\]
because \(z\) is not fixed.

\subsection{Multivariate Chain Rule}
In this section, all use of \(o\) notation is defined to be where all required \(\delta\) values approach 0.
We define the differential of a two-valued function \(f(x, y)\) to be
\begin{equation}\label{differential}
	\delta f = f(x + \delta x, y + \delta y) - f(x, y)
\end{equation}
We can evaluate this differential by rewriting \eqref{differential} as
\begin{align*}
	\delta f =\  & f(x + \delta x, y + \delta y) - f(x + \delta x, y)\ + \\
	             & f(x + \delta x, y) - f(x, y)
\end{align*}
We can move from \((x, y)\) to \((x + \delta x, y + \delta y)\) along the path \((x, y) \to (x + \delta x, y) \to (x + \delta x, y + \delta y)\).
If we move in this way, then we only need to worry about derivatives in the directions of our axes.
From here on in the derivation, the first line will always represent the path segment in the \(y\) direction, and the second line will represent the path segment in the \(x\) direction.

Now that we've separated the differential into these two axes, we can use Taylor series, treating each line as a single-valued function, to expand each of these path segments along the matching axis.
\begin{align*}
	\delta f =\  & f(x + \delta x, y) + \delta y\frac{\partial f}{\partial y}(x + \delta x, y) + o(\delta y) - f(x + \delta x, y)\ + \\
	             & f(x, y) + \delta x \frac{\partial f}{\partial x}(x, y) + o(\delta x) - f(x, y)
\end{align*}
We can now cancel the beginning and ending points of each segment of the path, leaving
\begin{align*}
	\delta f =\  & \delta y\frac{\partial f}{\partial y}(x + \delta x, y)\ + o(\delta y)+ \\
	             & \delta x \frac{\partial f}{\partial x}(x, y) + o(\delta x)
\end{align*}
We can reduce the remaining \(x+\delta x\) term to simply an \(x\) term by performing another Taylor expansion.
\begin{align*}
	\delta f =\  & \delta y\left[ \frac{\partial f}{\partial y}(x, y) + \delta x\frac{\partial^2 f}{\partial y^2}(x, y) + o(\delta x) \right] + o(\delta y)\ + \\
	             & \delta x \frac{\partial f}{\partial x}(x, y) + o(\delta x)
\end{align*}
Expanding out this bracket leaves
\begin{align*}
	\delta f =\  & \delta y\frac{\partial f}{\partial y}(x, y) + \delta x\delta y\frac{\partial^2 f}{\partial y^2}(x, y) + o(\delta x \delta y) + o(\delta y)\ + \\
	             & \delta x \frac{\partial f}{\partial x}(x, y) + o(\delta x)
\end{align*}
We will now change the meanings of each line.
Now, we will group terms by factors.
\begin{align*}
	\delta f =\  & \delta x \frac{\partial f}{\partial x}(x, y) + o(\delta x)\ +                  \\
	             & \delta y\frac{\partial f}{\partial y}(x, y) + o(\delta y)\ +                   \\
	             & \delta x\delta y\frac{\partial^2 f}{\partial y^2}(x, y) + o(\delta x \delta y)
\end{align*}
Because \(o(h)\) is significantly smaller than \(h\), we can eliminate all the \(o\) terms.
\begin{align*}
	\delta f =\  & \delta x \frac{\partial f}{\partial x}(x, y)\ +         \\
	             & \delta y\frac{\partial f}{\partial y}(x, y)\ +          \\
	             & \delta x\delta y\frac{\partial^2 f}{\partial y^2}(x, y)
\end{align*}
Finally, we can eliminate the \(\delta x \delta y\) term because it is vanishingly small as they tend to zero.
\begin{equation}
	\delta f = \delta x \frac{\partial f}{\partial x}(x, y) +
	\delta y\frac{\partial f}{\partial y}(x, y)
\end{equation}
This is the differential form of the multivariate chain rule.
We can take the result of this equation in the limit to create the infinitesimal form:
\begin{equation}\label{mvcr}
	\dd{f} = \dd{x} \frac{\partial f}{\partial x}(x, y) +
	\dd{y}\frac{\partial f}{\partial y}(x, y)
\end{equation}

\subsection{Integral form of Multivariate Chain Rule}
By integrating \eqref{mvcr}, we get
\[
	\int \dd{f} = \int \frac{\partial f}{\partial x}\ \dd{x} + \int \frac{\partial f}{\partial y}\ \dd{y}
\]
In definite integral form, we can write
\begin{align*}
	f(x_2 - x_1, y_2 - y_1) & = \int_{x_1}^{x_2} \frac{\partial f}{\partial x}(x, y_1)\ \dd{x} + \int_{y_1}^{y_2} \frac{\partial f}{\partial y}(x_2, y)\ \dd{y}    \\
	                        & = \int_{y_1}^{y_2} \frac{\partial f}{\partial y}(x_1, y)\ \dd{y} + \int_{x_1}^{x_2} \frac{\partial f}{\partial x}(x, y_2)\ \dd{x}    \\
	                        & \neq \int_{x_1}^{x_2} \frac{\partial f}{\partial x}(x, y_1)\ \dd{x} + \int_{y_1}^{y_2} \frac{\partial f}{\partial y}(x_1, y)\ \dd{y}
\end{align*}
Note that the first two examples of a right hand side go along the paths \((x_1, y_1) \to (x_2, y_1) \to (x_2, y_2)\) and \((x_1, y_1) \to (x_1, y_2) \to (x_2, y_2)\) by performing the integrals.
However, the last example does not follow a path from \((x_1, y_1)\) to \((x_2, y_2)\), so it is invalid.

\section{Absolute convergence}
\subsection{Space of linear maps}
Let \( V \) and \( W \) be \( F \)-vector spaces.
Consider the space of linear maps from \( V \) to \( W \).
Then \( L(V,W) = \qty{\alpha \colon V \to W \text{ linear}} \).
\begin{proposition}
	\( L(V,W) \) is an \( F \)-vector space under the operation
	\[
		(\alpha_1 + \alpha_2)(v) = \alpha_1(v) + \alpha_2(v);
	\]
	\[
		(\lambda \alpha)(v) = \lambda( \alpha(v) )
	\]
	Further, if \( V \) and \( W \) are finite-dimensional, then so is \( L(V,W) \) with
	\[
		\dim_F L(V,W) = \dim_F V \dim_F W
	\]
\end{proposition}
\begin{proof}
	Proving that \( L(V,W) \) is a vector space is left as an exercise.
	The dimensionality part is proven later.
\end{proof}

\subsection{Matrices}
\begin{definition}
	An \( m \times n \) matrix over \( F \) is an array of \( m \) rows and \( n \) columns, with entries in \( F \).
\end{definition}
We write \( M_{m \times n}(F) \) for the set of \( m \times n \) matrices over \( F \).
\begin{proposition}
	\( M_{m \times n}(F) \) is an \( F \)-vector space under
	\[
		((a_{ij}) + (b_{ij})) = (a_{ij} + b_{ij});
	\]
	\[
		\lambda (a_{ij}) = (\lambda a_{ij})
	\]
\end{proposition}
\begin{proposition}
	\( \dim_F M_{m,n}(F) = m n \).
\end{proposition}
\begin{proof}
	Consider the basis defined by, the `elementary matrix' for all \( i,j \):
	\[
		e_{pq} = \delta_{ip}\delta_{jq}
	\]
	Then \( (e_{ij}) \) is a basis of \( M_{m \times n}(F) \), since it spans \( M_{m \times n}(F) \) and we can show that it is free.
\end{proof}

\subsection{Linear maps as matrices}
Consider bases \( B \) of \( V \) and \( C \) of \( W \):
\[
	B = (v_1, \dots, v_n); C = (w_1, \dots, w_n)
\]
Then let \( v \in V \).
We have
\[
	v = \sum_{j=1}^n \lambda_j v_j \equiv [v]_B = \begin{pmatrix}
		\lambda_1 \\ \vdots \\ \lambda_n
	\end{pmatrix} \in F^n
\]
where the vector given is the coordinates in basis \( B \).
We can equivalently find \( [w]_C \), the coordinates of \( w \) in basis \( C \).
We can now define a matrix of some linear map \( \alpha \) in the \( B, C \) basis.
\begin{definition}
	\[
		[\alpha]_{B,C} = \begin{pmatrix}
			[\alpha(v_1)]_C, \dots, [\alpha(v_n)]_C
		\end{pmatrix} \in M_{m\times n}(F)
	\]
\end{definition}
Note that if \( [\alpha]_{BC} = (a_{ij}) \), then by definition
\[
	\alpha (v_j) = \sum_{i=1}^n a_{ij} w_i
\]
\begin{lemma}
	For all \( v \in V \),
	\[
		[\alpha(v)]_C = [\alpha]_{BC} \cdot [v]_{B}
	\]
\end{lemma}
\begin{proof}
	We have
	\[
		v = \sum_{i=1}^n \lambda_j v_j
	\]
	Hence
	\[
		\alpha\qty(\sum_{i=1}^n \lambda_j v_j) = \sum_{j=1}^n \lambda_j \alpha(v_j) = \sum_{j=1}^n \lambda_i \sum_{i=1}^m a_{ij} w_i = \sum_{i=1}^m \qty( \sum_{j=1}^n a_{ij} \lambda_j ) w_i
	\]
\end{proof}
\begin{lemma}
	Let \( \beta \colon U \to V \) and \( \alpha \colon V \to W \) bea linear.
	Then, if \( A,B,C \) are bases of \( U,V,W \) respectively, then
	\[
		[\alpha \circ \beta]_{A,C} = [\alpha]_{B,C} \cdot [\beta]_{A,B}
	\]
\end{lemma}
\begin{proof}
	Consider \( u \in A \).
	Then
	\[
		(\alpha \circ \beta)(u) = \alpha(\beta(u))
	\]
	giving
	\[
		\alpha\qty(\sum_j b_{jp} v_i) = \sum_j b_{jp} \alpha(v_j) = \sum_j b_{jp} \sum_i a_{ij} w_i = \sum_i ( \sum_j a_{ij} b_{jp} ) w_i
	\]
	where \( a_{ij} p_{jp} \) is the \( (i,j) \) element of \( AB \) by the definition of the product of matrices.
\end{proof}
\begin{proposition}
	If \( V, W \) are \( F \)-vector spaces, and \( \dim V = n, \dim W = m \), then
	\[
		L(V,W) \simeq M_{m \times n}(F)
	\]
	which implies the dimensionailty of \( L(V,W) \) in \( F \) is \( m \times n \).
\end{proposition}
\begin{proof}
	Consider two bases \( B, C \) of \( V, W \).
	We claim that
	\[
		\theta \colon L(V,W) \to M_{m \times n}(F)
	\]
	defined by \( \theta(\alpha) = [\alpha]_{B,C} \).
	is an isomorphism.
	First, note that \( \theta \) is linear.
	Then, \( \theta \) is surjective; consider any matrix \( A = (a_{ij}) \) and consider \( \alpha \colon v_j \mapsto \sum_{i=1}^m a_{ij} w_i \).
	Then this is certainly a linear map which extends uniquely by linearity to \( A \), giving \( [\alpha]_{B,C} = (a_{ij}) = A \).
	Now, \( \theta \) is injective since \( [\alpha]_{B,C} = 0 \implies \alpha = 0 \).
\end{proof}
\begin{remark}
	If \( B,C \) are bases of \( V,W \) respectively, and \( \varepsilon_B \colon V \to F^n \) is defined by \( v \mapsto [v]_B \), and analogously for \( \varepsilon_C \), then
	\[
		[\alpha]_{B,C} \circ \varepsilon_B = \varepsilon_C \circ \alpha
	\]
	so the operations commute.
\end{remark}
\begin{example}
	Let \( \alpha \colon V \to W \) be a linear map and \( Y \leq V \), where \( V, W \) are finite-dimensional.
	Then let \( \alpha(Y) = Z \leq W \).
	Consider a basis \( B \) of \( V \), such that \( B' = (v_1, \dots, v_k) \) is a basis of \( Y \) completed by \( B'' = (v_{k+1}, \dots, v_n) \) into \( B = B' \cup B'' \).
	Then let \( C \) be a basis of W, such that \( C' = (w_1, \dots, w_\ell) \) is a basis of \( Z \) completed by \( C'' = (w_{\ell + 1}, \dots, w_m) \) into \( C = C' \cup C'' \).
	Then
	\[
		[\alpha]_{B,C} = \begin{pmatrix}
			\alpha(v_1) & \dots & \alpha(v_k) & \alpha(v_{k+1}) & \dots & \alpha(v_n)
		\end{pmatrix}
	\]
	For \( 1 \leq i \leq k \), \( \alpha(v_i) \in Z \) since \( v_i \in Y, \alpha(Y) = Z \).
	So the matrix has an upper-left \( \ell \times k \) block \( A \) which is \( \alpha \colon Y \to Z \) on the basis \( B', C' \).
	We can show further that \( \alpha \) induces a map \( \overline{\alpha} \colon V / Y \to W / Z \) by \( v + Y \mapsto \alpha(v) + Z \).
	This is well-defined; \( v_1 + Y = v_2 + Y \) implies \( v_1 - v_2 \in Y \) hence \( \alpha(v_1 - v_2) \in Z \) as required.
	The bottom-right block is \( [\overline{\alpha}]_{B'', C''} \).
\end{example}

\section{Continuity}
\subsection{Stationary states}
\begin{definition}
	With the ansatz \( \psi(x,t) = \chi(x) T(t) \), we have found a particular class of solutions of the time-independent Schr\"odinger equation:
	\[
		\psi(x,t) = \chi(x) e^{-\frac{i E t}{\hbar}}
	\]
	where \( \chi(x) \) are the eigenfunctions of \( \hat H \) with eigenvalue \( E \).
	Such solutions are called stationary states.
\end{definition}
\noindent Note,
\[
	\rho(x,t) = \abs{\psi(x,t)}^2 = \abs{\chi(x)}
\]
This explains the naming of the states as `stationary', as their probability density is independent of time.
Now, suppose \( E \) is quantised.
Then, the general solution to the system is
\[
	\psi(x,t) = \sum_{n=1}^N a_n \chi_n(x) e^{-\frac{iE_n t}{\hbar}}
\]
where \( N \) can be finite or infinite.
In principle, we can also have a continuous energy state \( E_\alpha, \alpha \in \mathbb R \).
We can still use the same idea:
\[
	\psi(x,t) = \int_{\Delta \alpha} A(\alpha) \chi_\alpha(x) e^{-\frac{iE_\alpha t}{\hbar}} \dd{\alpha}
\]
Note that \( \abs{a_n}^2 \) and \( A(\alpha) \dd{\alpha} \) give the probability of measuring the particle energy to be \( E_n \) or \( E_\alpha \).

\subsection{One-dimensional solutions to Schr\"odinger equation}
Recall the equation
\[
	\hat H \chi(x) = -\frac{\hbar^2}{2m} \chi''(x) + U(x) \chi(x) = E \chi(x)
\]
We will solve this equation for:
\begin{itemize}
	\item bound states, such as potential wells;
	\item unbound states, including free particles and scattering of the potential;
	\item harmonic oscillators.
\end{itemize}

\subsection{Infinite potential well}
We define
\[
	U(x) = \begin{cases}
		0      & \text{for } \abs{x} \leq a \\
		\infty & \text{for } \abs{x} < a
	\end{cases}
\]
For \( \abs{x} > a \), we must have \( \chi(a) = 0 \).
Otherwise, \( \chi \cdot U = \infty \).
This gives us a boundary condition, \( \chi(\pm a) = 0 \).
For \( \abs{x} \leq a \), we seek solutions of the form
\[
	-\frac{\hbar^2}{2m} \chi''(x) = E \chi(x);\quad \chi(\pm a) = 0
\]
Equivalently,
\[
	\chi''(x) + k^2 \chi(x) = 0;\quad k = \sqrt{\frac{2mE}{\hbar^2}}
\]
Since \( E > 0 \),
\[
	\chi(x) = A \sin kx + B \cos kx
\]
Imposing boundary conditions,
\[
	A \sin ka + B \cos ka = 0;\quad A \sin ka - B \cos ka = 0
\]
Suppose \( A = 0 \), giving \( \chi(x) = B \cos kx \).
Then, imposing boundary conditions, \( \chi_n(x) = B \cos k_n x \) where \( k_n = \frac{n \pi}{2a} \), and \( n \) are odd positive integers.
These are even solutions.

Alternatively, suppose \( B = 0 \).
In this case, \( \chi(x) = A \sin kx \).
Thus, \( \chi_n(x) = A \sin k_n x \) where \( k_n = \frac{n \pi}{2a} \), and \( n \) are even non-zero positive integers.
These provide odd solutions.

We can also determine the normalisation constants by defining that the eigenfunctions of the Hamiltonian are normalised to unity.
Thus,
\[
	\int_{-a}^a \abs{\chi_n(x)}^2 = 1 \implies A = B = \sqrt{\frac{1}{a}}
\]
Hence, the general solution is given by the eigenvalues
\[
	E_n = \frac{\hbar^2}{2n} k_n^2 = \frac{\hbar^2 \pi^2 n^2}{2ma^2}
\]
and eigenfunctions
\[
	\chi_n(x) = \sqrt{\frac{1}{a}} \begin{cases}
		\cos(\frac{n \pi x}{2a}) & \text{if } n \text{ odd}  \\
		\sin(\frac{n \pi x}{2a}) & \text{if } n \text{ even}
	\end{cases}
\]
\begin{remark}
	Note that unlike classical mechanics, the ground state energy is not zero.
	Note also that \( \chi_n \) have \( (n+1) \) nodes in which \( \rho(x) = 0 \).
	When \( n \to \infty \), \( \rho_n(x) \) tends to a constant, which is like in classical mechanics.
	Eigenfunctions of the Hamiltonian in this case were either odd or even; we can in fact prove that this is the case in general.
\end{remark}
\begin{proposition}
	If we have a system of non-degenerate eigenstates (\( E_i \neq E_j \)),  then if \( U(x) = U(-x) \) the eigenfunctions of \( \hat H \) must be either odd or even.
\end{proposition}
\begin{proof}
	The time-independent Schr\"odinger equation is invariant under \( x \mapsto -x \) if \( U \) is even.
	Hence, if \( \chi(x) \) is a solution with eigenvalue \( E \), then \( \chi(-x) \) is also a solution.
	Since we have a non-degenerate solution, \( \chi(-x) = \chi(x) \) hence the solutions must be the same up to a normalisation factor.
	For consistency, \( \chi(x) = \chi(-(-x)) = \alpha \chi(-x) = \alpha^2 \chi(x) \).
	Hence \( \alpha = \pm 1 \), so \( \chi \) is either odd or even.
\end{proof}

\subsection{Finite potential well}
We define
\[
	U(x) = \begin{cases}
		0   & \text{for } \abs{x} \leq a \\
		U_0 & \text{for } \abs{x} < a
	\end{cases}
\]
Classically, if \( E < U_0 \), the particle has insufficient energy to escape the well.
We will only consider eigenstates with \( E < U_0 \) here, but we will find that it is possible in quantum mechanics to escape the well with positive probability.
We will search for even functions only, odd functions can be solved independently.
If \( \abs{x} \leq a \),
\[
	-\frac{\hbar^2}{2m} \chi''(x) = E\chi(x)
\]
Equivalently,
\[
	\chi''(x) + k^2 \chi(x) = 0;\quad k = \sqrt{\frac{2mE}{\hbar^2}}
\]
The solution becomes
\[
	\chi(x) = A \sin kx + B \cos kx \implies \chi(x) = B \cos kx
\]
since we are only looking for even solutions.
In the region \( \abs{x} > a \),
\[
	-\frac{\hbar^2}{2m} \chi''(x) + U_0 \chi(x) = E \chi(x)
\]
giving
\[
	\chi''(x) - \overline k^2 \chi(x) = 0;\quad \overline k = \sqrt{\frac{2m(U_0 - E)}{\hbar^2}}
\]

\section{Limit of a function}
\subsection{Separation of variables}
We wish to solve the wave equation subject to certain boundary and initial conditions.
Consider a possible solution of separable form:
\[
	y(x,t) = X(x) T(t)
\]
Substituting into the wave equation,
\[
	\frac{1}{c^2} \ddot y = y'' \implies \frac{1}{c^2} X \ddot T = X'' T
\]
Then
\[
	\frac{1}{c^2}\frac{\ddot T}{T} = \frac{X''}{X}
\]
However, \( \frac{\ddot T}{T} \) depends only on \( t \) and \( \frac{X''}{X} \) depends only on \( x \).
Thus, both sides must be equal to some \textit{separation constant} \( -\lambda \).
\[
	\frac{1}{c^2}\frac{\ddot T}{T} = \frac{X''}{X} = -\lambda
\]
Hence,
\[
	X'' + \lambda X = 0;\quad \ddot T + \lambda c^2 T = 0
\]

\subsection{Boundary conditions and normal modes}
We will begin by first solving the spatial part of the solution.
One of \( \lambda > 0, \lambda < 0, \lambda = 0 \) must be true.
The boundary conditions restrict the possible \( \lambda \).
\begin{enumerate}[(i)]
	\item First, suppose \( \lambda < 0 \).
	      Take \( \chi^2 = -\lambda \).
	      Then,
	      \[
		      X(x) = Ae^{\chi x} + Be^{-\chi x} = C \cosh (\chi x) + D \sinh (\chi x)
	      \]
	      The boundary conditions are \( x(0) = x(L) = 0 \), so only the trivial solution is possible: \( C = D = 0 \).
	\item Now, suppose \( \lambda = 0 \).
	      Then
	      \[
		      X(x) = Ax + B
	      \]
	      Again, the boundary conditions impose \( A = B = 0 \) giving only the trivial solution.
	\item Finally, the last possibility is \( \lambda > 0 \).
	      \[
		      X(x) = A \cos \qty(\sqrt{\lambda} x) + B \sin \qty(\sqrt{\lambda} x)
	      \]
	      The boundary conditions give
	      \[
		      A = 0;\quad B \sin \qty(\sqrt{\lambda} L) = 0 \implies \sqrt{\lambda} L = n \pi
	      \]
\end{enumerate}
\noindent The following are the eigenfunctions and eigenvalues.
\[
	X_n(x) = B_n \sin \frac{n \pi x}{L};\quad \lambda_n = \qty(\frac{n \pi}{L})^2
\]
These are also called the `normal modes' of the system.
The spatial shape in \( x \) does not change in time, but the amplitude may vary.
The fundamental mode is the lowest frequency of vibration, given by
\[
	n = 1 \implies \lambda_1 = \frac{\pi^2}{L^2}
\]
The second mode is the first overtone, and is given by
\[
	n = 2 \implies \lambda_2 = \frac{4\pi^2}{L^2}
\]

\subsection{Initial conditions and temporal solutions}
Substituting \( \lambda_n \) into the time ODE,
\[
	\ddot T + \frac{n^2 \pi^2 c^2}{L^2}T = 0
\]
Hence,
\[
	T_n(t) = C_n \cos \frac{n \pi c t}{L} + D_n \sin \frac{n \pi c t}{L}
\]
Therefore, a specific solution of the wave equation satisfying the boundary conditions is (absorbing the \( B_n \) into the \( C_n, D_n \)):
\[
	y_n(x,t) = T_n(t) X_n(x) = \qty(C_n \cos \frac{n \pi c t}{L} + D_n \sin \frac{n \pi c t}{L}) \sin \frac{n \pi x}{L}
\]
To find a particular solution for a given set of initial conditions, we must consider a linear superposition of all possible \( y_n \).
\[
	y(x,t) = \sum_{n=1}^\infty \qty(C_n \cos \frac{n \pi c t}{L} + D_n \sin \frac{n \pi c t}{L}) \sin \frac{n \pi x}{L}
\]
By construction, this \( y(x,t) \) satisfies the boundary conditions, so now we can impose the initial conditions.
\[
	y(x,0) = p(x) = \sum_{n=1}^\infty C_n \sin \frac{n \pi x}{L}
\]
We can find the \( C_n \) using standard Fourier series techniques, since this is exactly a half-range sine series.
Further,
\[
	\pdv{y(x,0)}{t} = q(x) = \sum_{n=1}^\infty \frac{n \pi c}{L} D_n \sin \frac{n \pi x}{L}
\]
Again we can solve for the \( D_n \) in a similar way.
In particular,
\[
	C_n = \frac{2}{L} \int_0^L p(x) \sin \frac{n \pi x}{L} \dd{x}
\]
\[
	D_n = \frac{2}{n \pi c} \int_0^L q(x) \sin \frac{n \pi x}{L} \dd{x}
\]
\begin{example}
	Consider the initial condition of a see-saw wave parametrised by \( \xi \), and let \( L = 1 \).
	This can be visualised as plucking the string at position \( \xi \).
	\[
		y(x,0) = p(x) = \begin{cases}
			x(1-\xi) & 0 \leq x < \xi \\
			\xi(1-x) & \xi \leq x < 1
		\end{cases}
	\]
	We also define
	\[
		\pdv{y(x,0)}{t} = q(x) = 0
	\]
	The Fourier series for \( p \) is given by
	\[
		C_n = \frac{2 \sin n \pi \xi}{(n \pi)^2};\quad D_n = 0
	\]
	Hence the solution to the wave equation is
	\[
		y(x,t) = \sum_{n=1}^\infty \frac{2}{(n \pi)^2} \sin n \pi \xi \sin n \pi x \cos n \pi c t
	\]
\end{example}

\subsection{Separation of variables methodology}
A general strategy for solving higher-dimensional partial differential equations is as follows.
\begin{enumerate}[(i)]
	\item Obtain a linear PDE system, using boundary and initial conditions.
	\item Separate variables to yield decoupled ODEs.
	\item Impose homogeneous boundary conditions to find eigenvalues and eigenfunctions.
	\item Use these eigenvalues (constants of separation) to find the eigenfunctions in the other variables.
	\item Sum over the products of separable solutions to find the general series solution.
	\item Determine coefficients for this series using the initial conditions.
\end{enumerate}
\begin{example}
	We will solve the wave equation instead in characteristic coordinates.
	Recall the sine and cosine summation identities:
	\begin{align*}
		y(x,t) & = \frac{1}{2} \sum_{n=1}^\infty \Bigg[ \qty(C_n \sin \frac{n \pi}{L}(x-ct) + D_n \cos \frac{n \pi}{L}(x-ct)) \\&
		+ \qty(C_n \sin \frac{n \pi}{L}(x+ct) - D_n \cos \frac{n \pi}{L}(x+ct)) \Bigg]                                        \\
		       & = f(x-ct) + g(x+ct)
	\end{align*}
	The standing wave solution can be interpreted as a superposition of a right-moving wave and a left-moving wave.
	A special case is \( q(x) = 0 \), implying \( f = g = \frac{1}{2} p \).
	Then,
	\[
		y(x,t) = \frac{1}{2}\qty[p(x-ct) + p(x+ct)]
	\]
\end{example}

\section{Bounds and inverses}
\subsection{Choosing Bases}
Note that $\{ \vb 0 \}$ is a trivial subspace of all vector spaces, and it has dimension zero since it requires a linear combination of no vectors.

\begin{proposition}
	Let $V$ be a vector space with finite subsets $Y = \{ \vb w_1, \cdots, \vb w_m \}$ that spans $V$, and $X = \{ \vb u_1, \cdots, \vb u_k \}$ that is linearly independent. Let $n = \dim V$. Then:
	\begin{enumerate}[(i)]
		\item A basis can be found as a subset of $Y$ by discarding vectors in $Y$ as necessary, and that $n \leq m$.
		\item $X$ can be extended to a basis by adding in additional vectors from $Y$ as necessary, and that $k \leq n$.
	\end{enumerate}
\end{proposition}
\begin{proof}
	This proof is non-examinable (without prompts).
	\begin{enumerate}[(i)]
		\item If $Y$ is linearly independent, then $Y$ is a basis and $m = n$. Otherwise, $Y$ is not linearly independent. So there exists some linear relation
		      \[ \sum_{i=1}^{m} \lambda_i \vb w_i = \vb 0 \]
		      where there is some $i$ such that $\lambda_i \neq 0$. Without loss of generality (because the order of elements in $Y$ does not matter) we will reorder $Y$ such that $\vb w_m \neq 0$. So we have
		      \[ \vb w_m = \frac{-1}{\lambda_m} \sum_{i=1}^{m-1} \lambda_i \vb w_i \]
		      So $\vecspan Y = \vecspan (Y \setminus \{ \vb w_m \})$. We can repeat this process of eliminating vectors from $Y$ until linear independence is achieved. We know that this process will end because $Y$ is a finite set. Clearly, in this case, $n < m$. So for all cases, $n \leq m$.

		\item If $X$ spans $V$, then $X$ is a basis and $k=n$. Else, there exists some $u_{k+1} \in V$ that is not in the span of $X$. Then, we will construct an arbitrary linear relation
		      \[ \sum_{i=1}^{k+1} \mu_i \vb u_i = \vb 0 \]
		      Note that this implies that $\mu_{k+1} = \vb 0$ because it is not in the span of $X$, and that $\mu_i = 0$ for all $i \leq k$ because the original $X$ was linearly independent. So we know that all the coefficients are zero, and therefore $X \cup \{ u_{k+1} \}$ is linearly independent.

		      Note that we can always choose this $u_{k+1}$ to be an element of $Y$ because we just need to ensure that $u_{k+1} \notin \vecspan X$. Suppose we cannot choose such a vector in $Y$. Then $Y \subseteq \vecspan X \implies \vecspan Y \subseteq \vecspan X \implies \vecspan X = V$, which is clearly false because $X$ does not span $V$. This is a contradiction, so we can always choose such a vector from $Y$. We can repeat this process of taking vectors from $Y$ and adding them to $X$ until we have a basis. This process will always terminate in a finite amount of steps because we are taking new vectors from a finite set $Y$. Therefore $k \leq n$, as we are adding vectors (increasing $k$) until $k=n$.
	\end{enumerate}
\end{proof}

\subsection{Infinite Dimensions}
It is perfectly possible to have a vector space that has infinite dimensionality. However, they will be rarely touched upon in this course apart from specific examples, like the following example. Let $V = \{ f: [0, 1] \to \mathbb R: f \text{ smooth}, f(0) = f(1) = 0\}$. Then let $S_n(x) = \sqrt 2 \sin(n \pi x)$ where $n$ is a natural number $1, 2, \cdots$. Clearly, $S_n \in V$ for all $n$. The inner product of two of these $S$ functions is given by
\begin{align*}
	\langle S_n, S_m \rangle & = 2 \int_0^1 \sin(n \pi x) \sin(m \pi x) \dd{x} \\
	                         & = \delta_{mn}
\end{align*}
So $S_n$ are orthonormal and therefore linearly independent. So we can continue adding more vectors until it becomes a basis. However, the set of all $S_n$ is already infinite --- so $V$ must have infinite dimensionality.

\subsection{Vectors in $\mathbb C^n$}
We define $\mathbb C^n$ by
\[ \mathbb C^n := \{ \vb z = (z_1, z_2, \cdots, z_n): \forall i, z_i \in \mathbb C \} \]
We define addition and scalar multiplication in obvious ways. Note that we have a choice over what the scalars are allowed to be. If we only allow scalars that are real numbers, $\mathbb C^n$ can be considered a real vector space with bases $(0, \cdots, 1, \cdots, 0)$ and $(0, \cdots, i, \cdots, 0)$ and dimension $2n$. Alternatively, if we let the scalars be any complex numbers, we don't need to have imaginary bases, thus giving us a complex vector space with bases $(0, \cdots, 1, \cdots, 0)$ and dimension $n$. We can say that $\mathbb C^n$ has dimension $2n$ over $\mathbb R$, and dimension $n$ over $\mathbb C$. From here on, unless stated otherwise, we treat $\mathbb C^n$ to be a complex vector space.

\subsection{Inner Product in $\mathbb C^n$}
We can define the inner product by
\[ \langle \vb z, \vb w \rangle := \sum_j \overline{z_j} w_j \]
The conjugate over the $z$ terms ensures that the inner product is positive definite. It has these properties, analogous to the properties of the inner product in the real vector space $\mathbb R^n$:
\begin{itemize}
	\item (Hermitian) $\langle \vb z, \vb w \rangle = \overline{\langle \vb w, \vb z \rangle}$
	\item (linear/antilinear) $\langle \vb z, \lambda \vb w + \lambda' \vb w' \rangle = \lambda \langle \vb z, \vb w \rangle + \lambda' \langle \vb z, \vb w' \rangle$ and $\langle \lambda \vb z + \lambda' \vb z', w \rangle = \overline{\lambda} \langle \vb z, \vb w \rangle + \overline{\lambda'} \langle \vb z', \vb w \rangle$
	\item (positive definite) $\langle \vb z, \vb z \rangle = \sum_j \abs{z_j}^2$ which is real and greater than or equal to zero, where the equality holds if and only if $\vb z = \vb 0$.
\end{itemize}
We can also define the norm of $\vb z$ to satisfy $\abs{\vb z} \geq 0$ and $\abs{\vb z}^2 = \langle \vb z, \vb z \rangle$. Note that the standard basis for $\mathbb C^n$ is orthonormal, since the inner product of any two basis vectors $\vb e_j$ and $\vb e_k$ is given by $\delta_{jk}$.

\subsection{Inner Product in Complex Plane}
Here is an example of the use of the complex inner product on $\mathbb C^1 = \mathbb C$. Note first that $\langle z, w \rangle = \overline z w$. Let $z = a_1 + ia_2$ and $w = b_1 + ib_2$ where $a_1, a_2, b_1, b_2 \in \mathbb R$. Then
\begin{align*}
	\langle z, w \rangle & = \overline z w                              \\
	                     & = (a_1 b_1 + a_2 b_2) + i(a_1 b_2 - a_2 b_1) \\
	                     & = (z \cdot w) + i[z, w]
\end{align*}
We can therefore use the inner product to compute two different scalar products at the same time.

\section{Differentiability}
\subsection{Scattering off a potential barrier}
Consider the potential
\[
	U(x) = \begin{cases}
		0   & x \leq 0, x \geq a \\
		U_0 & 0 < x < a
	\end{cases}
\]
When \( E < U_0 \), we define
\[
	k = \sqrt{\frac{2mE}{\hbar^2}} > 0;\quad \eta = \sqrt{\frac{2m(U_0 - E)}{\hbar^2}} > 0
\]
The solution is then
\[
	\chi(x) = \begin{cases}
		e^{ikx} + Ae^{-ikx}        & x \leq 0  \\
		Be^{-\eta x} + Ce^{\eta x} & 0 < x < a \\
		De^{ikx}                   & x \geq a
	\end{cases}
\]
since we can normalise the incoming flux to one.
The boundary conditions are that \( \chi(x) = \chi'(x) \) are both continuous at \( x = 0, x = a \).
This gives four conditions, which are enough to solve the problem.
\( \chi(x) \) and its derivative at zero give
\[
	1 + A = B + C;\quad ik - ikA = -\eta B + \eta C
\]
and the continuity at \( a \) gives
\[
	B e^{-\eta a} + C e^{\eta a} = D e^{ika};\quad -\eta B e^{-\eta a} + \eta C e^{\eta a} = ikD e^{ika}
\]
Solving the system gives
\[
	D = \frac{-4 i \eta k}{(\eta-ik)^2 \exp[(\eta+ik)a] - (\eta+ik)^2\exp[-(\eta-ik)a]}
\]
The transmitted flux is \( j_{\text{tr}} = \frac{\hbar k}{m} \abs{D}^2 \) and the incident flux is \( j_{\textit{inc}} = \frac{\hbar k}{m} \).
Hence, the transmission coefficient is \( T = \abs{D}^2 \).
This is
\[
	T = \frac{4 k^2 \eta^2}{(k^2+\eta^2)^2 \sinh^2(\eta a) + 4 k^2 \eta^2}
\]
If we take the limit as \( U_0 \gg E \), we have \( \eta a \gg 1 \).
Then
\[
	T \to \frac{16k^2 \eta^2}{(\eta^2 + k^2)^2} \exp[-2\eta a] \propto \exp[-\frac{2a}{k} \sqrt{2m(U_0 - E)}]
\]
So the probability decreases exponentially with the width of the barrier.

\subsection{Harmonic oscillator}
Consider a parabolic potential
\[
	U(x) = \frac{1}{2} kx^2 = \frac{1}{2} m \omega^2 x^2
\]
where \( k \) is an elastic constant and \( \omega = \sqrt{\frac{k}{m}} \) is the angular frequency of the harmonic oscillator.
Classically, we find the solution \( x = A \cos \omega t + B \sin \omega t \).
This gives a continuous energy spectrum.
The TDSE gives
\[
	-\frac{\hbar^2}{2m} \chi''(x) + \frac{1}{2} m\omega^2 x^2 \chi(x) = E \chi(X)
\]
Since this is a bound system, we will have a discrete set of eigenvalues.
The potential is symmetric so the eigenfunctions are odd or even.
We will make the change of variables
\[
	\xi^2 = \frac{m\omega}{\hbar} x^2;\quad \varepsilon = \frac{2E}{\hbar \omega}
\]
which reformulates the TDSE as
\[
	-dv[2]{\chi}{\xi} + \xi^2 \chi = \varepsilon \chi
\]
We will start by considering the solution for \( \varepsilon = 1 \).
In this case, \( E = \frac{\hbar \omega}{2} \).
The solution in this case is
\[
	\chi_0(\xi) = \exp[-\frac{\xi^2}{2}]
\]
So the first eigenfunction, \( \chi_0 \), is known in terms of \( x \), given by
\[
	\chi_0(x) = A \exp[-\frac{m\omega}{2\hbar}x^2];\quad E_0 = \frac{\hbar \omega}{2}
\]
To find the other eigenfunctions, we will take the general form
\[
	\chi(\xi) = f(\xi) \exp[-\frac{\xi^2}{2}]
\]
This works because we know we have a bound solution and \( \chi \) must tend to zero quickly as \( \xi \) tends to infinity, due to the differential equation in terms of \( \xi, \varepsilon \).
Using the above ansatz for \( \chi \) in the Schr\"odinger equation,
\[
	-\dv[2]{f}{\xi} + 2\xi \dv{f}{\xi} + (1-\varepsilon)f = 0
\]
Note that if \( \varepsilon = 1 \), a solution is \( f = 1 \).
We can find a power series solution to this differential equation, with \( \xi = 0 \) as a regular point.
\[
	f(\xi) = \sum_{n=0}^\infty a_n \xi^n
\]
We find
\[
	\xi \dv{f}{\xi} = \sum_{n=0}^\infty n a_n \xi^n;\quad \dv[2]{f}{\xi} = \sum_{n=0}^\infty n(n-1)a_n \xi^{n-2} = \sum_{n=0}^\infty (n+1)(n+2)a_{n+2}\xi^n
\]
Comparing coefficients of \( \xi^n \),
\[
	(n+1)(n+2) a_{n+2} - 2n a_n + (\varepsilon - 1) a_n = 0
\]
Hence,
\[
	a_{n+2} = \frac{2n - \varepsilon + 1}{(n+1)(n+2)} a_n
\]
Since the function must be either even or odd, exactly one of \( a_0 \) and \( a_1 \) must be zero.
\begin{proposition}
	If the series for \( f \) does not terminate, \( \chi \) is not normalisable.
\end{proposition}
\begin{proof}
	Suppose the series does not terminate.
	We will consider the asymptotic behaviour as \( n \to \infty \).
	\[
		\frac{a_{n+2}}{a_n} \to \frac{2}{n}
	\]
	But this is the same asymptotic behaviour as the function \( g(\xi) \) given by
	\[
		g(\xi) = \exp[\xi^2] = \sum_{m=0}^\infty \frac{\xi^{2m}}{m!} = \sum_{n=0}^\infty b_n \xi^n
	\]
	with
	\[
		b_n = \begin{cases}
			\frac{1}{m!} & n = 2m     \\
			0            & n = 2m + 1
		\end{cases}
	\]
	So asymptotically,
	\[
		\frac{b_{n+2}}{b_n} = \frac{\qty(\frac{n}{2})!}{\qty(\frac{n}{2} + 1)!} = \frac{2}{n+2} \to \frac{2}{n}
	\]
	Hence \( \chi \) would have a form asymptotically equal to
	\[
		\chi(\xi) \sim \exp[\frac{\xi^2}{2}]
	\]
	Hence \( \chi(\xi) \) would be not normalisable.
\end{proof}
Hence \( f \) must be a polynomial.
So there exists \( N \) such that \( a_{N+2} = 0 \) and \( a_N \neq 0 \).
So for this value,
\[
	2N - \varepsilon + 1 = 0 \implies \varepsilon = 2N + 1
\]
By the definition of \( \varepsilon \),
\[
	E_N = \qty(N + \frac{1}{2}) \hbar \omega
\]
In particular, \( E_{N+1} - E_N = \hbar \omega \).
The eigenfunctions are
\[
	\chi_N(\xi) = f_N(\xi) \exp[-\frac{\xi^2}{2}]
\]
with the property that
\[
	\chi_N(-\xi) = (-1)^N \chi_N(\xi)
\]
\begin{align*}
	f_0(\xi) & = 1                      \\
	f_1(\xi) & = \xi                    \\
	f_2(\xi) & = 1 - 2 \xi^2            \\
	f_3(\xi) & = \xi - \frac{2}{3}\xi^3 \\
	         & \vdots
\end{align*}

\section{Properties of the derivative}
\subsection{Derivation with Fourier's law}
In a volume \( V \), the overall heat energy \( Q \) is given by
\[
	Q = \int_V c_V \rho \theta \dd{V}
\]
where \( c_V \) is the specific heat of the material, \( \rho \) is the mass density, and \( \theta \) is the temperature.
The rate of change due to heat flow is
\[
	\dv{Q}{t} = \int_V c_V \rho \pdv{\theta}{t} \dd{V}
\]
Fourier's law for heat flow is
\[
	q = -k \grad{\theta}
\]
where \( q \) is the heat flux.
We will integrate this over the surface \( S = \partial V \), giving
\[
	-\dv{Q}{t} = \int_S q \cdot \hat n \dd{S}
\]
The negative sign is due to the normals facing outwards.
This is exactly
\[
	-\dv{Q}{t} = \int_S (-k \grad{\theta}) \cdot \hat n \dd{S} = \int_V -k \laplacian{\theta} \dd{V}
\]
Equating these two forms for \( \dv{Q}{t} \), we find
\[
	\int_V (c_V \rho \pdv{\theta}{t} - k \laplacian{\theta}) \dd{V} = 0
\]
Since \( V \) was arbitrary, the integrand must be zero.
So we have
\[
	\pdv{\theta}{t} - \frac{k}{c_V \rho} \laplacian{\theta} = 0
\]
Let \( D = \frac{k}{c_V \rho} \) be the diffusion constant.
Then we have the diffusion equation
\[
	\pdv{\theta}{t} - D \laplacian{\theta} = 0
\]

\subsection{Derivation with statistical dynamics}
We can derive this equation in another way, using statistical dynamics.
Gas particles diffuse by scattering every fixed time step \( \Delta t \) with probability density function \( p(\xi) \) of moving by a displacement \( \xi \).
On average, we have
\[
	\inner{\xi} = \int p(\xi) \xi \dd{\xi} = 0
\]
since there is no bias the direction in which any given particle is travelling.
Suppose that the probability density function after \( N\Delta t \) time is described by \( P_{N \Delta t}(x) \).
Then, for the next time step,
\[
	P_{(N+1)\Delta t}(x) = \int_{-\infty}^\infty p(\xi) P_{N \Delta t}(x - \xi) \dd{\xi}
\]
Using the Taylor expansion,
\begin{align*}
	P_{(N+1)\Delta t}(x) & \approx \int_{-\infty}^\infty p(\xi) \qty[P_{N \Delta t}(x) + P_{N \Delta t}'(x)(-\xi) + P_{N \Delta t}''(x)\frac{\xi^2}{2} + \cdots] \dd{\xi} \\
	                     & \approx P_{N \Delta t}(x) - P_{N \Delta t}'(x) \inner{\xi} + P_{N \Delta t}''(x) \frac{\inner{\xi^2}}{2} + \cdots                              \\
	                     & \approx P_{N \Delta t}(x) + P_{N \Delta t}''(x) \frac{\inner{\xi^2}}{2} + \cdots
\end{align*}
since \( \int p(\xi) \dd{\xi} = 1 \).
Identifying \( P_{N \Delta t}(x) = P(x, N\Delta t) \), we can write
\[
	P(x, (N+1)\Delta t) - P(x, N \Delta t) = \pdv[2]{x} P(x, N\Delta t) \frac{\inner{\xi^2}}{2}
\]
Assuming that the variance \( \frac{\inner{\xi^2}}{2} \) is proportional to \( D \Delta t \), then for small \( \Delta t \), we find
\[
	\pdv{P}{t} = D \pdv[2]{P}{x}
\]
which is exactly the diffusion equation.

\subsection{Similarity solutions}
The characteristic relation between the variance and time suggests that we seek solutions with a dimensionless parameter.
If we can a change of variables of the form \( \theta(\eta) = \theta(x,t) \), then it will likely be easier to solve.
Consider
\[
	\eta \equiv \frac{x}{2\sqrt{Dt}}
\]
Then,
\[
	\pdv{\theta}{t} = \pdv{\eta}{t} \pdv{\theta}{\eta} = \frac{-1}{2} \frac{x}{\sqrt{D} t^{3/2}} \theta' = \frac{-1}{2} \frac{\eta}{t} \theta'
\]
and
\[
	D \pdv[2]{\theta}{x} = D \pdv{x} \qty(\pdv{\eta}{x} \pdv{\theta}{\eta}) = D \pdv{x} \qty(\frac{1}{2\sqrt{Dt}} \theta') = \frac{D}{4Dt} \theta'' = \frac{1}{4t} \theta''
\]
Substituting into the diffusion equation,
\[
	\theta'' = -2 \eta \theta'
\]
Let \( \psi = \theta' \).
Then
\[
	\frac{\psi'}{\psi} = -2\eta \implies \ln \psi = -\eta^2 + \text{constant}
\]
Then, choosing a constant of \( c\frac{2}{\sqrt{\pi}} \),
\[
	\psi = c\frac{2}{\sqrt{\pi}} e^{-\eta^2} \implies \theta(\eta) = c\frac{2}{\sqrt{\pi}} \int_0^\eta e^{-u^2} \dd{u} = c \erf(\eta) = c \erf(\frac{x}{2\sqrt{Dt}})
\]
where
\[
	\erf(z) = \frac{2}{\sqrt{\pi}} \int_0^z e^{-u^2} \dd{u}
\]
This describes discontinuous initial conditions that spread over time.

\subsection{Heat conduction in a finite bar}
Suppose we have a bar of length \( 2L \) with \( -L \leq x \leq L \) and initial temperature
\[
	\theta(x,0) = H(x) = \begin{cases}
		1 & \text{if } 0 \leq x \leq L \\
		0 & \text{if } -L \leq x < 0
	\end{cases}
\]
with boundary conditions \( \theta(L, t) = 1 \), \( \theta(-L, t) = 0 \).
Currently the boundary conditions are not homoegeneous, so Sturm-Liouville theory cannot be used directly.
If we can identify a steady-state solution (time-independent) that reflects the late-time behaviour, then we can turn it into a homoegeneous set of boundary conditions.
We will try a solution of the form
\[
	\theta_s(x) = Ax + B
\]
since this certainly satisfies the diffusion equation.
To satisfy the boundary conditions,
\[
	A = \frac{1}{2L};\quad B = \frac{1}{2}
\]
Hence we have a solution
\[
	\theta_s = \frac{x + L}{2L}
\]
We will subtract this solution from our original equation for \( \theta \), giving
\[
	\hat \theta(x,t) = \theta(x,t) - \theta_s(x)
\]
with homogeneous boundary conditions
\[
	\hat \theta(-L, t) = \hat \theta(L, t) = 0
\]
and initial conditions
\[
	\theta(x,0) = H(x) - \frac{x+L}{2L}
\]
We will now separate variables in the usual way.
We will consider the ansatz
\[
	\hat \theta(x,t) = X(x) T(t) \implies X'' = - \lambda X; \dot T = -D \lambda T
\]
The boundary conditions imply \( \lambda > 0 \) and give the Fourier modes \( X(x) = A \cos \sqrt{\lambda} x + B \sin \sqrt{\lambda} x \).
For \( \cos \sqrt{\lambda} L = 0 \), we require \( \sqrt{\lambda_m} = \frac{m \pi}{2L} \) for \( m \) odd.
Also, \( \sin \sqrt{\lambda} L = 0 \) gives \( \sqrt{\lambda_n} = \frac{n \pi}{L} \) for \( n \) even.
Since \( \hat \theta \) is odd due to our initial conditions, we can take
\[
	X_n = B_n \sin \frac{n \pi x}{L}; \quad \lambda_n = \frac{n^2 \pi^2}{L^2}
\]
Substituting into \( \dot T = -D \lambda T \), we have
\[
	T_n(t) = c_n \exp(-\frac{Dn^2 \pi^2}{L^2} t )
\]
In general, the solution is
\[
	\hat \theta(x,t) = \sum_{n=1}^\infty b_n \sin \frac{n \pi x}{L} \exp(-\frac{Dn^2 \pi^2}{L^2} t )
\]

\section{Using the derivative}
\subsection{Conditions for local minimisers}
The Euler-Lagrange equation gives a necessary condition for a stationary point.
We cannot tell whether this leads to a minimum, a maximum, or a saddle point, just from the Euler-Lagrange equation.
We can analyse the nature of the stationary points by considering the second variation.
Consider the functional
\[
	F[y] = \int_\alpha^\beta f(x,y,y') \dd{x}
\]
where \( y \) is perturbed by a perturbation \( \varepsilon\eta \).
Let us assume that \( y \) is a solution to the Euler-Lagrange equation, so has no first variation.
We will then expand \( F[y+\varepsilon\eta] \) to second order.
\begin{align*}
	F[y+\varepsilon\eta]        & = \int_\alpha^\beta \qty[f(x,y+\varepsilon\eta,y'+\varepsilon\eta')] \dd{x}                                                                                  \\
	F[y+\varepsilon\eta] - F[y] & = \int_\alpha^\beta \qty[f(x,y+\varepsilon\eta,y'+\varepsilon\eta') - f(x,y,y')] \dd{x}                                                                      \\
	                            & = 0 + \varepsilon \underbrace{\int_\alpha^\beta \eta \qty( \pdv{f}{y} - \dv{x} \pdv{f}{y'} ) \dd{x}}_{\text{zero by Euler-Lagrange equation}}                \\
	                            & + \frac{1}{2} \varepsilon^2 \int_\alpha^\beta  \qty( \eta^2 \pdv[2]{f}{y} + \eta'^2 \pdv[2]{f}{(y')} + 2\eta\eta' \pdv{f}{y}{y'} ) \dd{x} + O(\varepsilon^3)
\end{align*}
The last term (excluding the \( \varepsilon^2 \) component) is called the second variation.
We write
\[
	\delta^2 F[y] \equiv \frac{1}{2}\int_\alpha^\beta \qty( \eta^2 \pdv[2]{f}{y} + \eta'^2 \pdv[2]{f}{(y')} + \dv{x} (\eta^2) \pdv{f}{y}{y'} ) \dd{x}
\]
Integrating the last term by parts, using \( \eta = 0 \) at \( \alpha, \beta \), we have
\[
	\delta^2 F[y] = \frac{1}{2}\int_\alpha^\beta \qty( Q\eta^2 + P(\eta')^2 ) \dd{x}
\]
where
\[
	P = \pdv[2]{f}{(y')};\quad Q = \pdv[2]{f}{y} - \dv{x} \qty(\pdv{f}{y}{y'})
\]
Thus, if \( y \) is a solution to the Euler-Lagrange equation, and also \( Q\eta^2 + P(\eta')^2 > 0 \) for all \( \eta \) vanishing at \( \alpha, \beta \), then \( y \) is a local minimiser of \( F \).

\begin{example}
	We will prove that the geodesic on a plane is a local minimiser of path length.
	The functional we will analyse is given by
	\[
		f = \sqrt{1 + (y')^2}
	\]
	Hence,
	\[
		P = \pdv[2]{f}{(y')} = \pdv{y'} \qty(\frac{y'}{\sqrt{1+(y')^2}}) = \frac{1}{(1+(y')^2)^{\frac{3}{2}}} > 0
	\]
	\[
		Q = 0
	\]
	Therefore the second variation is positive, so any \( y \) that satisfies the Euler-Lagrange equation minimises path length.
	In particular, straight lines minimise path length on the plane.
\end{example}

\subsection{Legendre condition for minimisers}
\begin{proposition}[Legendre condition]
	If \( y_0(x) \) is a local minimiser, then \( \eval{P}_{y=y_0} \geq 0 \).
\end{proposition}
\noindent We can say that the Legendre condition is a necessary condition for a minimiser.
In less formal terms, \( P \) is `more important' than \( Q \) when determining if a stationary point is a minimiser.
\begin{proof}
	This condition is not proven rigorously.
	However, the general idea of the proof is to construct a function \( \eta \) which is small everywhere (giving a small \( Q \) contribution), but oscillates very rapidly near some point \( x_0 \), at which \( P < 0 \).
	This gives a large \( P \) contribution which can overpower the \( Q \) contribution.
	Then this gives \( Q\eta^2 + P(\eta')^2 < 0 \) if there exists some \( x_0 \) where \( \eval{P}_{y=y_0} < 0 \).
\end{proof}

\noindent Note that the Legendre condition is not a sufficient condition for local minima, but \( P > 0 \) and \( Q \geq 0 \) is sufficient.

\begin{example}
	Consider again the brachistochrone problem.
	\[
		f = \sqrt{\frac{1 + (y')^2}{-y}}
	\]
	We have
	\[
		\pdv{f}{y} = -\frac{1}{2y}f
	\]
	\[
		\pdv{f}{y'} = \frac{y'}{\sqrt{1+(y')^2}\sqrt{-y}}
	\]
	Hence
	\[
		P = \frac{1}{(1+(y')^2)^\frac{3}{2} \sqrt{-y}} > 0
	\]
	\[
		Q = \frac{1}{2\sqrt{1 + (y^2)^2}y^2 \sqrt{-y}} > 0
	\]
	Hence the cycloid is a local minimiser of the time taken to travel between the two points.
\end{example}

\subsection{Associated eigenvalue problem}
When deriving the minimiser condition, we had the integrand
\[
	Q\eta^2 + P(\eta')^2
\]
We can integrate this by parts:
\[
	Q\eta^2 + \dv{x} (P \eta \eta') - \eta \dv{x} (P \eta')
\]
giving
\[
	\delta^2 F[y] = \frac{1}{2}\int_\alpha^\beta \eta \qty[-(P\eta')' + Q\eta]\dd{x}
\]
The bracketed term \( -(P\eta')' + Q\eta \) is known as the Sturm-Liouville operator acting on \( \eta \), denoted \( \mathcal L(\eta) \).
If there exists \( \eta \) such that \( \mathcal L(\eta) = -\omega^2\eta \), \( \omega \in \mathbb R\), and \( \eta(\alpha) = \eta(\beta) = 0 \), then \( y \) is not a minimiser, since the integrand will be \( -\omega^2\eta^2 < 0 \).

\begin{example}
	Consider
	\[
		F[y] = \int_0^\beta \qty((y')^2 - y^2)\dd{x}
	\]
	such that
	\[
		y(0) = y(\beta) = 0;\quad \beta \neq k\pi, k \in \mathbb N
	\]
	The Euler-Lagrange equation gives
	\[
		y'' + y = 0
	\]
	Thus, constrained to the boundary conditions, the only stationary point of \( F \) is
	\[
		y \equiv 0
	\]
	Analysing the second variation,
	\[
		\delta^2 F[0] = \frac{1}{2} \int_0^\beta \qty[\eta'^2 - \eta^2] \dd{x}
	\]
	giving
	\[
		P = 1 > 0;\quad Q < 0
	\]
	Let us now examine the eigenvalue problem, since we cannot find whether \( y \equiv 0 \) is a minimiser from what we know already.
	Consider the eigenvalue problem
	\[
		-\eta'' - \eta = -\omega^2 \eta;\quad \eta(0) = \eta(\beta) = 0
	\]
	Let us take
	\[
		\eta = A \sin(\frac{\pi x}{\beta})
	\]
	to give
	\[
		\qty(\frac{\pi}{\beta})^2 = 1 - \omega^2
	\]
	So this has a solution \( \omega > 0 \) if and only if \( \beta > \pi \).
	If \( P > 0 \), a problem may arise if the interval of integration is `too large' (in this case \( \beta > \pi \)).
	Next lecture we will make this notion precise.
\end{example}

\section{Extensions of the mean value theorem}
\subsection{Cauchy's mean value theorem}
\begin{theorem}
	If \(f, g \colon [a,b] \to \mathbb R\) are continuous, and differentiable on \((a, b)\), there exists \(t \in (a,b)\) such that
	\[
		(f(b) - f(a))g'(t) = f'(t)(g(b) - g(a))
	\]
\end{theorem}
\noindent We can recover the normal mean value theorem from Cauchy's generalisation by taking \(g(x) = x\).
\begin{proof}
	Let
	\[
		\phi(x) = \begin{vmatrix}
			1    & 1    & 1    \\
			f(a) & f(x) & f(b) \\
			g(a) & g(x) & g(b)
		\end{vmatrix}
	\]
	Certainly \(\phi(x)\) is continuous on \([a,b]\) and differentiable on \((a, b)\), by using previous results.
	Also, \(\phi(a) = \phi(b) = 0\) by observing the linear dependence of the columns.
	By Rolle's theorem, there exists \(t \in (a, b)\) such that \(\phi'(t) = 0\).
	We can expand \(\phi'(t)\) and this will show the required result.
	\[
		\phi'(x) = f'(x)g(b) - g'(x)f(b) + f(a)g'(x) - g(a)f'(x) = f'(x) [g(b) - g(a)] + g'(x) [f(a) - f(b)]
	\]
\end{proof}

\subsection{Example of l'H\^opital's rule}
The derivation of l'H\^opital's rule is on an example sheet, so in this subsection we will consider only a special case of it, using Cauchy's mean value theorem.
\[
	\ell = \lim_{x \to 0} \frac{e^x - 1}{\sin x}
\]
We can write
\[
	\ell = \lim_{x \to 0} \frac{e^x - e^0}{\sin x - \sin 0} = \frac{e^t}{\cos t}
\]
for some \(t \in (0, x)\).
So as \(x \to 0\), \(t \to 0\) and hence
\[
	\frac{e^t}{\cos t} \to 1
\]

\subsection{Taylor's theorem}
\begin{theorem}[Taylor's Theorem with Lagrange's Remainder]
	Suppose \(f\) and its derivatives up to order \(n-1\) are continuous in \([a, a+h]\), and \(f^{(n)}\) exists for \(x \in (a, a+h)\).
	Then
	\[
		f(a+h) = f(a) + hf'(a) + \frac{h^2}{2!} f''(a) + \dots + \frac{h^{n-1}}{(n-1)!}f^{(n-1)}(a) + \frac{h^n}{n!}f^{(n)}(a + \theta h)
	\]
	where \(\theta \in (0, 1)\).
\end{theorem}
\noindent Note that for \(n=1\), this is exactly the mean value theorem, so this can be seen as an \(n\)th order extension of the mean value theorem.
We commonly write \(R_n\) for the final error term \(\frac{h^n}{n!}f^{(n)}(a + \theta h)\).
This is known as Lagrange's form of the remainder.
\begin{proof}
	For \(0 \leq t \leq h\), we define
	\[
		\phi(t) = f(a+t) - f(a) - tf'(a) - \dots - \frac{t^{n-1}}{(n-1)!}f^{(n-1)}(a) - \frac{t^n}{n!}B
	\]
	where we choose \(B\) suitably such that \(\phi(h) = 0\).
	(Recall that in the proof of the mean value theorem, we used \(f(x) - kx\) and picked \(k\) suitably such that this allowed the use of Rolle's theorem.
	This is entirely analogous, but generalised to the \(n\)th derivative).
	Note that
	\[
		\phi(0) = \phi'(0) = \dots = \phi^{(n-1)}(0) = 0
	\]
	We can use Rolle's theorem inductively \(n\) times.
	Since \(\phi(0) = \phi(h) = 0\), there is a point \(0 < h_1 < h\) such that \(\phi'(h_1) = 0\).
	Since \(\phi'(0) = \phi'(h_1) = 0\), there is a point \(0 < h_2 < h_1\) such that \(\phi''(h_2) = 0\).
	This continues until we find a point \(0 < h_n < h\) such that \(\phi^{(n)}(h_n) = 0\).
	Hence \(h_n = \theta h\) for some \(0 < \theta < 1\).
	Now, \(\phi^{(n)}(t) = f^{(n)}(a + t) - B\).
	We can see now that \(B = f^{(n)}(a + \theta h)\), which gives the required result.
\end{proof}
\noindent We can prove an alternative version of Taylor's theorem with a different error term.
\begin{theorem}[Taylor's Theorem with Cauchy's Remainder]
	Suppose (equivalently to before) \(f\) and its derivatives up to order \(n-1\) are continuous in \([a, a+h]\), and \(f^{(n)}\) exists for \(x \in (a, a+h)\).
	Then
	\[
		f(a+h) = f(a) + hf'(a) + \frac{h^2}{2!} f''(a) + \dots + \frac{h^{n-1}}{(n-1)!}f^{(n-1)}(a) + R_n
	\]
	where
	\[
		R_n = \frac{(1 - \theta)^{n-1}h^n f^{(n)}(a + \theta h)}{(n-1)!}
	\]
	for \(\theta \in (0, 1)\).
\end{theorem}
\begin{proof}
	For simplicity, in this proof we let \(a = 0\), although the same argument applies when \(a \neq 0\).
	Let us define
	\[
		F(t) = f(h) - f(t) - (h-t)f'(t) - \dots - \frac{(h-t)^{n-1}f^{(n-1)}(t)}{(n-1)!}
	\]
	for \(t \in [0, h]\).
	Then
	\begin{align*}
		F'(t) & = -f'(t) + f'(t) - (h-t)f''(t) + (h-t)f''(t) - \frac{1}{2} (h-t)^2f'''(t) + \frac{1}{2} (h-t)^2f'''(t) \\
		      & - \dots - \frac{(h-t)^{n-1}}{(n-1)!}f^{(n)}(t)                                                         \\
		      & = - \frac{(h-t)^{n-1}}{(n-1)!}f^{(n)}(t)
	\end{align*}
	Let
	\[
		\phi(t) = F(t) - \left[ \frac{h-t}{h} \right]^p F(0)
	\]
	where \(p \in \mathbb N\) and \(1 \leq p \leq n\).
	Then
	\[
		\phi(0) = \phi(h) = 0
	\]
	By Rolle's theorem, there exists \(\theta \in (0, 1)\) such that
	\[
		\phi'(\theta h) = 0
	\]
	We can compute \(\phi'\) to find
	\[
		\phi'(\theta h) = F'(\theta h) + \frac{p(1-\theta)^{p-1}}{h} F(0) = 0
	\]
	Substituting everything back into \(F\) gives
	\[
		0 = \frac{-h^{n-1}(1-\theta)^{n-1}}{(n-1)!}f^{(n)}(\theta h) + \frac{p(1-\theta)^{p-1}}{h}\left[ f(h) - f(0) - h'(0) - \dots - \frac{h^{n-1}}{(n-1)!}f^{(n-1)}(0) \right]
	\]
	Hence
	\[
		f(h) = f(0) + hf'(0) + \frac{h^2}{2!} f''(0) + \dots + \frac{h^{n-1}}{(n-1)!}f^{(n-1)}(0) + \underbrace{\frac{h^n(1 - \theta)^{n-1}f^{(n)}(\theta h)}{(n-1)!\cdot p(1-\theta)^{p-1}}}_{R_n}
	\]
	By letting \(p = n\), we get Lagrange's remainder.
	If \(p=1\), we get Cauchy's remainder.
\end{proof}

\section{Applications of remainders in taylor's theorem}
\subsection{Abel's theorem}
Consider a second order homogeneous ODE:\@
\[
	y'' + p(x)y' + q(x)y = 0
\]
\begin{theorem}[Abel's Theorem]
	If \(p(x)\) and \(q(x)\) are continuous on an interval \(I\), then the Wro\'nskian \(W(x)\) is either zero or non-zero for all \(x \in I\).
\end{theorem}
\begin{proof}
	Let \(y_1, y_2\) be solutions to the equation.
	Then
	\begin{align}
		\label{abelproof1} y_2(y_1'' + p(x)y_1' + q(x)y_1) & = 0 \\
		\label{abelproof2} y_1(y_2'' + p(x)y_2' + q(x)y_2) & = 0
	\end{align}
	Now, calculating \eqref{abelproof2} \(-\) \eqref{abelproof1}, we get
	\begin{equation}\label{abelproof3}
		(y_1y_2'' - y_2y_1'') + p(x)(y_1y_2' - y_2y_1') = 0
	\end{equation}
	As we are solving a second order equation, \(W(x) = y_1y_2' - y_2y_1'\) and therefore
	\[
		\frac{\dd{W}}{\dd{x}} = y_1y_2'' + y_1'y_2' - y_2'y_1' - y_2y_1'' = y_1y_2'' - y_2y_1''
	\]
	Note that these are the coefficients in \eqref{abelproof3}.
	We have therefore
	\begin{equation}\label{abelproof4}
		W' + pW = 0
	\end{equation}
	Then by separating variables:
	\begin{align*}
		\frac{\dd{W}}{W}              & = -p(x)\dd{x}                         \\
		\int_{x_0}^x \frac{\dd{W}}{W} & = -\int_{x_0}^x p(u)\dd{u}            \\
		W(x)                          & = W(x_0)e^{-\int_{x_0}^x p(u) \dd{u}}
	\end{align*}
	This last equation is known as Abel's Identity, and is very important.
	Since \(p(x)\) is continuous on \(I\) with \(x \in I\), it is bounded and therefore integrable.
	Therefore \(e^{-\int_{x_0}^x p(u) \dd{u}} \neq 0\).
	It follows that if \(W(x_0) = 0\) then \(W(x) = 0\) for all \(x\).
	Likewise, if \(W(x_0) \neq 0\), then \(W(x) \neq 0\) for all \(x\) (on the interval).
\end{proof}
\begin{corollary}
	If \(p(x) = 0\), then \(W = W_0\) which is a constant.
\end{corollary}
Note that we can use this to find \(W(x)\) without actually solving the differential equation itself.
For example, Bessel's Equation
\[
	x^2y'' + xy' + (x^2 - n^2)y = 0
\]
has no closed form solutions, but the Wro\'nskian can be calculated be rewriting it as
\[
	y'' + \frac{1}{x}y' + \frac{x^2-n^2}{x^2}y = 0
\]
and by Abel's Identity,
\begin{align*}
	W(x) & = W_0 e^{-\int_{x_0}^x \frac{1}{u} \dd{u}} \\
	     & = W_0 e^{-\ln x}                           \\
	     & = \frac{W_0}{x}
\end{align*}

\subsection{Using Abel's identity}
We can find a second solution \(y_2\) given a solution \(y_1\) using a reduction of order method, but we can also use Abel's Identity.
\[
	y_1y_2' - y_2y_1' = W_0 e^{-\int_{x_0}^x p(u) \dd{u}}
\]
This is a first order ODE for \(y_2\) which we can now solve:
\[
	\frac{y_1y_2' - y_2y_1'}{y_1^2} =  \frac{W_0}{y_1^2} e^{-\int_{x_0}^x p(u) \dd{u}}
\]
The left hand side is exactly the quotient rule, giving
\[
	\dv{x}\frac{y_2}{y_1} = \frac{W_0}{y_1^2} e^{-\int_{x_0}^x p(u) \dd{u}}
\]
which can be solved to give \(y_2\) as a function of \(y_1\) and \(W\).

\subsection{Abel's theorem in higher dimensions}
Any linear \(n\)th order ODE can be written
\[
	\vb Y' + A(x) \vb Y = 0
\]
where \(A\) is a matrix; this converts an \(n\)th order ODE into a system of \(n\) first order ODEs.
This will be discussed later in the course.
It can be shown that this generalisation of Abel's Identity
\[
	W' + \tr(A)W = 0
\]
holds, and hence
\[
	W' = W_0e^{-\int_{x_0}^x \tr(A) \dd{u}}
\]
and Abel's theorem holds.
This is shown on Example Sheet 3, Question 7.

\subsection{Equidimensional equations}
An ODE is equidimensional if the differential operator is unaffected by a multiplicative scaling.
For example, rescaling
\[
	x \mapsto X = \alpha x
\]
where \(\alpha \in \mathbb R\).
The general form for a second order equidimensional equation is
\begin{equation}\label{equidimensional1}
	ax^2 y'' + bxy' + cy = f(x)
\end{equation}
where \(a, b, c\) are constant.
Note, \(\frac{\dd}{\dd{X}} = \frac{1}{\alpha}\frac{\dd}{\dd{x}}\), and \(\frac{\dd^2}{\dd{X}^2} = \frac{1}{\alpha^2}\frac{\dd}{\dd{x}^2}\), so plugging this into \eqref{equidimensional1} gives
\[
	aX^2\frac{\dd^2 y}{\dd{X}^2} + bX\frac{\dd{y}}{\dd{X}} + cy = f\left(\frac{X}{\alpha}\right)
\]
The left hand side was unaffected by this rescaling, so the equation is equidimensional.

There are two main methods for solving equidimensional equations.
\begin{enumerate}
	\item Note that \(y = x^k\) is an eigenfunction of the differential operator \(x\frac{\dd}{\dd{x}}\).
	      Inspired by this, to solve \eqref{equidimensional1} we will look for solutions of the form \(y=x^k\), so we have
	      \[
		      ak(k-1) + bk + c = 0
	      \]
	      We can simply solve this quadratic for two roots \(k_1\) and \(k_2\).
	      If \(k_1 \neq k_2\), then the complementary function is
	      \[
		      y_c = Ax^{k_1} + Bx^{k_2}
	      \]
	\item If \(k_1 = k_2\), then the substitution \(z = \ln x\) turns \eqref{equidimensional1} into an equation with constant coefficients.
	      \[
		      a \frac{\dd^2 y}{\dd{z}^2} + (b-a)\frac{\dd{y}}{\dd{z}} + cy = f(e^z)
	      \]
	      (TODO verify this).
	      Because this has constant coefficients, our complementary functions will be of the form \(y = e^{\lambda z}\), which can be solved as usual.
	      \[
		      y_c = Ae^{\lambda_1 z} + Be^{\lambda_2 z} = Ax^{\lambda_1} + Bx^{\lambda_2}
	      \]
	      which is the same form as above.
	      In this form, it is easier to see that if the two solutions \(\lambda_1\), \(\lambda_2\) are the same, then
	      \[
		      y_c = Ae^{\lambda_1 z} + Bze^{\lambda_1 z} = Ax^{k_1} + Bx^{k_1}\ln x
	      \]
\end{enumerate}

\section{Power series}
\subsection{Two Body Problem}
Consider two bodies of mass \(m_1\), \(m_2\) experiencing gravitational attraction to the other, with no external forces.
Let \(m_1\) be at position \(\vb r_1\), and \(m_2\) at \(\vb r_2\), with the centre of mass at \(\vb R\), and total mass \(M = m_1 + m_2\).
Then certainly,
\[
	\vb R = \frac{1}{M}(m_1 \vb r_1 + m_2 \vb r_2)
\]
We will define the separation vector \(\vb r = \vb r_1 - \vb r_2\).
We can then further say that
\[
	\vb r_1 = \vb R + \frac{m_2}{M}\vb r;\quad \vb r_2 = \vb R - \frac{m_1}{M}\vb r
\]
Since \(\vb F^\text{ext} = \vb 0\), the centre of mass \(\vb R\) does not accelerate; it moves with constant velocity.
Now, let us consider \(\vb r\).
\[
	\rddot = \rddot_1 + \rddot_2 = \frac{\vb F_{12}}{m_1} - \frac{\vb F_{21}}{m_2} = \vb F_{12} \left( \frac{1}{m_1} + \frac{1}{m_2} \right)
\]
Equivalently, we can write
\[
	\mu\rddot = \vb F_{12};\quad \mu = \frac{m_1 m_2}{m_1 + m_2}
\]
Notice that \(\mu\) has the dimension of mass; we call it the `reduced' mass since it is less than \(m_1\) and \(m_2\).
This can be seen as the equation of motion of a particle of mass \(\mu\) under the effect of force \(\vb F_{12}\).
In the case of a gravitational force, we have
\[
	\mu \rddot = \frac{-Gm_1m_2}{\abs{\vb r}^3}\vb r
\]
Hence,
\[
	\rddot = \frac{-G(m_1 + m_2)}{\abs{\vb r}^3}\vb r
\]
This is the motion of a particle under the effect of a gravitational force due to a mass \(m_1 + m_2\) fixed at the origin.
The total kinetic energy \(T\) is
\[
	T = \frac{1}{2}M\dot {\vb R}^2 + \frac{1}{2}\mu \rdot^2
\]
The total angular momentum \(\vb L\) is
\[
	\vb L = M \vb R \times \dot{\vb R} + \mu \vb r \times \dot{\vb r}
\]

\subsection{Example of Two Body Problem}
Let us consider the orbit of the earth and the sun.
Both particles move around the centre of mass, and both orbits have the same shape.
However, the sizes of the orbits are very different.
The ratio of masses is around \num{3e-4}, and the radius of orbit is approximately \SI{1.5e7}{\kilo\metre}.
Hence the displacement of the sun is around \SI{450}{\kilo\metre}.

\subsection{Variable Mass Problems and the Rocket Problem}
Consider a rocket which ejects mass (exhaust gases) at a high speed in order to propel itself forward.
We cannot apply Newton's second law to the rocket alone, since in this system mass is not conserved.
Consider the rocket moving in one dimension, with speed \(v(t)\) and mass \(m(t)\).
The mass is being expelled at velocity \(u\) relative to the rocket.
At time \(t\), the rocket has momentum \(v(t) m(t)\).
At time \(t + \delta t\), the momentum is \(v(t + \delta t) m(t + \delta t)\).
The exhaust gases emitted during \(\delta t\) have velocity \(v(t) - u + O(\delta t)\) and mass \(m(t) - m(t + \delta t)\).
The total momentum at \(t + \delta t\) is
\[
	v(t + \delta t) m(t + \delta t) + (v(t) - u + O(\delta t))(m(t) - m(t + \delta t))
\]
So the change in momentum is
\begin{align*}
	\delta p & = v(t + \delta t) m(t + \delta t) + (v(t) - u + O(\delta t))(m(t) - m(t + \delta t)) - v(t) m(t) \\
	         & = \left( \dv{m}{t} u + m \dv{v}{t} \right) \delta t + O(\delta t^2)                              \\
\end{align*}
But since momentum is conserved,
\[
	\dv{m}{t} u + m \dv{v}{t} = 0
\]
This is called the rocket equation.
We can generalise this to \(\dv{m}{t} u + m \dv{v}{t} = \vb F^\text{ext}\) in the presence of external forces.
In the absence of such external forces,
\begin{align*}
	\dv{m}{t} u   & = - m \dv{v}{t}                                  \\
	\implies v(t) & = v(0) + u \log \left( \frac{m(0)}{m(t)} \right)
\end{align*}

\section{Infinite differentiability of power series}
\subsection{Notation}
Let \( p(t) \) be a polynomial over \( F \).
We will write
\[
	p(t) = a_n t^n + \dots + a_0
\]
For a matrix \( A \in M_n(F) \), we write
\[
	p(A) = a_n A^n + \dots + a_0 \in M_n(F)
\]
For an endomorphism \( \alpha \in L(V) \),
\[
	p(\alpha) = a_n \alpha^n + \dots + a_0 I \in L(V);\quad \alpha^k \equiv \underbrace{\alpha \circ \dots \circ \alpha}_{k \text{ times}}
\]

\subsection{Sharp criterion of diagonalisability}
\begin{theorem}
	Let \( V \) be a vector space over \( F \) of finite dimension \( n \).
	Let \( \alpha \) be an endomorphism of \( V \).
	Then \( \alpha \) is diagonalisable if and only if there exists a polynomial \( p \) which is a product of \textit{distinct} linear factors, such that \( p(\alpha) = 0 \).
	In other words, there exist distinct \( \lambda_1, \dots, \lambda_k \) such that
	\[
		p(t) = \prod_{i=1}^n (t - \lambda_i) \implies p(\alpha) = 0
	\]
\end{theorem}
\begin{proof}
	Suppose \( \alpha \) is diagonalisable in a basis \( B \).
	Let \( \lambda_1, \dots, \lambda_k \) be the \( k \leq n \) \textit{distinct} eigenvalues.
	Let
	\[
		p(t) = \prod_{i=1}^k (t-\lambda_i)
	\]
	Let \( v \in B \).
	Then \( \alpha(v) = \lambda_i v \) for some \( i \).
	Then, since the terms in the following product commute,
	\[
		(\alpha - \lambda_i I)(v) = 0 \implies p(\alpha)(v) = \qty[\prod_{i=1}^k (\alpha - \lambda_i I)](v) = 0
	\]
	So for all basis vectors, \( p(\alpha)(v) \).
	By linearity, \( p(\alpha) = 0 \).

	Conversely, suppose that \( p(\alpha) = 0 \) for some polynomial \( p(t) = \prod_{i=1}^k (t-\lambda_i) \) with distinct \( \lambda_i \).
	Let \( V_{\lambda_i} = \ker(\alpha - \lambda_i I) \).
	We claim that
	\[
		V = \bigoplus_{i=1}^k V_{\lambda_i}
	\]
	Consider the polynomials
	\[
		q_j(t) = \prod_{i=1,i \neq j}^k \frac{t-\lambda_i}{\lambda_j - \lambda_i}
	\]
	These polynomials evaluate to one at \( \lambda_j \) and zero at \( \lambda_i \) for \( i \neq j \).
	Hence \( q_j(\lambda_i) = \delta_{ij} \).
	We now define the polynomial
	\[
		q = q_1 + \dots + q_k
	\]
	The degree of \( q \) is at most \( (k-1) \).
	Note, \( q(\lambda_i) = 1 \) for all \( i \in \qty{1, \dots, k} \).
	The only polynomial that evaluates to one at \( k \) points with degree at most \( (k-1) \) is exactly given by \( q(t) = 1 \).
	Consider the endomorphism
	\[
		\pi_j = q_j(\alpha) \in L(V)
	\]
	These are called the `projection operators'.
	By construction,
	\[
		\sum_{j=1}^k \pi_j = \sum_{j=1}^k q_j(\alpha) = I
	\]
	So the sum of the \( \pi_j \) is the identity.
	Hence, for all \( v \in V \),
	\[
		I(v) = v = \sum_{j=1}^k \pi_j(v) = \sum_{j=1}^k q_j(\alpha)(v)
	\]
	So we can decompose any vector as a sum of its projections \( \pi_j(v) \).
	Now, by definition of \( q_j \) and \( p \),
	\begin{align*}
		(\alpha - \lambda_j I) q_j(\alpha)(v) & = \frac{1}{\prod_{i \neq j} (\lambda_j - \lambda_i)} (\alpha - \lambda_j I) \qty[\prod_{i \neq j} (t - \lambda_i)] (\alpha) \\
		                                      & = \frac{1}{\prod_{i \neq j} (\lambda_j - \lambda_i)} \prod_{i=1}^k (\alpha - \lambda_i I)(v)                                \\
		                                      & = \frac{1}{\prod_{i \neq j} (\lambda_j - \lambda_i)} p(\alpha)(v)
	\end{align*}
	By assumption, this is zero.
	For all \( v \), we have \( (\alpha - \lambda_j I) q_j(\alpha)(v) \).
	Hence,
	\[
		(\alpha - \lambda_j I) \pi_j(v) = 0 \implies \pi_j(v) \in \ker(\alpha - \lambda_j I) = v_j
	\]
	We have then proven that, for all \( v \in V \),
	\[
		v = \sum_{j=1}^k \underbrace{\pi_j(v)}_{\in V_j}
	\]
	Hence,
	\[
		V = \sum_{j=1}^k V_j
	\]
	It remains to show that the sum is direct.
	Indeed, let
	\[
		v \in V_{\lambda_j} \cap \qty(\sum_{i \neq j} V_{\lambda_i})
	\]
	We must show \( v = 0 \).
	Applying \( \pi_j \),
	\[
		\pi_j(v) = q_j(\alpha)(v) = \prod_{i \neq j} \frac{(\alpha - \lambda_i I)(v)}{\lambda_j - \lambda_i}
	\]
	Since \( \alpha(v) = \lambda_j v \),
	\[
		\pi_j(v) = \prod_{i \neq j} \frac{(\lambda_j - \lambda_i)v}{\lambda_j - \lambda_i} = v
	\]
	Hence \( \pi_j \) really projects onto \( V_{\lambda_j} \).
	However, we also know \( v \in \sum_{i \neq j} V_{\lambda_i} \).
	So we can write \( v = \sum_{i \neq j} w_i \) for \( w \in V_{\lambda_i} \).
	Thus,
	\[
		\pi_j(w_i) = \prod_{m \neq j} \frac{(\alpha - \lambda_m I)(v)}{\lambda_m - \lambda_j}
	\]
	Since \( \alpha(w_i) = \lambda_i w_i \), one of the factors will vanish, hence
	\[
		\pi_j(w_i) = 0
	\]
	So
	\[
		v = \sum_{i \neq j} w_i \implies \pi_j(v) = \sum_{i \neq j} \pi_j(w_i) = 0
	\]
	But \( v = \pi_j(v) \) hence \( v = 0 \).
	So the sum is direct.
	Hence, \( B = (B_1, \dots, B_k) \) is a basis of \( V \), where the \( B_i \) are bases of \( V_{\lambda_i} \).
	Then \( [\alpha]_B \) is diagonal.
\end{proof}
\begin{remark}
	We have shown further that if \( \lambda_1, \dots, \lambda_k \) are distinct eigenvalues of \( \alpha \), then
	\[
		\sum_{i=1}^k V_{\lambda_i} = \bigoplus_{i=1}^k V_{\lambda_i}
	\]
	Therefore, the only way that diagonalisation fails is when this sum is not direct, so
	\[
		\sum_{i=1}^k V_{\lambda_i} < V
	\]
\end{remark}
\begin{example}
	Let \( F = \mathbb C \).
	Let \( A \in M_n(F) \) such that \( A \) has finite order; there exists \( m \in \mathbb N \) such that \( A^m = I \).
	Then \( A \) is diagonalisable.
	This is because
	\[
		t^m - 1 = p(t) = \prod_{j=1}^m (t - \xi_m^j);\quad \xi_m = e^{2 \pi i/m}
	\]
	and \( p(A) = 0 \).
\end{example}

\subsection{Simultaneous diagonalisation}
\begin{theorem}
	Let \( \alpha, \beta \) be endomorphisms of \( V \) which are diagonalisable.
	Then \( \alpha, \beta \) are \textit{simultaneously diagonalisable} (there exists a basis \( B \) of \( V \) such that \( [\alpha]_B, [\beta]_B \) are diagonal) if and only if \( \alpha \) and \( \beta \) commute.
\end{theorem}
\begin{proof}
	Two diagonal matrices commute.
	If such a basis exists, \( \alpha \beta = \beta \alpha \) in this basis.
	So this holds in any basis.

	Conversely, suppose \( \alpha \beta = \beta \alpha \).
	We have
	\[
		V = \bigoplus_{i=1}^k V_{\lambda_i}
	\]
	where \( \lambda_i, \dots, \lambda_k \) are the \( k \) distinct eigenvalues of \( \alpha \).
	We claim that \( \beta \qty(V_{\lambda_j}) \leq V_{\lambda_j} \).
	Indeed, for \( v in V_{\lambda_j} \),
	\[
		\alpha \beta(v) = \beta \alpha(v) = \beta(\lambda_j v) = \lambda_j \beta(v) \implies \alpha(\beta(v)) = \lambda_j \beta(v)
	\]
	Hence, \( \beta(v) \in V_{\lambda_j} \).
	By assumption, \( \beta \) is diagonalisable.
	Hence, there exists a polynomial \( p \) with distinct linear factors such that \( p(\beta) = 0 \).
	Now, \( \beta\qty(V_{\lambda_j}) \leq V_{\lambda_j} \) so we can consider \( \eval{\beta}_{V_{\lambda_j}} \).
	This is an endomorphism of \( V_{\lambda_j} \).
	We can compute
	\[
		p\qty(\eval{\beta}_{V_{\lambda_j}}) = 0
	\]
	Hence, \( \eval{\beta}_{V_{\lambda_j}} \) is diagonalisable.
	Let \( B_i \) be the basis of \( V_{\lambda_i} \) in which \( \eval{\beta}_{V_{\lambda_j}} \) is diagonal.
	Since \( V = \bigoplus V_{\lambda_i} \), \( B = (B_1, \dots, B_k) \) is a basis of \( V \).
	Then the matrices of \( \alpha \) and \( \beta \) in \( V \) are diagonal.
\end{proof}

\subsection{Minimal polynomials}
Recall from IB Groups, Rings and Modules the Euclidean algorithm for dividing polynomials.
Given \( a, b \) polynomials over \( F \) with \( b \) non-zero, there exist polynomials \( q, r \) over \( F \) with \( \deg r < \deg b \) and \( a = qb + r \).
\begin{definition}
	Let \( V \) be a finite dimensional \( F \)-vector space.
	Let \( \alpha \) be an endomorphism on \( V \).
	The \textit{minimal polynomial} \( m_\alpha \) of \( \alpha \) is the non-zero polynomial with smallest degree such that \( m_\alpha(\alpha) = 0 \).
\end{definition}
\begin{remark}
	If \( \dim V = n < \infty \), then \( \dim L(V) = n^2 \).
	In particular, the family \( \qty{I, \alpha, \dots, \alpha^{n^2}} \) cannot be free since it has \( n^2+1 \) entries.
	This generates a polynomial in \( \alpha \) which evaluates to zero.
	Hence, a minimal polynomial always exists.
\end{remark}
\begin{lemma}
	Let \( \alpha \in L(V) \) and \( p \in F[t] \) be a polynomial.
	Then \( p(\alpha) = 0 \) if and only if \( m_\alpha \) is a factor of \( p \).
	In particular, \( m_\alpha \) is well-defined and unique up to a constant multiple.
\end{lemma}
\begin{proof}
	Let \( p \in F[t] \) such that \( p(\alpha) = 0 \).
	If \( m_\alpha(\alpha) = 0 \) and \( \deg m_\alpha < \deg p \), we can perform the division \( p = m_\alpha q + r \) for \( \deg r < \deg m_\alpha \).
	Then \( p(\alpha) = m_\alpha(\alpha) q(\alpha) + r(\alpha) \).
	But \( m_\alpha(\alpha) = 0 \).
	But \( \deg r < \deg m_\alpha \) and \( m_\alpha \) is the smallest degree polynomial which evaluates to zero for \( \alpha \), so \( r \equiv 0 \) so \( p = m_\alpha q \).
	In particular, if \( m_1, m_2 \) are both minimal polynomials that evaluate to zero for \( \alpha \), we have \( m_1 \) divides \( m_2 \) and \( m_2 \) divides \( m_1 \).
	Hence they are equivalent up to a constant.
\end{proof}
\begin{example}
	Let \( V = F^2 \) and
	\[
		A= \begin{pmatrix}
			1 & 0 \\
			0 & 1
		\end{pmatrix};\quad B = \begin{pmatrix}
			1 & 1 \\
			0 & 1
		\end{pmatrix}
	\]
	We can check \( p(t) = (t-1)^2 \) gives \( p(A) = p(B) = 0 \).
	So the minimal polynomial of \( A \) or \( B \) must be either \( (t-1) \) or \( (t-1)^2 \).
	For \( A \), we can find the minimal polynomial is \( (t-1) \), and for \( B \) we require \( (t-1)^2 \).
	So \( B \) is not diagonalisable, since its minimal polynomial is not a product of distinct linear factors.
\end{example}

\section{Exponents, logarithms and powers}
\subsection{Expectation}
Consider a continuous random variable $X \colon \Omega \to \mathbb R$, with probability distribution function $F(x)$ and probability density function $f(x) = F'(x)$. We define the expectation of such a \textit{non-negative} random variable as
\[ \expect{X} = \int_0^\infty x f(x) \dd{x} \]
In this case, the expectation is either non-negative and finite, or positive infinity. Now, let $X$ be a general continuous random variable, that is not necessarily non-negative. Suppose $g \geq 0$. Then,
\[ \expect{g(X)} = \int_{-\infty}^\infty g(x) f(x) \dd{x} \]
We can define $X_+ = \max(X, 0)$ and $X_- = \min(-X, 0)$. If at least one of $\expect{X_+}$ or $\expect{X_-}$ is finite, then clearly
\[ \expect{X} := \expect{X_+} - \expect{X_-} = \int_{-\infty}^\infty xf(x) \dd{x} \]
It is easy to verify that the expectation is a linear function, due to the linearity property of the integral.

\subsection{Computing the Expectation}
\begin{claim}
	Let $X \geq 0$. Then
	\[ \expect{X} = \int_0^\infty \prob{X \geq x} \dd{x} \]
\end{claim}
\begin{proof}
	Using the definition of the expectation,
	\begin{align*}
		\expect{X} & = \int_0^\infty xf(x) \dd{x}                                           \\
		           & = \int_0^\infty \left( \int_0^x \dd{y} \right) f(x) \dd{x}             \\
		           & = \int_0^x \dd{y} \int_y^\infty f(x) \dd{x}                            \\
		           & = \int_0^\infty \dd{y} \left( 1 - \int_{-\infty}^y f(x) \dd{x} \right) \\
		           & = \int_0^\infty \dd{y} \prob{X \geq y}
	\end{align*}
\end{proof}
\noindent Here is an alternative proof.
\begin{proof}
	For every $\omega \in \Omega$, we can write
	\[ X(\omega) = \int_0^\infty 1(X(\omega) \geq x) \dd{x} \]
	Taking expectations, we get
	\[ \expect{X} = \expect{\int_0^\infty 1(X(\omega) \geq x) \dd{x}} \]
	We will interchange the integral and the expectation, although this step is not justified or rigorous.
	\begin{align*}
		\expect{X} & = \int_0^\infty \expect{1(X(\omega) \geq x)} \dd{x} \\
		           & = \int_0^\infty \prob{X \geq x} \dd{x}
	\end{align*}
\end{proof}

\subsection{Variance}
We define the variance of a continuous random variable as
\[ \Var{X} = \expect{(X - \expect{X})^2} = \expect{X^2} - \expect{X}^2 \]

\subsection{Uniform Distribution}
Consider the uniform distribution defined by $a, b \in\mathbb R$.
\[ f(x) = \begin{cases}
		\frac{1}{b-a} & x \in [a, b]     \\
		0             & \text{otherwise}
	\end{cases} \]
We write $X \sim U[a, b]$. For some $x \in [a,b]$, we can write
\[ \prob{X \leq x} = \int_a^x f(y) \dd{y} = \frac{x-a}{b-a} \]
Hence, for $x \in [a,b]$,
\[ F(x) = \begin{cases}
		1               & x > b       \\
		\frac{x-a}{b-a} & x \in [a,b] \\
		0               & x < a
	\end{cases} \]
Then,
\[ \expect{X} = \int_a^b \frac{x}{b-a} \dd{x} = \frac{a+b}{2} \]

\subsection{Exponential Distribution}
The exponential distribution is defined by $f(x) = \lambda e^{-\lambda x}$ for $\lambda > 0$, $x > 0$. We write $X \sim \mathrm{Exp}(\lambda)$.
\[ F(x) = \prob{X \geq x} = \int_0^x \lambda e^{-\lambda y} \dd{y} = 1 - e^{-\lambda x} \]
Further,
\[ \expect{X} = \int_0^\infty \lambda e^{-\lambda x} \dd{x} = \frac{1}{\lambda} \]
We can view the exponential distribution as a limit of geometric distributions. Suppose that $T \sim \mathrm{Exp}(\lambda)$, and let $T_n = \floor{nT}$ for all $n \in \mathbb N$. We have
\[ \prob{T_n \geq k} = \prob{T \geq \frac{k}{n}} = e^{-\lambda k / n} = \left( e^{-\lambda/n} \right)^k \]
Hence $T_n$ is a geometric distribution with parameter $p_n = e^{-\lambda/n}$. As $n \to \infty$, $p_n \sim \frac{\lambda}{n}$, and $\frac{T_n}{n} \sim T$. Hence the exponential distribution is the limit of a scaled version of the geometric distribution. A key property of the exponential distribution is that it has no memory. If $T \sim \mathrm{Exp}(\lambda)$, $\prob{T > t + s \mid T > s} = \prob{T > t}$. In fact, the distribution is uniquely characterised by this property.
\begin{proposition}
	Let $T$ be a positive continuous random variable not identically zero or infinity. Then $T$ has the memoryless property $\prob{T > t + s \mid T > s} = \prob{T > t}$ if and only if $T \sim \mathrm{Exp}(\lambda)$ for some $\lambda > 0$.
\end{proposition}
\begin{proof}
	Clearly if $T \sim \mathrm{Exp}(\lambda)$, then $\prob{T > t + s \mid T > s} = e^{-\lambda t} = \prob{T > t}$ as required. Now, given that $T$ has this memoryless property, for all $s$ and $t$, we have $\prob{T > t + s} = \prob{T > t} \prob{T > s}$. Let $g(t) = \prob{T > t}$; we would like to show that $g(t) = e^{-\lambda t}$. Then $g$ satisfies $g(t+s) = g(t)g(s)$. Then for all $m \in \mathbb N$, $g(mt) = (g(t))^m$. Setting $t=1$, $g(m) = g(1)^m$. Now, $g(m/n)^n = g(mn/n) = g(m)$ hence $g(m/n) = g(1)^{m/n}$. So for all rational numbers $q \in \mathbb Q$, $g(q) = g(1)^q$.

	Now, $g(1) = \prob{T > 1} \in (0, 1)$. Indeed, $g(1) \neq 0$ since in this case, for any rational number $q$ we would have $g(q) = 0$ contradicting the assumption that $T$ was not identically zero, and $g(1) \neq \infty$ because in this case $T$ would be identically infinity. Now, let $\lambda = -\log\prob{T > 1} > 0$. We have now proven that $g(t) = e^{-\lambda t}$ for all $t\in\mathbb Q$.

	Let $t \in \mathbb R_+$. Then for all $\varepsilon > 0$, there exist $r, s \in \mathbb Q$ such that $r \leq t \leq s$ and $\abs{r - s} \leq \varepsilon$. In this case, $e^{-\lambda s} = \prob{T > s} \leq \prob{T > t} \leq \prob{T > r} = e^{-\lambda r}$. Sending $\varepsilon \to 0$ finishes the proof, showing that $g(t) = e^{-\lambda t}$ for all positive reals.
\end{proof}

\subsection{Functions of Continuous Random Variables}
\begin{theorem}
	Suppose that $X$ is a continuous random variable with density $f$. Let $g$ be a monotonic continuous function (either strictly increasing or strictly decreasing), such that $g^{-1}$ is differentiable. Then $g(X)$ is a continuous random variable with density $fg^{-1}(x) \abs{\dv{x} g^{-1}(x)}$.
\end{theorem}
\begin{proof}
	Suppose that $g$ is strictly increasing. We have
	\[ \prob{g(X) \leq x} = \prob{X \leq g^{-1}(x)} = F(g^{-1}(x)) \]
	Hence,
	\[ \dv{x} \prob{g(X) \leq x} = F'(g^{-1}(x)) \cdot \dv{x} g^{-1}(x) = f(g^{-1}(x)) \dv{x}g^{-1}(x) \]
	Note that since $g$ is strictly increasing, so is $g^{-1}$. Now, suppose the $g$ is strictly decreasing. Since the random variable is continuous,
	\[ \prob{g(X) \leq x} = \prob{X \geq g^{-1}(x)} = 1 - F(g^{-1}(x)) \]
	Hence,
	\[ \dv{x} \prob{g(X) \leq x} = -F'(g^{-1}(x)) \cdot \dv{x} g^{-1}(x) = f(g^{-1}(x)) \abs{\dv{x}g^{-1}(x)} \]
	Likewise, in this case, $g$ is strictly decreasing.
\end{proof}

\subsection{Normal Distribution}
The normal distribution is characterised by $\mu \in \mathbb R$ and $\sigma > 0$. We define
\[ f(x) = \frac{1}{\sqrt{2 \pi \sigma^2}} \exp\qty{-\frac{(x-\mu)^2}{2\sigma^2}} \]
$f(x)$ is indeed a probability density function:
\[ I = \int_{-\infty}^\infty f(x) \dd{x} = \int_{-\infty}^\infty \frac{1}{\sqrt{2 \pi \sigma^2}} \exp\qty{-\frac{(x-\mu)^2}{2\sigma^2}} \dd{x} \]
Applying the substitution $x \mapsto \frac{x-\mu}{\sigma}$, we have
\[ I = \frac{1}{\sqrt{2 \pi}} \int_{-\infty}^\infty \exp\qty{-\frac{x^2}{2}} \dd{x} \]
We can evaluate this integral by considering $I^2$.
\[ I^2 = \frac{2}{\pi} \int_0^\infty \int_0^\infty e^{\frac{-(u^2 - v^2)}{2}} \dd{u}\dd{v} \]
Using polar coordinates $u = r\cos\theta$ and $v = r\sin\theta$, we have
\[ I^2 = \frac{2}{\pi} \int_0^\infty \dd{r} \int_0^{\frac{\pi}{2}} \dd{\theta} re^{-\frac{r^2}{2}} = 1 \implies I = \pm 1 \]
But clearly $I > 0$, so $I=1$. Hence $f$ really is a probability density function. Now, if $X \sim \mathrm{N}(\mu, \sigma^2)$,
\begin{align*}
	\expect{X} & = \int_{-\infty}^{\infty} \frac{x}{\sqrt{2\pi\sigma^2}} \exp\qty{-\frac{(x-\mu)^2}{2\sigma^2}} \dd{x}                                                                                                                                                                                                                      \\
	           & = \underbrace{\int_{-\infty}^{\infty} \frac{x - \mu}{\sqrt{2\pi\sigma^2}} \exp\qty{-\frac{(x-\mu)^2}{2\sigma^2}} \dd{x}}_{\text{odd function around } \mu \text{ hence } 0} + \mu\underbrace{\int_{-\infty}^{\infty} \frac{1}{\sqrt{2\pi\sigma^2}} \exp\qty{-\frac{(x-\mu)^2}{2\sigma^2}} \dd{x}}_{I = 1 \text{ by above}} \\
	           & = \mu                                                                                                                                                                                                                                                                                                                      \\
\end{align*}
We can also compute the variance, using the substitution $u = \frac{x - \mu}{\sigma}$, giving
\begin{align*}
	\Var{X} & = \int_{-\infty}^{\infty} \frac{(x - \mu)^2}{\sqrt{2\pi\sigma^2}} \exp\qty{-\frac{(x-\mu)^2}{2\sigma^2}} \dd{x} \\
	        & = \sigma^2 \int_{-\infty}^{\infty} \frac{u^2}{\sqrt{2\pi}} \exp\qty{-\frac{u^2}{2}} \dd{u}                      \\
	        & = \sigma^2
\end{align*}
In particular, when $\mu = 0$ and $\sigma^2 = 1$, we call the distribution $\mathrm{N}(\mu, \sigma^2) = \mathrm{N}(0, 1)$ the standard normal distribution. We define
\[ \Phi(x) = \int_{-\infty}^x \frac{1}{\sqrt{2\pi}} e^{-\frac{u^2}{2}} \dd{u};\quad \phi(x) = \Phi'(x) = \frac{1}{\sqrt{2\pi}} e^{-\frac{x^2}{2}} \]
Hence $\Phi(x) = \prob{X \leq x}$ if $X$ has the standard normal distribution. Since $\phi(x) = \phi(-x)$, we have $\Phi(x) + \Phi(-x) = 1$, hence $\prob{X \leq x} = 1 - \prob{X \leq -x}$.

\section{Trigonometric functions}
\subsection{Path-connectedness}
\begin{definition}
	Let \( X \) be a topological space.
	For points \( x, y \in X \), a \textit{path} from \( x \) to \( y \) in \( X \) is a continuous function \( \gamma \colon [0,1] \to X \) such that \( \gamma(0) = x, \gamma(1) = y \).
	We say that \( X \) is \textit{path-connected} if for all \( x, y \in X \), there exists a path from \( x \) to \( y \) in \( X \).
\end{definition}
\begin{example}
	In \( \mathbb R^n \), \( \mathcal D_r(x) \) is path-connected by a straight line segment between any two points in the ball.
	In particular, let \( \gamma(t) = (1-t)y + tz \).
	This is continuous and lies entirely inside \( \mathcal D_r(x) \), since
	\[
		\norm{\gamma(t) = x} = \norm{(1-t)t + tz - x} = \norm{((1-t)y+tz)-((1-t)x+tx)} \leq (1-t)\norm{y-x} + t\norm{z-x} < r
	\]
	In a similar way, any convex subset of \( \mathbb R^n \) is path-connected.
\end{example}
\begin{theorem}
	If \( X \) is path-connected, \( X \) is connected.
\end{theorem}
\begin{proof}
	Suppose \( X \) is not connected.
	Let \( U, V \) disconnect \( X \).
	Let \( x \in U, y \in V \), and suppose \( \gamma \colon [0,1] \to X \) is continuous with \( \gamma(0) = x \) and \( \gamma(1) = y \).
	Then \( \gamma^{-1}(U) \) and \( \gamma^{-1}(V) \) disconnect \( [0,1] \), which contradicts the connectedness of \( [0,1] \).
\end{proof}
\begin{example}
	The converse is false in general.
	Recall that the space
	\[
		X = \qty{\qty(x, \sin \frac{1}{x}) \colon x > 0} \cup \qty{(0,y) \colon -1 \leq y \leq 1}
	\]
	is connected.
	We will show \( X \) is not path-connected.
	Suppose \( \gamma \colon [0,1] \to X \) is continuous, and \( \gamma(0) = (0,0) \) and \( \gamma(1) = (1, \sin 1) \).
	Let \( \gamma = (\gamma_1, \gamma_2) \), so \( \gamma_1, \gamma_2 \) are continuous functions.
	Suppose \( t \in [0,1] \) such that \( \gamma_1(t) > 0 \).
	Then \( \gamma_1((0,t)) \supset (0, \gamma_1(t)) \) by the intermediate value theorem.
	In particular, there exists \( n \in \mathbb N \) such that \( \frac{1}{2\pi n} \in (0, \gamma_1(t)) \).
	Hence, there exists \( s < t \) such that \( \gamma_1(s) = \frac{1}{2\pi n} \) so \( \gamma_1(s) = 0 \).
	Similarly, \( \frac{1}{2\pi n + \frac{\pi}{2}} \in (0, \gamma_1(t)) \) so there exists a different \( s < t \) such that \( \gamma_1(s) = \frac{1}{2\pi n + \frac{\pi}{2}} \) hence \( \gamma_2(s) = 1 \).
	In both cases, \( \gamma_1(s) > 0 \).
	We can inductively find a sequence \( 1 > t_1 > t_2 > \dots > 0 \) such that \( \gamma_2(t_n) \) alternates between zero and one.
	But then \( t_n \to t \) since it is a decreasing bounded-below sequence, and \( \gamma_2 \) is continuous, so \( \gamma_2(t_n) \to \gamma_2(t) \) which is a contradiction.
\end{example}

\subsection{Gluing lemma}
\begin{lemma}
	Let \( X \) be a topological space.
	Suppose \( X = A \cup B \) where \( A, B \) are closed in \( X \).
	Let \( g \colon A \to Y \) and \( h \colon B \to Y \) be continuous where \( Y \) is a topological space, such that for \( A \cap B \), we have \( g = h \).
	Then \( f \colon X \to Y \) defined by
	\[
		f(x) = \begin{cases}
			g(x) & x \in A \\
			h(x) & x \in B
		\end{cases}
	\]
	is well defined and continuous.
\end{lemma}
\begin{proof}
	First, observe that if \( F \subset A \) and \( F \) is closed in \( A \), then there exists a closed set \( G \) in \( X \) such that \( F = A \cap G \).
	Since \( A \) is closed in \( X \), we must have \( F \) is closed in \( X \).
	The same holds for \( F \subset B \).
	Now, let \( V \) be a closed set in \( Y \).
	Then the inverse image of \( V \) under \( f \) is
	\[
		f^{-1}(V) = (f^{-1}(V) \cap A) \cup (f^{-1}(V) \cap B) = \underbrace{g^{-1}(V)}_{\text{closed in } A} \cup \underbrace{h^{-1}(V)}_{\text{closed in } B}
	\]
	So \( f^{-1}(V) \) is closed in \( X \).
	To prove continuity it suffices to show that the preimage of a closed set is closed, since that implies that the preimage of an open set is open.
\end{proof}
\begin{definition}
	Let \( X \) be a topological space.
	For \( x, y \in X \), we write \( x \sim y \) if there exists a path from \( x \) to \( y \) in \( X \).
	This is an equivalence relation:
	\begin{enumerate}[(i)]
		\item The constant function shows that \( x \sim x \) for all \( x \).
		\item If \( \gamma \colon [0,1] \to X \) is continuous and \( \gamma(0) = x \), \( \gamma(1) = y \), we define \( t \mapsto \gamma(1-t) \), which is a path from \( y \) to \( x \).
		\item Finally, if \( x \sim y \) and \( y \sim z \), we have continuous functions \( \gamma, \delta \) such that \( \gamma(0) = x, \gamma(1) = y = \delta(0), \delta(1) = z \).
		      Then let
		      \[
			      \eta(t) = \begin{cases}
				      \gamma(2t)   & t \in \qty[0, \frac{1}{2}] \\
				      \delta(2t-1) & t \in \qty[\frac{1}{2}, 1]
			      \end{cases}
		      \]
		      These intervals are closed on \( [0,1] \) and their union is \( [0,1] \).
		      On the intersection, they are equal.
		      By the gluing lemma, \( \eta \) is continuous, and now since \( \eta(0) = x, \eta(1) = z \) we have \( x \sim z \).
	\end{enumerate}
	We call the equivalence classes \textit{path-connected components} of \( X \).
\end{definition}
\begin{theorem}
	Let \( U \) be an open subset of \( \mathbb R^n \).
	Then \( U \) is connected if and only if \( U \) is path-connected.
\end{theorem}
\begin{proof}
	The converse is trivial.
	Suppose \( U \) is connected.
	Without loss of generality, suppose \( U \neq \varnothing \).
	Let \( x_0 \in U \).
	Let \( P = \qty{x \in U \colon x \sim x_0} \) be the equivalence class of \( x_0 \).
	We want to show \( P = U \).
	To do this, we will show that \( P \) is open and closed in \( U \).
	Then, \( P, U \setminus P \) will disconnect \( U \) unless \( P = \varnothing \) or \( P = U \).
	But we know \( x_0 \in P \), hence \( P = U \) will be the only possibility.

	To show \( P \) is open, let \( x \in U \).
	Since \( U \) is open, there exists \( r > 0 \) such that \( \mathcal D_r(x) \subset U \).
	Recall that for all \( y \in \mathcal D_r(x) \), we have \( y \sim x \).
	Now, if \( x \in P \), then we have \( y \sim x \) and \( x \sim x_0 \) so \( y \sim x_0 \).
	So \( \mathcal D_r(x) \subset P \).
	So \( P \) is open.

	Now, if \( x \in U \setminus P \) and \( y \in \mathcal D_r(x) \) has \( y \sim x_0 \), then by transitivity \( x \sim x_0 \).
	But this is a contradiction since \( x \not\in P \).
	Hence \( U \setminus P \) is open.
	So \( P \) is open and closed, so \( P = U \).
\end{proof}
\begin{theorem}
	For \( n \geq 2 \), \( \mathbb R \) and \( \mathbb R^n \) are not homeomorphic.
\end{theorem}
The generalisation \( \mathbb R^m \not\simeq \mathbb R^n \) is true, but significantly harder to prove and outside the scope of this course.
\begin{proof}
	Suppose \( f \colon \mathbb R \to \mathbb R^n \) is a homeomorphism.
	Let \( g = f^{-1} \).
	Then \( g \) is continuous.
	Then, \( \eval{f}_{\mathbb R \setminus \qty{0}} \) is a homeomorphism from \( \mathbb R \setminus \qty{0} \) to \( \mathbb R^n \setminus \qty{f(0)} \), with inverse \( \eval{g}_{\mathbb R^n \setminus \qty{f(0)}} \).
	But \( \mathbb R \setminus \qty{0} \) is disconnected, but \( \mathbb R^n \setminus \qty{f(0)} \) is connected since it is path-connected.
	This is a contradiction.
\end{proof}

\section{Integration}
\subsection{Injection, Surjection and Bijection}
\begin{definition}
	A function $f\colon A \to B$ is
	\begin{itemize}
		\item injective, if $\forall a, a' \in A$, we have $a \neq a' \implies f(a) \neq f(a')$, or equivalently, $f(a) = f(a') \implies a = a'$, or in words, `different points stay different' (e.g. example 6 above).
		\item surjective, if $\forall b \in B$, $\exists a \in A$ such that $f(a) = b$, or in words, `everything in $B$ is hit' (e.g. examples 6 and 8).
		\item bijective, if it is injective and surjective, or in words, `everything in $B$ is hit exactly once', or `$f$ pairs up elements of $A$ and elements of $B$' (e.g. example 6, or $f\colon \mathbb R \to \mathbb R$ given by $f(x) = x^3$).
	\end{itemize}
\end{definition}
\begin{definition}
	For a function $f\colon A \to B$, $A$ is the domain, $B$ is the range, and $\{ b \in B : \exists a \in A \st f(a) = b \}$ is the image.
\end{definition}
We must always provide the domain and range of a function; a function's properties depend on this. For example, is the function $f$ defined by $f(x) = f^2$ injective? If $f\colon \mathbb N \to \mathbb N$, then it is injective, but if $f\colon \mathbb Z \to \mathbb Z$, then it is not.

There are a number of properties that hold specifically for finite sets $A$, $B$:
\begin{enumerate}
	\item There is no surjection $A \to B$ if $\abs{B} > \abs{A}$.
	\item There is no injection $A \to B$ if $\abs{A} > \abs{B}$.
	\item For a function $f\colon A \to A$, $f$ injective $\iff$ $f$ surjective. Hence, if $f$ is either injective or surjective, it is bijective.
	\item There is no bijection from $A$ to any proper subset of $A$.
\end{enumerate}
As counterexamples for infinite sets:
\begin{enumerate}
	\item We define $f_0\colon \mathbb N \to \mathbb N$ by $f_0(x) = x+1$. Then, $f_0$ is injective but not surjective.
	\item We define $f_1\colon \mathbb N \to \mathbb N$ by $f_0(x) = x-1$, or 1 if $x=1$. Then, $f_0$ is surjective but not injective.
	\item We define $g\colon \mathbb N \to \mathbb N \setminus \{ 1 \}$ by $g(x) = x+1$. Then, $g$ is bijective between $\mathbb N$ and a proper subset of $\mathbb N$.
\end{enumerate}

\subsection{More Examples of Functions}
\begin{enumerate}
	\item For any set $X$ we have $1_X\colon X \to X$ defined by $1_X(x) = x$. This is known as the identity function on $X$.
	\item For any set $X$, and $A \subset X$, we have an indicator function (or characteristic function) $\chi_A\colon X \to \{ 0, 1 \}$ defined by
	      \[ \chi_A(x) = \begin{cases}
			      0 & x \notin A \\
			      1 & x \in A
		      \end{cases} \]
	\item A sequence of reals $x_1, x_2, \dots$ is a function $f\colon \mathbb N \to \mathbb R$ defined by $f(n) = x_n$.
	\item The operation $+$ on $\mathbb N$ is a function $\mathbb N^2 \to \mathbb N$.
	\item A set $X$ has size $n$ $\iff$ there is a bijection between $X$ and $\{ 1, 2, \dots, n \}$.
\end{enumerate}

\subsection{Composition of Functions}
Given $f\colon A \to B$ and $g\colon B \to C$, we define the composition $g\circ f \colon A \to C$, given by $(g\circ f)(a) = g(f(a))$. For example, if $f\colon \mathbb R \to \mathbb R$, $f(x) = 2x$, $g\colon \mathbb R \to \mathbb R$, $g(x) = x+1$, then $(f \circ g)(x) = 2(x+1)$, and $(g \circ f)(x) = 2x + 1$.

In general, the operation $\circ$ is not commutative, as we can see from this example. However, $\circ$ is associative. Given $f\colon A \to B$, $g\colon B \to C$, $h\colon C \to D$, we have $h \circ (g \circ f) = (h \circ g) \circ f$. Indeed, for any input $x \in A$,
\[ (h \circ (g \circ f))(x) = h((g \circ f)(x)) = h(g(f(x))) = (h \circ g)(f(x)) = ((h \circ g)\circ f)(x) \]
Thus $(h \circ (g \circ f))(x) = ((h \circ g)\circ f)(x)$ for every $x \in A$, so $h \circ (g \circ f) = (h \circ g)\circ f$.

\section{Classes of integrable functions}
\subsection{Discrete sampling and the Nyquist frequency}
Suppose a signal \( h(t) \) is sampled at equal times \( t_n = n\Delta \) with a time step \( \Delta \) and values \( h_n = h(t_n) = h(n\Delta) \), for all \( n \in \mathbb Z \).
The sampling frequency is therefore \( \Delta^{-1} \), so the sampling angular velocity is \( \omega_s = 2\pi f_s = \frac{2\pi}{\Delta} \).
The Nyquist frequency is \( f_c = \frac{1}{2\Delta} \), which is the highest frequency actually sampled at \( \Delta \).
Suppose we have a signal \( g_f \) with a given frequency \( f \).
We will write
\[
	g_f(t) = A \cos(2\pi f t + \varphi) = \Re \qty(A e^{2 \pi i f t + \varphi}) = \frac{1}{2} \qty(A e^{2 \pi i f t + \varphi}) + \frac{1}{2} \qty(A e^{-2 \pi i f t + \varphi})
\]
where \( A \in \mathbb R \).
Note that this signal has two `frequencies'; a positive and a negative frequency.
The combination of these frequencies gives the full wave.
Suppose we sample \( g_f(t) \) at the Nyquist frequency, so \( f = f_c \).
Then,
\begin{align*}
	g_{f_c}(t_n) & = A \cos(2 \pi \frac{1}{2\Delta} n \Delta + \varphi) \\
	             & = A \cos(\pi n + \varphi)                            \\
	             & = A \cos \pi n \cos \phi + A \sin \pi n \sin \phi    \\
	             & = A' \cos(2\pi f_c f_n)
\end{align*}
where \( A' = A \cos \phi \).
This has removed half of the information about the wave; the ampliude and the phase have become degenerate.
We can identify \( f_c \) with \( -f_c \) when considering the remaining information; we say that the two frequencies are \textit{aliased} together.
Now, suppose we sample at greater than the Nyquist frequency, in particular \( f = f_c + \delta f > f_c \), where for simplicity we let \( \delta f < f_c \).
We have
\begin{align*}
	g_f(t_n) & = A \cos(2\pi (f_c + \delta f)t_n + \varphi) \\
	         & = A \cos(2\pi (f_c - \delta f)t_n - \varphi)
\end{align*}
So frequencies above the Nyquist frequency are reinterpreted after the sampling as a frequency lower than the Nyquist frequency.
This aliases \( f_c + \delta f \) with \( f_c - \delta f \).

\subsection{Nyquist-Shannon sampling theorem}
\begin{definition}
	A signal \( g(t) \) is \textit{bandwidth-limited} if it contains no frequencies above \( \omega_{\max} = 2\pi f_{\max} \).
	In other words, \( \widetilde g(\omega) = 0 \) for all \( \abs{\omega} > \omega_{\max} \).
	In this case,
	\[
		g(t) = \frac{1}{2\pi} \int_{-\infty}^\infty \widetilde g(\omega) e^{i\omega t} \dd{\omega} = \frac{1}{2\pi} \int_{-\omega_{\max}}^{\omega_{\max}} \widetilde g(\omega) e^{i\omega t} \dd{\omega}
	\]
\end{definition}
\noindent Suppose we set the sampling rate to the Nyquist frequency, so \( \Delta = \frac{1}{2f_{\max}} \).
Then,
\[
	g_n \equiv g(t_n) = \frac{1}{2\pi} \int_{-\omega_{\max}}^{\omega_{\max}} \widetilde g(\omega) e^{i\pi n \omega / \omega_{\max}} \dd{\omega}
\]
This is a complex Fourier series coefficient \( c_n \), multiplied by \( \frac{\omega_{\max}}{\pi} \).
The Fourier series is periodic in \( \omega \) with period \( 2 \omega_{\max} \), not in space or time.
\[
	\widetilde g_\mathrm{per}(\omega) = \frac{\pi}{\omega_{\max}} \sum_{n=-\infty}^\infty g_n e^{-i \pi n \omega / \omega_{\max}}
\]
The actual Fourier transform \( \widetilde g \) is found by multiplying by a top hat window function
\[
	\widetilde h(\omega) = \begin{cases}
		1 & \abs{\omega} \leq \omega_{\max} \\
		0 & \text{otherwise}
	\end{cases}
\]
Hence,
\[
	\widetilde g(\omega) = \widetilde g_\mathrm{per}(\omega) \widetilde h(\omega)
\]
Note that this relation is exact.
Inverting this expression,
\begin{align*}
	g(t) & = \frac{1}{2\pi} \int_{-\infty}^\infty \widetilde g_\mathrm{per}(\omega) \widetilde h(\omega) e^{i \omega t} \dd{\omega}                                     \\
	     & = \frac{1}{2\omega_{\max}} \sum_{n=-\infty}^\infty g_n \int_{-\omega_{\max}}^{\omega_{\max}} \exp(i \omega\qty(t - \frac{n \pi}{\omega_{\max}})) \dd{\omega}
\end{align*}
Only the cosine term is even, hence
\[
	g(t) = \frac{1}{2\omega_{\max}} \sum_{n=-\infty}^\infty g_n \frac{\sin(\omega_{\max} t - \pi n)}{\omega_{\max} t - \pi n}
\]
Hence, \( g(t) \) can be written \textit{exactly} as a combination of countably many discrete sample points.

\section{Properties of the Riemann integral}
\subsection{Linear maps}
Let \( m, n \in \mathbb N \).
Recall that \( L(\mathbb R^m, \mathbb R^n) \) is the vector space of linear maps from \( \mathbb R^m \) to \( \mathbb R^n \).
This is isomorphic to \( M_{n,m} \), the space of \( n \times m \) real matrices.
There is also an isomorphism to \( \mathbb R^{mn} \).
Let \( e_1, \dots, e_m \) be the standard basis of \( \mathbb R^m \), and similarly let \( e_1', \dots, e_n' \) be the standard basis of \( \mathbb R^n \).
Then \( T \in L(\mathbb R^m, \mathbb R^n) \) is identified with the \( n \times m \) matrix \( (T_{ji}) \) where \( 1 \leq j \leq n \) and \( 1 \leq i \leq m \), such that \( T_{ji} = \inner{T e_i, e_j'} \).
We can therefore view \( L(\mathbb R^m, \mathbb R^n) \) as the \( mn \)-dimensional vector space \( \mathbb R^{mn} \) with the Euclidean norm.
So the norm of a linear map \( T \) is given by
\[
	\norm{T} = \sqrt{\sum_{i=1}^m \sum_{j=1}^n T_{ji}^2} = \sqrt{\sum_{i=1}^m \norm{Te_i}^2}
\]
where \( T e_i \) is the \( i \)th column of \( T \).
Thus, \( L(\mathbb R^m, \mathbb R^n) \) becomes a metric space together with the Euclidean distance \( d(S,T) = \norm{S-T} \).
\begin{lemma}
	For \( T \in L(\mathbb R^m, \mathbb R^n) \) and \( x \in \mathbb R^m \),
	\[
		\norm{Tx} \leq \norm{T} \cdot \norm{x}
	\]
	So \( T \) is a Lipschitz map and hence continuous.
	Further, if \( S \in L(\mathbb R^n, \mathbb R^p) \) then
	\[
		\norm{ST} \leq \norm{S} \cdot \norm{T}
	\]
\end{lemma}
\begin{proof}
	We can write
	\[
		x = \sum_{i=1}^m x_i e_i
	\]
	Hence,
	\[
		Tx = \sum_{i=1}^m x_i T e_i
	\]
	Thus,
	\[
		\norm{Tx} \leq \sum_{i=1}^m \abs{x_i} \norm{T e_i} \leq \qty(\sum_{i=1}^m x_i^2)^{1/2} \cdot \qty(\sum_{i=1}^m \norm{Te_i}^2)^{1/2} = \norm{T} \cdot \norm{x}
	\]
	Further, for \( x,y \in \mathbb R^m \) we have
	\[
		d(Tx, Ty) = \norm{Tx - Ty} = \norm{T(x-y)} \leq \norm{T} \cdot \norm{x-y} = \norm{T} d(x,y)
	\]
	So \( T \) is Lipschitz, and any Lipschitz function is continuous.
	Now,
	\[
		\norm{ST} = \qty(\sum_{i=1}^m \norm{STe_i}^2)^{1/2} \leq \qty(\sum_{i=1}^m \norm{S} \norm{Te_i}^2)^{1/2} = \norm{S} \qty(\sum_{i=1}^m \norm{Te_i}^2)^{1/2} = \norm{S} \cdot \norm{T}
	\]
\end{proof}

\subsection{Differentiation}
Recall from IA Analysis that a function \( f \colon \mathbb R \to \mathbb R \) is \textit{differentiable} at a point \( a \in \mathbb R \) if
\[
	\lim_{h \to 0} \frac{f(a+h) - f(a)}{h}
\]
exists.
The value of this limit is called the \textit{derivative} of \( f \) at \( a \), and denoted \( f'(a) \).
Note that \( f \) is differentiable at \( a \) if and only if there exists \( \lambda \in \mathbb R \) and \( \varepsilon \colon \mathbb R \to \mathbb R \) such that \( \varepsilon(0) = 0 \) and \( \varepsilon \) is continuous at \( 0 \), and
\[
	f(a+h) = f(a) + \lambda h + h \varepsilon(h)
\]
This is because we can define
\[
	\varepsilon(h) = \begin{cases}
		0                                 & h = 0    \\
		\frac{f(a+h) - f(a)}{h} - \lambda & h \neq 0
	\end{cases}
\]
Informally, this \( \varepsilon \) definition states that \( f \) is approximated very well (the error \( h\varepsilon(h) \) shrinks rapidly since \( \varepsilon \to 0 \)) by a linear function in a small neighbourhood of \( a \).
Recall that if \( f \) is \( n \) times differentiable at \( a \), then
\[
	f(a+h) = f(a) + \sum_{k=1}^n \frac{f^{(k)}(a)}{k!}h^k + o(h^n)
\]
\begin{definition}
	Let \( m, n \in \mathbb N \).
	Then \( f \colon \mathbb R^m \to \mathbb R^n \) and \( a \in \mathbb R^m \).
	We say that \( f \) is \textit{differentiable} at \( a \) if there exists a linear map \( T \in L(\mathbb R^m, \mathbb R^n) \) and a function \( \varepsilon \colon \mathbb R^m \to \mathbb R^n \) such that \( \varepsilon(0) = 0 \) and \( \varepsilon \) is continuous at \( 0 \), and
	\[
		f(a+h) = f(a) + T(h) + \norm{h} \varepsilon(h)
	\]
	Note that
	\[
		\varepsilon(h) = \begin{cases}
			0                                     & h = 0    \\
			\frac{f(a+h) - f(a) - T(h)}{\norm{h}} & h \neq 0
		\end{cases}
	\]
	So \( f \) is differentiable at \( a \) if and only if there exists \( T \in L(\mathbb R^m, \mathbb R^n) \) such that
	\[
		\frac{f(a+h) - f(a) - T(h)}{\norm{h}} \to 0
	\]
	as \( h \to 0 \).
	Such a \( T \) is unique.
	Indeed, suppose \( S, T \) satisfy the above limit.
	Then, by subtracting,
	\[
		\frac{S(h) - T(h)}{\norm{h}} \to 0
	\]
	For a fixed \( x \in \mathbb R^m \), \( x \neq 0 \), we have \( \frac{x}{k} \to 0 \) as \( k \to \infty \) so
	\[
		\frac{S\qty(\frac{x}{k}) - T\qty(\frac{x}{k})}{\norm{\frac{x}{k}}} \to 0 \implies \frac{S(x) - T(x)}{\norm{x}} = 0
	\]
	So \( Sx = Tx \).
	It follows that \( S = T \).
	We say that if a function \( f \) is differentiable at a point \( a \), \( T \) is the unique \textit{derivative} of \( f \) at \( a \).
	This is denoted \( f'(a) = Df(a) = \eval{Df}_a \).
	If \( f \colon \mathbb R^m \to \mathbb R^n \) is differentiable at \( a \in \mathbb R^m \) for every \( a \), we say that \( f \) is \textit{differentiable on} \( \mathbb R^m \).
	The function \( f' = D \colon \mathbb R^m \to L(\mathbb R^m, \mathbb R^n) \) mapping \( a \mapsto f'(a) \) is the derivative of \( f \).
\end{definition}
\begin{example}
	Constant functions are differentiable.
	Let \( f \colon \mathbb R^m \to \mathbb R^n \) such that \( f(x) = b \) for \( b \in \mathbb R^n \).
	Then for all \( a \in \mathbb R^m \), we have
	\[
		f(a+h) = f(a) + 0h + 0
	\]
	so \( f \) is differentiable at \( a \) and the derivative is zero.
\end{example}
\begin{example}
	Linear maps are differentiable.
	Let \( f \colon \mathbb R^m \to \mathbb R^n \) be defined by \( f(x) = Tx \) for a linear map \( T \in L(\mathbb R^m, \mathbb R^n) \).
	Then
	\[
		f(a+h) = f(a) + f(h) + 0
	\]
	so \( f \) is differentiable at \( a \) with derivative \( T = f \).
	So \( f' \) is a constant function.
\end{example}
\begin{example}
	Consider
	\[
		f(x) = \norm{x}^2
	\]
	For \( a \in \mathbb R^m \), we can find
	\[
		f(a+h) = \norm{a+h}^2 = \norm{a}^2 + 2\inner{a,h} + \norm{h}^2 = f(a) + 2\inner{a,h} + \norm{h} \varepsilon(h)
	\]
	Hence, \( f \) is differentiable with derivative
	\[
		f'(a)(h) = 2\inner{a,h}
	\]
	Note that \( f' \colon \mathbb R^m \to L(\mathbb R^m \to \mathbb R) \) is linear.
\end{example}

\section{Fundamental theorem of calculus}
\subsection{Breaking an Interval}
Let \(f\) be integrable on \([a, b]\).
If \(a < c < b\), then \(f\) is integrable over \([a, c]\) and \([c, b]\), with
\[
	\int_a^b f = \int_a^c f + \int_c^b f
\]
Conversely, if \(f\) is integrable on \([a, c]\) and \([c, b]\), then \(f\) is integrable over \([a, b]\) and the same equality holds for the combination of the integrals.
\begin{proof}
	We first make two observations.
	First, if \(\mathcal D_1\) is a dissection of \([a, c]\) and \(\mathcal D_2\) is a dissection of \([c, b]\), then \(\mathcal D = \mathcal D_1 \cup \mathcal D_2\) is a dissection of \([a, b]\), and
	\begin{equation}
		S(f, \mathcal D_1 \cup \mathcal D_2) = S\qty(\eval{f}_{[a, c]}, \mathcal D_1) + S\qty(\eval{f}_{[c, b]}, \mathcal D_2)
		\tag{\(\ast\)}
	\end{equation}
	Also, if \(\mathcal D\) is a dissection of \([a, b]\), then
	\begin{equation}
		S(f, \mathcal D) \geq S(f, \mathcal D \cup \{ c \}) = S\qty(\eval{f}_{[a, c]}, \mathcal D_1) + S\qty(\eval{f}_{[c, b]}, \mathcal D_2)
		\tag{\(\dagger\)}
	\end{equation}
	Now,
	\[
		(\ast) \implies I^\star(f) \leq I^\star\qty(\eval{f}_{[a, c]}) + I^\star\qty(\eval{f}_{[c, b]})
	\]
	Further,
	\[
		(\dagger) \implies I^\star(f) \geq I^\star\qty(\eval{f}_{[a, c]}) + I^\star\qty(\eval{f}_{[c, b]})
	\]
	Hence,
	\[
		I^\star(f) = I^\star\qty(\eval{f}_{[a, c]}) + I^\star\qty(\eval{f}_{[c, b]})
	\]
	This argument also applies for the lower integral, therefore
	\[
		0 \leq I^\star(f) - I_\star(f) = \underbrace{I^\star\qty(\eval{f}_{[a, c]}) - I_\star\qty(\eval{f}_{[a, c]})}_{A} + \underbrace{I^\star\qty(\eval{f}_{[c, b]}) + I_\star\qty(\eval{f}_{[c, b]})}_{B}
	\]
	Note that \(A, B \geq 0\).
	If \(f\) is integrable on \([a, c]\) and \([c, b]\), then \(A = B = 0\), hence \(I^\star(f) = I_\star(f)\) and it is integrable on \([a, b]\).
	If \(f\) is integrable on \([a, b]\), then we know \(I^\star(f) = I_\star(f)\), so \(A = B = 0\) so \(f\) is integrable on \([a, c]\) and \([c, b]\).
\end{proof}
\noindent Note that we take the following convention:
\[
	\int_a^b f = -\int_b^a f
\]
and if \(a=b\), then this value is zero.
With this convention, if \(f\) is bounded with \(\abs{f} \leq k\), then
\[
	\abs{\int_a^b f} \leq k\abs{b - a}
\]

\subsection{Fundamental Theorem of Calculus}
Suppose a function \(f \colon [a, b] \to \mathbb R\) is bounded and integrable.
Then since it is integrable on any sub-interval, we can define
\[
	F(x) = \int_a^x f(t) \dd{t}
\]
for \(x \in [a, b]\).
\begin{theorem}
	\(F\) is continuous.
\end{theorem}
\begin{proof}
	We know that
	\[
		F(x + h) - F(x) = \int_x^{x+h} f(t)\dd{t}
	\]
	We want this quantity to vanish as \(h \to 0\).
	We find, given that \(f\) is bounded by \(k\),
	\[
		\abs{F(x+h) - F(x)} = \abs{\int_x^{x+h} f(t) \dd{t}} \leq k\abs{h}
	\]
	So the result follows as \(h \to 0\).
\end{proof}
\begin{theorem}
	If in addition \(f\) is continuous at \(x\), then \(F\) is differentiable, with \(F'(x) = f(x)\).
\end{theorem}
\begin{proof}
	Consider
	\[
		\abs{\frac{F(x + h) - F(x)}{h} - f(x)}
	\]
	If this tends to zero, then the theorem holds.
	\[
		\abs{\frac{F(x + h) - F(x)}{h} - f(x)} = \frac{1}{\abs{h}} \abs{\int_x^{x+h} f(t) \dd{t} - hf(x)} = \frac{1}{\abs{h}} \abs{\int_x^{x+h} [f(t) - f(x)] \dd{t}}
	\]
	Since \(f\) is continuous at \(x\), given \(\varepsilon > 0\), \(\exists \delta > 0\) such that \(\abs{t - x} - \delta \implies \abs{f(t) - f(x)} < \varepsilon\).
	If \(\abs{h} < \delta\), then the integrand is bounded by \(\varepsilon\).
	Hence,
	\[
		\abs{\frac{F(x + h) - F(x)}{h} - f(x)} \leq \frac{1}{\abs{h}} \abs{h\varepsilon} = \varepsilon
	\]
	So we can make this value as small as we like.
	So the theorem holds.
\end{proof}
\noindent For example, consider the function
\[
	f(x) = \begin{cases}
		-1 & x \in [-1, 0] \\
		1  & x \in (0, 1]
	\end{cases}
\]
This is a bounded, integrable function, with
\[
	F(x) = -1 + \abs{x}
\]
Note that this \(F\) is not differentiable at \(x = 0\).
\begin{corollary}
	If \(f = g'\) is a continuous function on \([a, b]\), then
	\[
		\int_a^x f(t) \dd{t} = g(x) - g(a)
	\]
	is a differentiable function on \([a, b]\).
\end{corollary}
\begin{proof}
	From above, \(F - g\) has zero derivative in \([a, b]\), hence \(F - g\) is constant.
	Since \(F(a) = 0\), we get \(F(x) = g(x) - g(a)\).
\end{proof}
\noindent Note that every continuous function \(f\) has an `indefinite' integral (or `antiderivative') written \(\int f(x)\dd{x}\), which is determined uniquely up to the addition of a constant.
Note further that we have now essentially solved the differential equation
\[
	\left\{\begin{array}{l}
		y'(x) = f(x) \\
		y(a) = y_0
	\end{array}\right.
\]
and shown that there is a unique solution to this ordinary differential equation.



\section{Integration techniques and integrals in Taylor's theorem}
\subsection{Expectation and Variance}
We define the mean of a Gaussian vector \(X\) as
\[
	\mu = \expect{X} = \begin{pmatrix}
		\expect{X_1} \\ \vdots \\ \expect{X_n}
	\end{pmatrix};\quad \mu_i = \expect{X_i}
\]
We further define
\begin{align*}
	V & = \Var{X} = \expect{(X - \mu) (X - \mu)^\transpose} \\
	  & = \begin{pmatrix}
		\expect{(X_1 - \mu_1)^2}            & \expect{(X_1 - \mu_1)(X_2 - \mu_2)} & \cdots & \expect{(X_1 - \mu_1)(X_n - \mu_n)} \\
		\expect{(X_2 - \mu_2)(X_1 - \mu_1)} & \expect{(X_2 - \mu_2)^2}            & \cdots & \expect{(X_2 - \mu_2)(X_n - \mu_n)} \\
		\vdots                              & \vdots                              & \ddots & \vdots                              \\
		\expect{(X_n - \mu_n)(X_1 - \mu_1)} & \expect{(X_n - \mu_1)(X_n - \mu_2)} & \cdots & \expect{(X_n - \mu_n)^2}
	\end{pmatrix}
\end{align*}
Hence the components of \(V\) are
\[
	V_{ij} = \Cov{X_i, X_j}
\]
In particular, \(V\) is a symmetric matrix, and
\[
	\expect{u^\transpose X} = \expect{\sum_{i=1}^n u_i X_i} = \sum_{i=1}^n u_i \mu_i = u^\transpose \mu
\]
and
\[
	\Var{u^\transpose X} = \Var{\sum_{i=1}^n u_i X_i} = \sum_{i,j = 1}^n u_i \Cov{X_i, X_j} u_j = u^\transpose V u
\]
Hence \(u^\transpose X \sim \mathrm{N}(u^\transpose \mu, u^\transpose V u)\).
Further, \(V\) is a non-negative definite matrix.
Indeed, let \(u \in \mathbb R^n\).
Then \(\Var{u^\transpose X} = u^\transpose V u\).
Since \(\Var{u^\transpose X} \geq 0\), we have \(u^\transpose V u \geq 0\).

\subsection{Moment Generating Function}
We define the moment generating function of \(X\) by
\[
	m(\lambda) = \expect{e^{\lambda^\transpose X}}
\]
where \(\lambda \in \mathbb R^n\).
Then, we know that \(\lambda^\transpose X \sim \mathrm{N}(\lambda^\transpose \mu, \lambda^\transpose V \lambda)\).
Hence \(m(\lambda)\) is the moment generating function of a normal random variable with the above mean and variance, applied to the parameter \(\theta = 1\).
\[
	m(\lambda) = \exp(\lambda^\transpose \mu + \frac{\lambda^\transpose V\lambda}{2})
\]
Since the moment generating function uniquely characterises the distribution, it is clear that a Gaussian vector is uniquely characterised by its mean vector \(\mu\) and variance matrix \(V\).
In this case, we write \(X \sim \mathrm{N}(\mu, V)\).

\subsection{Constructing Gaussian Vectors}
Given a \(\mu\) and a \(V\) matrix, we might like to create a Gaussian vector that has this mean and variance.
Let \(Z_1, \dots, Z_n\) be a list of independent and identically distributed standard normal random variables.
Let \(Z = (Z_1, \dots, Z_n)^\transpose\).
Then \(Z\) is a Gaussian vector.
\begin{proof}
	For any vector \(u \in \mathbb R^n\), we have
	\[
		u^\transpose Z = \sum_{i=1}^n u_i Z_i
	\]
	Because the \(Z_i\) are independent, it is easy to take the moment generating function to get
	\[
		\expect{\exp(\lambda \sum_{i=1}^n u_i z_i)} = \expect{\prod_{i=1}^n \exp(\lambda u_i Z_i)} = \prod_{i=1}^n \expect{\exp(\lambda u_i Z_i)} = \prod_{i=1}^n \exp(\frac{(\lambda u_i)^2}{2}) = \exp(\frac{\lambda^2 \abs{u}^2}{2})
	\]
	So \(u^\transpose Z \sim \mathrm{N}(0, \abs{u}^2)\), which is normal as required.
\end{proof}
\noindent Now, \(\expect{Z} = 0\), and \(\Var{Z} = I\), the identity matrix.
We then write \(Z \sim \mathrm{N}(0, I)\).
Now, let \(\mu \in \mathbb R^n\), and \(V\) be a non-negative definite matrix.
We want to construct a Gaussian vector \(X\) such that its mean is \(\mu\) and its expectation is \(V\), by using \(Z\).
In the one-dimensional case, this is easy, since \(\mu\) is a single value, and \(V\) contains only one element, \(\sigma^2\).
In this case therefore, \(Z \sim \mathrm{N}(0, 1)\) so \(\mu + \sigma Z \sim \mathrm{N}(\mu, \sigma^2)\).
In the general case, since \(V\) is non-negative definite, we can write
\[
	V = U^\transpose DU
\]
where \(U^{-1} = U^\transpose\), and \(D\) is a diagonal matrix with diagonal entries \(\lambda_i \geq 0\).
We define the square root of the matrix \(V\) to be
\[
	\sigma = U^\transpose \sqrt{D} U
\]
where \(\sqrt{D}\) is the diagonal matrix with diagonal entries \(\sqrt{\lambda_i}\).
Then clearly,
\[
	\sigma^2 = U^\transpose \sqrt{D} U U^\transpose \sqrt{D} U = U^\transpose \sqrt{D} \sqrt{D} U = U^\transpose DU = V
\]
Now, let \(X = \mu + \sigma Z\).
We now want to show that \(X \sim \mathrm{N}(\mu, V)\).
\begin{proof}
	\(X\) is certainly Gaussian, since it is generated by a linear multiple of the Gaussian vector \(Z\), with an added constant.
	By linearity,
	\[
		\expect{X} = \mu
	\]
	and
	\begin{align*}
		\Var{X} & = \expect{(X - \mu)(X - \mu)^\transpose}           \\
		        & = \expect{(\sigma Z)(\sigma Z)^\transpose}         \\
		        & = \expect{\sigma Z Z^\transpose \sigma^\transpose} \\
		        & = \sigma \expect{Z Z^\transpose} \sigma^\transpose \\
		        & = \sigma \sigma^\transpose                         \\
		        & = \sigma \sigma                                    \\
		        & = V
	\end{align*}
\end{proof}

\subsection{Density}
We can calculate the density of such a Gaussian vector \(X \sim \mathrm{N}(\mu, V)\).
First, consider the case where \(V\) is positive definite.
Recall that in the one-dimensional case,
\[
	f_X(x) = f_Z(z) \abs{J};\quad x = \mu + \sigma z
\]
In general, since \(V\) is positive definite, \(\sigma\) is invertible.
So \(x = \mu + \sigma z\) gives \(z = \sigma^{-1}(x-\mu)\).
Hence,
\begin{align*}
	f_X(x) & = f_Z(z) \abs{J}                                                                  \\
	       & = \prod_{i=1}^n \frac{\exp(-\frac{z_i^2}{2})}{\sqrt{2\pi}} \abs{\det \sigma^{-1}} \\
	       & = \frac{1}{(2\pi)^{n/2}}\exp(-\frac{\abs{z}^2}{2}) \cdot \frac{1}{\sqrt{\det V}}  \\
	       & = \frac{1}{\sqrt{(2\pi)^n \det V}}\exp(-\frac{z^\transpose z}{2})
\end{align*}
Now,
\begin{align*}
	z^\transpose z & = (\sigma^{-1} (x-\mu))^\transpose (\sigma^{-1} (x-\mu))          \\
	               & = (x-\mu)^\transpose (\sigma^{-1})^\transpose \sigma^{-1} (x-\mu) \\
	               & = (x-\mu)^\transpose \sigma^{-2} (x-\mu)                          \\
	               & = (x-\mu)^\transpose V^{-1} (x-\mu)
\end{align*}
Hence,
\[
	f_X(x) = \frac{1}{\sqrt{(2\pi)^n \det V}}\exp(-\frac{(x-\mu)^\transpose V^{-1} (x-\mu)}{2})
\]
In the case where \(V\) is just non-negative definite (so it could have some zero eigenvalues), we can make an orthogonal change of basis, and assume that
\[
	V = \begin{pmatrix}
		U & 0 \\
		0 & 0
	\end{pmatrix};\quad \mu = \begin{pmatrix}
		\lambda \\ \nu
	\end{pmatrix}
\]
where \(U\) is an \(m \times m\) positive definite matrix, where \(m < n\), and where \(\lambda \in \mathbb R^m\), \(\nu \in \mathbb R^{n-m}\).
For \(U\), we can then apply the result above.
We can write
\[
	X = \begin{pmatrix}
		Y \\ \nu
	\end{pmatrix}
\]
where \(Y\) has density
\[
	f_Y(y) = \frac{1}{\sqrt{(2\pi)^n \det U}}\exp(-\frac{(y-\lambda)^\transpose U^{-1} (y-\lambda)}{2})
\]

\section{Improper integration}
\subsection{Introduction}
Intuitively, we might think that:
\begin{itemize}
	\item `\(A\) bijects with \(B\)' means `\(A\) has the same size as \(B\)'.
	\item `\(A\) injects into \(B\)' means `\(A\) is at most as large as \(B\)'.
	\item `\(A\) surjects onto \(B\)' means `\(A\) is at least as large as \(B\)'.
\end{itemize}
Of course, these analogies break down where \(B\) is zero, since there are no functions from \(A\) to \(B\) in this case.
For these to make sense, we require (for \(A, B\neq\varnothing\)) `\(A\) injects into \(B\)' to be true if and only if `\(B\) surjects onto \(A\)', and vice versa.
\begin{itemize}
	\item In the forward direction, we are given an injection \(f\colon A \to B\).
	      Pick some point \(a_0\) in \(A\), and define a surjective function \(g\colon B \to A\) given by
	      \[
		      b \mapsto \begin{cases}
			      a   & \text{if } \exists!\ a \in A, f(a) = b \\
			      a_0 & \text{otherwise}
		      \end{cases}
	      \]
	      Since the mapping \(f\) is injective, the first case will always provide a unique value of \(a\).
	\item Proving the converse, we are given a surjection \(g\colon B \to A\).
	      For each \(a\) in \(A\), we have some \(a' \in B\) with \(g(a') = a\) since \(g\) is a surjection.
	      Let \(f(a) = a'\) for each \(a\in A\), and \(f\) is injective.
\end{itemize}

\subsection{Schr\"oder-Bernstein Theorem}
Further, we must also have that if `\(A\) is at most as large as \(B\)' and `\(B\) is at most as large as \(A\)', then they must be the same size.
Otherwise this size intuition would not make sense.
\begin{theorem}[Schr\"oder-Bernstein Theorem]
	If \(f\colon A\to B\) and \(g\colon B\to A\) are injections, then there exists a bijection \(h\colon A\to B\).
\end{theorem}
\begin{proof}
	For \(a\in A\), we will write \(g^{-1}(a)\) to denote the unique \(b \in B\) such that \(g(b) = a\), if such a \(b\) exists (and similarly for \(f^{-1}(b)\)).
	The `ancestor sequence' of \(a \in A\) is \(g^{-1}(a), f^{-1}g^{-1}(a), g^{-1}f^{-1}g^{-1}(a), \dots\) which may terminate.
	So for any ancestor, after undergoing the relevant function \(f\) or \(g\) repeatedly, we will end up at \(a\).
	There are three possible behaviours:
	\begin{itemize}
		\item Let \(A_0\) be the subset of \(A\) such that the ancestor sequence stops at even time, i.e.\ the last ancestor is in \(A\);
		\item Let \(A_1\) be the subset of \(A\) such that the ancestor sequence stops at odd time, i.e.\ the last ancestor is in \(B\); and
		\item Let \(A_\infty\) be the subset of \(A\) such that the ancestor sequence does not terminate.
	\end{itemize}
	We specify 0 to be even, i.e.\ if \(a\in A\) has no ancestor \(g^{-1}(a)\), then \(a \in A_0\).
	We define similar subsets of \(B\): \(B_0\), \(B_1\), \(B_\infty\).
	Now:
	\begin{itemize}
		\item \(f\colon A \to B\) is a bijection between \(A_0\) and \(B_1\).
		      Clearly if some element \(a\) has an even number of ancestors, the ancestors of \(f(a)\) are exactly \(a\) and all of its ancestors, i.e.\ an odd number.
		      It is surjective because every element in \(B_1\) has an inverse \(f^{-1}(b) \in A_0\) by construction.
		\item \(g\colon B \to A\) is a bijection between \(B_0\) and \(A_1\) due to the same argument.
		\item \(f\) (or \(g\), both functions work for this proof) bijects \(A_\infty\) and \(B_\infty\).
		      It is surjective because for every element \(b \in B\), it has some ancestor \(f^{-1}(b) \in A_\infty\).
	\end{itemize}
	So the function \(h\colon A \to B\) is given by
	\[
		h(a) = \begin{cases}
			f(a)      & \text{if } a \in A_0      \\
			g^{-1}(a) & \text{if } a \in A_1      \\
			f(a)      & \text{if } a \in A_\infty
		\end{cases}
	\]
	is a bijection.
\end{proof}
Let us consider an example of this theorem in action.
Do \([0, 1]\) and \([0,1]\cup[2,3]\) biject?
All we need is to find an injection both ways.
\begin{itemize}
	\item Let \(f\colon [0,1] \to [0,1] \cup [2,3]\) be the identity map \(f(x) = x\).
	\item Let \(g\colon [0,1] \cup [2,3] \to [0,1]\) be given by \(g(x) = x/3\).
\end{itemize}

It would also be nice to have that, for any sets \(A\) and \(B\), either \(A\) injects into \(B\) or \(B\) injects into \(A\).
Then we can create a total ordering, rather than a partial ordering; we can compare any two sets.
This is proven to be true in the Part II course Logic and Set Theory.

\subsection{Injections into Power Sets}
We have the sets
\[
	\mathbb N, \mathcal P(\mathbb N), \mathcal P(\mathcal P(\mathbb N)), \dots, \mathcal P^k(\mathbb N), \dots
\]
Does every set \(X\) inject into one of those?
It seems like this might be true, but the set
\[
	X = \mathbb N \cup \mathcal P(\mathbb N) \cup \mathcal P(\mathcal P(\mathbb N)) \cup \dots
\]
is a counterexample.
Let us continue further with this approach.
\[
	X' = X \cup \mathcal P(X) \cup \mathcal P(\mathcal P(X)) \cup \dots
\]
\[
	X'' = X' \cup \mathcal P(X') \cup \mathcal P(\mathcal P(X')) \cup \dots
\]
and so on.
Now, does every set inject into one of these sets?
No, consider
\[
	Y = X \cup X' \cup X'' \cup X''' \cup \dots
\]
We can keep going forever.
So we can't construct a set that all sets inject into.

\subsection{What Happens Next?}
This is the end of the Numbers and Sets course.
Here are a few of the courses that feed from this course.
\begin{itemize}
	\item Factorisation is taken further in the IB Groups, Rings and Modules course.
	\item Fermat's Little Theorem, squares modulo \(p\) etc.
	      are taken further in the II Number Theory.
	\item The analysis chapter is extended by IA Analysis.
	\item Countability and sizes of sets are taken further in the II Logic and Set Theory course.
\end{itemize}

\section{Generalising Riemann integrability}
\subsection{Second derivatives and partial derivatives}
Let \( U \) be open in \( \mathbb R^n \), let \( f \colon U \to \mathbb R^n \), and let \( a \in U \).
Let \( f \) be twice differentiable at \( a \), so \( f \) is differentiable on some open neighbourhood \( V \) of \( a \) contained within \( U \), and \( f' \colon V \to L(\mathbb R^m, \mathbb R^n) \) is differentiable at \( a \).
Recall that
\[
	f'(a+h) = f'(a) + f''(a)(h) + o\qty(\norm{h})
\]
Evaluating at a fixed \( k \),
\[
	f'(a+h)(k) = f'(a)(k) + f''(a)(h,k) + o\qty(\norm{h})
\]
Let \( u, v \in \mathbb R^m \setminus \qty{0} \) be directions.
Let \( k = v \).
Then,
\[
	f'(a+h)(v) = D_v f(a+h) = D_v f(a) + f''(a)(h,v) + o\qty(\norm{h})
\]
Hence, the map \( D_v f \colon V \to \mathbb R^n \) maps \( x \mapsto D_v f(x) = f'(x)(v) \).
Then this map is differentiable at \( a \) and
\[
	(D_v f)'(a)(h) = f''(a)(h,v)
\]
Hence there exist directional derivatives.
\[
	D_u D_v f(a) \overset{\mathrm{def}}{=} D_u (D_v f)(a) = (D_v f)'(a)(u) = f''(a)(u,v)
\]
In particular, we have
\[
	D_i D_j f(a) = f''(a)(e_i, e_j)
\]
for \( 1 \leq i, j \leq m \).

\subsection{Symmetry of mixed directional derivatives}
\begin{theorem}
	Let \( U \) be open in \( \mathbb R^n \), let \( f \colon U \to \mathbb R^n \), and let \( a \in U \).
	Let \( f \) be twice differentiable on an open set \( V \) with \( a \in V \subset U \).
	Let \( f'' \colon V \to \mathrm{Bil}(\mathbb R^m \times \mathbb R^m, \mathbb R^n) \) be continuous at \( a \).
	Then, for all directions \( u,v \in \mathbb R^m \setminus \qty{0} \), we have
	\[
		D_u D_v f(a) = D_v D_u f(a)
	\]
	Equivalently,
	\[
		f''(a)(u,v) = f''(a)(v,u)
	\]
	In other words, \( f'' \) is a symmetric bilinear map.
\end{theorem}
\begin{proof}
	Without loss of generality we can let \( n = 1 \).
	Indeed, we have
	\[
		(D_u f)_j(x) = [D_u f(x)]_j = [f'(x)(u)]_j = f_j'(x)(u) = D_u f_j(x)
	\]
	Hence, \( (D_u f)_j = D_u f_j \).
	For \( v \):
	\[
		(D_v D_u f)_j = D_v (D_u f)_j = D_v D_u f_j
	\]
	So it is sufficient to show that \( D_v D_u f_j(a) = D_u D_v f_j(a) \).
	Now, consider
	\[
		\phi(s,t) = f(a+su+tv) - f(a+tv) - f(a+su) + f(a)
	\]
	for \( s, t \in \mathbb R \).
	Let \( s, t \) be fixed, and consider
	\[
		\psi(y) = f(a+yu+tv) - f(a+yu)
	\]
	Note that \( \phi(s,t) \) can be written as
	\[
		\phi(s,t) = \psi(s) - \psi(0)
	\]
	The term \( \psi(s) - \psi(0) \) can be interpreted as \( (f(a+su+tv) - f(a+tv)) - (f(a+su) - f(a)) \), which is the second difference given by the function when traversing the parallelogram with sides \( su, tv \).
	By the mean value theorem, there exists \( \alpha(s,t) \in (0,1) \) such that
	\[
		\phi(s,t) = \psi(s) - \psi(0) = s \psi'(\alpha s) = s \qty[ D_u f(a+\alpha s u+ tv) - D_u f(a + \alpha s u) ]
	\]
	Now, applying the mean value theorem to the function \( y \mapsto D_u f(a+\alpha s u + y v) \), we have
	\[
		\phi(s,t) = s t D_v D_u f (a+\alpha s u + \beta t v)
	\]
	for \( \beta(s,t) \in (0,1) \).
	Now,
	\[
		\frac{\phi(s,t)}{st} = D_v D_u f(a+\alpha su + \beta tv) = f''(a+\alpha su + \beta tv)(u,v)
	\]
	Since \( f'' \) is continuous at \( a \), we can let \( s, t \to 0 \) and find
	\[
		\frac{\phi(s,t)}{st} \to f''(a)(u,v)
	\]
	Now, we can repeat the above using
	\[
		\psi(y) = f(a+su) + yv) - f(a+yv)
	\]
	This calculates the second difference from above, but using the other path.
	We can find
	\[
		\frac{\phi(s,t)}{st} \to f''(a)(v,u)
	\]
	as required.
\end{proof}


\end{document}
