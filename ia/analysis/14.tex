\subsection{Definition}
A power series is a series of the form
\[
	\sum_{n=0}^\infty a_n z^n
\]
where \(z \in \mathbb C\), and the \(a_n\) is a given sequence of complex numbers.
We can also take a power series of the form
\[
	\sum_{n=0}^\infty a_n (z-z_0)^n
\]
but for simplicity we will take \(z_0 = 0\) in all of the analysis we will conduct on power series.

\subsection{Radius of Convergence}
\begin{lemma}
	If the series
	\[
		\sum_{n=0}^\infty a_n z_1^n
	\]
	converges for some point \(z_1\), and \(\abs{z} < \abs{z_1}\), then the series
	\[
		\sum_{n=0}^\infty a_n z^n
	\]
	also converges absolutely.
\end{lemma}
\begin{proof}
	Since \(\sum_{n=0}^\infty a_n z_1^n\) converges, \(a_n z_1^n \to 0\).
	Thus the sequence \(a_n z_1^n\) is bounded by some \(k > 0\), i.e.\ for all \(n\), \(\abs{a_n z_1^n}<k\).
	Then
	\[
		\abs{a_n z^n} \leq k\abs{\frac{z}{z_1}}^n
	\]
	Since the geometric series \(\sum_0^\infty \abs{\frac{z}{z_1}}^n\) converges, the lemma follows by comparison.
\end{proof}
\noindent Using this lemma, we can find that there exists a radius inside which any given power series converges absolutely.
This radius might be zero, and it might be infinite.
\begin{theorem}
	Any power series either
	\begin{enumerate}[(i)]
		\item converges absolutely for all \(z\), or
		\item converges absolutely for all \(z\) where \(\abs{z} < R\) and diverges for all \(z\) where \(\abs{z} > R\), or
		\item converges for \(z = 0\) only.
	\end{enumerate}
\end{theorem}
\noindent The circle \(\abs{z} = R\) is called the circle of convergence, and \(R\) is called the radius of convergence.
Note that this theorem does not make any claim about the behaviour \textit{on} the circle of convergence, just the behaviour inside it.
\begin{proof}
	Let
	\[
		S = \left\{ x \in \mathbb R \colon x \geq 0, \sum_0^\infty a_n x^n \text{ converges} \right\}
	\]
	Clearly, \(0 \in S\).
	By the above lemma, if \(x_1 \in S\), then \([0, x_1] \subseteq S\).
	If \(S = [0, \infty)\), then we have case (i) above due to the lemma. % chktex 9
	If \(S \neq [0, \infty)\), there exists a supremum \(0 \leq R = \sup S < \infty\). % chktex 9
	We must now just deal with case (ii), which is \(R > 0\).
	For all \(z_1\) with \(\abs{z_1} < R\) there exists \(R_0\) such that \(\abs{z_1} < R_0 < R\), and absolute convergence follows using the lemma.
	If \(\abs{z_1} > R\), there exists \(R_0\) such that \(\abs{z_1} > R_0 > R\).
	If the series with \(z_1\) converges, then by the lemma the same would be true for \(R_0\).
	But \(R_0\) does not converge, so this is a contradiction.
\end{proof}

\begin{lemma}
	If
	\[
		\abs{\frac{a_{n+1}}{a_n}} \to \ell
	\]
	as \(n \to \infty\), then \(R = \frac{1}{\ell}\).
\end{lemma}
\begin{proof}
	By the ratio test, we have absolute convergence if
	\[
		\abs{\frac{a_{n+1}}{a_n} \frac{z^{n+1}}{z^n}} < 1
	\]
	So we have absolute convergence if \(\abs{z} < \frac{1}{\ell}\) and divergence if \(\abs{z} > \frac{1}{\ell}\) as required.
\end{proof}
\begin{lemma}
	If
	\[
		\abs{a_n^{1/n}} \to \ell
	\]
	as \(n \to \infty\), then \(R = \frac{1}{\ell}\).
\end{lemma}
\noindent This can be shown similarly using the root test.

\subsection{Examples of Radii of Convergence}
\begin{enumerate}
	\item Consider the series \(\sum_0^\infty \frac{z^n}{n!}\).
	      Using the ratio test, the series converges absolutely everywhere.
	\item The geometric series \(\sum_0^\infty z^n\) gives \(R=1\) by the ratio test.
	      In this case, \(\abs{z} = 1\) gives divergence.
	\item The series \(\sum_0^\infty n!z^n\) has \(R=0\), which again can be seen using the ratio test.
	\item Consider \(\sum_1^\infty \frac{z^n}{n}\).
	      This also has \(R = 1\) by the ratio test.
	      Note that the series diverges for \(z=1\) since we get the harmonic series.
	      However, it converges when \(z = -1\) by the alternating series test.
	      To work out the behaviour at other points on the circle of convergence, we could consider the series \(\sum_1^\infty \frac{z^n}{n}(1-z)\), which converges exactly when the original series does.
	      The partial sums are
	      \begin{align*}
		      S_N & = \sum_1^N \left[ \frac{z_n - z^{n+1}}{n} \right]            \\
		          & = \sum_1^N \frac{z^n}{n} - \sum_1^N \frac{z^{n+1}}{n}        \\
		          & = \sum_1^N \frac{z^n}{n} - \sum_2^{N+1} \frac{z^n}{n-1}      \\
		          & = z - \frac{z^{N+1}}{N+1} + \sum_2^{N+1} \frac{-z^n}{n(n-1)} \\
	      \end{align*}
	      If \(\abs{z} = 1\), then the term \(\frac{z^{N+1}}{N+1}\) will vanish as \(N \to \infty\).
	      If \(z \neq 1\), the term \(\sum_2^{N+1} \frac{-z^n}{n(n-1)}\) converges as \(N \to \infty\).
	      So \(S_N\) does indeed converge for \(\abs{z} = 1\), \(z \neq 1\).
	\item Now, consider \(\sum_1^\infty \frac{z^n}{n^2}\).
	      This has \(R=1\) by the ratio test, but it converges for all \(z\) with \(\abs{z} = 1\).
	\item If we have \(\sum_0^\infty nz^n\), we have \(R=1\), but diverges for all \(\abs{z} = 1\).
\end{enumerate}
\noindent In conclusion, we cannot determine the behaviour at the boundary in the general case.
Inside the radius of convergence, power series will behave as if they were simply polynomials.

% chktex 17
