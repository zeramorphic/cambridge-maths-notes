\chapter[Analysis I \\ \textnormal{\emph{Lectured in Lent \oldstylenums{2021} by \textsc{Prof.\ G.\ Paternain}}}]{Analysis I}
\emph{\Large Lectured in Lent \oldstylenums{2021} by \textsc{Prof.\ G.\ Paternain}}

In this course, we rigorously define what it means for a sequence to approach a particular value; this is called a limit.
Limits can be used to define things like derivatives and integrals, without appealing to concepts such as infinitesimals.

We begin by using limits to make sense of infinite summations, also called series.
In general, a series may not have a sum (take \( 1 + 1 + \cdots \), for example), but many series do approach a value as we keep adding more terms (such as \( 1 + \frac{1}{2} + \frac{1}{4} + \frac{1}{8} + \cdots \to 2 \)).
We prove various facts about when series converge to a value, and when they do not.

Then, we define what it means for a function to approach a value as we get closer to a particular input.
We can use this to define the derivative of a function.
As with series, derivatives may not exist (for example, \( \abs{x} \) at \( x = 0 \)), so we need to be careful to restrict our analysis to differentiable functions.
We can similarly make a rigorous definition of the integral, and show the fundamental theorem of calculus: under suitable assumptions, the derivative of the integral of a function is the original function.

\subfile{../../ia/analysis/main.tex}
