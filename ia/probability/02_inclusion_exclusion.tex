\subsection{Inclusion-exclusion formula}
Suppose that \(A, B \in \mathcal F\).
Then \(\prob{A \cup B} = \prob{A} + \prob{B} - \prob{A \cap B}\).
Now let also \(C \in \mathcal F\).
Then
\begin{align*}
	\prob{A \cup B \cup C} & = \prob{A \cup B} + \prob{C} - \prob{(A \cup B) \cap C} \\
	                       & = \prob{A} + \prob{B} + \prob{C}                        \\
	                       & - \prob{A \cap B} - \prob{A \cap C} - \prob{B \cap C}   \\
	                       & + \prob{A \cap B \cap C}
\end{align*}
Let \(A_1, \dots, A_n\) be events in \(\mathcal F\).
Then
\begin{align*}
	\prob{\bigcup_{i=1}^n A_i} & = \sum_{i=1}^n \prob{A_i}                                                      \\
	                           & - \sum_{1 \leq i_1 < i_2 \leq n}\prob{A_{i_1} \cap A_{i_2}}                    \\
	                           & + \sum_{1 \leq i_1 < i_2 < i_3 \leq n}\prob{A_{i_1} \cap A_{i_2} \cap A_{i_3}} \\
	                           & - \cdots                                                                       \\
	                           & + (-1)^{n+1}\prob{A_1 \cap \dots \cap A_n}
\end{align*}
Or more concisely,
\[
	\prob{\bigcup_{i=1}^n A_i} = \sum_{k=1}^n (-1)^{k+1} \sum_{1 \leq i_1 < i_2 < \dots < i_k \leq n} \prob{A_{i_1} \cap \dots \cap A_{i_k}}
\]

\begin{proof}
	The case for \(n=2\) has been verified, so we can use induction on \(n\).
	Now, let us assume this holds for \(n-1\) events.
	\begin{align*}
		\prob{\bigcup_{i=1}^n A_i} & = \prob{\left( \bigcup_{i=1}^{n-1} A_i  \right) \cup A_n}                                                     \\
		                           & = \prob{\bigcup_{i=1}^{n-1} A_i} + \prob{A_n} - \prob{\left( \bigcup_{i=1}^{n-1} A_i  \right) \cap A_n}       \\
		                           & = \prob{\bigcup_{i=1}^{n-1} A_i} + \prob{A_n} - \prob{\bigcup_{i=1}^{n-1} (A_i \cap A_n)}                     \\
		\intertext{Let \(B_i = A_i \cap A_n\) for all \(i\).
			By the inductive hypothesis, we have}
		\prob{\bigcup_{i=1}^n A_i} & = \prob{\bigcup_{i=1}^{n-1} A_i} + \prob{A_n} - \prob{\bigcup_{i=1}^{n-1} B_n}                                \\
		                           & = \sum_{k=1}^{n-1} (-1)^{k+1} \sum_{1 \leq i_1 < \dots < i_k \leq n-1} \prob{A_{i_1} \cap \dots \cap A_{i_k}} \\
		                           & - \sum_{k=1}^{n-1} (-1)^{k+1} \sum_{1 \leq i_1 < \dots < i_k \leq n-1} \prob{B_{i_1} \cap \dots \cap B_{i_k}} \\
		                           & + \prob{A_n}
	\end{align*}
	which gives the claim as required.
\end{proof}

Let \((\Omega, \mathcal F, \mathbb P)\) be a probability space with \(\abs{\Omega} < \infty\) and \(\prob{A} = \frac{\abs{A}}{\abs{\Omega}}\).
Let \(A_1, \dots, A_n \in \mathcal F\).
Then
\[
	\abs{A_1 \cup \dots \cup A_n} = \sum_{k=1}^n (-1)^{k+1} \sum_{1 \leq i_1 < \dots < i_k \leq n} \abs{A_{i_1} \cap \dots \cap A_{i_k}}
\]

\subsection{Bonferroni inequalities}
Truncating the sum in the inclusion-exclusion formula at the \(r\)th term yields an estimate for the probability.
The Bonferroni Inequalities state that if \(r\) is odd, it is an overestimate, and if \(r\) is even, it is an underestimate.
\begin{align*}
	r \text{ odd}  & \implies \prob{\bigcup_{i=1}^n A_i} \leq \sum_{k=1}^r (-1)^{k+1} \sum_{1 \leq i_1 < \dots < i_k \leq n} \prob{A_{i_1} \cap \dots \cap A_{i_k}} \\
	r \text{ even} & \implies \prob{\bigcup_{i=1}^n A_i} \geq \sum_{k=1}^r (-1)^{k+1} \sum_{1 \leq i_1 < \dots < i_k \leq n} \prob{A_{i_1} \cap \dots \cap A_{i_k}}
\end{align*}
\begin{proof}
	Again, we will use induction.
	The \(n=2\) case is trivial.
	Suppose that \(r\) is odd.
	Then
	\begin{equation}
		\prob{\bigcup_{i=1}^n A_i} = \prob{\bigcup_{i=1}^{n-1} A_i} + \prob{A_n} - \prob{\bigcup_{i=1}^n B_i} \tag{\(\ast\)}
	\end{equation}
	where \(B_i = A_i \cap A_n\).
	Since \(r\) is odd,
	\[
		\prob{\bigcup_{i=1}^{n-1} A_i} \leq \sum_{k=1}^r (-1)^{k+1} \sum_{1 \leq i_1 < \dots < i_k \leq n-1} \prob{A_{i_1} \cap \dots \cap A_{i_k}}
	\]
	Since \(r-1\) is even, we can apply the inductive hypothesis to \(\prob{\bigcup_{i=1}^{n-1} B_i}\).
	\[
		\prob{\bigcup_{i=1}^{n-1} B_i} \geq \sum_{k=1}^{r-1} (-1)^{k+1} \sum_{1 \leq i_1 < \dots < i_k \leq n-1} \prob{B_{i_1} \cap \dots \cap B_{i_k}}
	\]
	We can substitute both bounds into (\(\ast\)) to get an overestimate.
\end{proof}

\subsection{Counting using inclusion-exclusion}
We can apply the Inclusion-Exclusion formula to count various things.
How many functions \(f \colon \{ 1, \dots, n \} \to \{ 1, \dots, m \}\) are surjective?
Let \(\Omega\) be the set of such functions, and \(A = \{ f \in \Omega : f \text{ is a surjection} \}\).
For all \(i \in \{ 1, \dots, m \}\), we define \(A_i = \{ f \in \Omega : i \notin \{ f(1), f(2), \dots, f(n) \} \}\).
Then \(A = \stcomp{A_1} \cap \stcomp{A_2} \cap \dots \cap \stcomp{A_m} = \stcomp{(A_1 \cup A_2 \cup \dots \cup A_m)}\).
Then
\[
	\abs{A} = \abs{\Omega} - \abs{A_1 \cup \dots \cup A_m} = m^n - \abs{A_1 \cup \dots \cup A_m}
\]
Now, let us use the Inclusion-Exclusion formula.
\begin{align*}
	\abs{A_1 \cup \dots \cup A_m} & = \sum_{k=1}^n (-1)^{k+1} \sum_{1 \leq i_1 < \dots < i_k \leq n} \abs{A_{i_1} \cap \dots \cap A_{i_k}} \\
	\intertext{Note that \(A_{i_1} \cap \dots \cap A_{i_k}\) is the set of functions where \(k\) distinct numbers are not included in the function's range.
		There are \((m-k)^n\) such functions.}
	\abs{A_1 \cup \dots \cup A_m} & = \sum_{k=1}^n (-1)^{k+1} \sum_{1 \leq i_1 < \dots < i_k \leq n} (m-k)^n                               \\
	                              & = \sum_{k=1}^n (-1)^{k+1} \binom{m}{k} (m-k)^n                                                         \\
	\abs{A}                       & = m^n -  \sum_{k=1}^n (-1)^{k+1} \binom{m}{k} (m-k)^n                                                  \\
	\abs{A}                       & = \sum_{k=0}^n (-1)^k \binom{m}{k} (m-k)^n
\end{align*}

\subsection{Counting derangements}
A derangement is a permutation which has no fixed point, i.e.\ \(\forall i, \sigma(i) \neq i\).
We will let \(\Omega\) be the set of permutations of \(\{1, \dots, n\}\), i.e.\ \(S_n\).
Let \(A\) be the set of derangements in \(\Omega\).
Let us pick a permutation \(\sigma\) at random from \(\Omega\).
What is the probability that it is a derangement?
We define \(A_i = \{ f \in \Omega \colon f(i) = i \}\), then \(A = \stcomp{A_1} \cap \dots \stcomp{A_n} = \stcomp{\left( \bigcup_{i=1}^n A_i  \right)}\), so \(\prob{A} = 1 - \prob{\bigcup_{i=1}^n A_i}\).
By the inclusion-exclusion formula,
\begin{align*}
	\prob{\bigcup_{i=1}^n A_i} & = \sum_{k=1}^n (-1)^{k+1} \sum_{1 \leq i_1 < i_2 < \dots < i_k \leq n} \prob{A_{i_1} \cap \dots \cap A_{i_k}} \\
	                           & = \sum_{k=1}^n (-1)^{k+1} \sum_{1 \leq i_1 < i_2 < \dots < i_k \leq n} \frac{(n-k)!}{\abs{\Omega}}            \\
	                           & = \sum_{k=1}^n (-1)^{k+1} \sum_{1 \leq i_1 < i_2 < \dots < i_k \leq n} \frac{(n-k)!}{n!}                      \\
	                           & = \sum_{k=1}^n (-1)^{k+1} \binom{n}{k} \frac{(n-k)!}{n!}                                                      \\
	                           & = \sum_{k=1}^n (-1)^{k+1} \frac{n!}{k!(n-k)!} \cdot \frac{(n-k)!}{n!}                                         \\
	                           & = \sum_{k=1}^n (-1)^{k+1} \frac{1}{k!}                                                                        \\
\end{align*}
So
\[
	\prob{A} = 1 - \prob{\bigcup_{i=1}^n A_i} = 1 - \sum_{k=1}^n \frac{(-1)^{k+1}}{k!} = \sum_{k=0}^n \frac{(-1)^k}{k!}
\]
As \(n \to \infty\), this value tends to \(e^{-1} \approx 0.3678\).
