\subsection{Strong Induction}
The induction axiom states that if we know
\begin{itemize}
	\item $p(1)$ is true, and
	\item $p(n) \implies p(n+1)$ for any $n \in \mathbb N$
\end{itemize}
then we can conclude that $p(n)$ is true for all $n \in \mathbb N$. We can in fact prove a stronger statement using this axiom, known as `strong induction'.
\begin{claim}
	If we know that
	\begin{itemize}
		\item $p(1)$ is true, and
		\item the fact that $p(k)$ is true for all $k\leq n$ implies that $p(n)$ is true
	\end{itemize}
	then $p(n)$ is true for all $n \in \mathbb N$.
\end{claim}
\begin{proof}
	Consider the predicate $q(n)$ defined as: `$p(k)$ is true for all $k \leq n$'. Given that $p(1)$ is true, $q(1)$ is trivially true since there are no $k$ below 1. Since $q(n) \implies q(n+1)$, we can use the induction axiom, showing that $q(n)$ is true for all $n$, so $p(n)$ is true for all $n$.
\end{proof}
This provides a very useful alternative way of looking at induction. Instead of just considering a process from $n$ to $n+1$, we can inject an inductive viewpoint into any proof. When proving something on the natural numbers, we can always assume that the hypothesis is true for smaller $n$ than what we are currently using. This allows us to write very powerful proofs because in the general case we are allowed to refer back to other smaller cases --- but not just $n-1$, any $k$ less than $n$.

We may rewrite the principle of strong induction in the following ways:
\begin{enumerate}
	\item If $p(n)$ is false for some $n$, there must be some $m$ where $p(m)$ is false and $p(k)$ is true for all $k<m$. In other words, if a counterexample exists, there must exist a minimal counterexample.
	\item If $p(n)$ is true for some $n$, then there is a smallest $n$ where $p(n)$. In other words, if an example exists, there must exist a minimal example. This is known as the `well-ordering principle'.
\end{enumerate}

\subsection{The Integers}
The integers $\mathbb Z$ consist of the set of natural numbers $\mathbb N$, their additive inverses, and an identity element denoted 0. In other words, $(\mathbb Z, +)$ is the group generated by $\mathbb N$ and the addition operator: $\mathbb Z = \genset{\mathbb N}$.

We define operations in a familiar way, for example $a < b \iff \exists c \in \mathbb N \st a+c = b$.

\subsection{The Rationals}
The rational numbers $\mathbb Q$ consist of all expressions denoted $\frac{a}{b}$ where $a, b \in \mathbb Z$ with $b \neq 0$; with $\frac{a}{b}$ regarded as the same as $\frac{c}{d}$ if and only if $ad=bc$.

We define, for example,
\[ \frac{a}{b} + \frac{c}{d} = \frac{ad + bc}{bd} \]
Note that is important to verify with each operation that it does not matter how you write a given rational number. For example, $\frac{1}{2} + \frac{1}{2} = \frac{2}{4} + \frac{3}{6}$. This means that operations such as $\frac{a}{b} \to \frac{a^3}{b^2}$ cannot exist because then it would depend on how you write the rational number.

\subsection{Prime Numbers}
\begin{proposition}
	Every $n \geq 2$ is expressible as a product of primes.
\end{proposition}
\begin{proof}
	We use induction on an integer $n$, starting at 2, a trivial case. Given $n > 2$, we have two cases:
	\begin{itemize}
		\item $n$ is prime. Therefore, $n$ is a product of primes as required.
		\item $n$ is composite. We know that $n$ can be split into two factors, denoted here as $a$, $b$. Using (strong) induction, we know that because both $a$ and $b$ are smaller than $n$, they are expressible as a product of primes. We simply multiply these products together to express $n$ as a product of primes.
	\end{itemize}
\end{proof}
