\subsection{Groups of Small Order}
We can completely classify groups of small order; we already know enough to classify all groups up to order 5 using Lagrange's Theorem.
\begin{proposition}
	If \(\abs{G} = 4\), then \(G \cong C_4\) or \(G \cong C_2 \times C_2\).
\end{proposition}
\begin{proof}
	By Lagrange's Theorem, the possible orders of elements of \(G\) with \(\abs{G} = 4\) are 1 (only the identity), 2 and 4.
	\begin{itemize}
		\item If there is an element \(g\) of order 4, then \(G = \genset{G}\) because \(e \neq g \neq g^2 \neq g^3\), so it is cyclic of order 4.
		\item If there is no such element, then all non-identity elements must have order 2.
		      \(G\) is abelian (by question 7 on example sheet 1).
		      Take two distinct elements \(b, c\) of order 2.
		      Then:
		      \begin{itemize}
			      \item \(\genset b \cap \genset c = \{ e, b \} \cap \{ e, c \} = \{ e \}\)
			      \item \(bc = cb\) as the group is abelian.
			      \item The element \(bc\) is not equal to \(b\) or \(c\) (\(bc = b \implies c = e\) which is an element of order 1).
			            It is also not equal to \(e\) because then \(b = c^{-1}\) which implies \(b = c\).
			            So the remaining element of \(G\) is simply \(bc\).
			            So any element in \(G\) may be written as the product of an element in \(\genset b\) multiplied by an element in \(\genset c\).
		      \end{itemize}
		      These are the three conditions of the direct product theorem, so \(G = \genset b \times \genset c \cong C_2 \times C_2\).
	\end{itemize}
\end{proof}
Now here is a list the first five smallest groups (we need more tools in order to classify larger groups):
\begin{enumerate}
	\item \(G = \{ e \}\)
	\item \(G \cong C_2\) because a group of prime order is cyclic.
	\item \(G \cong C_3\) for the same reason.
	\item \(G \cong C_4\) or \(G \cong C_2 \times C_2\) by the proof above.
	\item \(G \cong C_5\) because 5 is prime.
\end{enumerate}

\subsection{Normal Subgroups}
How and when does it make sense to divide one group by another?
\begin{definition}
	An subgroup \(N\) of a group \(G\) is \textit{normal} if \(\forall g \in G, gN = Ng\).
	We write \(N \trianglelefteq G\).
\end{definition}
\noindent The following equivalent definitions hold:
\begin{itemize}
	\item \(\forall g \in G, gN = Ng\)
	\item \(\forall g \in G, \forall n \in N, g^{-1}ng \in N\)
	\item \(\forall g \in G, g^{-1}Ng = N\)
\end{itemize}
\begin{proof}
	The first case is the definition.
	For the second case, clearly (from the first definition) \(ng \in gN\).
	So multiplying on the left by \(g^{-1}\), we have \(g^{-1}ng \in N\) as required.
	For the third case, we can simply multiply the first definition on the left by \(g^{-1}\).
	Note that these multiplications are distributed over each element in the coset: \(a(bC) = \{ abc : c \in C \}\).
\end{proof}

\begin{enumerate}
	\setcounter{enumi}{-1}
	\item \(\{ e \}\) and \(G\) are normal subgroups of \(G\).
	\item \(n\mathbb Z \trianglelefteq \mathbb Z\).
	      \(\forall a \in \mathbb Z\), we have \(a + n\mathbb Z = \{ a + nk : k \in \mathbb Z \} = \{ nk + a : k \in \mathbb Z \} = n\mathbb Z + a\).
	\item \(A_3 \trianglelefteq S_3\).
	      \begin{itemize}
		      \item \(eA_3 = A_3 = A_3 e\)
		      \item \((1\ 2\ 3)A_3 = A_3 = A_3(1\ 2\ 3)\)
		      \item \((1\ 3\ 2)A_3 = A_3 = A_3(1\ 3\ 2)\)
		      \item \((1\ 2)A_3 = \{ (1\ 2), (2\ 3), (1\ 3) \} = A_3(1\ 2)\)
	      \end{itemize}
	      and so on.
\end{enumerate}

\begin{proposition}
	\begin{enumerate}[(i)]
		\item Any subgroup of an abelian group is normal.
		\item Any subgroup of index 2 is normal.
	\end{enumerate}
\end{proposition}
\begin{proof}
	\begin{enumerate}[(i)]
		\item If \(G\) is abelian, then \(\forall g \in G, \forall n \in N, g^{-1}ng = n \in N\) which is stronger than required.
		\item If \(H \leq G\) with \(\abs{G : H} = 2\), then there are only 2 cosets.
		      \(H = eH = He\) is one of the two cosets.
		      Since cosets are disjoint, the other coset must be \(G \setminus H\).
		      This is true for both left and right cosets.
		      So the other left and right cosets must be equal, so \(H\) is normal.
	\end{enumerate}
\end{proof}

\begin{proposition}
	If \(\varphi: G \to H\) is a homomorphism, then \(\ker \varphi \trianglelefteq G\).
\end{proposition}
\begin{proof}
	We already know \(\ker \varphi\) is a subgroup of \(G\).
	Now we must check it is normal.
	Given some \(k \in \ker \varphi, g \in G\), we want to show that \(g^{-1} k g \in \ker \varphi\).
	We have \(\varphi(g^{-1} k g) = \varphi(g^{-1}) \varphi(k) \varphi(g) = \varphi(g^{-1}) e \varphi(g) = \varphi(g^{-1}g) = \varphi(e) = e\) so \(g^{-1} k g \in \ker \varphi\) as required.
\end{proof}
In fact, we will show later that normal subgroups are exactly kernels of homomorphisms and nothing else.

Here is now a less formal explanation of this theorem and its consequences.
Consider some subgroup \(K \leq G\).
There may be some property \(P\) that is true for every element of \(K\) and false for every other element of \(G\).
Then certainly, for example, given \(k_1, k_2 \in K\), we know that \(k_1k_2\) has the same property as it is within \(K\).
As another example, let \(k \in K\) and let \(g \in G \setminus K\).
Then \(kg\) does not have this property, as \(kg \notin K\).

We can encapsulate this behaviour by making a homomorphism from the whole group \(G\) to some other group --- it \textit{doesn't matter where we end up}, just as long as anything with this particular property maps to the new group's identity element.
Let \(\varphi: G \to H\), where \(H\) is some group that we don't really care about (apart from the identity).
This means that any element of \(K\), i.e.\ any element with property \(P\), is mapped to \(e_H\).
By the laws of homomorphisms, any product of \(k \in K\) with \(g \in G \setminus K\) does not give the identity element, so it does not have this property! This is exactly the behaviour we wanted.

If we can find such a homomorphism, then \(K\) is the kernel of this homomorphism.
Again, the image of this homomorphism is essentially irrelevant; all we care about is which elements map to the identity.
Now, note that by the laws of homomorphisms, given some element \(g \in G\) and \(k \in K\), \(\varphi(g^{-1}kg) = \varphi(g^{-1})\varphi(k)\varphi(g)\).
But since \(k\) has this desired property, the \(\varphi(k)\) term vanishes.
So we're left with the identity element.
This gives us the result that \(g^{-1}kg\) must be an element of \(K\), so it must have property \(P\).
This is a definition for a normal subgroup, so \(K\) must be normal in order for us to be able to find such a homomorphism \(\varphi\).

As another small aside, a normal subgroup in this context essentially means this: given some element \(k\) with property \(P\), the property is preserved when surrounding \(k\) with inverses.
This is just a `translation' of a definition of a normal subgroup: \(g^{-1}kg \in K\).

\begin{enumerate}[(i)]
	\item \(SL_n(\mathbb R) \trianglelefteq GL_n(\mathbb R)\), where \(GL_n(\mathbb R)\) is the group of invertible matrices of dimension \(n\), and where \(SL_n(\mathbb R)\) is the group of matrices of determinant 1.
	      This is because \(\det: GL_n(\mathbb R) \to \mathbb R^*\), and \(SL_n(\mathbb R) = \ker (\det)\).
	\item \(A_n \trianglelefteq S_n\) as \(A_n\) is the kernel of the sign homomorphism.
	      Alternatively, it is an index 2 subgroup so it must be normal.
	\item \(n\mathbb Z \trianglelefteq \mathbb Z\) as the kernel of \(\varphi: \mathbb Z \to \mathbb Z_n\), where \(\varphi(k) = k \mod n\), or since \(\mathbb Z\) is abelian.
\end{enumerate}
With this notion of normal subgroups, we can make some progress into categorising small groups.
\begin{proposition}
	If \(\abs{G} = 6\), then \(G \cong C_6\) or \(G \cong D_6\).
\end{proposition}
\begin{proof}
	By Lagrange's Theorem, the possible element orders are 1 (only the identity), 2, 3, 6.
	\begin{itemize}
		\item If there is an element \(g\) of order 6, then \(G = \genset g \cong C_6\).
		\item Otherwise, (again by question 7 on example sheet 1) there must be an element of the group not of order 2, because if we just had elements of order 2 then \(\abs{G}\)
		      would have to be a power of 2.
		      So there is an element \(r\) of order 3, so \(\abs{\genset{r}} = 3\), and by Lagrange's Theorem \(\abs{G} = 6 = \abs{G:\genset{r}} \cdot \abs{\genset{r}}\), so \(\abs{G:\genset{r}} = 2\).
		      So \(\genset{r} \trianglelefteq G\).
		      There must also be an element \(s\) of order 2, since \(\abs{G}\) is even (by question 8 from example sheet 1).
		      
		      So, what can \(s^{-1} r s\) be? Because \(\genset{r}\) is normal, then \(s^{-1} r s \in \genset{r}\).
		      So it is either \(e\), \(r\) or \(r^2\).
		      \begin{itemize}
			      \item If \(s^{-1}rs = e\) then \(r = e\) \contradiction{}
			      \item If \(s^{-1}rs = r\) then \(sr=rs\), and so \(sr\) has order \(\LCM(\ord s, \ord r) = \LCM(2, 3) = 6\) \contradiction{}
			      \item So \(s^{-1}rs = r^2\), then \(G = \genset{r, s}\) with \(r^3 = s^2 = e\) and \(sr = r^2 s = r^{-1}s\), which are the defining features of \(D_6\).
		      \end{itemize}
	\end{itemize}
\end{proof}
