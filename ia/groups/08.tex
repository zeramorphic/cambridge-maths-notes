\subsection{The M\"obius Group}
M\"obius groups are an analogous concept to permutation groups, but on the infinite set of the complex numbers.
A M\"obius transformation \(f\) is defined as follows:
\[
	f: \hat{\mathbb C} \to \hat{\mathbb C};\quad f(z) = \frac{az + b}{cz + d};\quad a, b, c, d \in \mathbb C;\quad ad-bc \neq 0
\]
The reason for the restriction that \(ad-bc\neq 0\) is that \(ad=bc\) implies that \(f\) is a constant value for all \(z\).
Note that \(\hat{\mathbb C}\) is known as the `extended complex plane', defined as the complex plane together with a point at infinity, denoted \(\infty\).
There are some special points related to M\"obius transformations:
\begin{itemize}
	\item \(f(\frac{-d}{c})\) is defined to be \(\infty\).
	      This is because the denominator of the fraction would be zero.
	\item \(f(\infty)\) is defined to be \(\frac{a}{c}\) if \(c \neq 0\).
	      This is because as the length of \(z\) increases to infinity, the constant terms \(b\) and \(d\) vanish.
	      However, if \(c = 0\), then the numerator explodes to infinity as the denominator remains constant, so \(f(\infty) = \infty\) in this case.
\end{itemize}

\begin{lemma}
	M\"obius transformations are bijections from \(\hat{\mathbb C} \to \hat{\mathbb C}\).
\end{lemma}
\begin{proof}
	We can prove this by evaluating \(f(f^{-1}(z))\) and \(f^{-1}(f(z))\) at all \(z\), taking into account all the special points.
	The entire proof is not written here, but it suffices to substitute every special point and a generic \(z\) into both of these expressions, and show that they equal \(z\) in all cases.
\end{proof}

\begin{theorem}
	The set \(\mathcal M\) of M\"obius maps forms a group under composition of functions.
\end{theorem}
\begin{proof}
	We must check each of the group axioms, and we begin with closure.
	Let \(f_1(z) = \frac{a_1 z + b_1}{c_1 z + d_1}; f_2(z) = \frac{a_2 z + b_2}{c_2 z + d_2}\).
	To compose these functions, we first ignore the special points and then check them individually later.
	\begin{align*}
		(f_2 \circ f_1)(z) & = f_2(f_1(z))                                                                                                               \\
		                   & = \frac{a_2 \left( \frac{a_1 z + b_1}{c_1 z + d_1} \right) + b_2}{c_2 \left( \frac{a_1 z + b_1}{c_1 z + d_1} \right) + d_2} \\
		                   & = \frac{(a_1 a_2 + b_2 c_1)z + (a_2 b_1 + b_2 d_1)}{(c_2 a_1 + d_2 c_1)z + (c_2 b_1 + d_1 d_2)}                             \\
		                   & =: \frac{az + b}{cz + d}
	\end{align*}
	Note that \(ad-bc = (a_1 a_2 + b_2 c_1)(c_2 b_1 + d_1 d_2) - (a_2 b_1 + b_2 d_1)(c_2 a_1 + d_2 c_1) = (a_1 d_1 - b_1 c_1)(a_2 d_2 - b_2 c_2)\) which is the product of two nonzero numbers, which is therefore nonzero.
	Now we will check all the special points.
	\begin{align*}
		(f_2 \circ f_1)(\infty)                                         & = f_2\left(\frac{a_1}{c_1}\right)                                                           \\
		                                                                & = \frac{a_2 \left( \frac{a_1}{c_1} \right) + b_2}{c_2 \left( \frac{a_1}{c_1} \right) + d_2} \\
		                                                                & = \frac{a_1 a_2 + b_2 c_1}{c_2 a_1 + d_2 c_1}                                               \\
		                                                                & = \frac{a}{c}                                                                               \\
		(f_2 \circ f_1)(\infty)                                         & = f_2(\infty) = \frac{a_2}{c_2}                                                             \\
		(f_2 \circ f_1)\left(f^{-1}\left(\frac{-d_2}{c_2}\right)\right) & = f_2\left( \frac{-d_2}{c_2} \right)                                                        \\
		                                                                & = \infty
	\end{align*}
	Note that each of these results matches up with our intuitive understanding of infinity in the limit, for instance \((f_2 \circ f_1)(\infty) = \frac{a}{c}\), where na\"\i{}vely we might assume \((f_2 \circ f_1)(\infty) = \frac{a \cdot \infty + b}{c \cdot \infty + d} = \frac{a}{c}\).

	Now we may prove the other group axioms hold for \(\mathcal M\).
	Clearly there is an identity element \(f(z) = \frac{1z + 0}{0z + 1}\).
	We know that there are always inverses because \(f\) is a bijection.
	Finally, we know that all M\"obius maps obey the associative law because function composition is always associative.
	So \(\mathcal M\) is a group.
\end{proof}

\subsection{Properties of the M\"obius Group}
When we are working with M\"obius groups, we use the following conventions:
\[
	\frac{1}{\infty} = 0;\quad \frac{1}{0} = \infty;\quad \frac{a\cdot\infty}{c\cdot\infty} = \frac{a}{c}
\]

Firstly, \(\mathcal M\) is not abelian.
For example, let \(f_1(z) = z + 1; f_2(z) = 2z\).
Then \((f_2 \circ f_1)(z) = 2z + 2\) and \((f_1 \circ f_2)(z) = 2z + 1\).

\begin{proposition}
	Every M\"obius transformation can be written as a composition of maps of the following forms:
	\begin{enumerate}[(i)]
		\item \(f(z) = az\) where \(a\neq 0\).
		      This is a dilation or rotation.
		\item \(f(z) = z + b\).
		      This is a translation by \(b\).
		\item \(f(z) = \frac{1}{z}\).
		      This is an inversion.
	\end{enumerate}
\end{proposition}
\begin{proof}
	Let \(f(z) = \frac{az + b}{cz + d}\).
	Then if \(c \neq 0\), \(f(z)\) is given by
	\[
		z \xrightarrow{\text{(ii)}} z + \frac{d}{c} \xrightarrow{\text{(iii)}} \frac{1}{z + \frac{d}{c}} \xrightarrow{\text{(i)}} \frac{(-ad+bc)c^{-2}}{z + \frac{d}{c}} \xrightarrow{\text{(ii)}} \frac{a}{c} + \frac{(-ad+bc)c^{-2}}{z + \frac{d}{c}} = \frac{az + b}{cz + d}
	\]
	If \(c = 0\), \(f(z)\) is given by
	\[
		z \xrightarrow{\text{(i)}} \frac{a}{d}z \xrightarrow{\text{(ii)}} \frac{a}{d}z + \frac{b}{d} = \frac{az + b}{d}
	\]
\end{proof}
Note therefore that the set \(\mathcal S\) of all dilations, rotations, translations and inversions generates \(\mathcal M\), or in symbolic form, \(\genset {\mathcal S} = \mathcal M\).

\subsection{Cosets}
Let \(H\) be a subgroup of some group \(G\), and let \(g \in G\).
Then a set of the form \(gH := \{ gh : h \in H \}\) is called a left coset of \(H\) in \(G\).
Also, a set of the form \(Hg := \{ hg : h \in H \}\) is called a right coset of \(H\) in \(G\).
Mostly we use left cosets, but right cosets can be seen in more specific scenarios.
Note that the order of group \(H\) is the same as the order of the cosets \(gH\) and \(Hg\); we can think of \(gH\) and \(Hg\) as translated copies of \(H\).
Note further that \(gH\) and \(Hg\) are not necessarily groups; in fact in general they are not groups.
We now consider some example cosets.
\begin{enumerate}
	\setcounter{enumi}{-1}
	\item Let \(H = \{ e \} \leq G\).
	      Then \(gH = \{ g \}\).
	\item Let \(H = 2\mathbb Z\) and let \(G = \mathbb Z\).
	      Then (where the cosets are written additively):
	      \begin{itemize}
		      \item \(0 + 2\mathbb Z = 2\mathbb Z\) which is the set of even integers.
		      \item \(1 + 2\mathbb Z = \{ 1 + k: k \in 2\mathbb Z \}\) which is the set of odd integers.
		      \item \(2 + 2\mathbb Z = 2\mathbb Z\).
		            There are only two distinct cosets of \(H\) in \(G\) here; every odd integer will create the set of odd integers, and every even integer will create the set of even integers.
	      \end{itemize}
	\item Let \(H = \{ e, (1\ 2) \}\), and let \(G = S_3\).
	      Then, each (left) coset of \(H\) in \(G\) is given by
	      \begin{itemize}
		      \item \(eH = \{ e, (1\ 2) \} = H\)
		      \item \((1\ 2)H = \{ (1\ 2), e \} = H\)
		      \item \((1\ 3)H = \{ (1\ 3), (1\ 2\ 3) \}\)
		      \item \((1\ 2\ 3)H = \{ (1\ 2\ 3), (1\ 3) \}\)
		      \item \((2\ 3)H = \{ (2\ 3), (1\ 3\ 2) \}\)
		      \item \((1\ 3\ 2)H = \{ (1\ 3\ 2), (2\ 3) \}\)
	      \end{itemize}
	      Note that:
	      \begin{itemize}
		      \item \(eH = H\)
		      \item \(\forall h \in H, hH = H\) as \(H\) is a group and therefore closed under multiplication with \(h\)
		      \item \(\abs{gH} = \abs{H}\)
		      \item \(\bigcup_{g\in G} gH = G\), and in this example in particular, each pair of cosets is equal and disjoint to any other pair
	      \end{itemize}
\end{enumerate}
