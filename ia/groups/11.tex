\subsection{Motivation for Quotients}
Let us consider \(n\mathbb Z \trianglelefteq \mathbb Z\). The cosets are \(0+n\mathbb Z\), \(1 + n\mathbb Z, \cdots, (n-1) + n\mathbb Z\). These cosets, although they are subsets of \(\mathbb Z\), behave a lot like the elements of the group \(\mathbb Z_n\). For example, if we try to define addition between the cosets:
\[ (k + n \mathbb Z) + (m + n \mathbb Z) := (k+m) + n \mathbb Z \]
which acts like addition modulo \(n\mathbb Z\). For a general subgroup \(H \leq G\), we could try to do the same.
\[ g_1 H \cdot g_2 H := g_1 g_2 H \]
But we can write the cosets on the left hand side in many ways, as the representation is dependent on the choice of representative for each coset, so this multiplication may not be well defined. We can guarantee that it is well defined (so that we can turn the set of cosets into a group) by ensuring that
\[ g_1' H = g_1 H;\; g_2' H = g_2 H \implies g_1'g_2'H = g_1g_2H \]
If \(g_1' H = g_1 H;\; g_2' H\), then \(g_1' = g_1h_1\) and \(g_2' = g_2h_2\) for some \(h_1 h_2 \in H\). So
\[ g_1'g_2'H = g_1h_1g_2\underbrace{h_2H}_{\mathclap{h_2H = \{ h_2h : h \in H \} = H}} \]
So in order to get \(g_1' g_2' H = g_1 g_2 H\), we need \(g_1 h_1 g_2 H = g_1 g_2 H\) for any elements \(g_1, g_2, h_1\) that we choose. Therefore:
\begin{align*}
	g_1h_1g_2H         & = g_1g_2 H \\
	g_2^{-1} h_1 h_2 H & = H        \\
	\text{or } g_2^{-1} h_1 g_2 \in H\;(\forall g_2 \in G, h_1 \in H)
\end{align*}
This is an equivalent condition for the subgroup to be normal.

\subsection{Definitions}
\begin{proposition}
	Let \(N \trianglelefteq G\). The set of (left) cosets of \(N\) in \(G\) forms a group under the operation \(g_1N\cdot g_2N = g_1g_2N\).
\end{proposition}
\begin{proof}
	The group operation is well defined as shown above. We now show the group axioms hold.
	\begin{itemize}
		\item (closure) If \(g_1N, g_2N\) are cosets, then \(g_1g_2N\) is also a coset.
		\item (identity) \(eN = N\)
		\item (inverses) \((gN)^{-1} = g^{-1}N\)
		\item (associativity) Follows from the associativity of \(G\): \((g_1 N \cdot g_2 N)\cdot g_3 N = g_1g_2 N \cdot g_3 N = g_1g_2g_3 N = g_1 N \cdot g_2g_3 N = g_1 N \cdot (g_2 N \cdot g_3 N)\)
	\end{itemize}
\end{proof}

\begin{definition}
	If \(N \trianglelefteq G\), the group of (left) cosets of \(N\) in \(G\) is called the quotient group of \(G\) by \(N\), written \(\frac{G}{N}\).
\end{definition}

This is a nice way of thinking about quotient groups. Imagine you have a group \(N\) of some distinct objects \(n_1, n_2, n_3\) and so on. Imagine lining them all up in a row of length \(\abs{N}\). Then the cosets of \(N\) in \(G\) can be thought of as `translated copies' of \(N\). For example, let the cosets of \(N\) in \(G\) be \(N, g_1N, g_2N\) and so forth. Now, picture these cosets as copies of \(N\), translated downwards on the page, so that they are like multiple rows, and that therefore there we have a grid containing all elements of \(G\). Now, we have formed a rectangle of area \(\abs{G}\) out of \(\abs{N}\) columns and \(c\) rows, where \(c\) is the amount of `copies' of \(N\). Therefore, \(c = \frac{\abs{G}}{\abs{N}}\), as the area of a rectangle is width multiplied by height.

Now, given some element in one of the cosets (i.e. in \(G\)) we can do some transformation \(g\) to take us to another element. But because we made cosets out of a normal subgroup, multiplying by \(g\) is the same as swapping some of the rows, then maybe moving around the order of the elements in each row. It keeps the identity of each row consistent --- all elements in a given row are transformed to the same output row. Remember that the word `row' basically means `coset'.

This means that we can basically forget about the individual elements in these cosets, all that we really care about is how the rows are swapped with each other under a given transformation. Note, the quotient of 5 in 100 is 20, because there are 20 copies of 5 in 100. So the quotient group of \(N\) in \(G\) is just all the copies of \(N\) in \(G\). The group operation is simply the transformation of rows. If we're talking about \(\frac{G}{N}\), ask the question: `how do the copies of \(N\) in \(G\) behave?'

\subsection{Examples and Properties}
\begin{enumerate}
	\item The cosets of \(n\mathbb Z\) in \(\mathbb Z\) give a group that behaves exactly like \(\mathbb Z_n\). We write \(\frac{\mathbb Z}{n\mathbb Z} \cong \mathbb Z_n\). In fact, these are the only quotients of \(\mathbb Z\), as these are the only subgroups of \(\mathbb Z\).
	\item \(A_3 \trianglelefteq S_3\) gives \(\frac{S_3}{A_3}\) which has only two elements since \(\abs{S_3 : A_3} = 2\), so it is isomorphic to \(C_2\). Note that in general, \(\abs{G:N} = \abs{\frac{G}{N}}\).
	\item If \(G = H \times K\), then both \(H\) and \(K\) are normal subgroups of \(G\). We have \(\frac{G}{H} \cong K\) and \(\frac{G}{K} \cong H\) (TODO: proof as exercise).
	\item Consider \(N := \genset{r^2} \trianglelefteq D_8\). We can check that it is normal by trying \(r^{-1}r^2r^{-1} \in N\), and also \(s^{-1}r^2s = r^{-2} = r^2 \in N\). Since \(\genset{r, s} = D_8\), and the generators obey this normal subgroup relation, it follows that \(g^{-1}ng\) for all \(g \in D_8\) (TODO proof as exercise). We know \(\abs{N} = 2\), so \(\abs{\frac{D_8}{N}} = \abs{D_8 : N} = \frac{\abs{D_8}}{\abs{N}}\) by Lagrange's Theorem. So \(\abs{\frac{D_8}{N}} = 4\). We know that any group of order 4 is isomorphic either to \(C_4\) or \(C_2 \times C_2\). We can check that the cosets are \(\frac{D_8}{N} = \{ N, sN, rN, srN \}\) which does not contain an element of order 4, so it is isomorphic to \(C_2 \times C_2\).
\end{enumerate}
We now show a non-example using the subgroup \(H := \genset{(1\ 2)} \leq S_3\) which is not normal, e.g. \((1\ 2\ 3) H \neq H (1\ 2\ 3)\). The cosets are
\[ H;\quad (1\ 2\ 3)H = \{(1\ 2\ 3), (1\ 3)\};\quad (1\ 3\ 2)H = \{(1\ 3\ 2), (2\ 3)\} \]
Attempting a multiplication gives
\[ (1\ 2\ 3)H \cdot (1\ 3\ 2) H = (1\ 2\ 3) (1\ 3\ 2) H = H \]
but using a different coset representative,
\[ (1\ 3)H \cdot (1\ 3\ 2) H = (1\ 3)(1\ 3\ 2) H = (2\ 3)H \neq H \]
so the multiplication is not well defined so we cannot form the quotient.

\begin{itemize}
	\item We can check that certain properties are inherited into quotient groups from the original group, such as being abelian and being finite.
	\item Quotients are not subgroups of the original group. They are associated with tha original group in a very different way to subgroups --- in general, a coset may not even be isomorphic to a subgroup in the group. The example with direct products above was an example that is not true in general.
	\item With normality, we need to specify in which group the subgroup is normal. For example, if \(K \leq N \leq G\), with \(K \trianglelefteq N\). This does not imply that \(K \trianglelefteq G\), this would require that \(g^{-1}Kg = K\) for all elements \(g\) in \(G\), but we only have that \(n^{-1}Kn = K\) for all elements \(n\) in \(N\), which is a weaker condition. Normality is not transitive --- for example, \(K \trianglelefteq N \trianglelefteq G \centernot\implies K \trianglelefteq G\).
	\item However, if \(N \leq H \leq G\) and \(N \trianglelefteq G\), then the weaker condition \(N \trianglelefteq H\) is true.
\end{itemize}

\begin{theorem}
	Given \(N \trianglelefteq G\), the function \(\pi: G \to \frac{G}{N}\), \(\pi(g) = gN\) is a surjective homomorphism called the quotient map. We have \(\ker \pi = N\).
\end{theorem}
\begin{proof}
	We prove that \(\pi\) is a homomorphism. \(\pi(g)\pi(h) = gN \cdot hN = (gh)N = \pi(gh)\) as required. Clearly it is surjective we we can create all possible cosets by applying the \(\pi\) function to a coset representative. Also, \(\pi(g) = gN = N\) if and only if \(g \in N\), so \(\ker \pi = N\).
\end{proof}
Therefore, normal subgroups are exactly kernels of homomorphisms. Using the idea of `properties' for normal subgroups above, the property in question here is `belonging to \(N\)'. Any element of \(N\) is in the coset \(N\), which is the identity coset of \(\frac{G}{N}\). Essentially, the first row of this quotient `grid' (as described above) is \(N\), which acts as the identity element in the \(\frac{G}{N}\) quotient group.
