\subsection{Definition and elementary properties}
\begin{definition}
	Let \((G, \ast_G)\), \((H, \ast_H)\) be groups.
	A function \(\varphi: H \to G\) is a homomorphism if
	\[
		\forall a, b \in H,\quad \varphi(a \ast_H b) = \varphi(a) \ast_G \varphi(b)
	\]
\end{definition}
A homomorphism \(\varphi: H \to G\) may have the following descriptions:
\begin{itemize}
	\item injective, if \(\varphi(a) = \varphi(b) \implies a = b\);
	\item surjective, if \(\forall g \in G, \exists h \in H \st \varphi(h) = g\); and
	\item bijective, if it is both injective and surjective.
\end{itemize}
A more intuitive interpretation of the descriptions is:
\begin{itemize}
	\item A function is injective if the outputs are unique;
	\item A function is surjective if all outputs are used;
	\item A function is bijective if there is a one-to-one relation between every element in the input and output sets.
\end{itemize}
Here are some examples, without proofs.
\begin{enumerate}
	\item Given any two groups \(G\) and \(H\), \(\varphi: H \to G\) defined by \(\varphi(h) = e_G\) is a homomorphism.
	\item The inclusion function \(\iota: H \to G\) where \(H \leq G\) is an injective homomorphism.
	      The inclusion function is defined as the identity function, simply transferring elements from a subgroup into the supergroup.
	\item \(\varphi: \mathbb Z \to \mathbb Z_n\) given that \(\varphi(k) = k \mod n\) is a surjective homomorphism.
	\item \(\varphi: (\mathbb R, +) \to (\mathbb R_{>0}, \cdot)\) where \(R_{>0} = \{r \in \mathbb R : r > 0\}\) and \(\varphi(x) = e^x\) is a bijective homomorphism, otherwise known as an isomorphism.
	\item \(\det : GL_2(\mathbb R) \to (\mathbb R^*, \cdot)\) is a surjective homomorphism.
\end{enumerate}

\begin{proposition}
	Let \(\varphi: H \to G\) be a homomorphism.
	Then, for all \(h \in H\):
	\begin{enumerate}
		\item \(\varphi(e_H) = e_G\)
		\item \(\varphi(h^{-1}) = \varphi(h)^{-1}\)
		\item Given another homomorphism \(\psi: G \to K\), \(\psi \circ \varphi: H \to K\) is a homomorphism.
	\end{enumerate}
\end{proposition}
\begin{proof}
	We prove each result in order.
	\begin{enumerate}
		\item Given the identity element of \(H\) is \(e_H\) and similarly for \(G\),
		      \begin{align*}
			      \varphi(e_H \ast e_H) & = \varphi(e_H) \ast \varphi(e_H) \\
			      \implies \varphi(e_H) & = \varphi(e_H) \ast \varphi(e_H) \\
			      e_G                   & = \varphi(e_H)
		      \end{align*}
		\item Consider \(\varphi(h) \ast \varphi(h^{-1}) = \varphi(h \ast h^{-1}) = \varphi(e_H) = e_G\) which is the defining property of the inverse.
		\item For all \(a, b \in H\):
		      \begin{align*}
			      (\psi \circ \varphi) (a \ast b) & = \psi(\varphi(a \ast b))                           \\
			                                      & = \psi(\varphi(a) + \varphi(b))                     \\
			                                      & = \psi(\varphi(a)) + \psi(\varphi(b))               \\
			                                      & = (\psi \circ \varphi)(a) + (\psi \circ \varphi)(b)
		      \end{align*}
	\end{enumerate}
\end{proof}

\subsection{Isomorphisms}
A bijective homomorphism is called an isomorphism.
If there exists an isomorphism \(\varphi: H \to G\), we say that \(H\) is isomorphic to \(G\), or \(H \cong G\).
\begin{enumerate}
	\item Consider a group \(G\) defined as \(\{ e^{\frac{2 \pi i k}{n}} : k \in \mathbb Z_n \}\) under multiplication.
	      Then, \((G, \cdot) \cong (\mathbb Z_n, +)\) where \(\varphi: \mathbb Z_n \to G\) is defined as \(\varphi(k) = e^{\frac{2\pi i k}{n}}\).
	\item \(\varphi: \mathbb Z \to n\mathbb Z\) for \(n \in \mathbb N\) given by \(\varphi(k) = nk\).
	      Note that all non-trivial subgroups of \(\mathbb Z\) are isomorphic to \(\mathbb Z\).
\end{enumerate}

\begin{proposition}
	Let \(\varphi: H \to G\) be an isomorphism.
	Then \(\varphi^{-1}: G \to H\) is an isomorphism.
\end{proposition}
\begin{proof}
	For all \(a, b \in G\),
	\begin{align*}
		\varphi^{-1}(a \ast b) & = \varphi^{-1}\left[ \varphi(\varphi^{-1}(a)) \ast \varphi(\varphi^{-1}(b)) \right] \\
		                       & = \varphi^{-1}\left[ \varphi(\varphi^{-1}(a) \ast \varphi^{-1}(b)) \right]          \\
		                       & = \varphi^{-1}(a) \ast \varphi^{-1}(b)
	\end{align*}
	So \(\varphi^{-1}\) is a homomorphism.
	But since \(\varphi\) is bijective, so is \(\varphi^{-1}\).
	So \(\varphi^{-1}\) is an isomorphism.
\end{proof}

\subsection{Images and kernels}
\begin{definition}
	Let \(\varphi: H \to G\) be a homomorphism.
	Then the image of \(\varphi\), denoted \(\Im \varphi\), is defined as \(\{ g \in G : g = \varphi(h) \text{ for some } h \in H \}\).
	The kernel of \(\varphi\), denoted \(\ker \varphi\), is defined as \(\{ h \in H : \varphi(h) = e_G \}\).
\end{definition}
Informally, we can say:
\begin{itemize}
	\item The image of \(\varphi\) is the set of outputs of \(\varphi\).
	\item The kernel of \(\varphi\) is the set of inputs that map to the identity element.
\end{itemize}
\begin{proposition}
	\(\Im \varphi \leq G\) and \(\ker \varphi \leq H\).
\end{proposition}
\begin{proof}
	To prove that \(\Im \varphi \leq G\), we check the group axioms (apart from associativity, since this is implicit).
	\begin{itemize}
		\item (closure) If \(a, b \in \Im \varphi\) then there exist some \(x, y \in H\) such that \(\varphi(x) = a\) and \(\varphi(y) = b\).
		      Therefore, \(\varphi(x)\varphi(y)=\varphi(xy)\) which is in the image by definition.
		\item (identity) \(\varphi(e_H) = e_G\)
		\item (inverses) Let \(x \in H\) such that \(\varphi(x) = a\).
		      Then, because \(x^{-1}\in H\), we know that \(\varphi(x^{-1}) = \varphi(x)^{-1} \in \Im H\) as required.
	\end{itemize}
	Now we prove a similar result for the kernel.
	\begin{itemize}
		\item (closure) If \(x, y \in \ker H\) then \(\varphi(xy) = \varphi(x) \varphi(y) = e_G e_G = e_G\), which is the requirement for being in the kernel, so \(xy \in \ker \varphi\).
		\item (identity) \(\varphi(e_H) = e_G\) so the identity element \(e_H\) is in the kernel.
		\item (inverses) \(\varphi(x^{-1}) = \varphi(x)^{-1}\).
		      So if \(x \in \ker \varphi\) then \(\varphi(x^{-1}) = e_G^{-1} = e_G\) so \(\varphi^{-1}\) is also in the kernel.
	\end{itemize}
\end{proof}
Here are a few examples of kernels and images of homomorphisms.
\begin{enumerate}
	\item If \(\varphi: H \to G\) is the trivial homomorphism (mapping every element to the identity) then:
	      \[
		      \Im \varphi = \{ e_G \};\quad \ker \varphi = H
	      \]
	\item If \(H \leq G\) then the inclusion homomorphism \(\iota: H \to G\) has
	      \[
		      \Im \iota = H;\quad \ker \iota = e_H
	      \]
	\item \(\varphi: \mathbb Z \to \mathbb Z_n\) where operations are performed modulo \(n\) has
	      \[
		      \Im \varphi = \mathbb Z_n;\quad \ker \varphi = n\mathbb Z
	      \]
\end{enumerate}

\begin{proposition}
	Let \(\varphi: H \to G\) be a homomorphism.
	Then
	\begin{itemize}
		\item \(\varphi\) is surjective if and only if \(\Im \varphi = G\); and
		\item \(\varphi\) is injective if and only if \(\ker \varphi = \{ e_H \}\).
	\end{itemize}
\end{proposition}
\begin{proof}
	The first case is trivial.
	After all, the definition of surjectivity is that all outputs are mapped onto by something, which means that image is equal to this output set.
	Now, let us prove the injectivity part.
	We start in the forward direction, then we prove the converse.

	Suppose that \(\varphi\) is injective.
	Then \(\varphi(a) = \varphi(b) \implies a = b\).
	We have that \(\varphi(e_H) = e_G\), so \(e_H\) must be the only element sent to \(e_G\).
	Therefore the kernel is simply \(\{ e_H \}\).

	Conversely, suppose that the kernel of \(\varphi\) is simply the identity element.
	Then, let us suppose there are two elements \(a, b\) in \(H\) such that \(\varphi(a) = \varphi(b)\).
	Then, \(\varphi(a b^{-1})=\varphi(a)\varphi(b)^{-1} = \varphi(b)\varphi(b)^{-1} = e_G\).
	Therefore, \(ab^{-1} = e_H\), so \(a = b\).
	So \(\varphi\) is injective.
\end{proof}
