\subsection{Cross-Ratios}
Recall that given distinct points \(z_1, z_2, z_3 \in \hat{\mathbb C}\), we have a unique M\"obius map \(f\) such that \(f(z_1) = 0\), \(f(z_2) = 1\), \(f(z_3) = \infty\).
\begin{definition}
	If \(z_1, z_2, z_3, z_4 \in \hat{\mathbb C}\) are distinct, then their cross-ratio \([z_1, z_2, z_3, z_4]\) is defined to be \(f(z_4)\) where \(f \in \mathcal M\) is the unique M\"obius map \(f\) such that \(f(z_1) = 0\), \(f(z_2) = 1\), \(f(z_3) = \infty\).
\end{definition}
In particular, \([0, 1, \infty, w] = w\).
We have the following formula for computing the cross-ratio.
\[
	[z_1, z_2, z_3, z_4] = \frac{(z_4 - z_1)(z_2 - z_3)}{(z_2 - z_1)(z_4 - z_3)}
\]
with special cases interpreted accordingly where \(z_i = \infty\).
This result follows from the proof that we can construct a map to send any three distinct points to \(0, 1, \infty\).
There are in fact \(4!
\) different conventions for the cross-ratio, depending on the order of \(0, 1, \infty\), so ensure that the correct convention is being used if referring to sources.
However, this potential ambiguity is mitigated by the following fact.
\begin{proposition}
	Double transpositions of the \(z_i\) fix the cross-ratio.
\end{proposition}
\begin{proof}
	By inspection of the formula, it it clear that this is true.
\end{proof}
\begin{theorem}
	M\"obius maps preserve the cross-ratio.
	\(\forall g \in \mathcal M\), \(\forall z_1, z_2, z_3, z_4 \in \hat{\mathbb C}\),
	\[
		[g(z_1), g(z_2), g(z_3), g(z_4)] = [z_1, z_2, z_3, z_4]
	\]
\end{theorem}
\begin{proof}
	Let \(f \in \mathcal M\) be the unique M\"obius map such that
	\[
		f(z_1) = 0;\quad f(z_2) = 1;\quad f(z_3) = \infty
	\]
	so therefore \(f(z_4) = [z_1, z_2, z_3, z_4]\).
	Now, consider \(f \circ g^{-1}\):
	\[
		(f \circ g^{-1})g(z_1) = 0;\quad (f \circ g^{-1})g(z_2) = 1;\quad (f \circ g^{-1})g(z_3) = \infty
	\]
	and \(f \circ g^{-1}\) is the unique map with this property.
	So the cross-ratio here is
	\[
		[g(z_1), g(z_2), g(z_3), g(z_4)] = (f \circ g^{-1})g(z_4) = f(z_4)
	\]
	as required.
\end{proof}
\begin{corollary}
	Four distinct points \(z_1, z_2, z_3, z_4 \in \hat{\mathbb C}\) lie on a circle if and only if their cross-ratio is real.
\end{corollary}
\begin{proof}
	Let \(f\) be the unique M\"obius map sending \((z_1, z_2, z_3) \mapsto (0, 1, \infty)\), so that \(f(z_4)\) is the required cross-ratio.
	The circle \(C\) passing through \(z_1, z_2, z_3\) is sent by \(f\) to the unique circle passing through \(0, 1, \infty\), i.e.\ the real line together with the point at infinity.
	So \(z_4\) lies on \(C\) if and only if \(f(z_4)\) lies on \(\mathbb R \cup \{ \infty \}\).
	But since \(f(z_3) = \infty\), \(f(z_4) \neq \infty\), so this condition is restricted only to \(\mathbb R\), excluding a point at infinity.
\end{proof}

\subsection{Matrix Groups}
We will look at various groups of matrices, their related actions, and study distance-preserving maps on \(\mathbb R^2\) and \(\mathbb R^3\).
Here are some examples of matrix groups.
\begin{itemize}
	\item \(M_{n \times n}(\mathbb F)\) is the set of \(n \times n\) matrices over the field \(\mathbb F\).
	\item \(GL_n(\mathbb F)\) is the set of \(n \times n\) matrices over \(\mathbb F\) which are invertible.
	      This is known as the general linear group over \(\mathbb F\).
	      \begin{itemize}
		      \item \(GL_n(\mathbb F)\) is a group under multiplication.
		      \item \(\det\colon GL_n(\mathbb F) \to \mathbb F^* := \mathbb F \setminus \{ 0 \}\) is a surjective homomorphism.
		      \item Given \(A \in GL_n(\mathbb R)\), \(A^\transpose\) is the matrix with entries \((A^\transpose)_{ij} = A_{ji}\).
		            It satisfies
		            \begin{itemize}
			            \item \((AB)^\transpose = B^\transpose A^\transpose\)
			            \item \((A^{-1})^\transpose = (A^\transpose)^{-1}\)
			            \item \(AA^\transpose = I \iff A^\transpose A = I \iff A^\transpose = A^{-1}\)
			            \item \(\det A^\transpose = \det A\)
		            \end{itemize}
	      \end{itemize}
	\item \(SL_n(\mathbb F) \leq GL_n(\mathbb F)\) is the kernel of the \(\det\) homomorphism.
	      This is the special linear group.
	\item \(O_n = O_n(\mathbb R) := \{ A \in GL_n(\mathbb R): A^\transpose A = I \}\) is the orthogonal group.
	      We can check the group axioms to verify it is a subgroup of \(GL_n(\mathbb R)\).
	\item \(SO_n \leq O_n\) is the kernel of the \(\det\) homomorphism.
	      This is the special orthogonal group.
\end{itemize}
\begin{proposition}
	\(\det \colon O_n \to \{ \pm 1 \}\) is a surjective homomorphism.
\end{proposition}
\begin{proof}
	If \(A \in O_n\), then \(A^\transpose A = I\).
	So \((\det A)^2 = \det A^\transpose \cdot \det A = \det(A^\transpose A) = \det I = 1\).
	So \(\det A = \pm 1\).
	It is surjective since \(\det I = 1\), and the determinant of the matrix similar to the identity but one of the diagonal entries is \(-1\) has determinant \(-1\).
\end{proof}

\subsection{M\"obius Maps via Matrices}
\begin{proposition}
	The function \(\varphi\colon SL_2(\mathbb C) \to \mathcal M\) mapping
	\[
		\begin{pmatrix}
			a & b \\ c & d
		\end{pmatrix} \mapsto f;\quad f(z) = \frac{az + b}{cz + d}
	\]
	is a surjective homomorphism with kernel \(\{ I, -I \}\).
\end{proposition}
\begin{proof}
	Firstly, \(\varphi\) is a homomorphism.
	If \(f_1(z) = \frac{a_1z+b_1}{c_1z+d_1}, f_2(z) = \frac{a_2z+b_2}{c_2z+d_2}\), then we have seen that \(f_2(f_1(z))\) can be written in the form \(\frac{az+b}{cz+d}\) where
	\[
		\begin{pmatrix}
			a & b \\ c & d
		\end{pmatrix} = \begin{pmatrix}
			a_2 & b_2 \\ c_2 & d_2
		\end{pmatrix}\begin{pmatrix}
			a_1 & b_1 \\ c_1 & d_1
		\end{pmatrix}
	\]
	So
	\[
		\varphi\left( \begin{pmatrix}
				a_2 & b_2 \\ c_2 & d_2
			\end{pmatrix}\begin{pmatrix}
				a_1 & b_1 \\ c_1 & d_1
			\end{pmatrix} \right) = \varphi \begin{pmatrix}
			a_2 & b_2 \\ c_2 & d_2
		\end{pmatrix} \cdot \varphi \begin{pmatrix}
			a_1 & b_1 \\ c_1 & d_1
		\end{pmatrix}
	\]
	Secondly, \(\varphi\) is surjective.
	If \(\frac{az+b}{cz+d}\) is a M\"obius map, then
	\[
		\begin{pmatrix}
			a & b \\ c & d
		\end{pmatrix} \in GL_2(\mathbb C)
	\]
	since \(ad-bc\neq 0\).
	But
	\[
		\det \begin{pmatrix}
			a & b \\ c & d
		\end{pmatrix}
	\]
	may not be 1, so we will take \(D^2\) to be this determinant, then we can consider
	\[
		\begin{pmatrix}
			a/D & b/D \\ c/D & d/D
		\end{pmatrix}
	\]
	This new matrix has determinant 1 and is equal to the original M\"obius map, so we have a matrix in \(SL_2(\mathbb C)\) that maps to any given M\"obius map.
	Finally, we want to find the kernel.
	\[
		\varphi \begin{pmatrix}
			a & b \\ c & d
		\end{pmatrix} = \text{id} \in \mathbb M \implies \frac{az+b}{cz+d} = z \iff c = d = 0; a = d
	\]
	But since this matrix has determinant 1, \(a = d = \pm 1\), and thus \(\ker \varphi = \{ I, -I \}\).
\end{proof}
\begin{corollary}
	\[
		\mathcal M \cong \frac{SL_2(\mathbb C)}{\{ I, -I \}}
	\]
\end{corollary}
\begin{proof}
	This is an immediate consequence of the first isomorphism theorem.
\end{proof}
The quotient \(\frac{SL_2(\mathbb C)}{\{ I, -I \}}\) is known as the projective special linear group \(PSL_2(\mathbb C)\).

\subsection{Actions of Matrices on Vector Spaces}
All of the groups defined above act on the corresponding vector spaces.
For example, \(GL_n(\mathbb F) \acts \mathbb F^n\).
As an example, let \(G \leq GL_2(\mathbb R) \acts \mathbb R^2\).
What are the orbits of this action?
Clearly, \(\{ \vb 0 \}\) is a singleton orbit since we are acting by linear maps.
\begin{itemize}
	\item If \(G = GL_2(\mathbb R)\), \(G\) acts transitively on \(\mathbb R^2 \setminus \{ 0 \}\).
	      We can complete any \(\vb v \neq 0\) to a basis and therefore we have an invertible change of basis matrix sending any basis to any basis.
	      So there are two orbits: \(\mathbb R^2 \setminus \{ 0 \}\) and \(\{ 0 \}\) itself.
	\item If \(G\) is the set of upper triangular matrices given by
	      \[
		      G = \left\{ \begin{pmatrix}
			      a & b \\ 0 & d
		      \end{pmatrix} \in GL_2(\mathbb R) \right\} = \left\{ \begin{pmatrix}
			      a & b \\ 0 & d
		      \end{pmatrix}: a, d \neq 0 \right\}
	      \]
	      We know that \(\Orb(\vb 0) = \{ \vb 0 \}\).
	      Further:
	      \[
		      \Orb\begin{pmatrix}
			      1 \\ 0
		      \end{pmatrix} = \left\{ \begin{pmatrix}
			      a & b \\ 0 & d
		      \end{pmatrix} \begin{pmatrix}
			      1 \\ 0
		      \end{pmatrix}: \begin{pmatrix}
			      a & b \\ 0 & d
		      \end{pmatrix} \in G \right\} = \left\{ \begin{pmatrix}
			      a \\ 0
		      \end{pmatrix}: a \neq 0 \right\}
	      \]
	      We haven't found all of the orbits yet so let us consider another point.
	      \[
		      \Orb\begin{pmatrix}
			      0 \\ 1
		      \end{pmatrix} = \left\{ \begin{pmatrix}
			      a & b \\ 0 & d
		      \end{pmatrix} \begin{pmatrix}
			      0 \\ 1
		      \end{pmatrix}: \begin{pmatrix}
			      a & b \\ 0 & d
		      \end{pmatrix} \in G \right\} = \left\{ \begin{pmatrix}
			      b \\ d
		      \end{pmatrix}: d \neq 0 \right\}
	      \]
	      We have found all of the orbits since the union gives \(\mathbb R^2\).
\end{itemize}
