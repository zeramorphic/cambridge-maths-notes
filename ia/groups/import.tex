\chapter[Groups \\ \textnormal{\emph{Lectured in Michaelmas \oldstylenums{2020} by \textsc{Dr.\ A.\ Khukhro}}}]{Groups}
\emph{\Large Lectured in Michaelmas \oldstylenums{2020} by \textsc{Dr.\ A.\ Khukhro}}

Many mathematical objects have lots of symmetry.
To study symmetry in an abstract way, we define the notion of a group.
Groups allow us to characterise all of the possible symmetries of an object, so understanding groups allows us to understand symmetry itself.
Shapes, numbers, and matrices all give rise to their own groups, which provide insight into how the objects are structured.

As well as studying groups on their own, we also study the ways in which groups can interact.
One example is a particular kind of function called a homomorphism, which preserves the structure of the groups in question.
The homomorphisms between groups allow us to study each group in more detail.

\subfile{../../ia/groups/main.tex}
