\subsection{Basic Setup}
Consider a system of $N$ particles of mass $m_i$ with position vectors $\vb r_i(t)$ and momentum $\vb p_i(t) = m_i \dot {\vb r}_i$. Newton's second law applies to the $i$th particle individually, but not necessarily to the whole group without any further derivation.
\[ m_i \rddot_i = \dot{\vb p}_i = \vb F_i \]
We will make a distinction between internal and external forces;
\[ \vb F_i = \vb F_i^\text{ext} + \sum_{j = 1}^N \vb F_{ij} \]
where the $\vb F_i^\text{ext}$ is the external force on the $i$th particle, and the $\vb F_{ij}$ is the force exerted on the $i$th particle by the $j$th particle. Note that $\vb F_{ii} = 0$ since particles do not affect themselves. Newton's third law gives a further constraint:
\[ \vb F_{ij} = -\vb F_{ji} \]

\subsection{Centre of Mass}
The total mass of the system, $M$, is given by
\[ M = \sum_{i=1}^N m_i \]
The centre of mass, $\vb R$, is given by
\[ \vb R = \frac{1}{M} \sum_{i=1}^N m_i \vb r_i \]
The total linear momentum, $\vb P$, is given by
\[ \vb P = \sum_{i=1}^N m_i \rdot_i = \sum_{i=1}^N \vb p_i = M\dot{\vb R} \]
which is the same momentum as a single particle of mass $M$ and position vector $\vb R$ would have. Then, by Newton's second law, taking into account the fact that the $\vb F_{ij}$ are antisymmetric,
\begin{align*}
	\dot{\vb P} & = M \ddot{\vb R}                                                         \\
	            & = \sum_{i=1}^N \dot{\vb p_i}                                             \\
	            & = \sum_{i=1}^N \vb F_i^\text{ext} + \sum_{i=1}^N \sum_{j=1}^N \vb F_{ij} \\
	            & = \sum_{i=1}^N \vb F_i^\text{ext}                                        \\
	            & = \vb F^\text{ext}
\end{align*}
So the centre of mass moves as if it were the position of a mass $M$ under the influence of a force $\vb F^\text{ext}$. This extends Newton's second law to a system of particles. If $\vb F^\text{ext} = \vb 0$ then we have conservation of the total momentum $\vb P$. In this case, there will be an inertial frame tracking the centre of mass at its origin.

\subsection{Motion Relative to the Centre of Mass}
Let $\vb r_i = \vb R + \vb s_i$, then $\vb s_i$ is the position vector of the $i$th particle relative to the centre of mass. Then
\[ \sum_{i=1}^N m_i \vb s_i = \sum_{i=1}^Nm_i (\vb r_i - \vb R) = \sum_{i=1}^N m_i \vb r_i - \sum_{i=1}^N m_i \vb R = \vb 0 \]
Further,
\[ \dv{t} \left( \sum_{i=1}^N m_i \vb s_i \right) = \vb 0 \]
The total linear momentum is
\[ \vb P = \sum_{i=1}^N m_i (\dot{\vb R} + \dot{\vb s}_i) = \sum_{i=1}^N m_i \dot{\vb R} = M\dot{\vb R} \]
as expected.

\subsection{Angular Momentum}
The total angular momentum $\vb L$ is defined as
\[ \vb L = \sum_{i=1}^N \vb r_i \times \vb p_i \]
Then
\begin{align*}
	\dot{\vb L} & = \sum_{i=1}^N \dot{\vb r}_i \times \vb p_i + \sum_{i=1}^N \vb r_i \times \dot{\vb p}_i                \\
	            & = \sum_{i=1}^N \vb r_i \times \dot{\vb p}_i                                                            \\
	            & = \sum_{i=1}^N \vb r_i \times \left( \vb F_i^\text{ext} + \sum_{j=1}^N \vb F_{ij} \right)              \\
	            & = \sum_{i=1}^N \vb r_i \times \vb F_i^\text{ext} + \sum_{i=1}^N \vb r_i \times \sum_{j=1}^N \vb F_{ij} \\
\end{align*}
The latter term is not necessarily zero, but for example if $\vb F_{ij} \parallel (\vb r_i - \vb r_j)$ then it is zero. If $\vb F_{ij} \parallel (\vb r_i - \vb r_j)$ then
\[ \dot{\vb L} = \sum_{i=1}^N \vb r_i \times \vb F_i^\text{ext} = \vb G^\text{ext} \]
where $\vb G^\text{ext}$ is the total external torque on the system. Relative to the centre of mass, we can write instead
\begin{align*}
	\vb L & = \sum_{i=1}^N m_i (\vb R + \vb s_i) \times (\dot{\vb R} + \dot{\vb s}_i)                                                                                                                                                   \\
	      & = \sum_{i=1}^N m_i \vb R \times \dot{\vb R} + \underbrace{\sum_{i=1}^N m_i \vb R \times \dot{\vb s}_i}_{=0} + \underbrace{\sum_{i=1}^N m_i \vb s_i \times \dot{\vb R}}_{=0} + \sum_{i=1}^N m_i \vb s_i \times \dot{\vb s}_i \\
	      & = \sum_{i=1}^N m_i \vb R \times \dot{\vb R} + \sum_{i=1}^N m_i \vb s_i \times \dot{\vb s}_i
\end{align*}
So the total angular momentum is essentially the sum of the angular momentum of a particle of mass $M$ at $\vb R$ moving with velocity $\dot{\vb R}$, and the angular momentum associated with the particles relative to the centre of mass.

\subsection{Energy}
The total kinetic energy $T$ is given by
\begin{align*}
	T & = \sum_{i=1}^N \frac{1}{2} m_i \dot{\vb r}_i^2                                                                                                                  \\
	  & = \sum_{i=1}^N \frac{1}{2} m_i (\dot{\vb R} + \dot{\vb s}_i)^2                                                                                                  \\
	  & = \frac{1}{2}\dot{\vb R}^2\sum_{i=1}^N m_i + \underbrace{\sum_{i=1}^N  m_i \dot{\vb R} \cdot \dot{\vb s}_i}_{=0} + \sum_{i=1}^N \frac{1}{2} m_i \dot{\vb s}_i^2 \\
	  & = \frac{1}{2}M\dot{\vb R}^2 + \sum_{i=1}^N \frac{1}{2} m_i \dot{\vb s}_i^2
\end{align*}
The total kinetic energy is the sum of the kinetic energy of a particle of mass $M$ at $\vb R$ moving with velocity $\dot{\vb R}$, and the kinetic energy associated with the particles relative to the centre of mass. Let us consider the rate of change of kinetic energy:
\begin{align*}
	\dv{T}{t} & = \dv{t} \sum_{i=1}^N \frac{1}{2} m_i \dot{\vb r}_i^2                                                                              \\
	          & = \sum_{i=1}^N \frac{1}{2} m_i \dot{\vb r}_i \cdot \ddot{\vb r}_i                                                                  \\
	          & = \sum_{i=1}^N \dot{\vb r}_i \cdot \vb F^\text{ext} + \sum_{i=1}^N \dot{\vb r}_i \cdot \sum_{j=1}^N \vb F_{ij}                     \\
	          & = \sum_{i=1}^N \dot{\vb r}_i \cdot \vb F^\text{ext} + \sum_{i=1}^N \sum_{j=i+1}^N (\dot{\vb r}_i - \dot{\vb r}_j) \cdot \vb F_{ij} \\
\end{align*}
If the external forces are defined by a potential
\[ \vb F_i^\text{ext} = -\grad_{\vb r_i} V_i^\text{ext} \]
and the internal forces are defined by a potential
\[ \vb F_{ij} = -\grad_{\vb r_i} V(\vb r_i - \vb r_j) \]
then
\[ \dv{T}{t} = -\dv{t} \sum_{i=1}^N V_i^\text{ext} - \dv{t} \sum_{i=1}^N \sum_{j=1+1}^N V(\vb r_i - \vb r_j) \]
Hence we have conservation of energy if the given properties are true.
