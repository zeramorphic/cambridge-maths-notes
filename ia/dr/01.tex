\subsection{Basic Concepts}
\begin{definition}
	A particle is an object which has negligible size.
	It therefore does not have an alignment or rotation.
	It has a finite mass \(m > 0\), and perhaps an electric charge \(q\) (which may be positive or negative).
	The position of the particle is described by a position vector \(\vb r(t)\) or \(\vb x(t)\), with respect to an origin \(O\).
\end{definition}
\begin{definition}
	The Cartesian components of this vector \(\vb r(t)\) are given by \((x, y, z)\), where \(\vb r = x \vu{\imath} + y \vu{\jmath} + z \vu{k}\), with \(\vu{\imath}, \vu{\jmath}, \vu{k}\) orthonormal.
	The choice of coordinate axes defines a frame of reference \(S\).
\end{definition}
\begin{definition}
	The velocity of a particle is \(\vb u(t) = \dot{\vb r} = \frac{\dd}{\dd{t}}\vb r(t)\).
	The velocity is tangential to the path, or \textit{trajectory}, of the particle.
\end{definition}
\begin{definition}
	The momentum of a particle is \(\vb p = m \vb u\).
\end{definition}
\begin{definition}
	The acceleration of a particle is \(\vb a = \dot{\vb u} = \ddot{\vb r}\).
\end{definition}
\begin{note}
	The time derivative of \(\vb u(t)\), for example, is defined using the limit definition:
	\[
		\dot{\vb u}(t) = \lim_{h \to 0} \frac{\vb u(t + h) - \vb u(t)}{h}
	\]
	with \(\vb u \to \vb u_0\) if and only if \(\abs{\vb u - \vb u_0} \to 0\).
	With Cartesian basis vectors, we can evaluate derivatives componentwise, bringing the differential operator inside each vector component.
\end{note}
The derivatives of scalar and vector functions interoperate as expected.
Suppose we have a scalar function \(f(t)\) and vector functions \(\vb g(t), \vb h(t)\), then for example we have
\[
	\frac{\dd}{\dd{t}}(f \vb g) = \frac{\dd{f}}{\dd{t}} \vb g + f \frac{\dd \vb g}{\dd{t}}
\]
\[
	\frac{\dd}{\dd{t}}(\vb g \cdot \vb h) = \frac{\dd \vb g}{\dd{t}}\cdot \vb h + \vb g \cdot \frac{\dd \vb h}{\dd{t}}
\]
\[
	\frac{\dd}{\dd{t}}(\vb g \times \vb h) = \frac{\dd \vb g}{\dd{t}}\times \vb h + \vb g \times \frac{\dd \vb h}{\dd{t}}
\]
Take note of the ordering of the terms involving \(\vb g\) and \(\vb h\) when using the vector product.

\subsection{Newton's Laws of Motion}
\begin{enumerate}
	\item (Galileo's Law of Inertia) There exist inertial frames of reference in which a particle remains at rest or moves in a straight line at constant speed (i.e.\ at constant velocity), unless it is acted on by a force.
	\item In an inertial frame of reference, the rate of change of momentum of a particle is equal to the force acting on it.
	\item To every action, there is an equal and opposite reaction.
	      The forces exerted between two particles are equal in magnitude and opposite in direction.
\end{enumerate}
Note that the second law is a statement about vectors.
All of these statements that we have made about particles can also be extended to finite bodies, composed of many particles.
