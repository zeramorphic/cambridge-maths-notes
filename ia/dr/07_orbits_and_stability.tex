\subsection{Motion in a central force field}
By Newton's second law, the force in a central force field is given by
\[
	m\ddot{\vb r} = \vb F = -\grad V = -\frac{\dd{V}}{\dd{r}} \vb e_r
\]
The results from the previous lecture give
\begin{equation}
	m\left(\ddot r - r\dot\theta^2\right) \vb e_r + m\left(2\dot r \dot \theta + r \ddot \theta\right) \vb e_\theta = \frac{-\dd{V}}{\dd{r}} \vb e_r \tag{\(\ast\)}
\end{equation}
But the right hand side has no angular component, so \(m\left(2\dot r \dot \theta + r \ddot \theta\right) = 0\).
Then
\[
	\frac{m}{r} \frac{\dd}{\dd{t}}\left(r^2 \dot\theta\right) = 0
\]
So the quantity \(h = r^2 \dot\theta\), known as the \textit{specific} angular momentum (since it contains no mass component) is constant.
Note that the angular momentum \(\vb L\) is given by
\[
	\vb L = m\vb r \times \dot{\vb r} = mr\vb e_r \times \left( \dot r\vb e_r + r\dot\theta \vb e_\theta \right) = mr^2 \dot\theta \vb e_z
\]
Hence the magnitude of the angular momentum is constant.
Now, let us consider the radial component in \((\ast)\).
\begin{align*}
	m\ddot r - mr\dot\theta^2 & = -\frac{\dd{V}}{\dd{r}}                    \\
	m\ddot r                  & = -\frac{\dd{V}}{\dd{r}} + \frac{mh^2}{r^3} \\
	m\ddot r                  & = -\frac{\dd{V}_\text{eff}}{\dd{r}}
\end{align*}
where
\[
	V_\text{eff}(r) = V(r) + \frac{mh^2}{2r^2}
\]
where \(V_\text{eff}\) is called the effective potential.
In other words, the motion of the particle is equivalent to one-dimensional motion under the influence of the effective potential.
The energy of the particle is given by
\[
	T + V(r) = \frac{1}{2}m\left( \dot r^2 + r^2 \dot\theta^2 \right) + V(r) = \frac{1}{2}m\dot r^2 + \frac{mh^2}{2r^2} + V(r) = \frac{1}{2}m\dot r^2 + V_\text{eff}(r)
\]
which is consistent with our description of the effective potential.

\subsection{Orbits under gravity}
As an example, let us consider
\[
	V(r) = \frac{-GMm}{r};\quad V_\text{eff}(r) = \frac{-GMm}{r} + \frac{mh^2}{2r^2}
\]
The effective potential has a single minimum point at \(r_\ast\), and a single root at \(r_0\).
In other words, \(V'_\text{eff}(r_\ast) = 0\) and \(V_\text{eff}(r_0) = 0\).
We can compute that
\[
	r_0 = \frac{h^2}{2GM};\quad r_\ast = \frac{h^2}{GM}
\]
The minimum energy is therefore
\[
	E_\text{min} = \frac{-m(GM)^2}{2h^2}
\]
What is the possible motion of the particle?
At \(E = E_\text{min}\), we have \(r(t) = r_\ast\), an equilibrium position.
Further, \(\dot\theta = \frac{h}{r_\ast^2}\) everywhere.
At \(E_\text{min} < E < 0\), then \(r(t)\) oscillates between a minimum point (periapsis/perihelion/perigee) and a maximum point (apoapsis/aphelion/apogee), and \(\dot\theta\) varies.
If \(E_\text{min} \geq 0\), the particle escapes to infinity.
This is sometimes called an unbound orbit.

\subsection{Stability of circular orbits}
Consider a general potential \(V(r)\).
Does a circular orbit exist, and is it stable?
We will assume that the angular momentum is given and nonzero.
For a circular orbit, the radius is a constant value \(r_\ast\), so \(\ddot r = 0\) and hence \(V'_\text{eff}(r_\ast) = 0\).
We know that we have a stable equilibrium if \(V_\text{eff}\) has a minimum at this point.
Correspondingly, it is unstable if this is a maximum.
So, for instance, it is stable if \(V''_\text{eff}(r_\ast) > 0\).
Now, let us rewrite these conditions in terms of \(V(r)\).
\[
	V'(r_\ast) - \frac{mh^2}{r_\ast^3} = 0;\quad V''(r_\ast) = V''(r_\ast) + \frac{3mh^2}{r_\ast^4} > 0
\]
We can combine these to give the condition for stability as
\[
	V''(r_\ast) + \frac{3V'(r_\ast)}{r_\ast} > 0
\]
Now let us consider an example,
\[
	V(r) = \frac{-km}{r^p}
\]
where \(p>0\), \(k>0\).
If \(p=1\), this is an example of an inverse square law.
We have a circular orbit if
\[
	\frac{pkm}{r_\ast^{p+1}} - \frac{mh^2}{r_\ast^3} = 0
\]
Hence,
\[
	r_\ast^{p-2} = \frac{pkm}{mh^2} \implies r_\ast = \left( \frac{pkm}{mh^2} \right)^{\frac{1}{p-2}}
\]
So there exists a circular orbit for all \(h\) provided \(p \neq 2\).
Is this a stable orbit?
\[
	V''(r_\ast) + \frac{3V'(r_\ast)}{r_\ast} = \frac{-kmp(p+1)}{r_\ast^{p+2}} + \frac{3kmp}{r_\ast^{p+2}} = \frac{p(2-p)km}{r_\ast^{p+2}}
\]
So this is greater than zero (stable) if \(0 < p < 2\) and less than zero (unstable) if \(p > 2\).

\subsection{The orbit equation}
What shape does a non-circular orbit trace out?
We could in principle find \(r(t)\) by the energy equation
\[
	E = \frac{1}{2}m \dot r^2 + V_\text{eff}(r) = \text{constant}
\]
Hence
\[
	t = \pm \sqrt{\frac{m}{2}} \int^r \frac{\dd{u}}{\sqrt{E - V_\text{eff}(u)}}
\]
Then we can use \(r(t)^2\dot \theta = h\) to deduce \(\theta(t)\).
However in practice, this is not useful.
An analytic solution is only possible for a small family of effective potential functions.
It is somewhat more convenient to find \(r\) in terms of \(\theta\), not in terms of \(t\).
We can write
\[
	\frac{\dd}{\dd{t}} = \frac{\dd{\theta}}{\dd{t}} \frac{\dd}{\dd{\theta}} = \frac{h}{r^2} \frac{\dd}{\dd{\theta}}
\]
Applying this to Newton's second law, we have
\[
	m\frac{h}{r^2} \frac{\dd}{\dd{\theta}}\left( \frac{h}{r^2} \frac{\dd}{\dd{\theta}} r \right) - \frac{mh^2}{r^3} = F(r)
\]
The \(\frac{h}{r^2} \frac{\dd}{\dd{\theta}} r\) term suggests using the substitution \(u = \frac{1}{r}\).
Then
\[
	mhu^2 \frac{\dd}{\dd{\theta}}\left( -h\frac{\dd{u}}{\dd{\theta}} \right) - mh^2u^3 = F(u^{-1})
\]
\[
	\frac{\dd^2 u}{\dd{\theta}^2} + u = \frac{-1}{mh^2u^2}F(u^{-1})
\]
This is known as the orbit equation.
We can solve this for \(u\) as a function of \(\theta\).

\subsection{The Kepler problem}
The Kepler problem is the orbit problem, specialised to the case of a gravitational central force.
The force is
\[
	F(r) = \frac{-mk}{r^2}
\]
where the constant \(k\) is equivalent to \(GM\).
Hence,
\[
	\dv[2]{u}{\theta} + u = \frac{-1}{mh^2u^2} \cdot -mku^2 = \frac{k}{h^2}
\]
This gives us a linear equation in \(u\), which is promising for solving this equation.
The solution is
\[
	u = \frac{k}{h^2} + A\cos(\theta - \theta_0)
\]
where \(A\) and \(\theta_0\) are specified by the initial conditions.
Without loss of generality we can let \(A \geq 0\).
If \(A = 0\), then \(u = \frac{k}{h^2}\) giving a circular orbit.
If \(A > 0\), then \(u\) is maximised (at periapsis) when \(\theta = \theta_0\), and \(u\) is minimised (at apoapsis) where \(\theta = \theta_0 + \pi\).
We will choose that \(\theta_0 = 0\) for convenience; we will simply need to change the origin of our coordinate system if this does not hold.
We will redefine other constants for convenience:
\[
	r = \frac{1}{u} = \frac{\ell}{1 + e\cos \theta};\quad \ell = \frac{h^2}{k};\quad e = A\frac{h^2}{k}
\]
This is the equation of a conic section.
Here, \(e\) is the eccentricity of the curve.
We can rewrite this in Cartesian form:
\begin{align}
	r(1 + e\cos \theta)         & = \ell \notag              \\
	r                           & = \ell - ex \notag         \\
	x^2 + y^2                   & = (\ell - ex)^2 \notag     \\
	(1-e^2)x^2 + y^2 + 2e\ell x & = \ell^2 \tag{\(\dagger\)}
\end{align}
By inspection we can see that the value of \((1-e^2)\) determines the shape of the conic section.
\begin{itemize}
	\item \((0 \leq e < 1)\) This forms an ellipse; the orbit is bounded by \(\frac{\ell}{1+e} \leq r \leq \frac{\ell}{1 - e}\).
	      We can rewrite \((\dagger)\) as
	      \[
		      \frac{(x+ea)^2}{a^2} + \frac{y^2}{b^2} = 1
	      \]
	      where \(a = \frac{\ell}{1 - e^2}\) and \(b = \frac{\ell}{\sqrt{1-e^2}}\), and therefore clearly \(b \leq a\).
	      Note that \(e=0\) is the special case of a circle.
	      The origin lies at one of the foci of the ellipse.
	\item \((e > 1)\) This forms a hyperbola.
	      This is an unbounded orbit, since there exists a value \(\alpha\) such that as \(\theta \to \alpha\), we have \(r \to \infty\).
	      Note that \(\alpha = \arccos(\frac{-1}{e}) \in \left(\frac{\pi}{2}, \pi\right)\) We can transform \((\dagger)\) as before:
	      \[
		      \frac{(x-ea)^2}{a^2} - \frac{y^2}{b^2} = 1
	      \]
	      where \(a = \frac{\ell}{e^2 - 1}\) and \(b = \frac{\ell}{\sqrt{e^2 - 1}}\).
	      This hyperbolic orbit represents an incoming body with large velocity, which is deflected by the gravitational force.
	      The asymptotes are
	      \[
		      y = \pm \frac{b}{a}(x-ea)
	      \]
	      In other words,
	      \[
		      bx \mp ay = eab
	      \]
	      The normal vectors to the asymptotes are
	      \[
		      \nhat = \frac{(b, \pm a)}{\sqrt{a^2 + b^2}}
	      \]
	      The (asymptotic) perpendicular distance between the incoming particle and the central mass (the origin) is given by
	      \[
		      \vb r \cdot \nhat = (x, y) \cdot \frac{(b, \pm a)}{\sqrt{a^2 + b^2}} = \frac{bx \mp ay}{\sqrt{a^2 + b^2}} = \frac{eab}{\sqrt{a^2 + b^2}} = b
	      \]
	      This is sometimes called the impact parameter, since it is the distance away from impacting the central mass.
	\item \((e = 1)\) This is the form of a parabola.
	      This can be seen as the `transitional' case between the ellipse and the hyperbola.
	      \[
		      r = \frac{\ell}{1 + \cos\theta}
	      \]
	      Hence, \(r \to \infty\) as \(\theta \to \pm \pi\).
	      In Cartesian coordinates,
	      \[
		      y^2 = \ell(\ell-2x)
	      \]
\end{itemize}
The other variables have very useful geometric interpretations; review diagrams of conic sections for more information.

\subsection{Energy and eccentricity}
Recall that
\[
	E = \frac{1}{2}m\left( \dot r^2 + r^2 \dot\theta^2 \right) - \frac{mk}{r}
\]
We can rewrite this in terms of \(u\), using \(\dot r = -h\dv{u}{\theta}\):
\begin{align*}
	E & = \frac{1}{2}mh^2\left( \left( \dv{u}{\theta} \right)^2 + u^2 \right) - mku                                      \\
	  & = \frac{1}{2}mh^2\left( e^2\sin^2\theta + (1+e\cos\theta)^2 \right)\frac{1}{e^2} - \frac{mk}{\ell}(1+\cos\theta) \\
	  & = \frac{mk}{2\ell}(e^2 - 1)
\end{align*}
So the energy is positive (unbounded orbits) if \(\abs{e} > 1\), and negative (bounded orbits) if \(\abs{e} < 1\).
The marginal case is at \(e = 1\), and \(E = 0\).

\subsection{Kepler's laws of planetary motion}
Kepler's Laws state:
\begin{enumerate}
	\item The orbit of a planet is an ellipse, with the sun at one focus.
	\item The line between a planet and the sun sweeps out equal area in equal times.
	\item The period \(P\) and the semi-major axis \(a\) are related: \(P^2 \propto a^3\).
\end{enumerate}
Note that the first law is consistent with our solution of the orbit equation for bound orbits.
The second law can be rewritten as approximating the sector area with \(\frac{1}{2}r^2\delta\theta\), giving a rate of change with respect to time of \(\frac{1}{2}r^2\dot\theta\), which is half of the angular momentum \(h\).
So this law can be seen as stating that the angular momentum is constant.
Using dimensional analysis, we can get close to the third law, but we need to be a little more precise to verify it completely.
The area of the ellipse is \(\pi a b = \frac{h}{2}P\) since in one period the line sweeps out the entire area of the ellipse.
We can then derive that \(P^2 = \frac{4\pi^2}{k}a^3\).
Note that two ellipses with equal semi-major but differing semi-minor axes have the same period.

\subsection{Rutherford scattering}
Consider a positive charge fired towards another, fixed, positive charge.
The particle will be deflected by the electrostatic force between the two particles.
What is the angle \(\beta\) by which the particle is deflected?
This is motion under a repulsive inverse square law force.
\[
	V(r) = \frac{mk}{r};\quad F(r) = \frac{mk}{r^2}
\]
We have already solved this problem for an attractive inverse square law force; this was the orbit equation.
We can replace \(k\) with \(-k\) to model a repulsive force.
\[
	u = \frac{-k}{h^2} + A\cos(\theta - \theta_0);\quad \theta_0 = 0, A \geq 0
\]
We can rewrite this as
\[
	r = \frac{\ell}{e\cos\theta - 1};\quad \ell = \frac{h^2}{k}, e = \frac{Ah^2}{k}
\]
Since we want \(r > 0\), we need \(e > 1\) such that for some \(\theta\), \(r > 0\).
Then, \(r \to \infty\) as \(\theta \to \pm \alpha\), with \(\arccos(e^{-1}) \in \left(0, \frac{\pi}{2}\right)\).
This gives a hyperbolic orbit.
We find
\[
	\frac{(x - ea)^2}{a^2} - \frac{y^2}{b^2} = 1;\quad a = \frac{\ell}{e^2 - 1}, b = \frac{\ell}{\sqrt{e^2 - 1}}
\]
\(h\) is given by \(\abs{\vb r \times \dot{\vb r}}\).
\(b\), the impact parameter, is the asymptotic distance of the moving particle from impacting the fixed particle.
On the incoming asymptote, \(\dot {\vb r} \approx v\vb e_\parallel\), and \(\vb r \approx b\vb e_\perp + z\vb e_\parallel\) for some \(z\).
Hence, \(h=bv\).
Since \(\tan \alpha = \sqrt{e^2 - 1}\), we have
\[
	b = \frac{\ell}{\tan \alpha} = \frac{\ell}{\tan\left( \frac{\pi}{2} - \beta \right)} = \frac{h^2}{k}\tan\left( \frac{\beta}{2} \right) = \frac{v^2b^2}{k}\tan\left( \frac{\beta}{2} \right)
\]
Hence
\[
	\beta = 2\arctan\left( \frac{k}{bv^2} \right)
\]
