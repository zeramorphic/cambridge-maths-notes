\subsection{Definition}
Friction is a contact force, unlike the forces we have discussed previously.
It is a convenient encapsulation of many complicated molecular phenomena; it is not a fundamental force.

\subsection{Dry friction}
The friction associated with solid bodies in contact is called `dry' friction.
It has two associated forces: the normal force \(\vb N\) perpendicular to the contact surface, which prevents objects from passing through each other, and the tangential force \(\vb F\) parallel to the contact surface, which resists the relative tangential motion of the bodies in contact.
When the two bodies are static, the empirically-derived formula relating the forces is
\[
	\abs{\vb F} \leq \mu_s \abs{\vb N}
\]
where \(\mu_s\) is the coefficient of static friction.
If the objects start to move relative to each other, this is kinetic friction.
In this case,
\[
	\abs{\vb F} = \mu_k \abs{\vb N}
\]
where \(\mu_k\) is the coefficient of kinetic friction.
Generally \(\mu_s > \mu_k > 0\).

\subsection{Fluid drag}
When a solid body moves through a fluid (a liquid or a gas), it experiences a drag force.
There are two important equations that model fluid drag.
The linear drag formula is
\[
	\vb F = -k_1 \vb u
\]
This formula is most relevant to `small' objects, moving through a viscous fluid.
Stokes' drag law for a moving sphere states that
\[
	k_1 = 6 \pi \eta R
\]
where \(\eta\) is the viscosity of the fluid, and \(R\) is the radius of the sphere.
The quadratic drag formula is
\[
	\vb F = -k_2 \abs{\vb u} \vb u
\]
This formula is more relevant to `large' objects, moving through a less viscous fluid.
Of course, \(k_1 \neq k_2\) since they have different dimensions.
Typically, we have
\[
	k_2 = \rho_{\text{fluid}} C_D R^2
\]
where \(C_D\) is the drag coefficient, and \(R^2\) is the size of the cross section.

\subsection{Work done by friction}
Note that since friction always acts in a direction opposite to a component of motion, the body loses kinetic energy if the fluid (or other body) is assumed to be at rest.
The rate of work under a fluid's drag force is
\[
	\vb F \cdot \vb u = \begin{cases}
		-k_1 \abs{\vb u}^2 & \text{linear drag}    \\
		-k_2 \abs{\vb u}^3 & \text{quadratic drag}
	\end{cases}
\]
In the latter case, the total work done is proportional to \(\abs{\vb u}^2\) multiplied by the total distance travelled.
The fluid gains energy, which may manifest as heat.

\subsection{Projectiles experiencing linear drag}
Let us consider the example of a projectile moving through the air, under uniform gravity and a linear drag force.
\[
	m \frac{\dd \vb u}{\dd{t}} = m\vb g - k\vb u
\]
Solving with an integrating factor, we have
\[
	\frac{\dd}{\dd{t}}\left( \vb u e^{kt/m} \right) = m\vb g e^{kt/m}
\]
\[
	\vb u = \frac{m\vb g}{k} + \vb C e^{-kt/m}
\]
We can find \(\vb C\) using the initial conditions, say at \(t=0\), \(\vb x = 0, \vb u = \vb U\).
\[
	\vb u = \frac{m\vb g}{k} + \left( \vb U - \frac{-\vb g}{k} \right) e^{-kt/m}
\]
Then
\begin{align*}
	\vb x & = \frac{m\vb g}{k}t - \frac{m}{k}\left( \vb U - \frac{-\vb g}{k} \right) e^{-kt/m} + D               \\
	      & =\frac{m\vb g}{k}t + \frac{m}{k}\left( \vb U - \frac{-\vb g}{k} \right) \left( 1 - e^{-kt/m} \right)
\end{align*}
Now, considering the components of \(\vb x = (x, y, z)\) and \(\vb u = (u, v, w)\), we can choose
\[
	\vb U = (U \cos \theta, 0, U\sin\theta);\quad \vb g = (0, 0, -g)
\]
Then
\[
	\begin{pmatrix}
		u \\ v \\ w
	\end{pmatrix} = \begin{pmatrix}
		U\cos\theta e^{-kt/m} \\
		0                     \\
		\left(U \sin\theta + \frac{mg}{k}\right)e^{-kt/m} - \frac{mg}{k}
	\end{pmatrix}
\]
Note that the terminal velocity is \((0, 0, -mg/k)\), achieved on a time scale of \(m/k\) (as seen from the exponential term).
Further,
\[
	\begin{pmatrix}
		x \\ y \\ z
	\end{pmatrix} = \begin{pmatrix}
		\frac{mU\cos\theta}{k}\left( 1 - e^{-kt/m} \right) \\
		0                                                  \\
		\frac{m}{k}\left( U \sin\theta + \frac{mg}{k} \right)\left( 1 - e^{-kt/m} \right) - \frac{mgt}{k}
	\end{pmatrix}
\]
There exists a range \(R\) of this particle, since initially the particle moves upwards, but as time increases the particle begins moving downwards again.
\(R\) is a function of \(U, \theta, m, k, g\).
We can construct the dimensionless group \(\frac{kU}{mg} = \frac{U/g}{m/k}\), which can be thought of as the gravitational time scale divided by the frictional time scale.
Dimensional analysis shows that
\[
	R = \frac{U^2}{g}F\left(\theta, \frac{kU}{mg}\right)
\]
When \(\frac{kU}{mg} \ll 1\), this is very small friction.
\[
	R = \frac{U^2}{g}\cdot 2\sin\theta\cos\theta
\]
When \(\frac{kU}{mg} \gg 1\), this is very large friction.
\[
	R = \frac{U^2}{g} \left( \frac{mg}{kU}\cos\theta \right)
\]
\(R\) is a decreasing function of \(\frac{kU}{mg}\).
