\documentclass{article}

\usepackage[UKenglish]{babel}
\usepackage[T1]{fontenc}
\usepackage[utf8]{inputenc}
\usepackage[a4paper]{geometry} % , margin=20mm
\usepackage{textcomp} % makes the "not defining \perthousand"/"\micro" errors go away by including this first
\usepackage{amsmath}
\usepackage{amssymb}
\usepackage{amsthm}
\usepackage{amsfonts}
\usepackage{bbm}
\usepackage{wrapfig}
\usepackage{physics}
\usepackage{bm}
\usepackage{tgpagella}
\DeclareDocumentCommand\mathbf{m}{\bm{\mathrm{#1}}} % make bold work for greek symbols
\DeclareDocumentCommand\vnabla{}{\nabla} % use non-bold nabla for \grad, \curl etc.
% Enabled to unify laplacian symbol between vector and scalar forms
\DeclareDocumentCommand\dotproduct{}{\cdot} % use non-bold dot for scalar product to unify notation
\DeclareDocumentCommand\crossproduct{}{\times} % use non-bold dot for scalar product to unify notation
\usepackage{gensymb}
\usepackage{enumerate}
\usepackage{mathtools}
\usepackage{centernot}
\usepackage{relsize}
\usepackage{mathrsfs}
\usepackage{siunitx}
\usepackage{booktabs}
\usepackage[ruled,vlined]{algorithm2e}
\usepackage{array}
\usepackage{multirow}
\usepackage{pgfplots}
\pgfplotsset{width=10cm,compat=1.9}
\usepgfplotslibrary{external}
\tikzexternalize[prefix=tikz/]
\usepackage[pdfa]{hyperref}
\hypersetup{
	colorlinks=true,
	linktoc=all,
	linkcolor=black,
}
\usepackage{minitoc}

\numberwithin{equation}{section} % make equations be numbered 1.1 not 1

\newcommand{\tableofcontentsnewpage}{\tableofcontents\newpage}

% create the theorem environments
\theoremstyle{definition}
\newtheorem*{definition}{Definition}

\newtheorem*{claim}{Claim}
\newtheorem*{theorem}{Theorem}
\newtheorem*{proposition}{Proposition}
\newtheorem*{lemma}{Lemma}
\newtheorem*{corollary}{Corollary}
\newtheorem*{example}{Example} % todo: convert `as an example...' to the example environment

\theoremstyle{remark}
\newtheorem*{note}{Note}
\newtheorem*{remark}{Remark}

\newcommand{\ddempty}{\mathrm{d}}
\newcommand{\dn}[2]{\mathrm{d}^#1#2}
\newcommand{\st}{\text{ s.t.
	}}
\newcommand{\contradiction}{\(\#\)}
\newcommand{\genset}[1]{\langle{} #1 \rangle}
\newcommand{\nhat}{\vu{n}}
\newcommand{\rdot}{\dot{\vb{r}}}
\newcommand{\rddot}{\ddot{\vb{r}}}
\newcommand{\transpose}{\intercal}
\newcommand{\acts}{\curvearrowright}
\newcommand{\adjugate}[1]{\widetilde{#1}}
\newcommand{\mathhuge}[1]{\mathlarger{\mathlarger{\mathlarger{#1}}}}
\newcommand{\stcomp}[1]{{#1}^c} % consider \complement?
% Personally I think this looks better, and it's what Wikipedia uses
\newcommand{\prob}[1]{\mathbb{P}\left({#1}\right)}
\newcommand{\psub}[2]{\mathbb{P}_{#1}\left({#2}\right)}
\newcommand{\psubx}[1]{\psub{x}{#1}}
\newcommand{\expect}[1]{\mathbb{E}\left[{#1}\right]}
\newcommand{\esub}[2]{\mathbb{E}_{#1}\left[{#2}\right]}
\newcommand{\esubx}[1]{\esub{x}{#1}}
\newcommand{\Var}[1]{\Varop\left({#1}\right)}
\newcommand{\Cov}[1]{\Covop\left({#1}\right)}
\newcommand{\Corr}[1]{\Corrop\left({#1}\right)}
\newcommand{\convdist}{\xrightarrow{d}}
\newcommand{\convprob}{\xrightarrow{\mathbb{P}}}
\newcommand{\wildcard}{{}\cdot{}}
\newcommand{\inner}[1]{\left\langle{#1}\right\rangle}
\newcommand{\Markov}[1]{\Markovop\left({#1}\right)}

\DeclareMathOperator{\vecspan}{span}
\DeclareMathOperator{\HCF}{HCF}
\DeclareMathOperator{\LCM}{LCM}
\DeclareMathOperator{\ord}{ord}
\DeclareMathOperator{\Sym}{Sym}
\DeclareMathOperator{\nullity}{null}
\DeclareMathOperator{\Orb}{Orb}
\DeclareMathOperator{\Stab}{Stab}
\DeclareMathOperator{\ccl}{ccl}
\DeclareMathOperator{\Varop}{Var}
\DeclareMathOperator{\Covop}{Cov}
\DeclareMathOperator{\Corrop}{Corr}
\DeclareMathOperator{\Markovop}{Markov}

\DeclarePairedDelimiter\ceil{\lceil}{\rceil}
\DeclarePairedDelimiter\floor{\lfloor}{\rfloor}

% for arrows in the middle of the line
\usetikzlibrary{decorations.markings}
\tikzset{->-/.style={decoration={
		markings,
		mark=at position #1 with {\arrow{>}}},postaction={decorate}}}


\title{Dynamics and Relativity}
\author{Cambridge University Mathematical Tripos: Part IA}

\begin{document}
\maketitle

\tableofcontents
\newpage

\section{Basic Definitions and Newton's Laws}
\subsection{Basic Concepts}
\begin{definition}
    A particle is an object which has negligible size. It therefore does not have an alignment or rotation. It has a finite mass $m > 0$, and perhaps an electric charge $q$ (which may be positive or negative). The position of the particle is described by a position vector $\bm r(t)$ or $\bm x(t)$, with respect to an origin $O$.
\end{definition}
\begin{definition}
    The Cartesian components of this vector $\bm r(t)$ are given by $(x, y, z)$, where $\bm r = x \hat{\bm \imath} + y \hat{\bm \jmath} + z \hat{\bm k}$, with $\hat{\bm \imath}, \hat{\bm \jmath}, \hat{\bm k}$ orthonormal. The choice of coordinate axes defines a frame of reference $S$.
\end{definition}
\begin{definition}
    The velocity of a particle is $\bm u(t) = \dot{\bm r} = \frac{\dd}{\dd t}\bm r(t)$. The velocity is tangential to the path, or \textit{trajectory}, of the particle.
\end{definition}
\begin{definition}
    The momentum of a particle is $\bm p = m \bm u$.
\end{definition}
\begin{definition}
    The acceleration of a particle is $\bm a = \dot{\bm u} = \ddot{\bm r}$.
\end{definition}
\begin{note}
    The time derivative of $\bm u(t)$, for example, is defined using the limit definition:
    \[ \dot{\bm u}(t) = \lim_{h \to 0} \frac{\bm u(t + h) - \bm u(t)}{h} \]
    with $\bm u \to \bm u_0$ if and only if $\abs{\bm u - \bm u_0} \to 0$. With Cartesian basis vectors, we can evaluate derivatives componentwise, bringing the differential operator inside each vector component.
\end{note}
The derivatives of scalar and vector functions interoperate as expected. Suppose we have a scalar function $f(t)$ and vector functions $\bm g(t), \bm h(t)$, then for example we have
\[ \frac{\dd}{\dd t}(f \bm g) = \frac{\dd f}{\dd t} \bm g + f \frac{\dd \bm g}{\dd t} \]
\[ \frac{\dd}{\dd t}(\bm g \cdot \bm h) = \frac{\dd \bm g}{\dd t}\cdot \bm h + \bm g \cdot \frac{\dd \bm h}{\dd t} \]
\[ \frac{\dd}{\dd t}(\bm g \times \bm h) = \frac{\dd \bm g}{\dd t}\times \bm h + \bm g \times \frac{\dd \bm h}{\dd t} \]
Take note of the ordering of the terms involving $\bm g$ and $\bm h$ when using the vector product.

\subsection{Newton's Laws of Motion}
\begin{enumerate}
    \item (Galileo's Law of Inertia) There exist inertial frames of reference in which a particle remains at rest or moves in a straight line at constant speed (i.e. at constant velocity), unless it is acted on by a force.
    \item In an inertial frame of reference, the rate of change of momentum of a particle is equal to the force acting on it.
    \item To every action, there is an equal and opposite reaction. The forces exerted between two particles are equal in magnitude and opposite in direction.
\end{enumerate}
Note that the second law is a statement about vectors. All of these statements that we have made about particles can also be extended to finite bodies, composed of many particles.

\section{Galilean Invariance}
\subsection{Boosts}
In an inertial frame, the acceleration of a particle is zero if the force acting on the particle is zero.
\[ \ddot{\bm r} = \bm 0 \iff \bm F = \bm 0 \]
There is no unique inertial frame of reference. If $S$ is an inertial frame, then any other frame $S'$ moving at constant velocity relative to $S$ is also an inertial frame. For example, suppose that $S'$ is moving at speed $v$ in the $x$ direction. Then here
\[ x'=x-vt;\quad y'=y;\quad z'=z;\quad t'=t \]
and
We can generalise this to $S'$ moving in an arbitrary direction relative to $S$, i.e.
\[ \bm r' = \bm r - \bm v t \]
where $\bm v$ is the velocity of $S'$ relative to $S$. This type of transformation is known as a `boost'. For a particle with position vector $\bm r(t)$ in $S$ (and position vector $\bm r'(t)$ in $S'$), we can compute the velocity $\bm u = \dot{\bm r}$ and acceleration $\bm a = \ddot{\bm r}$ as follows:
\[ \bm u' = \bm u - \bm v;\quad \bm a' = \bm a \]
This can be seen by taking the derivative of the `boost' formula.

\subsection{Galilean Transformations}
A general Galilean Transformation is any transformation that preserves inertial frames. They are combinations of:
\begin{itemize}
    \item boosts $\bm r' = \bm r - \bm vt$ where $\bm v$ is constant,
    \item translations of space (moving the origin) $\bm r' = \bm r - \bm r_0$ where $r_0$ is constant,
    \item translations of time $t' = t - t_0$ where $t_0$ is constant,
    \item rotations and reflections in space $\bm r' = R \bm r$ where $R$ is a constant orthogonal matrix.
\end{itemize}
This set generates the Galilean group. For any Galilean transformation we have
\[ \ddot{\bm r} = \bm 0 \iff \ddot{(\bm r')} = \bm 0 \]

The principle of Galilean relativity is that the laws of Newtonian physics are the same in all inertial frames. In other words, the laws of physics are always the same:
\begin{itemize}
    \item at any point in space
    \item at any point in time
    \item in any direction
    \item at any constant velocity
\end{itemize}
Any set of equations which describe Newtonian physics must preserve this Galilean invariant. This shows that measurement of velocity cannot be absolute, it must be relative to a specific inertial frame of reference --- but conversely, measurement of acceleration \textit{is} absolute.

\subsection{Newton's Second Law}
For any particle subject to a force $\bm F$, the momentum $\bm p$ of the particle satisfies
\[ \frac{\dd \bm p}{\dd t} = \bm F \]
where $\bm p = m \bm u$. For this part of the course, let us assume that $m$ is constant. Then $\bm F = \dot{\bm p} = m \bm a$. We can interpret this value $m$ as a measure of `reluctance to accelerate', i.e. its inertia. If $\bm F$ is specified as a function of $\bm r, \dot{\bm r}, t$, then we have a second order differential equation for $\bm r$. In order to solve this equation, we must provide two of initial conditions $\bm r_0$ and $\dot{\bm r}_0$ at some initial time $t_0$. The trajectory of the particle is then determined for all future and past times.

\subsection{Gravitational Force}
Consider two particles, one at $\bm r_1$ and one at $\bm r_2$. Newton's law of gravitation states that the gravitational force on $\bm r_1$ is given by
\[ \bm F_1 = \frac{- G m_1 m_2 (\bm r_1 - \bm r_2)}{\abs{\bm r_1 - \bm r_2}^3} \]
where $G$ is the gravitational constant, and $\bm F_2$ is given by $-\bm F_1$. Note that:
\begin{itemize}
    \item This is known as an inverse square law, since the magnitude of the output is proportional to the inverse of the square of the distance between the particles.
    \item This is an attractive force, since it is in the direction $\bm r_2 - \bm r_1$.
    \item This obeys Newton's Third Law, since $\bm F_2 = - \bm F_1$.
    \item By inspection, $G$ must have dimension $L^3 \cdot M^{-1} \cdot T^{-2}$, i.e. length cubed over mass over time squared.
\end{itemize}

\subsection{Electromagnetic Force}
Consider a particle with electric charce $q$, in the presence of an electric field $\bm E(\bm r, t)$ and a magnetic field $\bm B(\bm r, t)$. The Lorentz force law states that
\[ \bm F(\bm r, \dot{\bm r}, t) = q\left( \bm E + \dot{\bm r} \times \bm B \right) \]
As an example, let $\bm E = \bm 0$ everywhere, and let $\bm B$ be a constant vector. Then
\[ m \ddot{\bm r} = q \dot{\bm r} \times \bm B \]
We can solve this differential equation for $\bm r$. Let us choose axes such that $\bm B = B \hat{\bm z}$, i.e. $\bm B$ is in the $z$ direction. Evaluating the cross product, $m \ddot{z} = 0$, so $z = z_0 + ut$ where $z_0$ and $u$ are constants. Further,
\[ m \ddot x = qB\dot y;\quad m \ddot y = -qB\dot x \]
For convenience, let us define $\omega = qB/m$, and then
\[ x = x_0 - \alpha \cos(\omega(t - t_0));\quad x = y_0 + \alpha \sin(\omega(t - t_0)) \]
This describes circles in the $x$--$y$ plane, and constant velocity motion in the $z$ direction. This results in a helix in the direction of the magnetic field, clockwise when viewed from the direction of $\bm B$.

\end{document}