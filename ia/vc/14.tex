\subsection{Introduction and the Equations}
We will denote the magnetic field by \(\vb B(\vb x, t)\), and the electric field by \(\vb E(\vb x, t)\).
These fields will depend on the current density \(\vb J(\vb x, t)\), the electric current per unit area, and the charge density \(\rho(\vb x, t)\), the electric charge per unit volume.
\begin{align}
	\div{\vb E}                                      & = \frac{\rho}{\varepsilon_0} \tag{1} \\
	\div{\vb B}                                      & = 0                          \tag{2} \\
	\curl{\vb E} + \pdv{\vb B}{t}                    & = 0                          \tag{3} \\
	\curl{\vb B} - \mu_0\varepsilon_0 \pdv{\vb E}{t} & = \mu_0 \vb J \tag{4}
\end{align}
The constants \(\varepsilon_0\) and \(\mu_0\) denote the permittivity and permeability of free space, which obey
\[
	\frac{1}{\mu_0 \varepsilon_0} = c^2
\]
where \(c\) is the speed of light, \SI{299792458}{\metre\per\second}.
Note that if we take the divergence of equation (4), we find
\begin{align*}
	\mu_0 \varepsilon_0 \pdv{t}\left( \div{\vb E} \right) + \mu_0 \div{\vb J}              & = 0 \\
	(1) \implies \mu_0 \varepsilon_0 \pdv{t}\frac{\rho}{\varepsilon_0} + \mu_0 \div{\vb J} & = 0 \\
	\pdv{\rho}{t} + \div{\vb J}                                                            & = 0 \\
\end{align*}
which is a conservation law for charge.

\subsection{Integral Formulations of Maxwell's Equations}
Integrating \((1)\) over some volume \(V\), and applying the divergence theorem, gives
\begin{align*}
	\div{\vb E}                              & = \frac{\rho}{\varepsilon_0}                 \\
	\int_V \div{\vb E} \dd{V}                & = \frac{1}{\varepsilon_0} \int_V \rho \dd{V} \\
	\int_{\partial V} \vb E \cdot \dd{\vb S} & = \frac{Q}{\varepsilon_0}                    \\
\end{align*}
where \(Q\) is the total charge in \(V\).
This is known as Gauss' law.
For magnetic fields, we can integrate \((2)\):
\[
	\int_{V} \div{\vb B} \dd{V} = \int_{\partial V} \vb B \cdot \dd{\vb S} = 0
\]
Hence there is no net magnetic flux over any closed surface \(\partial V\).
This implies that we cannot have a magnetic field with only a north pole or only a south pole.
Integrating \((3)\) over a surface, and applying Stokes' theorem, gives
\begin{align*}
	\curl{\vb E} + \pdv{\vb B}{t}                                                      & = 0                                     \\
	\int_S \qty(\curl{\vb E} + \pdv{\vb B}{t}) \cdot \dd{\vb S}                        & = 0                                     \\
	\oint_{\partial S} \vb E \cdot \dd{\vb x} + \int_S \pdv{\vb B}{t} \cdot \dd{\vb S} & = 0                                     \\
	\oint_{\partial S} \vb E \cdot \dd{\vb x}                                          & = -\dv{t} \int_S \vb B \cdot \dd{\vb S}
\end{align*}
So a change in the magnetic flux through a surface \(S\) induces a circulation in \(\vb E\) about the boundary.
Integrating \((4)\) over a surface, again using Stokes' theorem, we have
\begin{align*}
	\int_S \qty(\curl{\vb B} - \mu_0\varepsilon_0 \pdv{\vb E}{t}) \cdot \dd{\vb S} & = \int_S \mu_0 \vb J \cdot \dd{\vb S}                                                             \\
	\oint_{\partial S} \vb B \cdot \dd{\vb x}                                      & = \int_S \mu_0 \vb J \cdot \dd{\vb S} + \int_S \mu_0\varepsilon_0 \pdv{\vb E}{t} \cdot \dd{\vb S} \\
	\oint_{\partial S} \vb B \cdot \dd{\vb x}                                      & = \mu_0 \int_S \vb J \cdot \dd{\vb S} + \mu_0\varepsilon_0 \dv{t} \int_S \vb E \cdot \dd{\vb S}   \\
\end{align*}
So if an electric current flows through a wire, this generates a circulation of the magnetic field around the wire.

\subsection{Electromagnetic Waves}
In empty space, \(\rho = 0\) and \(\vb J = \vb 0\).
Maxwell's equations show that
\begin{align}
	\div{\vb E}                                       & = 0 \tag{1} \\
	\div{\vb B}                                       & = 0 \tag{2} \\
	\curl{\vb E} + \pdv{\vb B}{t}                     & = 0 \tag{3} \\
	\curl{\vb B} - \mu_0 \varepsilon_0 \pdv{\vb E}{t} & = 0 \tag{4}
\end{align}
Recall that the Laplacian of a vector field \(\vb F\) is
\[
	\laplacian \vb F = \grad(\div{\vb F}) - \curl(\curl{\vb F})
\]
We can deduce that
\begin{align*}
	\laplacian \vb E & = \grad(\div{\vb E}) - \curl(\curl{\vb E})  \\
	                 & = \grad(0) - \curl(-\pdv{\vb B}{t})         \\
	                 & = \curl(\pdv{\vb B}{t})                     \\
	                 & = \dv{t} \curl{\vb B}                       \\
	                 & = \dv{t} \mu_0 \varepsilon_0 \pdv{\vb E}{t} \\
	                 & = \frac{1}{c^2} \pdv[2]{\vb E}{t}
\end{align*}
\[
	\therefore \laplacian \vb E - \frac{1}{c^2} \pdv[2]{\vb E}{t} = \vb 0
\]
which is the wave equation for waves travelling at speed \(c\).
Hence, in a vacuum, the electric field propagates at speed \(c\).
Similarly, for the magnetic field,
\begin{align*}
	\laplacian \vb B & = \grad(\div{\vb B}) - \curl(\curl{\vb B})             \\
	                 & = \grad(0) - \curl(\mu_0 \varepsilon_0 \pdv{\vb E}{t}) \\
	                 & = -\mu_0\varepsilon_0 \dv{t} \curl{\vb E}              \\
	                 & = \mu_0\varepsilon_0 \dv{t} \pdv{\vb B}{t}             \\
	                 & = \frac{1}{c^2} \pdv[2]{\vb B}{t}
\end{align*}
\[
	\therefore \laplacian \vb B - \frac{1}{c^2} \pdv[2]{\vb B}{t} = \vb 0
\]
Hence the magnetic field also propagates at speed \(c\).
So in general, we can say that electromagnetic waves always travel at speed \(c\) in a vacuum.

\subsection{Electrostatics and Magnetostatics}
Suppose that all fields and source terms are independent of \(t\).
Then Maxwell's equations decouple into
\begin{align}
	\div{\vb E}  & = \frac{\rho}{\varepsilon_0} \tag{1} \\
	\div{\vb B}  & = 0                          \tag{2} \\
	\curl{\vb E} & = 0                          \tag{3} \\
	\curl{\vb B} & = \mu_0 \vb J \tag{4}
\end{align}
which gives one system of equations for \(\vb E\), and one for \(\vb B\).
When considering the whole of \(\mathbb R^3\), which is 2-connected, then equations (2) and (3) imply
\[
	\vb E = -\grad \phi;\quad \vb B = \curl \vb A
\]
where \(\phi\) is the electric potential, and \(\vb A\) is the magnetic potential.
Substituting into the other two equations, we have
\begin{align*}
	(1) \implies -\div{\grad \phi} & = \frac{\rho}{\varepsilon_0} \\
	-\laplacian \phi               & = \frac{\rho}{\varepsilon_0}
\end{align*}
and
\[
	(4) \implies \curl(\curl{\vb A}) = \mu_0 \vb J
\]
