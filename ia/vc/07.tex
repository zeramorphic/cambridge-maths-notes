\subsection{Example and Change of Variables}
To integrate over regions \(V\) in \(\mathbb R^3\), we can use similar ideas to those discussed in the previous lecture.
\[
	\int_V f(\vb x) \dd{V} = \lim_{\varepsilon \to 0} \sum_{i,j,k} f(x_i, y_j, z_k) \,\delta V_{ijk}
\]
where the \(\delta V_{ijk}\) partition \(V\), and each contain the point \((x_i, y_j, z_k)\).
In this case, the volume element satisfies
\[
	\dd{V} = \dd{x}\dd{y}\dd{z}
\]
The integrals may be computed in any order.
As an example, consider the simplex defined by
\[
	V = \{ x > 0,\, y > 0,\, z > 0,\, x+y+z < 1 \}
\]
We can compute the volume using the integral
\begin{align*}
	I & = \int_{z=0}^1 \int_{y=0}^{1-z} \int_{x=0}^{1-y-z} 1 \dd{x}\dd{y}\dd{z}                                                        \\
	  & = \int_{z=0}^1 \int_{y=0}^{1-z} (1-y-z)\dd{y}\dd{z}                                                                            \\
	  & = \int_{z=0}^1 \left((1-z) - \frac{1}{2}(1-z)^2 - (1-z)z\right) \dd{z}                                                         \\
	  & = \left[ z - \frac{1}{2}z^2 - \frac{1}{2}z + \frac{1}{2}z^2 - \frac{1}{6}z^3 - \frac{1}{2}z^2 + \frac{1}{3}z^3 \right]_{z=0}^1 \\
	  & = \frac{1}{6}
\end{align*}
We can compute things like the centre of mass, assuming it has constant density \(\rho = 1\).
Then
\[
	\vb X = \frac{1}{m} \int_V \rho \vb x \dd{V} = \frac{1}{4}\begin{pmatrix}
		1 \\1\\1
	\end{pmatrix}
\]
\begin{proposition}
	Let \(x(u, v, w), y(u, v, w), z(u, v, w)\) be a continuously differentiable bijection with a continuously differentiable inverse, that maps the volume \(V'\) to \(V\).
	The integral
	\[
		\iiint_V f(x, y, z)\dd{x}\dd{y}\dd{z} = \iiint_{V'} f(x(u, v, w), y(u, v, w), z(u, v, w))\,\abs{J}\dd{u}\dd{v}\dd{w}
	\]
	where
	\[
		J = \det\left( \frac{\partial \vb x}{\partial u} \,\middle|\, \frac{\partial \vb x}{\partial v} \,\middle|\, \frac{\partial \vb x}{\partial w} \right)
	\]
	More concisely,
	\[
		\dd{x}\dd{y}\dd{z} = \abs{J}\dd{u}\dd{v}\dd{w}
	\]
\end{proposition}
\noindent The Jacobian comes from the fact that the volume of a parallepiped generated by the vectors
\[
	\frac{\partial \vb x}{\partial u} \delta u,\,\frac{\partial \vb x}{\partial v} \delta v,\,\frac{\partial \vb x}{\partial w} \delta w
\]
is precisely the determinant of the Jacobian matrix multiplied by \(\delta u\,\delta v\,\delta w\).
The rest of this proof follows from the two-dimensional case.
As an example, let us consider cylindrical polar coordinates \((u, v, w) = (\rho, \phi, z)\).
\[
	\dd{V} = \rho \, \dd \rho \, \dd \phi \, \dd{z};\quad \abs{J} = \rho
\]
In spherical polar coordinates \((u, v, w) = (r, \theta, \phi)\),
\[
	\dd{V} = r^2 \sin\theta \dd{r} \,\dd \theta \,\dd \phi;\quad \abs{J} = r^2\sin\theta
\]

\subsection{Calculating Volumes}
We can use the volume element to calculate, for example, the volume of a ball of radius \(R\).
To begin, let us use Cartesian coordinates.
\begin{align*}
	\int_V \dd{V} & = \int_{z=-R}^R \dd{z} \int_{y = -\sqrt{R^2 - z^2}}^{\sqrt{R^2 - z^2}} \dd{y} \int_{x = -\sqrt{R^2 - z^2 - y^2}}^{\sqrt{R^2 - z^2 - y^2}} \dd{x}                             \\
	              & = \int_{z=-R}^R \dd{z} \int_{y = -\sqrt{R^2 - z^2}}^{\sqrt{R^2 - z^2}} \dd{y} \left[ 2\sqrt{R^2 - z^2 - y^2} \right]                                                         \\
	              & = \int_{z=-R}^R \dd{z} \left[ y\sqrt{R^2 - z^2 - y^2} + (R^2 - z^2) \arctan \left( \frac{y}{\sqrt{R^2 - z^2 - y^2}} \right)_{y=-\sqrt{R^2 - z^2}}^{\sqrt{R^2 - z^2}} \right] \\
	              & = \int_{z=-R}^R \dd{z} \left[ \pi (R^2 - z^2) \right]                                                                                                                        \\
	              & = \frac{4}{3}\pi R^3
\end{align*}
We can alternatively use spherical polar coordinates.
\begin{align*}
	\int_V \dd{V} & = \int_{r=0}^R \dd{r} \int_{\theta=0}^\pi \dd \theta \int_{\phi=0}^{2\pi} \dd \phi \cdot r^2 \sin\theta        \\
	              & = \int_{r=0}^R r^2\dd{r} \int_{\theta=0}^\pi \sin\theta\, \dd \theta \int_{\phi=0}^{2\pi} \dd \phi             \\
	              & = \int_{r=0}^R r^2\dd{r} \cdot \int_{\theta=0}^\pi \sin\theta\, \dd \theta \cdot \int_{\phi=0}^{2\pi} \dd \phi \\
	              & = \frac{1}{3}R^3 \cdot 2 \cdot 2 \pi                                                                           \\
	              & = \frac{4}{3}\pi R^3
\end{align*}
This is clearly a much cleaner computation.
Now, consider the a ball of radius \(a\) with cylinder of radius \(b<a\) removed from the centre aligned with the \(z\) axis.
To calculate this volume, the symmetry of the problem suggests we might want to use cylindrical polar coordinates.
\[
	V = \{ (\rho, \phi, z) \colon 0 < \rho^2 + z^2 < a^2,\, b < \rho < a \}
\]
\begin{align*}
	\int_V \dd{V} & = \int_{\rho=b}^a \rho\,\dd \rho \int_{\phi=0}^{2\pi} \dd \phi \int_{z=-\sqrt{a^2 - \rho^2}}^{\sqrt{a^2 - \rho^2}} \dd{z} \\
	              & = 2 \pi \int_b^a 2\rho\sqrt{a^2 - \rho^2}\,\dd \rho                                                                       \\
	              & = \frac{4}{3}\pi (a^2 - b^2)^{\frac{3}{2}}
\end{align*}
