\chapter[Vector Calculus \\ \textnormal{\emph{Lectured in Lent \oldstylenums{2021} by \textsc{Dr.\ A.\ Ashton}}}]{Vector Calculus}
\emph{\Large Lectured in Lent \oldstylenums{2021} by \textsc{Dr.\ A.\ Ashton}}

This course brings the tools of calculus to higher dimensions.
We move away from one-dimensional graphs and towards curves and surfaces, building the foundation for a subject called differential geometry.
These new kinds of objects have different ways of calculating derivatives, giving rise to various differential operators.
One such operator, the gradient operator, shows how the value of a function changes when the input point is moved slightly in all possible directions in space.
These differential operators show up in many formulas in mathematics and physics, and we explore various tools to solve equations involving them.

We also study tensors, which can be thought of as a step up from vectors, matrices, or bilinear maps.
We can also apply differential operators to tensors.
Tensors can be seen in physics, such as the linear strain tensor which explains some features of how an elastic body deforms, or the inertia tensor which shows how mass is concentrated in a rigid body.

\subfile{../../ia/vc/main.tex}
