\subsection{Additive categories}
In this section, we will study categories enriched over \( (\mathbf{AbGp}, \otimes, \mathbb Z) \); these are called \emph{additive} categories.
We will also consider other weaker enrichments: a category enriched over \( (\mathbf{Set}_\star, \wedge, 2) \) is called \emph{pointed}, and a category enriched over \( (\mathbf{CMon}, \otimes, \mathbb N) \), where \( \mathbf{CMon} \) is the category of commutative monoids, is called \emph{semi-additive}.

In a pointed category \( \mathcal C \), each \( \mathcal C(A, B) \) has a distinguished element 0, and all composites with zero morphisms are zero morphisms.
In a semi-additive category \( \mathcal C \), each \( \mathcal C(A, B) \) has a binary addition operation which is associative, commutative, and has an identity \( 0 \).
Composition in a semi-additive category is bilinear, so \( (f + g)(h + k) = fh + gh + fk + gk \) whenever the composites are defined.
In an additive category, each morphism \( f \in \mathcal C(A, B) \) has an additive inverse \( -f \in \mathcal C(A, B) \).
\begin{lemma}
    \begin{enumerate}
        \item For an object \( A \) in a pointed category \( \mathcal C \), the following are equivalent.
        \begin{enumerate}
            \item \( A \) is a terminal object of \( \mathcal C \).
            \item \( A \) is an initial object of \( \mathcal C \).
            \item \( 1_A = 0 : A \to A \).
        \end{enumerate}
        \item For objects \( A, B, C \) in a semi-additive category \( \mathcal C \), the following are equivalent.
        \begin{enumerate}
            \item there exist morphisms \( \pi_1 : C \to A \) and \( \pi_2 : C \to B \) making \( C \) into a product of \( A \) and \( B \);
            \item there exist morphisms \( \nu_1 : A \to C \) and \( \nu_2 : B \to C \) making \( C \) into a coproduct of \( A \) and \( B \);
            \item there exist morphisms \( \pi_1 : C \to A, \pi_2 : C \to B, \nu_1 : A \to C, \nu_2 : B \to C \) satisfying
            \[ \pi_1 \nu_1 = 1_A;\quad \pi_2 \nu_2 = 1_B;\quad \pi_1 \nu_2 = 0;\quad \pi_2 \nu_1 = 0;\quad \nu_1 \pi_1 + \nu_2 \pi_1 = 1_C \]
        \end{enumerate}
    \end{enumerate}
\end{lemma}
