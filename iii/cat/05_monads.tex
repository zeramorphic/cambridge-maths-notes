\subsection{???}
Suppose \( F \dashv G \) is an adjunction with \( F : \mathcal C \to \mathcal D \) and \( G : \mathcal D \to \mathcal C \), where \( \mathcal C \) is a well-understood category, but \( \mathcal D \) is not.
We can study \( \mathcal D \) indirectly inside the context of \( \mathcal C \) by using the adjunction.
We have the composite \( T = GF : \mathcal C \to \mathcal C \), and we have the unit \( \eta : 1_{\mathcal C} \to T \).
The counit is not directly accessible from \( \mathcal C \), but we have \( \mu = G\epsilon_F : T^2 \to T \).
The triangular identities give rise to identities linking \( \eta \) and \( \mu \).
\[\begin{tikzcd}
	T & {T^2} \\
	& T
	\arrow["T\eta", from=1-1, to=1-2]
	\arrow["\mu", from=1-2, to=2-2]
	\arrow["{1_T}"', from=1-1, to=2-2]
\end{tikzcd}\quad\quad\begin{tikzcd}
	T & {T^2} \\
	& T
	\arrow["{\eta_T}", from=1-1, to=1-2]
	\arrow["\mu", from=1-2, to=2-2]
	\arrow["{1_T}"', from=1-1, to=2-2]
\end{tikzcd}\]
In addition, naturality of \( \epsilon \) gives
\[\begin{tikzcd}
	{T^3} & {T^2} \\
	{T^2} & T
	\arrow["T\mu", from=1-1, to=1-2]
	\arrow["\mu", from=1-2, to=2-2]
	\arrow["{\mu_T}"', from=1-1, to=2-1]
	\arrow["\mu"', from=2-1, to=2-2]
\end{tikzcd}\]
\begin{definition}
    A \emph{monad} on a category \( \mathcal C \) is a triple \( \mathbb T = (T, \eta, \mu) \) where \( T \) is a functor \( \mathbb C \to \mathbb C \), and \( \eta : 1_{\mathcal C} \to T \) and \( \mu : T^2 \to T \) are natural transformations satisfying the following commutative diagrams.
    \[\begin{tikzcd}
        T & {T^2} \\
        & T
        \arrow["T\eta", from=1-1, to=1-2]
        \arrow["\mu", from=1-2, to=2-2]
        \arrow["{1_T}"', from=1-1, to=2-2]
    \end{tikzcd}\quad\quad\begin{tikzcd}
        T & {T^2} \\
        & T
        \arrow["{\eta_T}", from=1-1, to=1-2]
        \arrow["\mu", from=1-2, to=2-2]
        \arrow["{1_T}"', from=1-1, to=2-2]
    \end{tikzcd}\quad\quad\begin{tikzcd}
        {T^3} & {T^2} \\
        {T^2} & T
        \arrow["T\mu", from=1-1, to=1-2]
        \arrow["\mu", from=1-2, to=2-2]
        \arrow["{\mu_T}"', from=1-1, to=2-1]
        \arrow["\mu"', from=2-1, to=2-2]
    \end{tikzcd}\]
    \( \eta \) is the \emph{unit} of the monad, and \( \mu \) is the \emph{multiplication} of the monad.
\end{definition}
The dual notion is called a \emph{comonad}.
\begin{example}
    \begin{enumerate}
        \item Let \( M \) be a monoid.
        The functor \( M \times (-) : \mathbf{Set} \to \mathbf{Set} \) has a monad structure.
        The unit \( \eta_A : A \to M \times A \) maps each \( a \) to \( (1, a) \), and the multiplication \( \mu_A : M \times M \times A \to M \times A \) maps \( (m, m', a) \) to \( (mm', a) \).
        These maps are natural.
        The required commutative diagrams encode precisely the left and right unit laws and the associativity law of a monoid.
        In fact, monoids correspond precisely to monads on \( \mathbf{Set} \) whose underlying functors have right adjoints.
        \item Let \( P : \mathbf{Set} \to \mathbf{Set} \) be the covariant power-set functor.
        This can be given a monad structure.
        The unit \( \eta_A : A \to PA \) maps \( a \) to its singleton \( \qty{a} \), and the multiplication \( \mu_A : PPA \to PA \) is the union operation mapping \( S \) to \( \bigcup S \).
        One can check that the required laws are satisfied.
    \end{enumerate}
\end{example}
These examples both arise as a result of adjunctions.
Example (a) arises from the free \( M \)-set functor \( F : \mathbf{Set} \to [M, \mathbf{Set}] \) and the forgetful functor \( U : [M, \mathbf{Set}] \to \mathbf{Set} \), where \( F \dashv U \).
For example (b), there is a forgetful functor \( U : \mathbf{CSLat} \to \mathbf{Set} \) from the category of complete (join-)semilattices.
This has a left adjoint \( P : \mathbf{Set} \to \mathbf{CSLat} \), which is the free complete semilattice on \( A \).
Indeed, given any \( f : A \to UB \), there is a unique extension of \( f \) to a join-preserving map \( \overline f : PA \to B \) given by
\[ \overline f(A') = \bigvee \qty{f(a') \mid a' \in A'} \]
Note that an \( M \)-set is a set \( A \) equipped with a map \( \alpha : M \times A \to A \), and a complete semilattice is a set \( A \) equipped with a map \( \bigvee : PA \to A \).
So the elements of the other category can be defined in terms of the monad.

This holds in general: every monad arises from an adjunction.
