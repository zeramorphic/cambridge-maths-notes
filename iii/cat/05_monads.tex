\subsection{Definition}
Suppose \( F \dashv G \) is an adjunction with \( F : \mathcal C \to \mathcal D \) and \( G : \mathcal D \to \mathcal C \), where \( \mathcal C \) is a well-understood category, but \( \mathcal D \) is not.
We can study \( \mathcal D \) indirectly inside the context of \( \mathcal C \) by using the adjunction.
We have the composite \( T = GF : \mathcal C \to \mathcal C \), and we have the unit \( \eta : 1_{\mathcal C} \to T \).
The counit is not directly accessible from \( \mathcal C \), but we have \( \mu = G\epsilon_F : T^2 \to T \).
The triangular identities give rise to identities linking \( \eta \) and \( \mu \).
\[\begin{tikzcd}
	T & {T^2} \\
	& T
	\arrow["T\eta", from=1-1, to=1-2]
	\arrow["\mu", from=1-2, to=2-2]
	\arrow["{1_T}"', from=1-1, to=2-2]
\end{tikzcd}\quad\quad\begin{tikzcd}
	T & {T^2} \\
	& T
	\arrow["{\eta_T}", from=1-1, to=1-2]
	\arrow["\mu", from=1-2, to=2-2]
	\arrow["{1_T}"', from=1-1, to=2-2]
\end{tikzcd}\]
In addition, naturality of \( \epsilon \) gives
\[\begin{tikzcd}
	{T^3} & {T^2} \\
	{T^2} & T
	\arrow["T\mu", from=1-1, to=1-2]
	\arrow["\mu", from=1-2, to=2-2]
	\arrow["{\mu_T}"', from=1-1, to=2-1]
	\arrow["\mu"', from=2-1, to=2-2]
\end{tikzcd}\]
\begin{definition}
    A \emph{monad} on a category \( \mathcal C \) is a triple \( \mathbb T = (T, \eta, \mu) \) where \( T \) is a functor \( \mathbb C \to \mathbb C \), and \( \eta : 1_{\mathcal C} \to T \) and \( \mu : T^2 \to T \) are natural transformations satisfying the following commutative diagrams.
    \[\begin{tikzcd}
        T & {T^2} \\
        & T
        \arrow["T\eta", from=1-1, to=1-2]
        \arrow["\mu", from=1-2, to=2-2]
        \arrow["{1_T}"', from=1-1, to=2-2]
    \end{tikzcd}\quad\quad\begin{tikzcd}
        T & {T^2} \\
        & T
        \arrow["{\eta_T}", from=1-1, to=1-2]
        \arrow["\mu", from=1-2, to=2-2]
        \arrow["{1_T}"', from=1-1, to=2-2]
    \end{tikzcd}\quad\quad\begin{tikzcd}
        {T^3} & {T^2} \\
        {T^2} & T
        \arrow["T\mu", from=1-1, to=1-2]
        \arrow["\mu", from=1-2, to=2-2]
        \arrow["{\mu_T}"', from=1-1, to=2-1]
        \arrow["\mu"', from=2-1, to=2-2]
    \end{tikzcd}\]
    \( \eta \) is the \emph{unit} of the monad, and \( \mu \) is the \emph{multiplication} of the monad.
\end{definition}
The dual notion is called a \emph{comonad}.
\begin{example}
    \begin{enumerate}
        \item Let \( M \) be a monoid.
        The functor \( M \times (-) : \mathbf{Set} \to \mathbf{Set} \) has a monad structure.
        The unit \( \eta_A : A \to M \times A \) maps each \( a \) to \( (1, a) \), and the multiplication \( \mu_A : M \times M \times A \to M \times A \) maps \( (m, m', a) \) to \( (mm', a) \).
        These maps are natural.
        The required commutative diagrams encode precisely the left and right unit laws and the associativity law of a monoid.
        In fact, monoids correspond precisely to monads on \( \mathbf{Set} \) whose underlying functors have right adjoints.
        \item Let \( P : \mathbf{Set} \to \mathbf{Set} \) be the covariant power-set functor.
        This can be given a monad structure.
        The unit \( \eta_A : A \to PA \) maps \( a \) to its singleton \( \qty{a} \), and the multiplication \( \mu_A : PPA \to PA \) is the union operation mapping \( S \) to \( \bigcup S \).
        One can check that the required laws are satisfied.
    \end{enumerate}
\end{example}
These examples both arise as a result of adjunctions.
Example (a) arises from the free \( M \)-set functor \( F : \mathbf{Set} \to [M, \mathbf{Set}] \) and the forgetful functor \( U : [M, \mathbf{Set}] \to \mathbf{Set} \), where \( F \dashv U \).
For example (b), there is a forgetful functor \( U : \mathbf{CSLat} \to \mathbf{Set} \) from the category of complete (join-)semilattices.
This has a left adjoint \( P : \mathbf{Set} \to \mathbf{CSLat} \), which is the free complete semilattice on \( A \).
Indeed, given any \( f : A \to UB \), there is a unique extension of \( f \) to a join-preserving map \( \overline f : PA \to B \) given by
\[ \overline f(A') = \bigvee \qty{f(a') \mid a' \in A'} \]
Note that an \( M \)-set is a set \( A \) equipped with a map \( \alpha : M \times A \to A \), and a complete semilattice is a set \( A \) equipped with a map \( \bigvee : PA \to A \).
So the elements of the other category can be defined in terms of the monad.

This holds in general: every monad arises from an adjunction.
We present two constructions.

\subsection{Eilenberg--Moore algebras}
\begin{definition}
    Let \( \mathbb T = (T, \eta, \mu) \) be a monad on \( \mathcal C \).
    An \emph{Eilenberg--Moore algebra} or \emph{\( \mathbb T \)-algebra} is a pair \( (A, \alpha) \) where \( A \) is an object in \( \mathcal C \), and \( \alpha : TA \to A \) is a morphism satisfying
    % https://q.uiver.app/#q=WzAsMyxbMCwwLCJBIl0sWzEsMCwiVEEiXSxbMSwxLCJBIl0sWzAsMSwiXFxldGFfQSJdLFsxLDIsIlxcYWxwaGEiXSxbMCwyLCIxX0EiLDJdXQ==
\[\begin{tikzcd}
	A & TA \\
	& A
	\arrow["{\eta_A}", from=1-1, to=1-2]
	\arrow["\alpha", from=1-2, to=2-2]
	\arrow["{1_A}"', from=1-1, to=2-2]
\end{tikzcd}\quad\quad\begin{tikzcd}
	T^2A & TA \\
	TA & A
	\arrow["T\alpha", from=1-1, to=1-2]
	\arrow["\alpha", from=1-2, to=2-2]
	\arrow["\mu"', from=1-1, to=2-1]
	\arrow["\alpha"', from=2-1, to=2-2]
\end{tikzcd}\]
    A homomorphism of algebras \( f : (A, \alpha) \to (B, \beta) \) is a morphism \( f : A \to B \) such that the following diagram commutes.
    % https://q.uiver.app/#q=WzAsNCxbMCwwLCJUQSJdLFsxLDAsIlRCIl0sWzEsMSwiQiJdLFswLDEsIkEiXSxbMCwxLCJUZiJdLFsxLDIsIlxcYmV0YSJdLFswLDMsIlxcYWxwaGEiLDJdLFszLDIsImYiLDJdXQ==
\[\begin{tikzcd}
	TA & TB \\
	A & B
	\arrow["Tf", from=1-1, to=1-2]
	\arrow["\beta", from=1-2, to=2-2]
	\arrow["\alpha"', from=1-1, to=2-1]
	\arrow["f"', from=2-1, to=2-2]
\end{tikzcd}\]
    This forms a category of \( \mathbb T \)-algebras, denoted \( \mathcal C^{\mathbb T} \).
\end{definition}
\begin{proposition}
    The forgetful functor \( G^{\mathbb T} : \mathcal C^{\mathbb T} \to \mathcal C \) has a left adjoint \( F^{\mathbb T} \), and the adjunction \( F^{\mathbb T} \dashv G^{\mathbb T} \) induces the monad \( \mathbb T \) on \( \mathcal C \).
\end{proposition}
\begin{proof}
    We define the \emph{free algebra} of an object \( A \) to be \( F^{\mathbb T}A = (TA, \mu_A) \).
    This defines an algebra structure on \( TA \) for every \( A \) by the monad laws.
    For \( f : A \to B \), we define \( F^{\mathbb T}f = Tf \); this is a homomorphism by naturality of \( \mu \).
    This is functorial as \( T \) is functorial.

    We have \( G^{\mathbb T} F^{\mathbb T} = T \).
    For the unit of the adjunction, we use the unit of the monad \( \eta \).
    For the counit, we define
    \[ \eta_{(A,\alpha)} = \alpha : F^{\mathbb T} A \to (A, \alpha) \]
    This is a homomorphism by the definition of an algebra, and it is a natural transformation by the definition of homomorphisms of algebras.
    It suffices to verify the triangular identities, which follows from the remaining unused diagrams.
    One can check that the multiplication induced by this monad is equal to that of \( \mathbb T \).
\end{proof}

\subsection{Kleisli categories}
If \( F \dashv G \) with \( F : \mathcal C \to \mathcal D \) and \( G : \mathcal D \to \mathcal C \) is an adjunction inducing \( \mathbb T \), then \( F' \dashv G' \) with \( F' : \mathcal C \to \mathcal D' \) and \( G' : \mathcal D' \to \mathcal C \), where \( \mathcal D' \) is the full subcategory of \( \mathcal D \) on objects in the image of \( F \).
Thus, when finding a construction for \( \mathcal D \), we can assume that \( F \) is surjective (or, indeed, bijective) on objects.
Then, the morphisms \( FA \to FB \) must correspond to morphisms \( A \to GFB \) under the adjunction, but \( GF = T \).
\begin{definition}
    Let \( \mathbb T = (T, \mu, \eta) \) be a monad on \( \mathcal C \).
    The \emph{Kleisli category} \( \mathcal C_{\mathbb T} \) is the category where the objects are precisely the objects of \( \mathcal C \), and the morphisms from \( A \) to \( B \) in \( \mathcal C_{\mathbb T} \) are the morphisms \( A \to TB \) in \( \mathcal C \).
    To avoid confusion, we will denote morphisms from \( A \) to \( B \) in this category by \( A \rightdotarrow B \).
    The identity \( A \rightdotarrow A \) is \( \eta_A : A \to TA \).
    The composite of
    % https://q.uiver.app/#q=WzAsMyxbMCwwLCJBIl0sWzEsMCwiQiJdLFsyLDAsIkMiXSxbMCwxLCJmIiwwLHsic3R5bGUiOnsiYm9keSI6eyJuYW1lIjoiZG90dGVkIn19fV0sWzEsMiwiZyIsMCx7InN0eWxlIjp7ImJvZHkiOnsibmFtZSI6ImRvdHRlZCJ9fX1dXQ==
\[\begin{tikzcd}
	A & B & C
	\arrow["f", dotted, from=1-1, to=1-2]
	\arrow["g", dotted, from=1-2, to=1-3]
\end{tikzcd}\]
    is
    % https://q.uiver.app/#q=WzAsNCxbMCwwLCJBIl0sWzEsMCwiVEIiXSxbMiwwLCJUVEMiXSxbMywwLCJUQyJdLFswLDEsImYiXSxbMSwyLCJUZyJdLFsyLDMsIlxcbXVfQyJdXQ==
\[\begin{tikzcd}
	A & TB & T^2C & TC
	\arrow["f", from=1-1, to=1-2]
	\arrow["Tg", from=1-2, to=1-3]
	\arrow["{\mu_C}", from=1-3, to=1-4]
\end{tikzcd}\]
    These satisfy the unit and associativity laws.
    % https://q.uiver.app/#q=WzAsNCxbMCwwLCJBIl0sWzEsMCwiVEIiXSxbMiwwLCJUVEIiXSxbMiwxLCJUQiJdLFswLDEsImYiXSxbMSwyLCJUXFxldGFfQiJdLFsyLDMsIlxcbXVfQiJdLFsxLDMsIjFfe1RCfSJdXQ==
\[\begin{tikzcd}
	A & TB & T^2B \\
	&& TB
	\arrow["f", from=1-1, to=1-2]
	\arrow["{T\eta_B}", from=1-2, to=1-3]
	\arrow["{\mu_B}", from=1-3, to=2-3]
	\arrow["{1_{TB}}"', from=1-2, to=2-3]
\end{tikzcd}\quad\quad\begin{tikzcd}
	A & TA \\
	TB & T^2B \\
	& TB
	\arrow["{\eta_A}", from=1-1, to=1-2]
	\arrow["Tf", from=1-2, to=2-2]
	\arrow["{\mu_B}", from=2-2, to=3-2]
	\arrow["f"', from=1-1, to=2-1]
	\arrow["{\eta_{TB}}", from=2-1, to=2-2]
	\arrow["{1_{TB}}"', from=2-1, to=3-2]
\end{tikzcd}\]
\[\begin{tikzcd}
	A & TB & {T^2C} & {T^3D} & {T^2D} \\
	&& TC & {T^2D} & TD
	\arrow["f", from=1-1, to=1-2]
	\arrow["Tg", from=1-2, to=1-3]
	\arrow["{T^2h}", from=1-3, to=1-4]
	\arrow["{T\mu_D}", from=1-4, to=1-5]
	\arrow["{\mu_D}", from=1-5, to=2-5]
	\arrow["{\mu_{TD}}", from=1-4, to=2-4]
	\arrow["{\mu_D}"', from=2-4, to=2-5]
	\arrow["{\mu_C}", from=1-3, to=2-3]
	\arrow["Th"', from=2-3, to=2-4]
\end{tikzcd}\]
where in the last diagram, the upper composite is \( (hg)f \) and the lower composite is \( h(gf) \) in \( \mathcal C_{\mathbb T} \).
\end{definition}
\begin{proposition}
    There is an adjunction \( F_{\mathbb T} \dashv G_{\mathbb T} \) where \( F_{\mathbb T} : \mathcal C \to \mathcal C_{\mathbb T} \) and \( G_{\mathbb T} : \mathcal C_{\mathbb T} \to \mathcal C \) that induces the monad \( \mathbb T \).
\end{proposition}
\begin{proof}
    We define \( F_{\mathbb T} A = A \), and for \( f : A \to B \), define \( F_{\mathbb T} f = \eta_B f \).
    This preserves identities as \( 1_{F_{\mathbb T} A} = \eta_A \), and preserves composites since% https://q.uiver.app/#q=WzAsNyxbMCwwLCJBIl0sWzEsMCwiQiJdLFsyLDAsIlRCIl0sWzIsMSwiVEMiXSxbMywwLCJUXjJDIl0sWzMsMSwiVEMiXSxbMSwxLCJDIl0sWzAsMSwiZiJdLFsxLDIsIlxcZXRhX0IiXSxbMiwzLCJUZyIsMl0sWzMsNCwiVFxcZXRhX0MiLDFdLFs0LDUsIlxcbXVfQyJdLFszLDUsIjFfe1RDfSIsMl0sWzEsNiwiZyJdLFs2LDMsIlxcZXRhX0MiXV0=
    \[\begin{tikzcd}
        A & B & TB & {T^2C} \\
        & C & TC & TC
        \arrow["f", from=1-1, to=1-2]
        \arrow["{\eta_B}", from=1-2, to=1-3]
        \arrow["Tg"', from=1-3, to=2-3]
        \arrow["{T\eta_C}", from=2-3, to=1-4]
        \arrow["{\mu_C}", from=1-4, to=2-4]
        \arrow["{1_{TC}}"', from=2-3, to=2-4]
        \arrow["g", from=1-2, to=2-2]
        \arrow["{\eta_C}", from=2-2, to=2-3]
    \end{tikzcd}\]
    commutes.
    For \( G_{\mathbb T} \), we define \( G_{\mathbb T} A = TA \), and for \( f : A \rightdotarrow B \), we define \( G_{\mathbb T} f \) to be the composite
    % https://q.uiver.app/#q=WzAsMyxbMCwwLCJUQSJdLFsxLDAsIlRUQiJdLFsyLDAsIlRCIl0sWzAsMSwiVGYiXSxbMSwyLCJcXG11X0IiXV0=
\[\begin{tikzcd}
	TA & T^2B & TB
	\arrow["Tf", from=1-1, to=1-2]
	\arrow["{\mu_B}", from=1-2, to=1-3]
\end{tikzcd}\]
    Note that \( G_{\mathbb T} \) preserves identities by the unit law and preserves composites as
    % https://q.uiver.app/#q=WzAsNyxbMCwwLCJUQSJdLFsxLDAsIlReMkIiXSxbMiwwLCJUXjNDIl0sWzMsMCwiVF4yQyJdLFszLDEsIlRDIl0sWzIsMSwiVF4yQyJdLFsxLDEsIlRCIl0sWzAsMSwiVGYiXSxbMSwyLCJUXjJnIl0sWzIsMywiVFxcbXVfQyJdLFszLDQsIlxcbXVfQyJdLFsyLDUsIlxcbXVfe1RDfSJdLFs1LDQsIlxcbXVfQyIsMl0sWzEsNiwiXFxtdV9CIl0sWzYsNSwiVGciLDJdXQ==
\[\begin{tikzcd}
	TA & {T^2B} & {T^3C} & {T^2C} \\
	& TB & {T^2C} & TC
	\arrow["Tf", from=1-1, to=1-2]
	\arrow["{T^2g}", from=1-2, to=1-3]
	\arrow["{T\mu_C}", from=1-3, to=1-4]
	\arrow["{\mu_C}", from=1-4, to=2-4]
	\arrow["{\mu_{TC}}", from=1-3, to=2-3]
	\arrow["{\mu_C}"', from=2-3, to=2-4]
	\arrow["{\mu_B}", from=1-2, to=2-2]
	\arrow["Tg"', from=2-2, to=2-3]
\end{tikzcd}\]
    commutes.
    Then \( G_{\mathbb T} \) is a functor, and \( G_{\mathbb T} F_{\mathbb T} = T \).
    The unit of the adjunction is the unit of the monad \( \eta \).
    For the counit \( \epsilon_A : TA = F_{\mathbb T} G_{\mathbb T} A \rightdotarrow A \), we use the identity \( 1_{TA} \).
    This is natural, as given \( f : A \rightdotarrow B \), the diagram
    % https://q.uiver.app/#q=WzAsNCxbMCwwLCJUQSJdLFsxLDAsIlRCIl0sWzEsMSwiQiJdLFswLDEsIkEiXSxbMCwxLCJGX3tcXG1hdGhiYiBUfSBHX3tcXG1hdGhiYiBUfWYiLDAseyJzdHlsZSI6eyJib2R5Ijp7Im5hbWUiOiJkb3R0ZWQifX19XSxbMSwyLCJcXGVwc2lsb25fQiIsMCx7InN0eWxlIjp7ImJvZHkiOnsibmFtZSI6ImRvdHRlZCJ9fX1dLFswLDMsIlxcZXBzaWxvbl9BIiwyLHsic3R5bGUiOnsiYm9keSI6eyJuYW1lIjoiZG90dGVkIn19fV0sWzMsMiwiZiIsMix7InN0eWxlIjp7ImJvZHkiOnsibmFtZSI6ImRvdHRlZCJ9fX1dXQ==
\[\begin{tikzcd}[column sep=large]
	TA & TB \\
	A & B
	\arrow["{F_{\mathbb T} G_{\mathbb T}f}", dotted, from=1-1, to=1-2]
	\arrow["{\epsilon_B}", dotted, from=1-2, to=2-2]
	\arrow["{\epsilon_A}"', dotted, from=1-1, to=2-1]
	\arrow["f"', dotted, from=2-1, to=2-2]
\end{tikzcd}\]
    commutes, as the paths are% https://q.uiver.app/#q=WzAsNSxbMCwwLCJUQSJdLFsxLDAsIlReMkIiXSxbMiwwLCJUQiJdLFszLDAsIlReMkIiXSxbNCwwLCJUQiJdLFswLDEsIlRmIl0sWzEsMiwiXFxtdV9CIl0sWzIsMywiXFxldGFfe1RCfSJdLFszLDQsIlxcbXVfQiJdXQ==
    \[\begin{tikzcd}
        TA & {T^2B} & TB & {T^2B} & TB
        \arrow["Tf", from=1-1, to=1-2]
        \arrow["{\mu_B}", from=1-2, to=1-3]
        \arrow["{\eta_{TB}}", from=1-3, to=1-4]
        \arrow["{\mu_B}", from=1-4, to=1-5]
    \end{tikzcd}\]
    and
    % https://q.uiver.app/#q=WzAsMyxbMCwwLCJUQSJdLFsxLDAsIlReMkIiXSxbMiwwLCJUQiJdLFswLDEsIlRmIl0sWzEsMiwiXFxtdV9CIl1d
\[\begin{tikzcd}
	TA & {T^2B} & TB
	\arrow["Tf", from=1-1, to=1-2]
	\arrow["{\mu_B}", from=1-2, to=1-3]
\end{tikzcd}\]
    which coincide.
    One can show that both triangular identities reduce to a unit law.
    It suffices to verify that the multiplication of the induced monad is correct.
    The multiplication law is \( G_{\mathbb T} \epsilon_{F_{\mathbb T} A} \), which is
    % https://q.uiver.app/#q=WzAsMyxbMCwwLCJUXjJBIl0sWzEsMCwiVF4yQSJdLFsyLDAsIlRBIl0sWzAsMSwiVDFfe1RBfSJdLFsxLDIsIlxcbXVfQSJdXQ==
\[\begin{tikzcd}
	{T^2A} & {T^2A} & TA
	\arrow["{T1_{TA}}", from=1-1, to=1-2]
	\arrow["{\mu_A}", from=1-2, to=1-3]
\end{tikzcd}\]
    which is equal to \( \mu_A \), as required.
\end{proof}

\subsection{Universal properties}
\begin{definition}
    Let \( \mathbb T = (T, \eta, \mu) \) be a monad on \( \mathcal C \).
    Then \( \operatorname{Adj}(\mathbb T) \) is the category of adjunctions \( F \dashv G \) which induce \( \mathbb T \), where the morphisms \( F \dashv G \) to \( F' \dashv G' \) are the functors \( K : \mathcal D \to \mathcal D' \) satisfying \( KF = F \) and \( G' K = G \).
\end{definition}
\begin{theorem}
    The Kleisli adjunction \( F_{\mathbb T} \dashv G_{\mathbb T} \) is initial in \( \operatorname{Adj}(\mathbb T) \), and the Eilenberg--Moore adjunction \( F^{\mathbb T} \dashv G^{\mathbb T} \) is terminal in \( \operatorname{Adj}(\mathbb T) \).
\end{theorem}
\begin{proof}
    We will first do the case of the Eilenberg--Moore adjunction.
    Let \( F \dashv G \) be an adjunction inducing \( \mathbb T \).
    We define \( K : \mathcal D \to \mathcal C^{\mathbb T} \) by \( KB = (GB, G\epsilon_B) \).
    This is an algebra by the triangular identities and naturality of \( \epsilon \).
    On morphisms \( f : B \to C \) in \( \mathcal D \), we define \( Kg = Gg \), which is a homomorphism as \( \epsilon \) is a natural transformation.
    Clearly \( G^{\mathbb T}K = G \), and \( KFA = (GFA, G\epsilon_{FA}) = F^{\mathbb T}A \), and for \( f : A \to A' \), \( KFf = GFf = Tf = F^{\mathbb T} f \).
    So \( K \) is a morphism of \( \operatorname{Adj}(\mathbb T) \).

    For uniqueness, suppose \( K' \) were another such morphism.
    Then \( K'B = (GB, \beta_B) \), and \( K'g = Gg \) for \( g : B \to C \).
    Note that \( \beta \) must be a natural transformation \( GFG \to G \).
    Also, \( \beta_{FA} = G\epsilon_{FA} \) for all \( A \), as \( K'F = F^{\mathbb T} \).
    But we have naturality squares
    % https://q.uiver.app/#q=WzAsNCxbMCwwLCJHRkdGR0IiXSxbMCwxLCJHRkdCIl0sWzEsMCwiR0ZHQiJdLFsxLDEsIkdCIl0sWzAsMSwiXFxiZXRhX3tGR0J9IiwyLHsib2Zmc2V0IjoyfV0sWzAsMiwiR0ZHXFxlcHNpbG9uX0IiXSxbMSwzLCJHXFxlcHNpbG9uX0IiLDJdLFswLDEsIkdcXGVwc2lsb25fe0ZHQn0iLDAseyJvZmZzZXQiOi0yfV0sWzIsMywiXFxiZXRhX0IiLDIseyJvZmZzZXQiOjJ9XSxbMiwzLCJHXFxlcHNpbG9uX0IiLDAseyJvZmZzZXQiOi0yfV1d
\[\begin{tikzcd}[column sep=large]
	GFGFGB & GFGB \\
	GFGB & GB
	\arrow["{\beta_{FGB}}"', shift right=2, from=1-1, to=2-1]
	\arrow["{GFG\epsilon_B}", from=1-1, to=1-2]
	\arrow["{G\epsilon_B}"', from=2-1, to=2-2]
	\arrow["{G\epsilon_{FGB}}", shift left=2, from=1-1, to=2-1]
	\arrow["{\beta_B}"', shift right=2, from=1-2, to=2-2]
	\arrow["{G\epsilon_B}", shift left=2, from=1-2, to=2-2]
\end{tikzcd}\]
    where the left edges are equal and the top edge is a split epimorphism, so the right edges are equal.
    Thus \( K \) is unique.
\end{proof}
