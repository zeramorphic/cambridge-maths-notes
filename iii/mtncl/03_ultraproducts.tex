\subsection{Products}
We will use the symbol \( \lambda \) to define functions without giving them explicit names.
The syntax \( \lambda x.\, y \) represents the function \( f \) such that \( f(x) = y \).

Let \( \qty{\mathcal M_i}_{i \in I} \) be a set of \( \mathcal L \)-structures.
The \emph{product} \( \prod_{i \in I} \mathcal M_i \) of this family is the \( \mathcal L \)-structure with carrier set
\[ \prod_{i \in I} \mathcal M_i = \qty{\alpha : I \to \bigcup M_i \midd \alpha(i) \in \mathcal M_i} \]
such that
\begin{itemize}
    \item an \( n \)-ary function symbol \( f \) is interpreted as
    \[ f^{\prod_I \mathcal M_i} : \qty(\prod_I \mathcal M_i)^n \to \prod_I \mathcal M_i \]
    given by
    \[ (\alpha_1, \dots, \alpha_n) \mapsto \lambda i.\, f^{\mathcal M_i}(\alpha_1(i), \dots, \alpha_n(i)) \]
    \item an \( n \)-ary relation symbol \( R \) is interpreted as the subset
    \[ R^{\prod_I \mathcal M_i} \subseteq \qty(\prod_I \mathcal M_i)^n \]
    given by
    \[ R^{\prod_I \mathcal M_i} = \qty{(\alpha_1, \dots, \alpha_n) \in \qty(\prod_I \mathcal M_i)^n \midd \forall i \in I.\, (\alpha_1(i), \dots, \alpha_n(i)) \in R^{\mathcal M_i}} \]
\end{itemize}
The relation symbols in this kind of product are not particularly useful.
We want to construct a different kind of product in such a way that \( \varphi \) holds in the product if the set of \( \mathcal M_i \) that model \( \varphi \) is `large'.

\subsection{Lattices}
\begin{definition}
    A \emph{lattice} is a set \( L \) equipped with binary operations \( \wedge \) and \( \vee \) that are associative and commutative, and satisfy the \emph{absorption laws}
    \[ a \vee (a \wedge b) = a;\quad a \wedge (a \vee b) = a \]
    A lattice is called
    \begin{itemize}
        \item \emph{distributive}, if \( a \wedge (b \vee c) = (a \wedge b) \vee (a \wedge c) \);
        \item \emph{bounded}, if there are elements \( \bot \) and \( \top \) such that \( a \vee \bot = a \) and \( a \wedge \top = a \);
        \item \emph{complemented}, if it is bounded and for each \( a \in L \) there exists \( a^\star \in L \) called its \emph{complement} such that \( a \wedge a^\star = \bot \) and \( a \vee a^\star = \top \);
        \item a \emph{Boolean algebra}, if it is distributive, bounded, and complemented.
    \end{itemize}
\end{definition}
\begin{remark}
    \begin{enumerate}
        \item Distributive lattices model the fragment of a deduction system with only the conjunction and disjunction operators.
        Boolean algebras model classical propositional logic.
        \item Every lattice has an ordering, defined by \( a \leq b \) when \( a \wedge b = a \).
        This ordering models the provability relation between propositions.
    \end{enumerate}
\end{remark}
\begin{example}
    \begin{enumerate}
        \item Let \( I \) be a set.
        The power set \( \mathcal P(I) \) can be made into a Boolean algebra by taking \( \wedge = \cap \) and \( \vee = \cup \).
        \item More generally, let \( X \) be a topological space.
        The set of closed and open sets of \( X \) form a Boolean algebra; they can also be thought of as the propositions in classical logic.
        In fact, all Boolean algebras are of this form.
        This result is known as Stone's representation theorem.
        \item For any \( \mathcal L \)-structure \( \mathcal M \) and subset \( B \subseteq \mathcal M \), the set \( \qty{\varphi(\mathcal M) \mid \varphi(\vb x) \in \mathcal L_B} \) of definable subsets with parameters in \( B \) is a Boolean algebra.
    \end{enumerate}
\end{example}

\subsection{Filters}
\begin{definition}
    Let \( X \) be a lattice.
    A \emph{filter} \( \mathcal F \) on \( X \) is a subset of \( X \) such that
    \begin{enumerate}
        \item \( \mathcal F \neq \varnothing \);
        \item \( \mathcal F \) is \emph{upward closed}: if \( f \leq x \) and \( f \in \mathcal F \) then \( x \in \mathcal F \);
        \item \( \mathcal F \) is \emph{downward directed}: if \( x, y \in \mathcal F \), then \( x \wedge y \in \mathcal F \).
    \end{enumerate}
\end{definition}
For property (ii), we might also say that \( \mathcal F \) is a \emph{terminal segment} of \( X \).
% large things get "stuck in the filter"; filters measure largeness
