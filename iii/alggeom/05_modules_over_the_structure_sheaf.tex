\subsection{Definitions}
\begin{example}
    Let \( \mathbb CP^n \) be the variety \( \mathbb C^{n+1} \setminus \qty{0} \) modulo scaling by \( \mathbb C \).
    We have a structure sheaf \( \mathcal O_{\mathbb CP^n} \), where if \( U \subseteq \mathbb CP^n \) is Zariski open, we define
    \[ \mathcal O_{\mathbb CP^n}(U) = \qty{\frac{P(\vb x)}{Q(\vb x)} \midd P, Q \text{ homogeneous of the same degree, and the ratio is regular at all } p \in U} \]
    For any integer \( d \), we can consider a sheaf \( \mathcal O_{\mathbb CP^n}(d) \) given by
    \[ \mathcal O_{\mathbb CP^n}(d)(U) = \qty{\frac{P(\vb x)}{Q(\vb x)} \midd P, Q \text{ homogeneous, } \deg P - \deg Q = d\text{, and regular at all } p \in U} \]
    This is a sheaf of groups, but not a sheaf of rings as it is not closed under multiplication for \( d \neq 0 \).
    Note that \( \mathcal O_{\mathbb CP^n}(d)(U) \) is a module over \( \mathcal O_{\mathbb CP^n}(U) \), and the multiplication commutes with restriction.
\end{example}
\begin{example}
    Let \( A \) be a ring, and let \( M \) be an \( A \)-module.
    We define the sheaf \( \mathcal F_M \) on \( \Spec A \) as follows.
    If \( U \subseteq \Spec A \) is a distinguished open \( U = U_f \), then we set
    \[ \mathcal F_M(U) = M_f \]
    which is the module \( M \) localised at \( f \).
    This defines a sheaf on a base, and hence extends to a unique sheaf on \( \Spec A \).
\end{example}
\begin{definition}
    Let \( (X, \mathcal O_X) \) be a ringed space.
    A \emph{sheaf of \( \mathcal O_X \)-modules} on \( X \) is a sheaf \( \mathcal F \) of abelian groups together with a multiplication \( \mathcal F(U) \times \mathcal O_X(U) \to \mathcal F(U) \) that makes \( \mathcal F(U) \) into an \( \mathcal O_X(U) \)-module, that is compatible with restriction.
    % https://q.uiver.app/#q=WzAsNCxbMCwwLCJcXG1hdGhjYWwgRihWKSBcXHRpbWVzIFxcbWF0aGNhbCBPX1goVikiXSxbMSwwLCJcXG1hdGhjYWwgRihWKSJdLFsxLDEsIlxcbWF0aGNhbCBGKFUpIl0sWzAsMSwiXFxtYXRoY2FsIEYoVSkgXFx0aW1lcyBcXG1hdGhjYWwgT19YKFUpIl0sWzAsMV0sWzEsMl0sWzAsM10sWzMsMl1d
\[\begin{tikzcd}
	{\mathcal F(V) \times \mathcal O_X(V)} & {\mathcal F(V)} \\
	{\mathcal F(U) \times \mathcal O_X(U)} & {\mathcal F(U)}
	\arrow[from=1-1, to=1-2]
	\arrow[from=1-2, to=2-2]
	\arrow[from=1-1, to=2-1]
	\arrow[from=2-1, to=2-2]
\end{tikzcd}\]
\end{definition}
Similarly, we can define a sheaf of \( \mathcal O_X \)-algebras.
A morphism between sheaves of modules \( \varphi : \mathcal F \to \mathcal G \) on \( X \) is a homomorphism of sheaves of abelian groups that is compatible with multiplication.
