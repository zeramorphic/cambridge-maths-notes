\subsection{Morphisms of ringed spaces}
Let \( (X, \mathcal O_X) \) be a scheme.
The stalks \( \mathcal O_{X,\mathfrak p} \) are local rings: they have a unique maximal ideal, which is the set of all non-unit elements.
Given \( f \in \mathcal O_X(U) \), we can meaningfully ask whether \( f \) vanishes at \( \mathfrak p \); that is, if the image of \( f \) in \( \mathcal O_{X, \mathfrak p} \) is contained in the maximal ideal.
\begin{definition}
    A morphism of ringed spaces \( f : (X, \mathcal O_X) \to (Y, \mathcal O_Y) \) consists of a continuous function \( f : X \to Y \) and a morphism \( f^\sharp : \mathcal O_Y \to f_\star \mathcal O_X \) between sheaves of rings on \( Y \).
\end{definition}
\( f^\sharp \) represents function composition with \( f^{-1} \), although the ring \( \mathcal O_X \) may not be a ring of functions.
It is possible to find a morphism \( (f, f^\sharp) \) between schemes \( (X, \mathcal O_X) \) and \( (Y, \mathcal O_Y) \) such that there exists \( q \in U \subseteq Y \) and \( h \in \mathcal O_Y(U) \) such that \( h \) vanishes at \( q \) but \( f^\sharp(h) \in \mathcal O_X(f^{-1}(U)) \) does not vanish at some \( p \in X \) with \( f(p) = q \).
This motivates the definition of a morphism of schemes.

Let \( f : X \to Y \) be a morphism of ringed spaces.
Given any point \( p \in X \), there is an induced map \( f^\sharp : \mathcal O_{Y,f(p)} \to \mathcal O_{X,p} \).
Explicitly, given \( s \in \mathcal O_{Y,f(p)} \), we can represent it by \( (s_U, U) \) where \( U \) is open, \( f(p) \in U \), and \( s_U \in \mathcal O_Y(U) \).
Now, \( f^\sharp(s_U) \in \mathcal O_X(f^{-1}(U)) \), so the pair \( (f^\sharp(s_U), f^{-1}(U)) \) defines an element of \( \mathcal O_{X,p} \).
\begin{definition}
    A ringed space \( (X, \mathcal O_X) \) is called a \emph{locally ringed space} if for all \( p \in X \), the stalk \( \mathcal O_{X,p} \) is is a local ring.
    A morphism of locally ringed spaces \( (f, f^\sharp) : (X, \mathcal O_X) \to (Y, \mathcal O_Y) \) is a morphism of ringed spaces such that if \( \mathfrak m_p \) denotes the maximal ideal in \( \mathcal O_{X,p} \), then \( f^\sharp(\mathfrak m_{f(p)}) \subseteq \mathfrak m_p \).
\end{definition}
This encapsulates the idea that functions vanishing on the codomain must also vanish on the domain after the inverse image, as the maximal ideal represents functions vanishing at the point.

\subsection{Morphisms of schemes}
Note that all schemes are locally ringed spaces.
\begin{definition}
    A \emph{morphism of schemes} \( X \to Y \) is a morphism of locally ringed spaces \( X \to Y \).
\end{definition}
\begin{theorem}
    There is a natural bijection
    \[ \qty{\text{morphisms of schemes } \Spec B \to \Spec A} \leftrightarrow \qty{\text{homomorphisms of rings } A \to B} \]
\end{theorem}
\begin{proof}
    First, recall that a section \( s \) of a sheaf \( \mathcal F \) on \( U \) is a coherent collection of elements of the stalks \( s(p) \in \mathcal F_p \) for all \( p \in U \).
    We will construct a map of schemes \( \Spec B \to \Spec A \) for every ring homomorphism \( A \to B \), and then show that every morphism of schemes arises in this way.

    Let \( \varphi : A \to B \) be a ring homomorphism.
    Let \( \varphi^{-1} : \Spec B \to \Spec A \) be the map of topological spaces; this is a continuous function.
    We now build
    \[ \varphi^\sharp : \mathcal O_{\Spec A} \to \varphi_\star^{-1} \mathcal O_{\Spec B} \]
    At the level of stalks, the map \( A_{\varphi^{-1}(\mathfrak p)} \to B_{\mathfrak p} \) is induced by \( \varphi \) by mapping \( \frac{a}{s} \) to \( \frac{\varphi(a)}{\varphi(s)} \).
    This is well-defined, as for \( s \notin \varphi^{-1}(\mathfrak p) \), then \( \varphi(s) \notin \mathfrak p \).
    Observe that this is automatically a local homomorphism.
    
    We must now show that this choice of maps on stalks extends to a map between sheaves.
    Given \( U \subseteq \Spec A \), we need to define
    \[ \varphi^\sharp : \mathcal O_{\Spec A}(U) \to \mathcal O_{\Spec B}((\varphi^{-1})^{-1}(U)) \]
    An element \( s \in \mathcal O_{\Spec A}(U) \) is a collection of assignments \( (\mathfrak p \mapsto s(\mathfrak p))_{\mathfrak p \in U} \) for \( \mathfrak p \in U \) and \( s(\mathfrak p) \in A_{\mathfrak p} \).
    We then define \( \varphi^\sharp \) by
    \[ (\mathfrak p \mapsto s(\mathfrak p))_{\mathfrak p \in U} \mapsto (\mathfrak q \mapsto \varphi_{\mathfrak q}(s(\varphi^{-1}(\mathfrak q))))_{\mathfrak q \in (\varphi^{-1})^{-1}(U)} \]
    One can check that the gluing conditions are satisfied.

    Conversely, suppose \( (f, f^\sharp) : \Spec B \to \Spec A \) is a morphism of schemes.
    Using the fact that we have a map of global sections \( \mathcal O_{\Spec A}(\Spec A) \to \mathcal O_{\Spec B}(\Spec B) \), we obtain a ring homomorphism \( g : A \to B \).
    We must check that \( g^{-1} : \Spec B \to \Spec A \) gives the correct map \( f \) on topological spaces, and that the construction above yields the correct map \( f^\sharp \) on sheaves.
    The maps on stalks are compatible with restriction, so the following diagram commutes for all \( \mathfrak p \in \Spec B \).
    % https://q.uiver.app/#q=WzAsNCxbMCwwLCJcXEdhbW1hKFxcU3BlYyBBLCBcXG1hdGhjYWwgT197XFxTcGVjIEF9KSJdLFsxLDAsIlxcR2FtbWEoXFxTcGVjIEIsIFxcbWF0aGNhbCBPX3tcXFNwZWMgQn0pIl0sWzEsMSwiXFxtYXRoY2FsIE9fe1xcU3BlYyBCLCBcXG1hdGhmcmFrIHB9Il0sWzAsMSwiXFxtYXRoY2FsIE9fe1xcU3BlYyBBLCBmKFxcbWF0aGZyYWsgcCl9Il0sWzAsMV0sWzEsMl0sWzAsM10sWzMsMl1d
\[\begin{tikzcd}
	{\Gamma(\Spec A, \mathcal O_{\Spec A})} & {\Gamma(\Spec B, \mathcal O_{\Spec B})} \\
	{\mathcal O_{\Spec A, f(\mathfrak p)}} & {\mathcal O_{\Spec B, \mathfrak p}}
	\arrow[from=1-1, to=1-2]
	\arrow[from=1-2, to=2-2]
	\arrow[from=1-1, to=2-1]
	\arrow[from=2-1, to=2-2]
\end{tikzcd}\]
    Equivalently, the following diagram commutes for all \( \mathfrak p \in \Spec B \).
    % https://q.uiver.app/#q=WzAsNCxbMCwwLCJBIl0sWzEsMCwiQiJdLFsxLDEsIkJfe1xcbWF0aGZyYWsgcH0iXSxbMCwxLCJBX3tmKFxcbWF0aGZyYWsgcCl9Il0sWzAsMV0sWzEsMl0sWzAsM10sWzMsMl1d
\[\begin{tikzcd}
	A & B \\
	{A_{f(\mathfrak p)}} & {B_{\mathfrak p}}
	\arrow[from=1-1, to=1-2]
	\arrow[from=1-2, to=2-2]
	\arrow[from=1-1, to=2-1]
	\arrow[from=2-1, to=2-2]
\end{tikzcd}\]
    Since the morphism is local, \( (f^\sharp)^{-1}(\mathfrak p B_{\mathfrak p}) = f(\mathfrak p) A_{f(\mathfrak p)} \).
    As the above diagram commutes, \( g^{-1} = f \) as maps of topological spaces, and the maps of structure sheaves agree at the level of stalks by construction so they must agree everywhere.
\end{proof}

\subsection{Immersions}
\begin{definition}
    Let \( X, Y \) be schemes.
    A morphism of schemes \( f : X \to Y \) is an \emph{open immersion} if \( f \) induces an isomorphism of \( X \) onto an open subscheme \( \qty(U, \eval{\mathcal O_Y}_U) \) of \( Y \).
    A morphism \( f : X \to Y \) is a \emph{closed immersion} if \( f \) is a homeomorphism onto a closed subset of \( Y \), and \( g^\sharp : \mathcal O_Y \to g_\star \mathcal O_X \) is surjective.
\end{definition}
\begin{example}
    Let \( k[t] \to \faktor{k[t]}{(t^2)} \).
    The induced map \( \Spec \faktor{k[t]}{(t^2)} \to \Spec k[t] \) is a closed immersion.
    More generally, let \( A \) be a ring and \( I \) be an ideal in \( A \).
    Then the induced map \( \Spec \faktor{A}{I} \to \Spec A \) is a closed immersion.
\end{example}
\begin{definition}
    Let \( Y \) be a scheme.
    A \emph{closed subscheme} of \( Y \) is an equivalence class of closed immersions \( X \to Y \), where we say \( f : X \to Y \) and \( f' : X' \to Y \) are equivalent if there is a commutative triangle
    % https://q.uiver.app/#q=WzAsMyxbMCwwLCJYIl0sWzIsMCwiWCciXSxbMSwxLCJZIl0sWzAsMSwiXFxzaW0iLDAseyJzdHlsZSI6eyJ0YWlsIjp7Im5hbWUiOiJhcnJvd2hlYWQifX19XSxbMSwyLCJmJyJdLFswLDIsImYiLDJdXQ==
\[\begin{tikzcd}
	X && {X'} \\
	& Y
	\arrow["\sim", tail reversed, from=1-1, to=1-3]
	\arrow["{f'}", from=1-3, to=2-2]
	\arrow["f"', from=1-1, to=2-2]
\end{tikzcd}\]
\end{definition}

\subsection{Fibre products}
The notion of fibre product will simultaneously generalise the notions of product, intersections of closed subschemes, and inverse images of subschemes (such as points) along morphisms.
\begin{definition}
    Consider a diagram
    % https://q.uiver.app/#q=WzAsMyxbMSwwLCJYIl0sWzEsMSwiUyJdLFswLDEsIlkiXSxbMCwxXSxbMiwxXV0=
\[\begin{tikzcd}
	& X \\
	Y & S
	\arrow[from=1-2, to=2-2]
	\arrow[from=2-1, to=2-2]
\end{tikzcd}\]
    The \emph{fibre product} is a scheme \( X \times_S Y \) making the following diagram commute:
    % https://q.uiver.app/#q=WzAsNCxbMSwwLCJYIl0sWzEsMSwiUyJdLFswLDEsIlkiXSxbMCwwLCJYIFxcdGltZXNfUyBZIl0sWzAsMV0sWzIsMV0sWzMsMiwicF9ZIiwyXSxbMywwLCJwX1giXV0=
\[\begin{tikzcd}
	{X \times_S Y} & X \\
	Y & S
	\arrow[from=1-2, to=2-2]
	\arrow[from=2-1, to=2-2]
	\arrow["{p_Y}"', from=1-1, to=2-1]
	\arrow["{p_X}", from=1-1, to=1-2]
\end{tikzcd}\]
    such that for any other scheme \( Z \) together with morphisms \( q_X, q_Y \) completing the square, there is a unique factorisation through \( X \times_S Y \), making the following diagram commute.
    % https://q.uiver.app/#q=WzAsNSxbMiwxLCJYIl0sWzIsMiwiUyJdLFsxLDIsIlkiXSxbMSwxLCJYIFxcdGltZXNfUyBZIl0sWzAsMCwiWiJdLFswLDFdLFsyLDFdLFszLDIsInBfWSJdLFszLDAsInBfWCIsMl0sWzQsMCwicV9YIiwwLHsiY3VydmUiOi0yfV0sWzQsMiwicV9ZIiwyLHsiY3VydmUiOjJ9XSxbNCwzLCIiLDEseyJzdHlsZSI6eyJib2R5Ijp7Im5hbWUiOiJkYXNoZWQifX19XV0=
\[\begin{tikzcd}
	Z \\
	& {X \times_S Y} & X \\
	& Y & S
	\arrow[from=2-3, to=3-3]
	\arrow[from=3-2, to=3-3]
	\arrow["{p_Y}", from=2-2, to=3-2]
	\arrow["{p_X}"', from=2-2, to=2-3]
	\arrow["{q_X}", curve={height=-12pt}, from=1-1, to=2-3]
	\arrow["{q_Y}"', curve={height=12pt}, from=1-1, to=3-2]
	\arrow[dashed, from=1-1, to=2-2]
\end{tikzcd}\]
\end{definition}
Note that as this is a definition by universal property, if \( X \times_S Y \) exists, it is unique up to unique isomorphism.
The fibre product is schemes is the category-theoretic \emph{pullback}.
\begin{example}
    \begin{enumerate}
        \item In the category of sets, the fibre product of the diagram
        % https://q.uiver.app/#q=WzAsMyxbMSwwLCJYIl0sWzEsMSwiUyJdLFswLDEsIlkiXSxbMCwxLCJyX1giXSxbMiwxLCJyX1kiLDJdXQ==
    \[\begin{tikzcd}
        & X \\
        Y & S
        \arrow["{r_X}", from=1-2, to=2-2]
        \arrow["{r_Y}"', from=2-1, to=2-2]
    \end{tikzcd}\]
        is the set
        \[ X \times_S Y = \qty{(x, y) \in X \times Y \mid r_X(x) = r_Y(y)} \]
        \item In the category of topological spaces, the fibre product is defined to be the same set, assigning \( X \times_S Y \) the subspace topology as a subset of \( X \times Y \).
        \item Let \( r_X : X \to S \) be a map of sets, and let \( Y = \qty{\star} \) with \( r_Y(\star) = s \in S \).
        Then
        \[ X \times_S Y = r_X^{-1}(s) \]
        \item Let \( r_X : X \to S \) and \( r_Y : Y \to S \) be inclusions of subsets.
        Then
        \[ X \times_S Y = X \cap Y \]
    \end{enumerate}
\end{example}
\begin{theorem}
    Fibre products of schemes exist.
\end{theorem}
\begin{proof}[Proof sketch]
    \emph{Step 1.}
    Let \( X, Y, S \) be affine schemes, with associated rings \( A, B, R \).
    Then the fibre product \( X \times_S Y \) exists, and is isomorphic to \( \Spec (A \otimes_R B) \).
    Note that the tensor product is the category-theoretic pushout in the category of rings.
    We must now check that the universal property of the fibre product is satisfied.
    Consider the commutative square
    % https://q.uiver.app/#q=WzAsNCxbMCwwLCJaIl0sWzEsMCwiWCJdLFsxLDEsIlMiXSxbMCwxLCJZIl0sWzAsMV0sWzEsMl0sWzAsM10sWzMsMl1d
\[\begin{tikzcd}
	Z & X \\
	Y & S
	\arrow[from=1-1, to=1-2]
	\arrow[from=1-2, to=2-2]
	\arrow[from=1-1, to=2-1]
	\arrow[from=2-1, to=2-2]
\end{tikzcd}\]
    If \( Z \) is an affine scheme, the result holds.
    It is a general fact that a map of schemes \( Z \to \Spec (A \otimes_R B) \) is the same data as a map \( A \otimes_R B \to \Gamma(Z, \mathcal O_Z) \).

    \emph{Step 2.}
    Let \( X, Y, S \) be arbitrary schemes.
    If \( X \times_S Y \) exists and \( U \subseteq X \) is an open subscheme, then \( U \times_S Y \) also exists, by taking the inverse image of \( U \) under the projection \( X \times_S Y \to X \) endowed with the structure of an open subscheme.

    \emph{Step 3.}
    If \( X \) is covered by open subschemes \( \qty{X_i} \), then if \( X_i \times_S Y \) exists for all \( i \), then \( X \times_S Y \) exists, by gluing each of the \( X_i \times_S Y \) together.
    Note that the ability to glue these schemes together relies on Step 2, and the fact that there is no cocycle condition.

    \emph{Step 4.}
    If \( Y \) and \( S \) are affine, then \( X \times_S Y \) exists by Step 3, by covering \( X \) by affine subschemes.
    As \( X \) and \( Y \) are interchangeable, \( X \times_S Y \) exists for any \( X \) and \( Y \) as long as \( S \) is affine.

    \emph{Step 5.}
    Now, cover \( S \) by affine subschemes \( \qty{S_i} \).
    Let \( X_i, Y_i \) be the preimages of of \( S_i \) in \( X \) and \( Y \) respectively.
    Now, \( X_i \times_{S_i} Y_i \) exists.
    Observe by the universal property that \( X_i \times_{S_i} Y_i = X_i \times S Y_i \).
    Finally, gluing gives \( X \times_S Y \) as required. 
\end{proof}
\begin{example}
    \begin{enumerate}
        \item We have
        \[ \mathbb P^n_{\mathbb C} = \mathbb P^n_{\mathbb Z} \times_{\Spec \mathbb Z} \Spec \mathbb C \]
        where the map \( \Spec \mathbb C \to \Spec \mathbb Z \) is induced by the ring homomorphism \( \mathbb Z \to \mathbb C \), and the map \( \mathbb P^n_{\mathbb Z} \to \Spec \mathbb Z \) is induced locally by the inclusion \( \mathbb Z \to \mathbb Z\qty[\frac{x_0}{x_i}, \dots, \frac{x_n}{x_i}] \).
        Note also that
        \[ \mathbb Z[\vb x] \otimes_{\mathbb Z} \mathbb C = \mathbb C[\vb x] \]
        \item Let \( C = \Spec \faktor{\mathbb C[x, y]}{(y - x^2)} \) and \( L = \Spec \faktor{\mathbb C[x, y]}{(y)} \).
        We have natural closed immersions \( C \to \mathbb A^2_{\mathbb C} \) and \( L \to \mathbb A^2_{\mathbb C} \).
        One can show that
        \[ C \times_{\mathbb A^2_{\mathbb C}} L = \Spec \faktor{\mathbb C[x]}{(x^2)} \]
        representing the intersection.
    \end{enumerate}
\end{example}

\subsection{Schemes over a base}
In scheme theory, we often fix a scheme \( S \) called the \emph{base scheme}, and consider other schemes with a fixed map to \( S \).
These form a category of schemes \emph{over \( S \)}, where the morphisms are the morphisms of schemes \( f : X \to Y \) such that the following diagram commutes.
% https://q.uiver.app/#q=WzAsMyxbMCwwLCJYIl0sWzEsMSwiUyJdLFsyLDAsIlkiXSxbMCwxXSxbMCwyLCJmIl0sWzIsMV1d
\[\begin{tikzcd}
	X && Y \\
	& S
	\arrow[from=1-1, to=2-2]
	\arrow["f", from=1-1, to=1-3]
	\arrow[from=1-3, to=2-2]
\end{tikzcd}\]
This is known as Grothendieck's \emph{relative point of view}.
Typically, \( S \) is the spectrum of a field or a ring.
Note that every scheme has a unique morphism to \( \Spec \mathbb Z \), so the category of schemes is isomorphic to the category of schemes over \( \Spec \mathbb Z \).
The product of \( X \) and \( Y \) in the category of schemes over \( S \) is the fibre product \( X \times_S Y \).
Analogously, in commutative algebra, we often consider algebras of a fixed ring, and the category of rings is isomorphic to the category of \( \mathbb Z \)-algebras.

% TODO: Define reduced, integral, irreducible, noetherian... from ES2/Hartshorne

\subsection{Separatedness}
Recall that a topological space \( X \) is Hausdorff if and only if the diagonal \( \Delta_X \subseteq X \times X \) is closed.
\begin{definition}
    Let \( X \to S \) be a morphism of schemes.
    Then the \emph{diagonal} is the morphism \( \Delta_{X/S} : X \to X \times_S X \) induced using the universal property by the following diagram.
    % https://q.uiver.app/#q=WzAsNSxbMCwwLCJYIl0sWzEsMSwiWCBcXHRpbWVzX1MgWCJdLFsyLDEsIlgiXSxbMiwyLCJTIl0sWzEsMiwiWCJdLFswLDEsIiIsMCx7InN0eWxlIjp7ImJvZHkiOnsibmFtZSI6ImRhc2hlZCJ9fX1dLFsxLDJdLFsyLDNdLFsxLDRdLFs0LDNdLFswLDIsIlxcaWRfWCIsMCx7ImN1cnZlIjotMn1dLFswLDQsIlxcaWRfWCIsMix7ImN1cnZlIjoyfV1d
\[\begin{tikzcd}
	X \\
	& {X \times_S X} & X \\
	& X & S
	\arrow[dashed, from=1-1, to=2-2]
	\arrow[from=2-2, to=2-3]
	\arrow[from=2-3, to=3-3]
	\arrow[from=2-2, to=3-2]
	\arrow[from=3-2, to=3-3]
	\arrow["{\id_X}", curve={height=-12pt}, from=1-1, to=2-3]
	\arrow["{\id_X}"', curve={height=12pt}, from=1-1, to=3-2]
\end{tikzcd}\]
\end{definition}
We write \( \Delta \) for \( \Delta_{X/S} \) if \( X \) and \( S \) are clear from context.
\begin{remark}
    If \( U, V \) are open subschemes of \( X \) and \( S = \Spec k \) for a field \( k \), then
    \[ \Delta^{-1}(U \times_S V) = U \cap V \]
\end{remark}
\begin{definition}
    A morphism \( X \to S \) is \emph{separated} if \( \Delta_{X/S} : X \to X \times_S X \) is a closed immersion.
\end{definition}
\begin{example}
    Let \( X = \Spec \mathbb C[t] \), let \( S = \Spec \mathbb C \), and induce the map \( X \to S \) by the \( \mathbb C \)-algebra homomorphism \( \mathbb C \to \mathbb C[t] \).
    Then
    \[ X \times_S X = \Spec(\mathbb C[t] \otimes_{\mathbb C} \mathbb C[t]) \]
    and the diagonal map \( \Delta \) is induced by the multiplication map
    \[ \mathbb C[t] \otimes_{\mathbb C} \mathbb C[t] \to \mathbb C[t] \]
    Note that \( \Delta \) is closed, as the map \( \mathbb C[t] \otimes_{\mathbb C} \mathbb C[t] \to \mathbb C[t] \) is surjective.
\end{example}
\begin{proposition}
    Let \( g : X \to S \) be a morphism of schemes.
    Then there is a factorisation of \( \Delta_{X/S} \) as follows.
    % https://q.uiver.app/#q=WzAsMyxbMCwxLCJYIl0sWzEsMCwiVSJdLFsyLDEsIlggXFx0aW1lc19TIFgiXSxbMCwxLCJcXHRleHR7Y2xvc2VkIGltbWVyc2lvbn0iXSxbMSwyLCJcXHRleHR7b3BlbiBpbW1lcnNpb259Il0sWzAsMiwiXFxEZWx0YV97WC9TfSIsMl1d
\[\begin{tikzcd}
	& U \\
	X && {X \times_S X}
	\arrow["{\text{closed immersion}}", from=2-1, to=1-2]
	\arrow["{\text{open immersion}}", from=1-2, to=2-3]
	\arrow["{\Delta_{X/S}}"', from=2-1, to=2-3]
\end{tikzcd}\]
    We say that \( g : X \to S \) is a \emph{locally closed immersion}.
\end{proposition}
\begin{proof}
    Let \( S \) be covered by open affine subschemes \( \qty{V_i} \), and suppose \( X \) is covered by open affine subschemes \( \qty{U_{ij}} \), where for some fixed \( i \), the \( U_{ij} \) cover \( g^{-1}(V_i) \).
    We have morphisms \( U_{ij} \to V_i \) induced by
    % https://q.uiver.app/#q=WzAsNSxbMCwwLCJVX3tpan0iXSxbMSwwLCJnXnstMX0oVl9pKSJdLFsyLDAsIlZfaSJdLFsyLDEsIlMiXSxbMSwxLCJYIl0sWzAsMV0sWzEsMl0sWzIsM10sWzEsNF0sWzQsM11d
\[\begin{tikzcd}
	{U_{ij}} & {g^{-1}(V_i)} & {V_i} \\
	& X & S
	\arrow[from=1-1, to=1-2]
	\arrow[from=1-2, to=1-3]
	\arrow[from=1-3, to=2-3]
	\arrow[from=1-2, to=2-2]
	\arrow[from=2-2, to=2-3]
\end{tikzcd}\]
    where the commutative square is a fibre product.
    Observe that \( U_{ij} \times_{V_i} U_{ij} \) is affine and open in \( X \times_S X \), and their union contains the image of the diagonal \( \Delta_{X/S} \).
    Also,
    \[ \Delta^{-1}(U_{ij} \times_{V_i} U_{ij}) = U_{ij} \subseteq X \]
    Let \( U \) be the union of the \( U_{ij} \times_{V_i} U_{ij} \) over all \( i, j \).
    Then the second map in the statement is clearly an open immersion.
    Observe that to check if \( f : T \to T' \) is a closed immersion, it suffices to check locally on the codomain.
    For each \( U_{ij} \), the diagonal is a map \( U_{ij} \to U_{ij} \times_{V_i} U_{ij} \), which one can show is a closed immersion.
\end{proof}
\begin{proposition}
    If \( X \to S \) is a morphism of affine schemes, then \( \Delta_{X/S} \) is a closed immersion.
\end{proposition}
\begin{proof}
    Let \( X = \Spec A, S = \Spec B \), and let the map \( X \to S \) be given by a map \( B \to A \).
    Then the map \( A \otimes_B A \to A \) is surjective as required.
\end{proof}
Thus every morphism of affine schemes is separated.
\begin{example}
    \begin{enumerate}
        \item Recall the bug-eyed line
        \[ \faktor{\mathbb A^1_k \sqcup \mathbb A^1_k} {\sim} \]
        where if \( U = \mathbb A^1_k \setminus \qty{0} \subseteq \mathbb A^1_k \) and \( V \) is defined similarly, we define the isomorphism \( V \to U \) by the map \( u \mapsto t : k[u,u^{-1}] \to k[t,t^{-1}] \).
        We claim that the bug-eyed line is not separated over \( \Spec k \).
        We can compute \( X \times_S X \) by the gluing construction of the fibre product.
        This is a plane with doubled axes and four origins.
        The diagonal only contains two of the four origins, and this is not a closed subset.
        \item Open and closed immersions are are always separated.
    \end{enumerate}
\end{example}
\begin{corollary}
    Let \( X \to S \) be a morphism of schemes.
    If the image of \( \Delta_{X/S} \) is closed as a topological subspace, then \( X \to S \) is separated.
\end{corollary}
\begin{proof}
    A locally closed immersion onto a closed subset is a closed immersion.
\end{proof}
\begin{proposition}
    Let \( A \) be a ring.
    The morphism \( \mathbb P^n_A \to \Spec A \) is separated.
\end{proposition}
\begin{proposition}
    Let \( k \) be a field, and let \( X \to \Spec k \) be a morphism of schemes.
    Let \( U, V \subseteq X \) be affine open subschemes.
    If \( X \to \Spec k \) is separated, then \( U \cap V \) is also affine.
\end{proposition}
