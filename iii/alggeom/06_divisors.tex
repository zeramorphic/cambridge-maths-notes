\subsection{Height and dimension}
Recall that for a prime ideal \( \mathfrak p \) in \( R \), its \emph{height} is the largest \( n \) such that there exists a chain of inclusions of prime ideals
\[ \mathfrak p_0 \subsetneq \mathfrak p_1 \subsetneq \dots \subsetneq \mathfrak p_n = \mathfrak p \]
For example, if \( R \) is an integral domain, a prime ideal is of height 1 if and only if no nonzero prime ideal is strictly contained within it.
\begin{example}
    \begin{enumerate}
        \item In any integral domain, \( (0) \) has height 0.
        \item In \( \mathbb C[x, y] \), the ideal \( (x) \) has height 1, and the ideal \( (x, y) \) has height 2.
    \end{enumerate}
\end{example}
It can be shown that in a unique factorisation domain, every prime ideal of height 1 is principal.

We will globalise the notion of height 1 prime ideals, giving \emph{Weil divisors}, and also the notion of principal ideals, giving \emph{Cartier divisors}.
In the case of Weil divisors, we will assume that the ambient scheme \( X \) is Noetherian, integral, separated, and \emph{regular in codimension 1}.

If \( X \) is integral and \( U = \Spec A \) is an open affine, then the ideal \( (0) \subseteq A \) is called the \emph{generic point} of \( X \).
The generic points given by each \( U \) coincide in \( X \).
This point is often denoted by \( \eta \) or \( \eta_X \).
\begin{definition}
    Let \( X \) be a scheme.
    \begin{enumerate}
        \item The \emph{dimension} of \( X \) is the length \( n \) of the longest chain of nonempty closed irreducible subsets
        \[ Z_0 \subsetneq Z_1 \subsetneq \dots \subsetneq Z_n \]
        \item Let \( Z \subseteq X \) be closed and irreducible.
        The \emph{codimension} of \( X \) is the length \( n \) of the longest chain
        \[ Z = Z_0 \subsetneq Z_1 \subsetneq \dots \subsetneq Z_n \]
        \item If \( X \) is a \emph{Noetherian topological space}, so every decreasing sequence of closed subsets stabilises, then every closed \( Z \subseteq X \) has a decomposition into finitely many irreducible closed subsets.
        \item Suppose \( X \) is Noetherian, integral, and separated.
        We say that \( X \) is \emph{regular in codimension 1} if for every subspace \( Y \subseteq X \) that is closed, irreducible, and of codimension 1, if \( \eta_Y \) denotes the generic point of \( Y \), then \( \mathcal O_{X, \eta_Y} \) is a discrete valuation ring, or equivalently a local principal ideal domain.
    \end{enumerate}
\end{definition}

\subsection{Weil divisors}
\begin{definition}
    Let \( X \) be Noetherian, integral, separated, and regular in codimension 1.
    A \emph{prime divisor} on \( X \) is an integral closed subscheme of codimension 1.
    A \emph{Weil divisor} on \( X \) is an element of the free abelian group \( \operatorname{Div}(X) \) generated by the prime divisors.
\end{definition}
We will write \( D \in \operatorname{Div}(X) \) as \( \sum_i n_{Y_i} [Y_i] \) where the \( Y_i \) are prime divisors.
\begin{definition}
    A Weil divisor \( \sum_i n_{Y_i} [Y_i] \) is \emph{effective} if all \( n_{Y_i} \) are nonnegative.
\end{definition}
If \( X \) is integral, for \( \Spec A = U \subseteq X \), the local ring \( \mathcal O_{X, \eta} \) is a field, as it is in particular the fraction field of \( A \).
Indeed, because \( \eta \) is contained in every open affine, \( \mathcal O_{X, \eta} \) permits arbitrary denominators.

Let \( f \in \mathcal O_{X, \eta_X} = k(X) \).
Since for every prime divisor \( Y \subseteq X \), the ring \( \mathcal O_{X, \eta_Y} \) is a discrete valuation ring, we can calculate the valuation \( \nu_Y(f) \) of \( f \) in this ring.
We thus define the divisor
\[ \operatorname{div}(f) = \sum_{Y \subseteq X \text{ prime}} \nu_Y(f) [Y] \]
We claim that this is a Weil divisor; that is, the sum is finite.
