\subsection{Introduction}
The idea behind forcing is to widen a given model of \( \mathsf{ZFC} \) to `add lots of reals'.
But if we work over \( \mathrm{V} \), we already have added all of the sets, so there is nothing left to add.
Instead, we will work over countable transitive set models of \( \mathsf{ZFC} \).
However, this means that we will not immediately get \( \Con(\mathsf{ZF}) \to \Con(\mathsf{ZFC} + \neg\mathsf{CH}) \).
We will then use the reflection theorem to obtain this result.

If \( M \) is such a countable transitive model, we want to add \( \omega_2^M \)-many reals to \( M \).
We will try to do this a `minimal way'; for example, we do not want to add any ordinals.
This gives us much more control over the model that we build.

Recall the argument that the sentence \( \varphi(x) \equiv \exists x.\, x^2 = 2 \) is independent of the axioms of fields: we began with a field in which the sentence failed, namely \( \mathbb Q \), and then extended it in a minimal way to \( \mathbb Q\qty[\sqrt{2}] \).
The model \( \mathbb Q\qty[\sqrt{2}] \) does not just contain \( \mathbb Q \cup \qty{\sqrt{2}} \), it also contains everything that can be built from \( \mathbb Q \) and \( \sqrt{2} \) using the axioms of fields.
The field \( \mathbb Q\qty[\sqrt{2}] \) is the minimal field extension of \( \mathbb Q \) satisfying \( \varphi \).

We may encounter some difficulties when adding arbitrary reals to our model.
Suppose that \( M \) is of the form \( \mathrm{L}_\gamma \), where \( \gamma \) is a countable ordinal.
Then \( \gamma \) can be coded as a subset \( c \) of \( \omega \), which can be viewed as a real.
If we added \( c \) to \( M \), we could decode it to form \( \gamma = \mathrm{Ord} \cap M \).
This would violate the principle of not adding any new ordinals.

Suppose we enumerate all formulas as \( \qty{\varphi_n \mid n \in \omega} \).
Let \( r = \qty{n \mid M \vDash \varphi_n} \).
If we added \( r \) to \( M \), we could then build a truth predicate for \( M \).
This would cause problems due to Tarski.

The main issues we must overcome are the following.
\begin{enumerate}
    \item We need a method to choose the \( \omega_2^M \)-many subsets of \( M \) to be added.
    \item Given these, we need to ensure that the extension satisfies \( \mathsf{ZFC} \).
    \item We must ensure that \( \omega_1^M \) and \( \omega_2^M \) are still cardinals in the extension.
\end{enumerate}
We will build these reals from within \( M \) itself.
Note that if \( r \) is a real, then each of its finite decimal approximations is already in \( M \).
The issue is that from within \( M \), we do not know what the real we want to add is.
So we may not know from within \( M \) which reals we will add.
Instead, we will add a \emph{generic} real.
To be generic, we will not specify any particular digits, but its decimal expansion will contain every finite sequence.
We will call a specification \emph{dense} if any finite approximation can be extended to one satisfying the specification.
For example, `beginning with a \( 7 \)' is not dense, but `containing the subsequence \( 746 \)' is dense.
It turns out that a real is generic precisely when it meets every dense specification.

The axiom of choice is not needed in the basic machinery of forcing, so we will work primarily over \( \mathsf{ZF} \) and state explicitly where choice is used.

\subsection{Forcing posets}
\begin{definition}
    A \emph{preorder} is a pair \( (\mathbb P, \leq) \) such that
    \begin{itemize}
        \item \( \mathbb P \) is nonempty;
        \item \( \leq \) is a binary relation on \( \mathbb P \);
        \item \( \leq \) is transitive, so \( p \leq q \) and \( q \leq r \) implies \( p \leq r \);
        \item \( \leq \) is reflexive, so \( p \leq p \).
    \end{itemize}
    A preorder is called a \emph{partial order} if \( \leq \) is antisymmetric, so \( p \leq q \) and \( q \leq p \) implies \( p = q \).
\end{definition}
\begin{definition}
    A \emph{forcing poset} is a triple \( (\mathbb P, \leq_{\mathbb P}, \Bbbone_{\mathbb P}) \), where \( (\mathbb P, \leq_{\mathbb P}) \) is a preorder and \( \Bbbone_{\mathbb P} \) is a maximal element.
    Elements of \( \mathbb P \) are called \emph{conditions}, and we say \( q \) is \emph{stronger} than \( p \) or an \emph{extension} of \( p \) if \( q \leq p \).
    We say that \( p, q \) are \emph{compatible}, written \( p \mathrel{\|}_{\mathbb P} q \), if there exists \( r \) such that \( r \leq_{\mathbb P} p, q \).
    Otherwise, we say they are \emph{incompatible}, written \( p \perp q \).
\end{definition}
\begin{remark}
    In some texts, the partial order is reversed.
    This is called \emph{Jerusalem notation}.
\end{remark}
The notation \( \mathbb P \in M \) abbreviates \( (\mathbb P, \leq_{\mathbb P}, \Bbbone_{\mathbb P}) \in M \).
Note that by transitivity if \( \mathbb P \) is an element of \( M \), then \( \Bbbone_{\mathbb P} \in M \), but we do not necessarily have \( \leq_{\mathbb P} \in M \).
\begin{definition}
    A preorder is \emph{separative} if whenever \( q \nleq p \), there is \( r \leq q \) such that \( r \perp p \).
\end{definition}
\begin{proposition}
    If \( (\mathbb P, \leq) \) is a separative preorder, it is a partial order.
\end{proposition}
\begin{proposition}
    Suppose that \( (\mathbb P, \leq) \) is a preorder.
    Define \( p \sim q \) by
    \[ p \sim q \leftrightarrow \forall r \in P.\, (r \mathrel\| p \leftrightarrow r \mathrel\| q) \]
    Then there is a separative preorder on \( \faktor{\mathbb P}{\sim} \) such that
    \[ [p] \perp [q] \leftrightarrow p \perp q \]
    and if \( \mathbb P \) has a maximal element, so does \( \faktor{\mathbb P}{\sim} \).
\end{proposition}
\begin{example}
    For sets \( I, J \), we let \( F_n(I, J) \) denote the set of all finite partial functions from \( I \) to \( J \).
    \[ F_n(I, J) = \qty{p \mid \abs{p} < \omega \wedge p \text{ is a function} \wedge \dom p \subseteq I \wedge \operatorname{ran} p \subseteq J} \]
    We let \( \leq \) be the reverse inclusion on \( F_n(I, J) \), so \( q \leq p \) if and only if \( q \supseteq p \).
    The maximal element \( \Bbbone \) is the empty set.
    Then \( (F_n(I, J), \leq, \varnothing) \) is a forcing poset, and moreover, the preorder is separative.
\end{example}
\begin{remark}
    When \( \alpha \) is an ordinal, the forcing poset \( F_n(\alpha \times \omega, 2) \) is often written \( \operatorname{Add}(\omega, \alpha) \), denoting the idea that we are adding \( \alpha \)-many subsets of \( \omega \).
\end{remark}
