\begin{definition}
    Let \( A \) be a \( k \)-algebra, and let \( x_1, \dots, x_n \in A \).
    We say that \( x_1, \dots, x_n \) are \emph{\( k \)-algebraically independent} if for every nonzero polynomial \( p \in k[T_1, \dots, T_n] \), we have \( p(x_1, \dots, x_n) \neq 0 \).
    Equivalently, the \( k \)-algebra homomorphism \( k[T_1, \dots, T_n] \to A \) given by \( T_i \mapsto x_i \) is injective.
\end{definition}
\begin{theorem}[Noether's normalisation theorem]
    Let \( k \) be a field, and let \( A \neq 0 \) be a finitely generated \( k \)-algebra.
    Then there exist \( x_1, \dots, x_n \in A \) which are \( k \)-algebraically independent and \( A \) is finite over \( A' = k[x_1, \dots, x_n] \).
\end{theorem}
We first present an example of the method used in the proof.
\begin{example}
    Let \( A = k[T, T^{-1}] \simeq \faktor{k[X,Y]}{(XY - 1)} \).
    We claim that \( k[T] \subseteq k[T, T^{-1}] \) is not a finite extension.
    Indeed, suppose it were finite.
    Then \( T^{-1} \) would be integral over \( k[T] \), so
    \[ (T^{-1})^n \in \vecspan_{k[T]}\qty{(T^{-1})^0, \dots, (T^{-1})^{n-1}} \]
    Multiplying by \( T^n \), we have
    \[ 1 \in \vecspan_{k[T]}(T^n, \dots, T) \]
    which is false.
    However, if \( c \in k \) is a scalar which we will choose later,
    \[ A = k[T, T^{-1}] = k[T, T^{-1} - cT] \]
    We claim that \( k[T^{-1} - cT] \subseteq A \) is a finite extension for most values of \( c \), and in particular, for at least one.
    First, note \( T^{-1} T - 1 = 0 \), and then change variables to
    \[ ((T^{-1} - cT) + cT) T - 1 = 0 \implies \underbrace{c}_{\in k} T^2 + \underbrace{(T^{-1} - cT)}_{\in k[T^{-1} - ct]} T - \underbrace{1}_{\in k[T^{-1} - cT]} = 0 \]
    Hence if \( c \neq 0 \), \( T \) is integral over \( k[T^{-1} - cT] \).
\end{example}
\begin{proof}
    In this proof, we will assume \( k \) is infinite, although the theorem is true even if \( k \) if finite.
    We will proceed by induction on the minimal number of generators of \( A \) as a \( k \)-algebra, which we will denote \( m \).
    For the case \( m = 0 \), we have \( A = k \), so we can take \( A' = k \).

    Suppose that \( A \) is generated as a \( k \)-algebra by \( x_1, \dots, x_m \in A \).
    If \( x_1, \dots, x_m \) are algebraically independent, then we can take \( A' = A \).
    Otherwise, we claim that there are \( c_1, \dots, c_{m-1} \in k \) such that \( x_m \) is integral over
    \[ B = k[x_1 - c_1 x_m, \dots, x_{m-1} - c_{m-1} x_m] \]
    Assuming that this holds, we have \( A = B[x_m] \), so \( B \subseteq A \) is a finite extension.
    But \( B \) is generated by \( m - 1 \) elements, so by induction \( B \) contains \( z_1, \dots, z_n \in B \) which are \( k \)-algebraically independent, and \( B \) is finite over \( A' = k[z_1, \dots, z_n] \).
    Then \( A \) is finite over \( A' \) by transitivity of finiteness.
    
    We now prove the claim.
    As \( x_1, \dots, x_m \) are not algebraically independent over \( k \), there is a nonzero polynomial \( f \in k[T_1, \dots, T_m] \) such that \( f(x_1, \dots, x_m) = 0 \).
    We want to show that \( x_m \) is integral over \( B \).
    Write \( f \) as the sum of its homogeneous parts, and let \( F \) be the part of highest degree \( \deg f = r \).
    For scalars \( c_1, \dots, c_{m-1} \in k \) which will be chosen later, we define
    \begin{align*}
        g(T_1, \dots, T_m) &= f(T_1 + c_1 T_m, \dots, T_{m-1} + c_{m-1} T_m, T_m) \\
        &= \underbrace{F(c_1, \dots, c_m, 1)}_{\in k} T_m^r + \text{terms of lower degree in \( T_m \) with coefficients in \( k[T_1, \dots, T_{m-1}] \)}
    \end{align*}
    Note that
    \[ g(x_1 - c_1 x_m, \dots, x_{m-1} - c_{m-1} x_m, x_m) = f(x_1, \dots, x_m) = 0 \]
    but as a polynomial in \( T_m \) over \( k[T_1, \dots, T_{m-1}] \), it has degree at most \( r \), and the coefficient of \( T_m^r \) is \( F(c_1, \dots, c_m, 1) \).
    As \( F(T_1, \dots, T_m) \) is a nonzero homogeneous polynomial, \( F(T_1, \dots, T_{m-1}, 1) \) is not the zero polynomial.
    Thus there are \( c_1, \dots, c_{m-1} \) such that \( F(c_1, \dots, c_{m-1}, 1) \neq 0 \) as \( k \) is an infinite field.
\end{proof}
