\subsection{Modules}
In this course, a \emph{ring} stands for a commutative unital ring \( R \).
We do however allow for one noncommutative exception, the endomorphism ring \( \End(M) \) of an abelian group \( M \).
This is a ring where composition is the multiplication operation.
\begin{definition}
    An \( R \)-module is an abelian group \( M \) with a fixed ring homomorphism \( \rho : R \to \End(M) \).
    If \( r \in R \) and \( m \in M \), we define \( r \cdot m = \rho(r)(m) \).    
\end{definition}
\begin{remark}
    Note that as \( \rho(r) \) is a group homomorphism,
    \[ r(m_1 + m_2) = \rho(r)(m_1 + m_2) = \rho(r)(m_1) + \rho(r)(m_2) = r \cdot m_1 + r \cdot m_2 \]
    Also, as \( \rho \) is a ring homomorphism,
    \[ (r_1 + r_2)m = \rho(r_1 + r_2)(m) = (\rho(r_1) + \rho(r_2))m = r_1 \cdot m + r_2 \cdot m \]
\end{remark}
\begin{example}
    \begin{enumerate}
        \item Let \( k \) be a field.
        Then a \( k \)-module is a \( k \)-vector space.
        \item Every abelian group \( M \) is a \( \mathbb Z \)-module in a unique way, because the morphism \( \mathbb Z \to \End M \) must map \( 1 \) to \( \id \).
        \item Every ring \( R \) is an \( R \)-module, by taking \( \rho(r) = r_0 \mapsto r_0 r \).
        \item More generally, the direct sum \( R^{\oplus \mathbb N} \) of many copies of \( R \) and the direct product \( R^{\mathbb N} \) are also \( R \)-modules.
        Note that \( R^{\oplus \mathbb N} \) is the subset of \( R^{\mathbb N} \) containing elements of only finite support.
    \end{enumerate}
\end{example}

\subsection{Noetherian and Artinian modules}
\begin{definition}
    An \( R \)-module \( M \) is \emph{Noetherian} if one of the following conditions holds.
    \begin{enumerate}
        \item Every ascending chain of submodules \( M_0 \subseteq M_1 \subseteq \cdots \) inside \( M \) stabilises.
        That is, for some \( k \), every \( j \) has \( M_{k+j} = M_k \).
        \item Every nonempty set \( \Sigma \) of submodules of \( M \) has a maximal element.
    \end{enumerate}
\end{definition}
Note that we can use the axiom of choice to show that the two given conditions are equivalent.
\begin{definition}
    \( M \) is \emph{Artinian} if one of the following conditions holds.
    \begin{enumerate}
        \item Every descending chain of submodules \( M_0 \supseteq M_1 \supseteq \cdots \) inside \( M \) stabilises.
        \item Every nonempty set \( \Sigma \) of submodules of \( M \) has a minimal element.
    \end{enumerate}
\end{definition}
Again, both conditions are equivalent.
\begin{lemma}
    An \( R \)-module \( M \) is Noetherian if and only if every submodule of \( M \) is finitely generated.
\end{lemma}
Note that every Noetherian module is finitely generated.
Let \( R = \mathbb Z[T_1, T_2, \dots] \), and let \( M = R \) as an \( R \)-module.
\( M \) is generated by \( 1_R \), so in particular it is finitely generated.
But it has a submodule \( \langle T_1, T_2, \dots \rangle \) that is not finitely generated.
So in the above lemma we indeed must check every submodule.
\begin{definition}
    A ring \( R \) is Noetherian (respectively Artinian) if \( R \), as an \( R \)-module is Noetherian (resp.\ Artinian).
\end{definition}
\begin{example}
    \begin{enumerate}
        \item \( \mathbb Z \) over itself is a Noetherian module as it is a principal ideal domain, but it is not an Artinian module because we can take the chain \( (2) \supsetneq (4) \supsetneq (8) \supsetneq \cdots \).
        \item \( \mathbb Z \) is similarly a Noetherian ring but not an Artinian ring.
        \item \( \faktor{\mathbb Z\qty[\frac{1}{2}]}{\mathbb Z} \) is an Artinian module but not a Noetherian module.
        \item In fact, a ring \( R \) is Artinian if and only if \( R \) is Noetherian and \( R \) has \emph{Krull dimension} 0.
    \end{enumerate}
\end{example}

\subsection{Exact sequences}
\begin{definition}
    A sequence
    \[\begin{tikzcd}
        \cdots & {M_{i-1}} & {M_i} & {M_{i+1}} & \cdots
        \arrow[from=1-1, to=1-2]
        \arrow["{f_i}", from=1-2, to=1-3]
        \arrow["{f_{i+1}}", from=1-3, to=1-4]
        \arrow[from=1-4, to=1-5]
    \end{tikzcd}\]
    is \emph{exact} if the image of \( f_i \) is equal to the kernel of \( f_{i+1} \) for each \( i \), where the \( M_i \) are modules and the \( f_i \) are module homorphisms.
\end{definition}
\begin{definition}
    A \emph{short exact sequence} is an exact sequence of the form
    \[\begin{tikzcd}
        0 & {M'} & M & {M''} & 0
        \arrow[from=1-1, to=1-2]
        \arrow["{\text{injective}}", from=1-2, to=1-3]
        \arrow["{\text{surjective}}", from=1-3, to=1-4]
        \arrow[from=1-4, to=1-5]
    \end{tikzcd}\]
    In this situation, \( M'' \simeq \faktor{M}{i(M')} \).
    This is a way to encode \( M'' \) as a quotient by a submodule.
\end{definition}
\begin{lemma}
    Let
    \[\begin{tikzcd}
        0 & N & M & L & 0
        \arrow[from=1-1, to=1-2]
        \arrow[from=1-2, to=1-3]
        \arrow[from=1-3, to=1-4]
        \arrow[from=1-4, to=1-5]
    \end{tikzcd}\]
    be a short exact sequence of \( R \)-modules.
    Then \( M \) is Noetherian (resp.\ Artinian) if and only if both \( N \) and \( L \) are Noetherian (resp.\ Artinian).
\end{lemma}
\begin{corollary}
    If \( M_1, \dots, M_n \) are Noetherian (resp.\ Artinian) modules, then so is \( M_1 \oplus \dots \oplus M_n \).
\end{corollary}
Recall that a module homomorphism between \( R \)-modules
\[ \varphi : M_1 \oplus \dots \oplus M_n \to L \]
is the same as a collection of module homomorphisms
\[ \varphi_1 : M_1 \to L, \dots, \varphi_n : M_n \to L \]
This is also the case for an infinite direct sum, but not necessarily from an infinite direct product.
\begin{proposition}
    For a Noetherian (resp.\ Artinian) ring \( R \), every finitely generated \( R \)-module is Noetherian (resp.\ Artinian).
\end{proposition}
\begin{proof}
    \( M \) is finitely generated if and only if there is a surjective module homomorphism \( \varphi : R^n \to M \) for some \( n \geq 0 \).
    That is, \( M \) is a quotient of \( R^n \).
    The fact that \( R^n \) is Noetherian (or Artinian) passes through to its quotients.
\end{proof}

\subsection{Algebra}
\begin{definition}
    An \emph{\( R \)-algebra} is a ring \( A \) together with a fixed ring homomorphism \( \rho : R \to A \).
\end{definition}
\begin{example}
    The map \( k \to k[T_1, \dots, T_n] \) makes the polynomial ring \( k[T_1, \dots, T_n] \) a \( k \)-algebra.
\end{example}
We will write \( ra = \rho(r) a \).
Note that \( \rho(r) = \rho(r) \cdot 1_A = r \cdot 1_A \), so we can write \( r \cdot 1_A \) for \( \rho(r) \).
\begin{remark}
    Every \( R \)-algebra is an \( R \)-module.
\end{remark}
\begin{example}
    As a \( k \)-module, \( k[T_1, \dots, T_n] \) is infinite-dimensional.
    As a \( k \)-algebra, \( k[T_1, \dots, T_n] \) is generated by the \( n \) elements \( T_1, \dots, T_n \).
\end{example}
\begin{definition}
    \( \varphi : A \to B \) is an \emph{\( R \)-algebra homomorphism} if \( \varphi \) is a ring homomorphism and preserves all elements of \( R \).
    That is, \( \varphi(r \cdot 1_A) = r \cdot 1_B \).
\end{definition}
An \( R \)-algebra \( A \) is finitely generated if and only if there is some \( n \geq 0 \) and a surjective algebra homomorphism \( R[T_1, \dots, T_n] \to A \).
