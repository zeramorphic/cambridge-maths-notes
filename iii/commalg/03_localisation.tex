\subsection{Definitions}
\begin{definition}
    A \emph{multiplicative set} or \emph{multiplicatively closed set} \( S \subseteq R \) is a subset such that \( 1 \in S \) and if \( a, b \in S \), then \( ab \in S \).
    If \( U \subseteq R \) is any set, its \emph{multiplicative closure} \( S \) of \( U \) is the set
    \[ \qty{\prod_{i = 1}^n u_i \mid n \geq 0, u_i \in U} \]
    which is the smallest multiplicatively closed set containing \( U \).
\end{definition}
\begin{example}
    \begin{enumerate}
        \item If \( R \) is an integral domain, then \( S = R \setminus \qty{0} \) is multiplicative.
        \item More generally, if \( \mathfrak p \) is a prime ideal in \( R \), then \( S = R \setminus \mathfrak p \) is multiplicative.
        \item If \( x \in R \), then the set \( \qty{x^n \mid n \geq 0} \) is multiplicative.
    \end{enumerate}
\end{example}
\begin{remark}
    \( \mathbb Q \) is obtained from \( \mathbb Z \) by adding inverses for the elements of the multiplicative subset \( \mathbb Z \setminus \qty{0} \).
    We have a ring homomorphism \( \mathbb Z \hookrightarrow \mathbb Q \).
    We generalise this construction to arbitrary rings and multiplicative sets.
    In general, injectivity of the ring homomorphism in question may fail.
\end{remark}
\begin{definition}
    Let \( S \subseteq R \) be a multiplicative set, and let \( M \) be an \( R \)-module.
    Then the \emph{localisation} of \( M \) by \( S \) is the set \( S^{-1} M = \faktor{M \times S}{\sim} \) where \( (m_1, s_1) \sim (m_2, s_2) \) if and only if there exists \( u \in S \) such that \( u(s_2 m_1 - s_1 m_2) = 0 \).
    We write \( \frac{m}{s} \) for the equivalence class corresponding to \( (m, s) \).
    We make \( S^{-1} M \) into an \( R \)-module by defining
    \[ \frac{m_1}{s_1} + \frac{m_2}{s_2} = \frac{m_1 s_2 + m_2 s_1}{s_1 s_2};\quad r \cdot \frac{m}{s} = \frac{rm}{s} \]
    We can make \( S^{-1} R \) into a ring by defining
    \[ \frac{r_1}{s_1} \cdot \frac{r_2}{s_2} = \frac{r_1 r_2}{s_1 s_2} \]
\end{definition}
