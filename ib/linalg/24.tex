\subsection{???}
We wish to extend the previous statements about spectral theory into statements about bilinear forms.
\begin{corollary}
	Let \( A \in M_n(\mathbb R) \) (or \( M_n(\mathbb C) \)) be a symmetric (or respectively Hermitian) matrix.
	Then there exists an orthonormal (respectively unitary) matrix \( P \) such that \( P^\transpose A P \) (or \( P^\dagger A P \)) is diagonal with real-valued entries.
\end{corollary}
\begin{proof}
	Using the standard inner product, \( A \in L(F^n) \) is self-adjoint and hence there exists an orthonormal basis \( B \) of \( F^n \) such that \( A \) is diagonal in this basis.
	Let \( P = (v_1, \dots, v_n) \) be the matrix of this basis.
	Since \( B \) is orthonormal, \( P \) is orthogonal (or unitary).
	The result follows from the fact that \( P^{-1} A P \) is diagonal.
	The eigenvalues are real, hence the diagonal matrix is real.
\end{proof}
\begin{corollary}
	Let \( V \) be a finite-dimensional real (or complex) inner product space.
	Let \( \phi \colon V \times V \to F \) be a symmetric (or Hermitian) bilinear form.
	Then, there exists an orthonormal basis \( B \) of \( V \) such that \( [\phi]_B \) is diagonal.
\end{corollary}
\begin{proof}
	\( A^\transpose = A \) (or respectively \( A^\dagger = A \)), hence there exists an orthogonal (respectively unitary) matrix \( P \) such that \( P^{-1} A P \) is diagonal.
	Let \( (v_i) \) be the \( i \)th row of \( P^{-1} = P^\transpose \) (or \( P^\dagger \)).
	Then \( (v_1, \dots, v_n) \) is an orthonormal basis \( B \) of \( V \) such that \( [\phi]_V \) is this diagonal matrix.
\end{proof}
\begin{remark}
	The diagonal entries of \( P^{-1} A P \) are the eigenvalues of \( A \).
	Moreover, we can define the signature \( s(\phi) \) to be the difference between the number of positive eigenvalues of \( A \) and the number of negative eigenvalues of \( A \).
\end{remark}

\subsection{Simultaneous diagonalisation}
\begin{corollary}
	Let \( V \) be a finite-dimensional real (or complex) vector space.
	Let \( \phi, \psi \) be symmetric (or Hermitian) bilinear forms on \( V \).
	Let \( \phi \) be positive definite.
	Then there exists a basis \( (v_1, \dots, v_n) \) of \( V \) with respect to which \( \phi \) and \( \psi \) are represented with a diagonal matrix.
\end{corollary}
\begin{proof}
	Since \( \phi \) is positive definite, \( V \) equipped with \( \phi \) is a finite-dimensional inner product space where \( \inner{u,v} = \phi(u,v) \).
	Hence, there exists a basis of \( V \) in which \( \psi \) is represented by a diagonal matrix, which is orthonormal with respect to the inner product defined by \( \phi \).
	Then, \( \phi \) in this basis is represented by the identity matrix given by \( \phi(v_i, v_j) = \inner{v_i, v_j} = \delta_{ij} \), which is diagonal.
\end{proof}
\begin{corollary}
	Let \( A, B \in M_n(\mathbb R) \) (or \( \mathbb C \)) which are symmetric (or Hermitian).
	Suppose for all \( x \neq 0 \) we have \( x^\dagger A x > 0 \), so \( A \) is positive definite.
	Then there exists an invertible matrix \( Q \in M_n(\mathbb R) \) (or \( \mathbb C \)) such that \( Q^\transpose A Q \) (or \( Q^\transpose A \overline{Q} \)) and \( Q^\transpose B Q \) (or \( Q^\transpose B \overline{Q} \)) are diagonal.
\end{corollary}
\begin{proof}
	\( A \) induces a quadratic form \( Q(x) = x^\dagger A x \) which is positive definite by assumption.
	Similarly, \( \widetilde Q(x) = x^\dagger B x \) is induced by \( B \).
	Then we can apply the previous corollary and change basis.
\end{proof}
