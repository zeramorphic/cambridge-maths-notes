\subsection{Projection maps}
\begin{definition}
	Suppose \( V = U \oplus W \), so \( U \) is a complement of \( W \) in \( V \).
	Then, we define \( \pi \colon V \to W \) which maps \( v = u + w \) to \( w \).
	This is well defined, since the sum is direct.
	\( \pi \) is linear, and \( \pi^2 = \pi \).
	We say that \( \pi \) is the \textit{projection} operator onto \( W \).
\end{definition}
\begin{remark}
	The map \( \iota - \pi \) is the projection onto \( U \), where \( \iota \) is the identity map.
\end{remark}
\begin{lemma}
	Let \( V \) be an inner product space.
	Let \( W \leq V \) be a finite-dimensional subspace.
	Let \( (e_1, \dots, e_k) \) be an orthonormal basis for \( W \).
	Then,
	\begin{enumerate}[(i)]
		\item \( \pi(v) = \sum_{i=1}^k \inner{v, e_i} e_i \); and
		\item for all \( v \in V, w in W \), \( \norm{v - \pi(v)} \leq \norm{v - w} \) with equality if and only if \( w = \pi(v) \), hence \( \pi(v) \) is the point in \( W \) closest to \( v \).
	\end{enumerate}
\end{lemma}
\begin{proof}
	We define \( \pi(v) = \sum_{i=1}^k \inner{v, e_i} e_i \).
	Since \( W = \genset{\qty{e_k}} \), \( \pi(v) \in W \) for all \( v \in V \).
	Then, \( v = (v - \pi(v)) + \pi(v) \) has a term in \( W \).
	We claim that the remaining term is in the orthogonal; \( v - \pi(v) \in W^\perp \).
	Indeed, we must show \( \inner{v - \pi(v), w} = 0 \) for all \( w \in W \).
	Equivalently, \( \inner{v - \pi(v), e_i} = 0 \) for all basis vectors \( e_i \) of \( W \).
	We can explicitly compute
	\[
		\inner{v - \pi(v), e_j} = \inner{v, e_j} - \inner{\sum_{i=1}^k \inner{v, e_i} e_i, e_j} = \inner{v, e_j} - \sum_{i=1}^k \inner{v, e_i} \inner{e_i, e_j} = \inner{v, e_j} - \inner{v, e_j} = 0
	\]
	Hence, \( v = (v - \pi(v)) + \pi(v) \) is a decomposition into \( W \) and \( W^\perp \).
	Since \( W \cap W^\perp = \qty{0} \), we have \( V = W \overset{\perp}{\oplus} W^\perp \).
	For the second part, let \( v \in V \), \( w \in W \), and we compute
	\[
		\norm{v - w}^2 = \norm{\underbrace{v - \pi(v)}_{\in W^\perp} + \underbrace{\pi(v) - w}_{\in W}}^2 = \norm{v - \pi(v)}^2 + \norm{\pi(v) - w}^2 \geq \norm{v - \pi(v)}^2
	\]
	with equality if and only if \( w = \pi(v) \).
\end{proof}

\subsection{Adjoint maps}
\begin{definition}
	Let \( V, W \) be finite-dimensional inner product spaces.
	Let \( \alpha \in L(V, W) \).
	Then there exists a unique linear map \( \alpha^\star \colon W \to V \) such that for all \( v, w \in V, W \),
	\[
		\inner{\alpha(v), w} = \inner{v, \alpha^\star(w)}
	\]
	Moreover, if \( B \) is an orthonormal basis of \( V \), and \( C \) is an orthonormal basis of \( W \), then
	\[
		[\alpha^\star]_{C, B} = \qty(\overline{[\alpha]_{B, C}})^\transpose
	\]
\end{definition}
\begin{proof}
	Let \( B = (v_1, \dots, v_n) \) and \( C = (w_1, \dots, w_m) \) and \( A = [\alpha]_{B, C} = (a_{ij}) \).
	To check existence, we define \( [\alpha^\star]_{C, B} = \overline{A}^\transpose = (c_{ij}) \) and explicitly check the definition.
	By orthogonality,
	\[
		\inner{\alpha\qty(\sum \lambda_i v_i), \sum \mu_j w_j} = \inner{\sum_{i,k} \lambda_i a_{ki} w_k, \sum_j \mu_j w_j} = \sum_{i,j} \lambda_i a_{ji} \overline{\mu_j}
	\]
	Then,
	\[
		\inner{\sum \lambda_i v_i, \alpha^\star\qty(\sum \mu_j w_j)} = \inner{\sum_i \lambda_i v_i, \sum_{j,k} \mu_j c_{kj} v_k} = \sum_{i,j} \lambda_i \overline{c_{ij}} \overline{\mu_j}
	\]
	So equality requires \( \overline{c_{ij}} = a_{ji} \).
	Uniqueness follows from the above; the expansions are equivalent for any vector if and only if \( \overline{c_{ij}} = a_{ji} \).
\end{proof}
\begin{remark}
	The same notation, \( \alpha^\star \), is used for the adjoint as just defined, and the dual map as defined before.
	If \( V, W \) are real product inner spaces and \( \alpha \in L(V,W) \), we define \( \psi \colon V \to V^\star \) such that \( \psi(v)(x) = \inner{x,v} \) and similarly for \( W \).
	Then we can check that the adjoint for \( \alpha \) is given by the composition of \( \psi \) from \( V \to V^\star \), then applying the dual, then applying the inverse of \( \psi \) for \( W \).
\end{remark}

\subsection{Self-adjoint and isometric maps}
\begin{definition}
	Let \( V \) be a finite-dimensional inner product space, and \( \alpha \) be an endomorphism of \( V \).
	Let \( \alpha^\star \in L(V) \) be the adjoint map.
	Then,
	\begin{enumerate}[(i)]
		\item the condition \( \inner{\alpha v, w} = \inner{v, \alpha w} \) is equivalent to the condition \( \alpha = \alpha^\star \), and such an \( \alpha \) is called \textit{self-adjoint} (for \( \mathbb R \) we call such endomoprhisms \textit{symmetric}, and for \( \mathbb C \) we call such endomorphisms Hermitian);
		\item the condition \( \inner{\alpha v, \alpha w} = \inner{v, w} \) is equivalent to the condition \( \alpha^\star = \alpha^{-1} \), and such an \( \alpha \) is called an \textit{isometry} (for \( \mathbb R \) it is called \textit{orthogonal}, and for \( \mathbb C \) it is called \textit{unitary}).
	\end{enumerate}
\end{definition}
\begin{proposition}
	The conditions for isometries defined as above are equivalent.
\end{proposition}
\begin{proof}
	Suppose \( \inner{\alpha v, \alpha w} = \inner{v,w} \).
	Then for \( v = w \), we find \( \norm{\alpha v}^2 = \norm{v}^2 \), so \( \alpha \) preserves the norm.
	In particular, this implies \( \ker \alpha = \qty{0} \).
	Since \( \alpha \) is an endomorphism and \( V \) is finite-dimensional, \( \alpha \) is bijective.
	Then for all \( v, w \in V \),
	\[
		\inner{v, \alpha^\star(w)} = \inner{\alpha v, w} = \inner{\alpha v, \alpha\qty(\alpha^{-1}(w))} = \inner{v, \alpha^{-1}(w)}
	\]
	Hence \( \alpha^\star = \alpha^{-1} \).
	Conversely, if \( \alpha^\star = \alpha^{-1} \) we have
	\[
		\inner{\alpha v, \alpha w} = \inner{v, \alpha^\star(\alpha w)} = \inner{v, w}
	\]
	as required.
\end{proof}
\begin{remark}
	Using the polarisation identity, we can show that \( \alpha \) is isometric if and only if for all \( v \in V \), \( \norm{\alpha(v)} = \norm{v} \).
\end{remark}
\begin{lemma}
	Let \( V \) be a finite-dimensional real (or complex) inner product space.
	Then for \( \alpha \in L(V) \),
	\begin{enumerate}[(i)]
		\item \( \alpha \) is self-adjoint if and only if for all orthonormal bases \( B \) of \( V \), we have \( [\alpha]_B \) is symmetric (or Hermitian);
		\item \( \alpha \) is an isometry if and only if for all orthonormal bases \( B \) of \( V \), we have \( [\alpha]_B \) is orthogonal (or unitary).
	\end{enumerate}
\end{lemma}
\begin{proof}
	Let \( B \) be an orthonormal basis for \( V \).
	Then we know \( [\alpha^\star]_B = [\alpha]_B^\dagger \).
	We can then check that \( [\alpha]_B^\dagger = [\alpha]_B \) and \( [\alpha]_B^\dagger = [\alpha]_B^{-1} \) respectively.
\end{proof}
\begin{definition}
	For \( F = \mathbb R \), we define the \textit{orthogonal group} of \( V \) by
	\[
		O(V) = \qty{ v \in L(V) \colon \alpha \text{ is an isometry} }
	\]
	Note that \( O(V) \) is bijective with the set of orthogonal bases of \( V \).
	For \( F = \mathbb C \), we define the \textit{unitary group} of \( V \) by
	\[
		U(V) = \qty{ v \in L(V) \colon \alpha \text{ is an isometry} }
	\]
	Again, note that \( U(V) \) is bijective with the set of orthogonal bases of \( V \).
\end{definition}
