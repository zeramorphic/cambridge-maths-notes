\subsection{Geodesics on surfaces}
Consider a surface \( \Sigma \) in \( \mathbb R^3 \), given by
\[
	\Sigma = \qty{ \vb x \colon g(\vb x) = 0 }
\]
Consider two points \( A, B \) on \( \Sigma \).
What are the geodesics (shortest paths on the surface) between the two points, if one exists at all?
Consider a parametrisation of such a path given by \( t \in [0, 1] \) where \( A = \vb x(0), B = \vb x(1) \).
We wish to extremise
\[
	\Phi[\vb x, \lambda] = \int_0^1 \qty{ \sqrt{\dot x^2 + \dot y^2 + \dot z^2} - \lambda(t) g(\vb x) } \dd{t}
\]
The Lagrange multiplier, a function of \( t \), since we want the entire curve (for all \( t \)) to lie on \( \Sigma \).
We substitute the integrand \( h \) in the Euler-Lagrange equation.
Considering the variation with respect to \( \lambda \), we have
\[
	\dv{t} \pdv{h}{\dot\lambda} - \pdv{h}{\lambda} = 0
\]
But \( h \) does not depend on \( \dot\lambda \), hence \( \pdv{h}{\lambda} = 0 \), giving \( g(\vb x) = 0 \) for all \( \vb x \).
Considering the variation with respect to \( x_i \), we have
\[
	\dv{t} \pdv{h}{\dot x_i} - \pdv{h}{x_i} = 0
\]
Hence
\[
	\dv{t}\qty(\frac{\dot x_i}{\sqrt{\dot x_1^2 + \dot x_2^2 + \dot x_3^2}}) + \lambda \pdv{g}{x_i} = 0
\]
We could alternatively solve the constraint \( g = 0 \), and parametrise the surface according to this solution.

\subsection{Multiple independent variables}
In the most general case, we may have multiple independent variables in a variational problem.
This converts the Euler-Lagrange equation into a partial differential equation.
Suppose \( \phi \colon \mathbb R^n \to \mathbb R^m \).
If \( n = 3 \), for example, we have
\[
	F[\phi] = \iiint_{\mathcal D} f(\underbrace{x, y, z}_{\mathclap{\text{independent}}}, \phi, \phi_x, \phi_y, \phi_z) \dd{x}\dd{y}\dd{z}
\]
where \( \mathcal D \subset \mathbb R^3 \), and \( \phi_{x_i} := \pdv*{\phi}{x_i} \).
Suppose there exists some extremum \( \phi \), and consider a small variation \( \phi \mapsto \phi(x,y,z) + \varepsilon \eta(x,y,z) \) where \( \eta = 0 \) on \( \partial \mathcal D \).
Evaluating the functional on this perturbed \( \phi \) gives
\[
	F[\phi + \varepsilon\eta] - F[\phi] = \varepsilon \iiint_{\mathcal D} \qty{ \eta \pdv{f}{\phi} + \eta_x \pdv{f}{\phi_x} + \eta_y \pdv{f}{\phi_y} + \eta_z \pdv{f}{\phi_z} } \dd{x}\dd{y}\dd{z} + O(\varepsilon^2)
\]
\[
	= \varepsilon \iiint_{\mathcal D} \qty{ \eta \pdv{f}{\phi} + \underbrace{\div(\eta\qty(\pdv{f}{\phi_x}, \pdv{f}{\phi_y}, \pdv{f}{\phi_z}))}_{\mathclap{\text{apply divergence theorem since \( \eta \) vanishes on \( \partial \mathcal D \)}}} - \eta \div(\pdv{f}{\phi_x}, \pdv{f}{\phi_y}, \pdv{f}{\phi_z}) } \dd{x}\dd{y}\dd{z} + O(\varepsilon^2)
\]
\[
	= \varepsilon \iiint_{\mathcal D} \eta \qty{ \pdv{f}{\phi} - \div(\pdv{f}{\phi_x}, \pdv{f}{\phi_y}, \pdv{f}{\phi_z}) } \dd{x}\dd{y}\dd{z} + O(\varepsilon^2)
\]
% Doesn't fit on page if we are in an align environment.
Now, we can apply the fundamental lemma to give the Euler-Lagrange equation for multiple independent variables.
\[
	\pdv{f}{\phi} - \div(\pdv{f}{\phi_x},\pdv{f}{\phi_y},\pdv{f}{\phi_z}) = 0
\]
Or, in suffix notation (with the summation convention),
\[
	\pdv{f}{\phi} - \partial_i \pdv{f}{(\partial_i \phi)} = 0
\]
This result applies for any \( n \).
Note that this is now a partial differential equation for \( \phi \), instead of an ordinary differential equation.

\subsection{Potential energy and the Laplace equation}
Consider the functional
\[
	F[\phi] = \iint_{\mathcal D \subset \mathbb R^2} \frac{1}{2} \qty[\phi_x^2 + \phi_y^2] \dd{x}\dd{y}
\]
Note that \( \pdv{f}{\phi} = 0 \) and \( \pdv{f}{\phi_x} = \phi_x; \pdv{f}{\phi_y} = \phi_y \).
The Euler-Lagrange equation becomes
\[
	\pdv{x}\phi_x + \pdv{y}\phi_y = 0 \implies \phi_{xx} + \phi_{yy} = 0
\]
This produces the Laplace equation.

\subsection{Minimal surfaces}
Consider minimising the area of a surface \( \Sigma \subset \mathbb R^3 \), where we want the surface to have two boundaries defined by fixed closed curves.
This is sometimes known as Plateau's problem.
We will let \( \Sigma = \qty{ \vb x = \mathbb R^3 \colon k(x, y, z) = 0 } \), and assume there exists a parametrisation of \( \Sigma \) given by \( z = \phi(x, y) \).
The line element is given by
\[
	\dd{s}^2 = \dd{x}^2 + \dd{y}^2 + \dd{z}^2
\]
We have \( \dd{z} = \phi_x \dd{x} + \phi_y \dd{y} \) hence
\[
	\dd{s}^2 = ( 1 + \phi_x^2 ) \dd{x}^2 + ( 1 + \phi_y^2 ) \dd{y}^2 + 2 \phi_x \phi_y \dd{x}\dd{y}
\]
This is a quadratic form in the differentials \( \dd{x}, \dd{y} \), known as the first fundamental form (also the Riemannian metric).
Alternatively,
\[
	\dd{s}^2 = g_{ij}\dd{x^i}\dd{x^j}
\]
where
\[
	g = \begin{pmatrix}
		1 + \phi_x^2  & \phi_x \phi_y \\
		\phi_x \phi_y & 1 + \phi_y^2
	\end{pmatrix}
\]
From this, we can compute the area element, which is defined as
\[
	\dd{A} = \sqrt{\det g} \dd{x}\dd{y}
\]
We will extremise the area functional
\[
	A[\phi] = \int_{\mathcal D} \sqrt{1 + \phi_x^2 + \phi_y^2}\dd{x}\dd{y}
\]
Let the integrand be \( h \), and apply the Euler-Lagrange equation.
\[
	\pdv{h}{\phi_x} = \frac{\phi_x}{\sqrt{1 + \phi_x^2 + \phi_y^2}};\quad \pdv{h}{\phi_y} = \frac{\phi_y}{\sqrt{1 + \phi_x^2 + \phi_y^2}}
\]
Hence
\[
	\partial_x \qty(\frac{\phi_x}{\sqrt{1 + \phi_x^2 + \phi_y^2}}) + \partial_y \qty(\frac{\phi_x}{\sqrt{1 + \phi_x^2 + \phi_y^2}}) = 0
\]
which can be expanded to give
\[
	(1 + \phi_y^2)\phi_{xx} + (1 + \phi_x^2)\phi_{yy} - 2 \phi_x \phi_y \phi_{xy} = 0
\]
This is known as the minimal surface equation.
We will solve a special case, where there is circular (cylindrical) symmetry, so \( z = \phi(r) \).
Since \( r = \sqrt{x^2 + y^2} \), we can find that
\[
	\phi_x = z' \frac{x}{r};\quad \phi_y = z' \frac{y}{r}
\]
and we can analogously compute \( \phi_{xx}, \phi_{yy}, \phi_{xy} \).
This gives
\[
	rz'' + z' + (z')^3 = 0
\]
