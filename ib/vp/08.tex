\subsection{Central Forces}
\begin{example}
	Consider a central force in the plane.
	The Lagrangian is
	\[
		L = T - V = \frac{1}{2}m\qty(\dot r^2 + r^2 \dot \theta^2) - V(r)
	\]
	The Euler-Lagrange equation gives
	\begin{align*}
		\dv{t}\pdv{L}{\dot r} - \pdv{L}{r}           & = 0 \\
		\dv{t}\pdv{L}{\dot \theta} - \pdv{L}{\theta} & = 0
	\end{align*}
	Since \( \pdv{L}{\theta} = 0 \), we have a first integral form:
	\[
		\pdv{L}{\dot \theta} = mr^2 \dot \theta = \text{constant}
	\]
	This can be interpreted physically as the law of conservation of angular momentum.
	Further, we have \( \pdv{L}{t} = 0 \) so we have another first integral:
	\begin{align*}
		\dot r \pdv{L}{\dot r} + \dot\theta \pdv{L}{\dot\theta} - L                              & = \text{constant} \\
		m\dot r^2 + mr^2\dot\theta^2 - \frac{1}{2}m\dot r^2 - \frac{1}{2}mr^2\dot\theta^2 + V(r) & = \text{constant} \\
		\frac{1}{2} m \qty( \dot r^2 + r^2\dot\theta^2 ) + V(r)                                  & = \text{constant}
	\end{align*}
	The left hand side is the total energy of the system, denoted \( E \).
	This is the law of conservation of energy.
\end{example}

\subsection{Configuration Space and Generalised Coordinates}
\begin{example}
	Consider \( N \) particles moving in \( \mathbb R^3 \).
	Typically we represent each point as a distinct vector in \( \mathbb R^3 \) that changes over time.
	We can alternatively consider a point in \( \mathbb R^{3N} \), which contains the information about every point.
	This is called the configuration space.
	The Lagrangian in configuration space is
	\[
		L = L(q_i, \dot {q_i}, t)
	\]
	where \( \vb q \) is the combined position vector of all \( N \) points, and likewise \( \dot{\vb q} \) is the combined velocity.
\end{example}

\subsection{Noether's Theorem}
Consider a functional
\[
	F[\vb y] = \int_\alpha^\beta f(y_i, y_i', x) \dd{x};\quad i = 1,\dots,n
\]
Suppose there exists a one-parameter family of transformations
\[
	y_i(x) \mapsto Y_i(x,s);\quad Y_i(x,0) = y_i(x)
\]
This can be thought of as a change of variables parametrised by \( s \in \mathbb R \), where \( s = 0 \) implies no change of variables.
This family is called a \textit{continuous symmetry} of the Lagrangian \( f \) if
\[
	\dv{s} f(Y_i(x,s), Y_i'(x,s), x) = 0
\]
In this course, we only consider continuous symmetries, so they may be abbreviated as just `symmetries'.
\begin{theorem}[Noether's Theorem]
	Given a continuous symmetry \( Y_i(x,s) \) of \( f \),
	\[
		\eval{\pdv{f}{y_i'}\pdv{Y_i}{s}}_{s=0}
	\]
	is a first integral of the Euler-Lagrange equation (where the summation convention applies).
\end{theorem}
\begin{proof}
	\begin{align*}
		0                          & = \eval{\dv{s} f}_{s=0}                                                                           \\
		                           & = \eval{\pdv{f}{y_i} \dv{Y_i}{s}}_{s=0} + \eval{\pdv{f}{y_i'}\pdv{Y_i'}{s}}_{s=0}                 \\
		                           & = \eval{\qty[\dv{x} \qty(\pdv{f}{y_i'})\dv{Y_i}{s} + \pdv{f}{y_i'}\dv{x}\qty(\dv{Y_i}{s})]}_{s=0} \\
		                           & = \dv{x}\eval{\qty[\pdv{f}{y_i'}\pdv{Y_i}{s}]}_{s=0}                                              \\
		\therefore \text{constant} & = \pdv{f}{y_i'}\pdv{Y_i}{s}
	\end{align*}
\end{proof}

\subsection{Conservation of Momentum}
\begin{example}
	Consider a vector \( \vb y = (y,z) \) and the function
	\[
		f = \frac{1}{2}y'^2 + \frac{1}{2}z'^2 - V(y-z)
	\]
	Consider the symmetry
	\begin{align*}
		Y = y + s                  & \implies Y' = y'      \\
		Z = z + s                  & \implies Z = z'       \\
		\therefore V(Y-Z) = V(y-z) & \implies \dv{s} f = 0
	\end{align*}
	Then from Noether's theorem,
	\[
		\text{constant} = \eval{\qty[\pdv{f}{y'}\dv{Y}{s} + \pdv{f}{z'}\dv{Z}{s}]}_{s=0} = y' + z'
	\]
	This can be thought of as a conserved momentum in the \(y+z\) direction.
\end{example}

\subsection{Conservation of Angular Momentum under Central Force}
\begin{example}
	Suppose \( \Theta = \theta + s, R = r \).
	Our space is isotropic, so \( \dv{L}{s} = 0 \), hence
	\[
		\eval{\qty[\pdv{L}{\dot\theta}\pdv{\Theta}{s} + \pdv{L}{\dot r}\pdv{R}{s}]}_{s=0} = mr^2\dot\theta
	\]
	which shows that angular momentum is conserved.
\end{example}

\subsection{Convex Functions}
This section reviews certain parts of calculus on \( \mathbb R^n \).
\begin{definition}
	A set \( S \subset \mathbb R^n \) is convex if \( \forall \vb x, \vb y \in S, \forall t \in [0,1], (1-t)\vb x + t \vb y \in S \).
\end{definition}
\begin{definition}
	The graph of a function \( f \colon \mathbb R^n \to \mathbb R \) is the surface \( \qty{ (\vb x, z)\in \mathbb R^{n+1} \colon z - f(\vb x) = 0} \).
\end{definition}
\begin{definition}
	A chord of a function \( f \colon \mathbb R^n \to \mathbb R \) is a line segment connecting two points on the graph of \( f \).
\end{definition}
\begin{definition}
	A function \( f \colon \mathbb R^n \to \mathbb R \) is convex if
	\begin{enumerate}[(i)]
		\item the domain of \( f \) is a convex set; and
		\item \( \forall \vb x, \vb y \in S, \forall t \in [0,1], f((1-t)x + ty) \leq (1-t)f(x) + tf(y) \)
	\end{enumerate}
	Equivalently, \( f \) is convex if the graph of \( f \) lies below (or on) all of its chords.
\end{definition}
