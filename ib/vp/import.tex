\chapter[Variational Principles \\ \textnormal{\emph{Lectured in Easter \oldstylenums{2021} by \textsc{Dr.\ M.\ Dunajski}}}]{Variational Principles}
\emph{\Large Lectured in Easter \oldstylenums{2021} by \textsc{Dr.\ M.\ Dunajski}}

In this course, we solve problems of the form `find the optimal function such that\dots'.
Examples include `find the shortest path between points \( A \) and \( B \) on surface \( \Sigma \)', or `find the shape of a wire under the influence of gravity between points \( A \) and \( B \) in the plane'.
The latter is called the brachistochrone problem, and is of central importance in motivating the subject.

In the same way that turning points of functions can often be located by setting the derivative to zero, optimal functions can be located by setting the functional derivative to zero.
This is called the Euler--Lagrange equation, and is a main tool that we use to find solutions to such problems.
An application of the Euler--Lagrange equation is Noether's theorem, which roughly states that any symmetry of a physical system gives rise to a conserved quantity.
For example, uniformity of space in the laws of physics shows that momentum is conserved, and uniformity of time shows that energy is conserved.

\subfile{../../ib/vp/main.tex}
