\subsection{The argument principle}
\begin{proposition}
	If \( f \) has a zero (or pole) of order \( k \geq 1 \) at \( z = a \), then \( \frac{f'}{f} \) has a simple pole at \( z = a \) with residue \( k \) (or \( -k \), respectively).
\end{proposition}
\begin{proof}
	If \( z = a \) is a zero of order \( k \), there is a disk \( D(a,r) \) such that \( f(z) = (z-a)^k g(z) \) for \( z \in D(a,r) \) where \( g \colon D(a,r) \to \mathbb C \) is holomorphic with \( g(z) \neq 0 \) for all \( z \in D(a,r) \).
	Hence,
	\[ f'(z) = k(z-a)^{k-1} g(z) + (z-a)^k g'(z) \]
	and
	\[ \frac{f'(z)}{f(z)} = \frac{k}{z-a} + \frac{g'(z)}{g(z) \]
	for all \( z \in D(a,r) \setminus \qty{a} \).
	Since \( \frac{g'}{g} \) is holomorphic in \( D(a,R) \), the claim follows.
	A similar argument holds for poles.
\end{proof}
\begin{definition}
	The order of a zero or pole \( a \) of a holomorphic function \( f \) is denoted \( \ord_f(a) \).
\end{definition}
\begin{theorem}[the argument principle]
	Let \( f \) be a meromorphic function on a domain \( U \) with finitely many zeroes \( a_1, \dots, a_k \) and finitely many poles \( b_1, \dots, b_\ell \).
	If \( \gamma \) is a closed curve in \( U \) homologous to zero in \( U \), and if \( a_i, b_j \not\in \Im \gamma \) for all \( i,j \), then
	\[ \frac{1}{2\pi i} \int_\gamma \frac{f'(z)}{f(z)} \dd{z} = \sum_{i=1}^k I(\gamma;a_i) \ord_f(a_i) - \sum_{j=1}^\ell I(\gamma;b_j) \ord_f(b_j) \]
\end{theorem}
\begin{proof}
	Apply the residue theorem to \( g = \frac{f'}{f} \).
	If \( z_0 \in U \) is not a pole of \( f \), then \( f \) and hence \( f' \) are holomorphic near \( z_0 \).
	If additionally \( z_0 \) is not a zero of \( f \), \( g \) is holomorphic near \( z_0 \).
	So the set of singularities of \( g \) is precisely \( \qty{a_1, \dots, a_k} \cup \qty{b_1, \dots, b_\ell} \).
	By the previous proposition, their residues are known, and the result follows.
\end{proof}
\begin{remark}
	Let \( f, \gamma \) be as in the theorem, and let \( \Gamma(t) = f(\gamma(t)) \).
	Then \( \Gamma(t) \) is a closed curve with image \( \Im \Gamma \subset \mathbb C \setminus \qty{0} \), since no zeroes or poles of \( f \) are in \( \Im \gamma \).
	Moreover, if \( [a,b] \) is the domain of \( \gamma \), we have
	\[ I(\gamma;0) = \frac{1}{2\pi i} \int_\Gamma \frac{\dd{z}}{z} = \frac{1}{2\pi i} \int_a^b \frac{\Gamma'(t)}{\Gamma(t)} \dd{t} = \frac{1}{2\pi i} \int_a^b \frac{f'(\gamma(t)) \gamma'(t)}{f(\gamma(t))} \dd{t} = \frac{1}{2 \pi i} \int_\gamma \frac{f'(z)}{f(z)} \dd{z} \]
	Thus, \( \frac{1}{2 \pi i} \int_\gamma \frac{f'(z)}{f(z)} \) is the number of times the image curve \( f \circ \gamma \) winds around zero as we move along \( \gamma \).
\end{remark}
\begin{definition}
	Let \( \Omega \) be a domain, and let \( \gamma \) be a closed curve in \( \mathbb C \).
	We say that \( \gamma \) \textit{bounds} \( \Omega \) if \( I(\gamma;w) = 1 \) for all \( w \in \Omega \), and \( I(\gamma;w) = 0 \) for all \( w \in \mathbb C \setminus \qty(\Omega \cup \Im \gamma) \).
\end{definition}
\begin{example}
	\( \partial D(0,1) \) bounds \( D(0,1) \), but does not bound \( D(0,1) \setminus \qty{0} \).
\end{example}
\begin{remark}
	If \( \gamma \) bounds \( \Omega \), then
	\begin{enumerate}[(i)]
		\item \( \Omega \) is bounded.
			Indeed, let \( D(a,R) \) such that \( \Im \gamma \subseteq D(a,R) \).
			Then \( I(\gamma;w) = 0 \) for \( w \in \mathbb C \setminus D(a,R) \).
			Since \( I(\gamma;w) = 1 \) for all \( w \in \Omega \), we must have \( \Omega \subset D(a,R) \).
		\item the topological boundary \( \partial \Omega \) is contained within \( \Im \gamma \), but it need not be the case that \( \partial \Omega = \Im \gamma \).
	\end{enumerate}
	There is a large class of closed curves that bound domains, namely, \textit{simple closed curves}, which are curves \( \gamma \colon [a,b] \to \mathbb C \) with \( \gamma(a) = \gamma(b) \), and such that \( \gamma(t_1) = \gamma(t_2) \) implies \( t_1 = t_2 \) or \( t_1, t_2 \in \qty{a,b} \).
	That a simple closed curve bounds a domain is a highly non-trivial fact guaranteed by the Jordan curve theorem: if \( \gamma \) is a simple closed curve, then \( \mathbb C \setminus \Im \gamma \) consists precisely of two connected components, one of which is bounded and the other unbounded, and moreover, \( \gamma \) (or \( -\gamma \)) bounds the bounded component, and \( \Im \gamma \) is the boundary of each of the two components.
	Thus, if \( \Omega_1 \) is the bounded component and \( \Omega_2 \) is the unbounded component, then after possibly changing the orientation of \( \gamma \), we have \( I(\gamma;w) = 1 \) for \( w \in \Omega_1 \), and \( I(\gamma;w) = 0 \) for \( w \in \Omega_2 \).
	This last assertion is simply that for any disk \( D(a,R) \supset \Im \gamma \), we have \( I(\gamma;w) = 0 \) for all \( w \in \mathbb C \setminus D(a,R) \).
\end{remark}
For a domain bounded by a closed curve, the argument principle gives the following.
\begin{corollary}
	Let \( \gamma \) be a closed curve bounding a domain \( \Omega \), and let \( f \) be meromorphic in an open set \( U \) with \( \Omega \cup \Im \gamma \subseteq U \).
	Suppose that \( f \) has no zeroes or poles on \( \Im \gamma \), and precisely \( N \) zeroes and \( P \) poles in \( \Omega \), both counted with multiplicity.
	Then \( N \) and \( P \) are finite, and
	\[ N - P = \frac{1}{2 \pi i} \int_\gamma \frac{f'(z)}{f(z)} \dd{z} = I(\Gamma;0) \]
	where \( \Gamma = f \circ \gamma \).
\end{corollary}
