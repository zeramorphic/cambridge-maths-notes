\subsection{Scattering off a potential barrier}
Consider the potential
\[
	U(x) = \begin{cases}
		0   & x \leq 0, x \geq a \\
		U_0 & 0 < x < a
	\end{cases}
\]
When \( E < U_0 \), we define
\[
	k = \sqrt{\frac{2mE}{\hbar^2}} > 0;\quad \eta = \sqrt{\frac{2m(U_0 - E)}{\hbar^2}} > 0
\]
The solution is then
\[
	\chi(x) = \begin{cases}
		e^{ikx} + Ae^{-ikx}        & x \leq 0  \\
		Be^{-\eta x} + Ce^{\eta x} & 0 < x < a \\
		De^{ikx}                   & x \geq a
	\end{cases}
\]
since we can normalise the incoming flux to one.
The boundary conditions are that \( \chi(x) = \chi'(x) \) are both continuous at \( x = 0, x = a \).
This gives four conditions, which are enough to solve the problem.
\( \chi(x) \) and its derivative at zero give
\[
	1 + A = B + C;\quad ik - ikA = -\eta B + \eta C
\]
and the continuity at \( a \) gives
\[
	B e^{-\eta a} + C e^{\eta a} = D e^{ika};\quad -\eta B e^{-\eta a} + \eta C e^{\eta a} = ikD e^{ika}
\]
Solving the system gives
\[
	D = \frac{-4 i \eta k}{(\eta-ik)^2 \exp[(\eta+ik)a] - (\eta+ik)^2\exp[-(\eta-ik)a]}
\]
The transmitted flux is \( j_{\text{tr}} = \frac{\hbar k}{m} \abs{D}^2 \) and the incident flux is \( j_{\textit{inc}} = \frac{\hbar k}{m} \).
Hence, the transmission coefficient is \( T = \abs{D}^2 \).
This is
\[
	T = \frac{4 k^2 \eta^2}{(k^2+\eta^2)^2 \sinh^2(\eta a) + 4 k^2 \eta^2}
\]
If we take the limit as \( U_0 \gg E \), we have \( \eta a \gg 1 \).
Then
\[
	T \to \frac{16k^2 \eta^2}{(\eta^2 + k^2)^2} \exp[-2\eta a] \propto \exp[-\frac{2a}{k} \sqrt{2m(U_0 - E)}]
\]
So the probability decreases exponentially with the width of the barrier.

\subsection{Harmonic oscillator}
Consider a parabolic potential
\[
	U(x) = \frac{1}{2} kx^2 = \frac{1}{2} m \omega^2 x^2
\]
where \( k \) is an elastic constant and \( \omega = \sqrt{\frac{k}{m}} \) is the angular frequency of the harmonic oscillator.
Classically, we find the solution \( x = A \cos \omega t + B \sin \omega t \).
This gives a continuous energy spectrum.
The TDSE gives
\[
	-\frac{\hbar^2}{2m} \chi''(x) + \frac{1}{2} m\omega^2 x^2 \chi(x) = E \chi(X)
\]
Since this is a bound system, we will have a discrete set of eigenvalues.
The potential is symmetric so the eigenfunctions are odd or even.
We will make the change of variables
\[
	\xi^2 = \frac{m\omega}{\hbar} x^2;\quad \varepsilon = \frac{2E}{\hbar \omega}
\]
which reformulates the TDSE as
\[
	-dv[2]{\chi}{\xi} + \xi^2 \chi = \varepsilon \chi
\]
We will start by considering the solution for \( \varepsilon = 1 \).
In this case, \( E = \frac{\hbar \omega}{2} \).
The solution in this case is
\[
	\chi_0(\xi) = \exp[-\frac{\xi^2}{2}]
\]
So the first eigenfunction, \( \chi_0 \), is known in terms of \( x \), given by
\[
	\chi_0(x) = A \exp[-\frac{m\omega}{2\hbar}x^2];\quad E_0 = \frac{\hbar \omega}{2}
\]
To find the other eigenfunctions, we will take the general form
\[
	\chi(\xi) = f(\xi) \exp[-\frac{\xi^2}{2}]
\]
This works because we know we have a bound solution and \( \chi \) must tend to zero quickly as \( \xi \) tends to infinity, due to the differential equation in terms of \( \xi, \varepsilon \).
Using the above ansatz for \( \chi \) in the Schr\"odinger equation,
\[
	-\dv[2]{f}{\xi} + 2\xi \dv{f}{\xi} + (1-\varepsilon)f = 0
\]
Note that if \( \varepsilon = 1 \), a solution is \( f = 1 \).
We can find a power series solution to this differential equation, with \( \xi = 0 \) as a regular point.
\[
	f(\xi) = \sum_{n=0}^\infty a_n \xi^n
\]
We find
\[
	\xi \dv{f}{\xi} = \sum_{n=0}^\infty n a_n \xi^n;\quad \dv[2]{f}{\xi} = \sum_{n=0}^\infty n(n-1)a_n \xi^{n-2} = \sum_{n=0}^\infty (n+1)(n+2)a_{n+2}\xi^n
\]
Comparing coefficients of \( \xi^n \),
\[
	(n+1)(n+2) a_{n+2} - 2n a_n + (\varepsilon - 1) a_n = 0
\]
Hence,
\[
	a_{n+2} = \frac{2n - \varepsilon + 1}{(n+1)(n+2)} a_n
\]
Since the function must be either even or odd, exactly one of \( a_0 \) and \( a_1 \) must be zero.
\begin{proposition}
	If the series for \( f \) does not terminate, \( \chi \) is not normalisable.
\end{proposition}
\begin{proof}
	Suppose the series does not terminate.
	We will consider the asymptotic behaviour as \( n \to \infty \).
	\[
		\frac{a_{n+2}}{a_n} \to \frac{2}{n}
	\]
	But this is the same asymptotic behaviour as the function \( g(\xi) \) given by
	\[
		g(\xi) = \exp[\xi^2] = \sum_{m=0}^\infty \frac{\xi^{2m}}{m!} = \sum_{n=0}^\infty b_n \xi^n
	\]
	with
	\[
		b_n = \begin{cases}
			\frac{1}{m!} & n = 2m     \\
			0            & n = 2m + 1
		\end{cases}
	\]
	So asymptotically,
	\[
		\frac{b_{n+2}}{b_n} = \frac{\qty(\frac{n}{2})!}{\qty(\frac{n}{2} + 1)!} = \frac{2}{n+2} \to \frac{2}{n}
	\]
	Hence \( \chi \) would have a form asymptotically equal to
	\[
		\chi(\xi) \sim \exp[\frac{\xi^2}{2}]
	\]
	Hence \( \chi(\xi) \) would be not normalisable.
\end{proof}
Hence \( f \) must be a polynomial.
So there exists \( N \) such that \( a_{N+2} = 0 \) and \( a_N \neq 0 \).
So for this value,
\[
	2N - \varepsilon + 1 = 0 \implies \varepsilon = 2N + 1
\]
By the definition of \( \varepsilon \),
\[
	E_N = \qty(N + \frac{1}{2}) \hbar \omega
\]
In particular, \( E_{N+1} - E_N = \hbar \omega \).
The eigenfunctions are
\[
	\chi_N(\xi) = f_N(\xi) \exp[-\frac{\xi^2}{2}]
\]
with the property that
\[
	\chi_N(-\xi) = (-1)^N \chi_N(\xi)
\]
\begin{align*}
	f_0(\xi) & = 1                      \\
	f_1(\xi) & = \xi                    \\
	f_2(\xi) & = 1 - 2 \xi^2            \\
	f_3(\xi) & = \xi - \frac{2}{3}\xi^3 \\
	         & \vdots
\end{align*}
