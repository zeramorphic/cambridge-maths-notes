\subsection{Wave-like behaviour of particles}
De Broglie hypothesised that any particle of any mass is associated with a wave with
\[
	\omega = \frac{E}{\hbar}; \quad \vb k = \frac{\vb p}{\hbar}
\]
This hypothesis made sense of the quantisation of electron angular momentum; if the electron lies on a circular orbit then \( 2 \pi r = n \lambda \) where \( \lambda \) is the wavelength of the electron.
However,
\[
	p = \hbar k = \hbar \frac{2 \pi}{\lambda} \implies L = m_e v r = p r = \hbar \frac{2 \pi}{\lambda} \frac{n \lambda}{2 \pi} = n \hbar
\]
Hence the angular momentum must be quantised.
The electron diffraction experiment showed that this hypothesis was true, by showing that electrons behaved sufficiently like waves.
Interference patterns were observed with \( \lambda = \frac{2 \pi}{\abs{\vb k}} = \frac{2 \pi k}{\abs{\vb p}} \) compatible with the De Broglie hypothesis.

\subsection{Wavefunctions and probabilistic interpretation}
In classical mechanics, we can describe a particle with \( \vb x, \dot{\vb x} \) or \( \vb p = m \dot{\vb x} \).
In quantum mechanics, we need the state \( \psi \) described by \( \psi(\vb x, t) \) called the wavefunction.
\begin{remark}
	Note that the state is an abstract entity, while \( \psi(\vb x, t) \) is the representation of \( \psi \) in the space of \( \vb x \).
	In some sense, \( \psi(\vb x, t) \) is the complex coefficient of \( \psi \) in the continuous basis of \( \vb x \).
	In other words, \( \psi(\vb x, t) \) is \( \psi \) in the \( \vb x \) representation.
	In this course, we always work in the \( \vb x \) representation.
\end{remark}
\begin{definition}
	A wavefunction \( \psi(\vb x, t) \colon \mathbb R^3 \to \mathbb C \) that satisfies certain mathematical properties (defined later) dictated by its physical interpretation.
	\( t \) is considered a fixed external parameter, so it is not included in the function's type.
\end{definition}
The physical interpretation of a wavefunction is called Born's rule.
The probability density for a particle to be at some point \( \vb x \) at \( t \) is given by \( \abs{\psi(\vb x, t)}^2 \).
We write the probability density as \( \rho \), hence \( \rho(\vb x, t) \dd{V} \) is the probability that the particle lies in some small volume \( V \) centred at \( \vb x \).
Now, since the particle must be somewhere, the wave function must be \textit{normalisable}, or \textit{square-integrable} in \( \mathbb R^3 \):
\[
	\int_{\mathbb R^3} \psi^\star(\vb x, t) \psi(\vb x, t) \dd{V} = \int_{\mathbb R^3} \abs{\psi(\vb x, t)}^2 \dd{V} = N \in (0, \infty)
\]
Since we want the total probability to be 1, we must normalise the wavefunction by defining
\[
	\overline{\psi}(\vb x, t) = \frac{1}{\sqrt{N}} \psi(\vb x, t) \iff \int_{\mathbb R^3} \abs{\overline{\psi}(\vb x, t)}^2 \dd{V} = 1
\]
Hence, \( \rho(\vb x,t) = \abs{\overline{\psi}(\vb x,t)}^2 \) really is a probability density.
From now, we will not use the bar for denoting normalisation, since normalisation is evident from context.
