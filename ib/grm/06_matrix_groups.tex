\subsection{Definitions}
\begin{definition}
	Let \( F \) be a field, such as \( \mathbb C \) or \( \faktor{\mathbb Z}{p \mathbb Z} \).
	Let \( GL_n(F) \) be set of \( n \times n \) invertible matrices over \( F \), which is called the \textit{general linear group}.
	Let \( SL_n(F) \) be set of \( n \times n \) matrices with unit determinant over \( F \), which is called the \textit{special linear group}.
	\( SL_n(F) \) is the kernel of the determinant homomorphism on \( GL_n(F) \), so \( SL_n(F) \triangleleft GL_n(F) \).
	
	Let \( Z \triangleleft GL_n(F) \) denote the subgroup of \textit{scalar matrices}, the group of non-zero multiples of the identity.
	The group \( PGL_n(F) = \faktor{GL_n(F)}{Z} \) is called the \textit{projective general linear group}.
	Let \( PSL_n(F) = \faktor{SL_n(F)}{Z \cap SL_n(F)} \).
	By the second isomorphism theorem, \( PSL_n(F) \) is isomorphic to \( \faktor{Z \cdot SL_n(F)}{Z} \), which is a subgroup of \( PGL_n(F) \).
\end{definition}
\begin{example}
	Consider the finite group \( G = GL_n\qty(\faktor{\mathbb Z}{p\mathbb Z}) \).
	A list of \( n \) vectors in \( \faktor{\mathbb Z}{p\mathbb Z} \) are the columns of a matrix \( A \in G \) if and only if the vectors are linearly independent.
	Hence, by considering dimensionality of subspaces generated by each column,
	\[ \abs{G} = (p^n - 1)(p^n - p)(p^n - p^2) \cdots (p^n - p^{n-1}) = p^{1+2+\dots+(n-1)} (p^n - 1)(p^{n-1} - 1) \cdots (p - 1) = p^{\binom{n}{2}} \prod_{i=1}^n (p^i - 1) \]
	Hence the Sylow \( p \)-subgroups have size \( p^{\binom{n}{2}} \).
	Let \( U \) be the set of upper triangular matrices with unit diagonal.
	This forms a Sylow \( p \)-subgroup of \( G \), since there are \( \binom{n}{2} \) entries in a given upper triangular matrix, and there are \( p \) choices for such an entry.
\end{example}

\subsection{M\"obius maps in modular arithmetic}
Recall that \( PGL_2(\mathbb C) \) acts on \( \mathbb C \cup \qty{\infty} \) by M\"obius transformations.
Likewise, \( PGL_2\qty(\faktor{\mathbb Z}{p\mathbb Z}) \) acts on \( \faktor{\mathbb Z}{p\mathbb Z} \cup \qty{\infty} \) by M\"obius transformations.
For a matrix
\[ A = \begin{pmatrix}
	a & b \\
	c & d
\end{pmatrix} \in GL_2\qty(\faktor{\mathbb Z}{p\mathbb Z});\quad A \colon z \mapsto \frac{az+b}{cz+d} \]
Since the scalar matrices act trivially, we obtain an action on the projective general linear group instead of the general linear group.
We can represent \( \infty \) as an integer, say, \( p \), for the purposes of constructing a permutation representation.
\begin{lemma}
	The permutation representation \( PGL_2\qty(\faktor{\mathbb Z}{p\mathbb Z}) \to S_{p+1} \) is injective (and is an isomorphism if \( p = 2 \) or \( p = 3 \)).
\end{lemma}
\begin{proof}
	Suppose that \( \frac{az+b}{cz+d} = z \) for all \( z \in \faktor{\mathbb Z}{p\mathbb Z} \cup \qty{\infty} \).
	Since \( z = 0 \), we have \( b = 0 \).
	Since \( z = \infty \), we find \( c = 0 \).
	Thus the matrix is diagonal.
	Finally, since \( z = 1 \), \( \frac{a}{d} = 1 \) hence \( a = d \).
	Thus the matrix is scalar.
	So the permutation representation from \( PGL_2\qty(\faktor{\mathbb Z}{p \mathbb Z}) \) has trivial kernel, giving injectivity as required.

	If \( p = 2 \) or \( p = 3 \) we can compute the orders of relevant groups manually and show that the permutation representation is an isomorphism.
\end{proof}
