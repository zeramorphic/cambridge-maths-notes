\subsection{Definitions}
\begin{definition}
	A \textit{ring} is a triple \( (R, +, \cdot) \) where \( R \) is a set and \( +, \cdot \) are binary operations \( R \times R \to R \), satisfying the following axioms.
	\begin{enumerate}[(i)]
		\item \( (R, +) \) is an abelian group, and we will denote the identity element 0 and the inverse of \( x \) as \( -x \);
		\item \( (R, \cdot) \) satisfies the group axioms except for the invertibility axiom, and we will denote the identity element 1 and the inverse of \( x \) as \( x^{-1} \);
		\item for all \( x, y, z \in R \) we have \( x \cdot (y+z) = x\cdot y + x\cdot z \) and \( (y+z)\cdot x = y \cdot x + z \cdot x \).
	\end{enumerate}
	If multiplication is commutative, we say that \( R \) is a \textit{commutative} ring.
	In this course, we will study only commutative rings.
\end{definition}
\begin{remark}
	For all \( x \in R \),
	\[ 0 \cdot x = (0+0) \cdot x = 0 \cdot x + 0 \cdot x \implies 0 \cdot x = 0 \]
	Further,
	\[ 0 = 0 \cdot x = (1 + -1) \cdot x = x + (-1 \cdot x) \implies -1 \cdot x = -x \]
\end{remark}
\begin{definition}
	A subset \( S \subseteq R \) is a \textit{subring}, denoted \( S \leq R \), if \( (S, +, \cdot) \) is a ring with the same identity elements.
\end{definition}
\begin{remark}
	It suffices to check the closure axioms for addition and multiplication; the other properties are inherited.
\end{remark}
\begin{example}
	\( \mathbb Z \leq \mathbb Q \leq \mathbb R \leq \mathbb C \) are rings.
	The set \( \mathbb Z[i] = \qty{a+bi \colon a,b\in\mathbb Z} \) is a subring of \( \mathbb C \).
	This is known as the ring of Gaussian integers.
	The set \( \mathbb Q[\sqrt{2}] = \qty{a+b\sqrt{2} \colon a,b \in\mathbb Q} \) is a subring of \( \mathbb R \).
\end{example}
\begin{example}
	The set \( \faktor{\mathbb Z}{n\mathbb Z} \) is a ring.
\end{example}
\begin{example}
	Let \( R, S \) be rings.
	Then the \textit{product} \( R \times S \) is a ring under the binary operations
	\[ (a,b) + (c,d) = (a+c,b+d);\quad (a,b) \cdot (c,d) = (a\cdot c,b\cdot d) \]
	The additive identity is \( (0_R, 0_S) \) and the multiplicative identity is \( (1_R, 1_S) \).
	Note that the subset \( R \times \qty{0} \) is preserved under addition and multiplication, so it is a ring, but it is not a subring because the multiplicative identity is different.
\end{example}

\subsection{Polynomials}
\begin{definition}
	Let \( R \) be a ring.
	A \textit{polynomial} \( f \) over \( R \) is an expression
	\[ f = a_0 + a_1 X + a_2 X^2 + \dots + a_n X^n \]
	for \( a_i \in R \).
	The term \( X \) is a formal symbol, no substitution of \( X \) for a value will be made.
	We could alternatively define polynomials as finite sequences of terms in \( R \).
	The \textit{degree} of a polynomial \( f \) is the largest \( n \) such that \( a_n \neq 0 \).
	A degree-n polynomial is \textit{monic} if \( a_n = 1 \).
	We write \( R[X] \) for the set of all such polynomials over \( R \).
	Let \( g = b_0 + b_1 X + \dots + b_n X^n \).
	Then we define
	\[ f + g = (a_0 + b_0) + (a_1 + b_1) X + \dots + (a_n + b_n) X^n;\quad f \cdot g = \sum_i \qty(\sum_{j=0}^i a_j b_{i-j}) X^i \]
	Then \( (R[X], +, \cdot) \) is a ring.
	The identity elements are the constant polynomials \( 0 \) and \( 1 \).
	We can identify the ring \( R \) with the subring of \( R[X] \) of constant polynomials.
\end{definition}
\begin{definition}
	An element \( r \in R \) is a \textit{unit} if \( r \) has a multiplicative inverse.
	The units in a ring, denoted \( R^\times \), form an abelian group under multiplication.
	For instance, \( \mathbb Z^\times = \qty{\pm 1} \) and \( \mathbb Q^\times = \mathbb Q \setminus \qty{0} \).
\end{definition}
\begin{definition}
	A \textit{field} is a ring where all nonzero elements are units and \( 0 \neq 1 \).
\end{definition}
\begin{example}
	\( \faktor{\mathbb Z}{n\mathbb Z} \) is a field only if \( n \) is a prime.
\end{example}
\begin{remark}
	If \( R \) is a ring such that \( 0 = 1 \), then every element in the ring is equal to zero.
	Indeed, \( x = 1\cdot x = 0\cdot x = 0 \).
	Thus, the exclusion of rings with \( 0 = 1 \) in the definition of a field simply excludes the trivial ring.
\end{remark}
\begin{proposition}
	Let \( f, g \in R[X] \) such that the leading coefficient of \( g \) is a unit.
	Then there exist polynomials \( q, r \in R[X] \) such that \( f = qg + r \), where the degree of \( r \) is less than the degree of \( g \).
\end{proposition}
\begin{remark}
	This is the Euclidean algorithm for division, adapted to polynomial rings.
\end{remark}
\begin{proof}
	Let \( n \) be the degree of \( f \) and \( m \) be the degree of \( g \), so
	\[ f = a_n X^n + \dots + a_0;\quad g = b_m X^m + \dots + b_0 \]
	By assumption, \( b_m \in R^\times \).
	If \( n < m \) then let \( q = 0 \) and \( r = f \).
	Conversely, we have \( n \geq m \).
	Consider the polynomial \( f_1 = f - a_n b_m^{-1} g X^{n-m} \).
	This has degree at most \( n - 1 \).
	Hence, we can use induction on \( n \) to decompose \( f_1 \) as \( f_1 = q_1 g + r \).
	Thus \( f = (q_1 + a_n b_m^{-1} X^{n-m}) g + r \) as required.
\end{proof}
\begin{remark}
	If \( R \) is a field, then every nonzero element of \( R \) is a unit.
	Therefore, the above algorithm can be applied for all polynomials \( g \) unless \( g \) is the constant polynomial zero.
\end{remark}
\begin{example}
	Let \( R \) be a ring and \( X \) be a set.
	Then the set of functions \( X \to R \) is a ring under
	\[ (f + g)(x) = f(x) + g(x);\quad (f \cdot g)(x) = f(x) \cdot g(x) \]
	The set of continuous functions \( \mathbb R \to \mathbb R \) is a subring of the ring of all functions \( \mathbb R \to \mathbb R \), since they are closed under addition and multiplication.
	The set of polynomial functions \( \mathbb R \to \mathbb R \) is also a subring, and we can identify this with the ring \( \mathbb R[X] \).
\end{example}
\begin{example}
	Let \( R \) be a ring.
	Then the \textit{power series ring} \( R\Brackets{X} \) is the set of power series on \( X \).
	This is defined similarly to the polynomial ring, but we permit infinitely many nonzero elements in the expansion.
	The power series is defined formally; we cannot actually carry out infinitely many additions in an arbitrary ring.
	We instead consider the power series as a sequence of numbers.
\end{example}
\begin{example}
	Let \( R \) be a ring.
	Then the ring of \textit{Laurent polynomials} is \( R[X,X^{-1}] \) with the restriction that \( a_i \neq 0 \) for finitely many \( i \).
\end{example}

\subsection{Homomorphisms}
\begin{definition}
	Let \( R \) and \( S \) be rings.
	A function \( \varphi \colon R \to S \) is a \textit{ring homomorphism} if
	\begin{enumerate}[(i)]
		\item \( \varphi(r_1 + r_2) = \varphi(r_1) + \varphi(r_2) \);
		\item \( \varphi(r_1 \cdot r_2) = \varphi(r_1) \cdot \varphi(r_2) \);
		\item \( \varphi(1_R) = 1_S \).
	\end{enumerate}
	We can derive that \( \varphi(0_R) = 0_S \) from (i).

	A ring homomorphism is an \textit{isomorphism} if it is bijective.
	The \textit{kernel} of a ring homomorphism is \( \ker \varphi = \qty{r \in R : \varphi(r) = 0} \).
\end{definition}
\begin{lemma}
	Let \( R, S \) be rings.
	Then a ring homomorphism \( \varphi \colon R \to S \) is injective if and only if \( \ker \varphi = \qty{0} \).
\end{lemma}
\begin{proof}
	Let \( \varphi \colon (R, +) \to (S, +) \) be the induced group homomorphism on addition.
	The result then follows from the corresponding fact about group homomorphisms.
\end{proof}

\subsection{Ideals}
\begin{definition}
	A subset \( I \subseteq R \) is an \textit{ideal}, written \( I \trianglelefteq R \), if
	\begin{enumerate}[(i)]
		\item \( I \) is a subgroup of \( (R, +) \);
		\item if \( r \in R \) and \( x \in I \), then \( rx \in I \).
	\end{enumerate}
	We say that an ideal is \textit{proper} if \( I \neq R \).
\end{definition}
\begin{lemma}
	Let \( \varphi \colon R \to S \) be a ring homomorphism.
	Then \( \ker \varphi \) is an ideal of \( R \).
\end{lemma}
\begin{proof}
	We know that \( \ker \varphi \) is a subgroup by the equivalent fact from groups.
	If \( r \in R \) and \( x \in \ker \varphi \), then
	\[ \varphi(rx) = \varphi(r) \varphi(x) = \varphi(r) \cdot 0 = 0 \]
	Hence \( rx \in \ker \varphi \).
\end{proof}
\begin{remark}
	If \( I \) contains a unit, then the multiplicative identity lies in \( I \).
	Then all elements lie in \( I \).
	In particular, if \( I \) is a proper ideal, \( 1 \not\in I \).
	Hence a proper ideal \( I \) is not a subring of \( R \).
\end{remark}
\begin{lemma}
	The ideals in \( \mathbb Z \) are precisely the subsets of the form \( n\mathbb Z \) for any \( n = 0, 1, 2, \dots \).
\end{lemma}
\begin{proof}
	First, we can check directly that any subset of the form \( n\mathbb Z \) is an ideal.
	Now, let \( I \) be any nonzero ideal of \( \mathbb Z \) and let \( n \) be the smallest positive element.
	Then \( n\mathbb Z \subseteq I \).
	Let \( m \in I \).
	Then by the Euclidean algorithm, \( m = qn+r \) for \( q,r \in \mathbb Z \) and \( r \in \qty{0, 1, \dots, n-1} \).
	Then \( r = m - qn \).
	We know \( qn \in I \) since \( n \in I \), so \( r \in I \).
	If \( r \neq 0 \), this contradicts the minimality of \( n \) as chosen above.
	So \( I = n\mathbb Z \) exactly.
\end{proof}
\begin{definition}
	For an element \( a \in R \), we write \( (a) \) to denote the subset of \( R \) given by multiples of \( a \); that is, \( (a) = \qty{ra \colon r\in R} \).
	This is an ideal, known as the ideal \textit{generated by \( a \)}.
	More generally, if \( a_1, \dots, a_n \in R \), then \( (a_1, \dots, a_n) \) is the set of elements in \( R \) given by linear combinations of the \( a_i \).
	This is also an ideal.
\end{definition}
\begin{definition}
	Let \( I \trianglelefteq R \).
	Then \( I \) is \textit{principal} if there exists some \( a \in R \) such that \( I = (a) \).
\end{definition}

\subsection{Quotients}
\begin{theorem}
	Let \( I \trianglelefteq R \).
	Then the set \( \faktor{R}{I} \) of cosets of \( I \) in \( (R, +) \) forms the \textit{quotient ring} under the operations
	\[ (r_1 + I) + (r_2 + I) = (r_1 + r_2) + I;\quad (r_1 + I) \cdot (r_2 + I) = (r_1 \cdot r_2) + I \]
	This ring has the identity elements
	\[ 0_{\faktor{R}{I}} = 0_R + I;\quad 1_{\faktor{R}{I}} = 1_R + I \]
	Further, the map \( R \to \faktor{R}{I} \) defined by \( r \mapsto r + I \) is a ring homomorphism called the \textit{quotient map}.
	The kernel of the quotient map is \( I \).
	Hence any ideal is the kernel of some homomorphism.
\end{theorem}
\begin{proof}
	From the analogous result from groups, the addition defined on the set of cosets yields the group \( \qty(\faktor{R}{I}, +) \).
	If \( r_1 + I = r_1' + I \) and \( r_2 + I = r_2' + I \), then \( r_1' = r_1 + a_1 \) and \( r_2' = r_2 + a_2 \) for some \( a_1, a_2 \in I \).
	Then
	\[ r_1' r_2' = (r_1 + a_1)(r_2 + a_2) = r_1 r_2 + a_1 r_2 + r_1 a_2 + a_1 a_2 \]
	Hence \( (r_1' r_2') + I = (r_1 r_2) + I \).
	The remainder of the proof is trivial.
\end{proof}
\begin{example}
	In the integers \( \mathbb Z \), the ideals are \( n\mathbb Z \).
	Hence we can form the quotient ring \( \faktor{\mathbb Z}{n\mathbb Z} \).
	The ring \( \faktor{\mathbb Z}{n\mathbb Z} \) has elements \( n\mathbb Z, 1 + n\mathbb Z, \dots, (n-1) + n\mathbb Z \).
	Addition and multiplication behave like in modular arithmetic modulo \( n \).
\end{example}
\begin{example}
	Consider the ideal \( (X) \) inside the polynomial ring \( \mathbb C[X] \).
	This ideal is the set of polynomials with zero constant term.
	Let \( f(X) = a_n X^n + \dots + a_0 \) be an arbitrary element of \( \mathbb C[X] \).
	Then \( f(X) + X = a_0 + X \).
	Thus, there exists a bijection between \( \faktor{\mathbb C[X]}{(X)} \) and \( \mathbb C \), defined by \( f(x) + (X) \mapsto f(0) \), with inverse \( a \mapsto a + (X) \).
	This bijection is a ring homomorphism, hence \( \faktor{\mathbb C[X]}{(X)} \cong \mathbb C \).
\end{example}
\begin{example}
	Consider \( (X^2 + 1) \triangleleft \mathbb R[X] \).
	For \( f(X) = a_n X^n + \dots + a_0 \in \mathbb R[X] \), we can apply the Euclidean algorithm to write \( f(X) \) as \( q(X) (X^2 + 1) + r(X) \) where the degree of \( r \) is less than two.
	Hence \( r(X) = a+bX \) for some real numbers \( a \) and \( b \).
	Thus, any element of \( \faktor{\mathbb R[X]}{(X^2 + 1)} \) can be written \( a+bX + (X^2 + 1) \).
	Suppose a coset can be represented by two representatives: \( a+bX + (X^2+1) = a' + b'X + (X^2 + 1) \).
	Then,
	\[ a+bX - a' - b'X = (a-a') - (b-b')X = g(X) (X^2 + 1) \]
	Hence \( g(X) = 0 \), giving \( a-a' = 0 \) and \( b-b' = 0 \).
	Hence the coset representative is unique.
	Consider the bijection \( \varphi \) between this quotient ring and the complex numbers given by \( a+bX + (X^2 + 1) \mapsto a+bi \).
	We can show that \( \varphi \) is a ring homomorphism.
	Indeed, it preserves addition, and \( 1 + (X^2 + 1) \mapsto 1 \), so it suffices to check that multiplication is preserved.
	\begin{align*}
		\varphi((a+bX + (X^2 + 1)) \cdot (c+dX + (X^2 + 1))) &= \varphi((a+bX)(c+dX) + (X^2 + 1)) \\
		&= \varphi(ac + (ad + bc)X + bd (X^2 + 1) - bd + (X^2 + 1)) \\
		&= \varphi(ac-bd+(ad+bc)X + (X^2 + 1)) \\
		&= ac - bd + (ad + bc)i \\
		&= (a+bi)(c+di) \\
		&= \varphi((a+bX) + (X^2 + 1)) \varphi((c+dX) + (X^2 + 1))
	\end{align*}
	Thus \( \faktor{\mathbb R[X]}{(X^2 + 1)} \cong \mathbb C \).
\end{example}

\subsection{Isomorphism theorems}
\begin{theorem}[first isomorphism theorem]
	Let \( \varphi \colon R \to S \) be a ring homomorphism.
	Then,
	\[ \ker \varphi \triangleleft R;\quad \Im \varphi \leq S;\quad \faktor{R}{\ker \varphi} \cong \Im \varphi \]
\end{theorem}
\begin{proof}
	We have \( \ker \varphi \triangleleft R \) from above.
	We know that \( \Im \varphi \leq (S, +) \).
	Now we show that \( \Im \varphi \) is closed under multiplication.
	\[ \varphi(r_1) \varphi(r_2) = \varphi(r_1 r_2) \in \Im \varphi \]
	Finally,
	\[ 1_S = \varphi(1_R) \in \Im \varphi \]
	Hence \( \Im \varphi \) is a subring of \( S \).
	Let \( K = \ker \varphi \).
	Then, we define \( \Phi \colon \faktor{R}{K} \to \Im \varphi \) by \( r+K \mapsto r \).
	By appealing to the first isomorphism theorem from groups, this is well-defined, a bijection, and a group homomorphism under addition.
	It therefore suffices to show that \( \Phi \) preserves multiplication and maps the multiplicative identities to each other.
	\[ \Phi(1_R + K) = \varphi(1_R) = 1_S;\quad \Phi((r_1+K)(r_2+K)) = \Phi(r_1 r_2 +K) = \varphi(r_1 r_2) = \varphi(r_1) \varphi(r_2) \]
	The result follows as required.
\end{proof}
\begin{theorem}[second isomorphism theorem]
	Let \( R \leq S \) and \( J \triangleleft S \).
	Then,
	\[ R \cap J \triangleleft R;\quad R + J = \qty{r+a \colon r\in R, a \in J} \leq S;\quad \faktor{R}{R \cap J} \cong \faktor{R+J}{J} \leq \faktor{S}{J} \]
\end{theorem}
\begin{proof}
	By the second isomorphism theorem for groups, \( R+J \leq (S, +) \).
	Further, \( 1_S = 1_S + 0_S \), and since \( R \) is a subring, \( 1_S + 0_S \in R + J \) hence \( 1_S \in R \cap J \).
	If \( r_1, r_2 \in R \) and \( a_1, a_2 \in J \), we have
	\[ (r_1 + a_1)(r_2 + a_2) = \underbrace{r_1 r_2}_{\in R} + \underbrace{r_1 a_2}_{\in J} + \underbrace{r_2 a_1}_{\in J} + \underbrace{r_2 a_2}_{\in J} \in R + J \]
	Hence \( R+J \) is closed under multiplication, giving \( R+J \leq S \).

	Let \( \varphi \colon R \to \faktor{S}{J} \) be defined by \( r \mapsto r + J \).
	This is a ring homomorphism, since it is the composite of the inclusion homomorphism \( R \subseteq S \) and the quotient map \( S \to \faktor{S}{J} \).
	The kernel of \( \varphi \) is the set \( \qty{r \in R \colon r+J = J} = R \cap J \).
	Since this is the kernel of a ring homomorphism, \( R \cap J \) is an ideal in \( R \).
	The image of \( \varphi \) is \( \qty{r+J \mid r \in R} = \faktor{R+J}{J} \leq \faktor{S}{J} \).
	By the first isomorphism theorem, \( \faktor{R}{R \cap J} \cong \faktor{R+J}{J} \) as required.
\end{proof}
\begin{remark}
	If \( I \triangleleft R \), there exists a bijection between ideals in \( \faktor{R}{I} \) and the ideals of \( R \) containing \( I \).
	Explicitly,
	\[ K \mapsto \qty{r \in R \mid r + I \in K};\quad J \mapsto \faktor{J}{I} \]
\end{remark}
\begin{theorem}[third isomorphism theorem]
	Let \( I \triangleleft R \) and \( J \triangleleft R \) with \( I \subseteq J \).
	Then,
	\[ \faktor{J}{I} \triangleleft \faktor{R}{I};\quad \faktor{R/I}{J/I} \cong \faktor{R}{J} \]
\end{theorem}
\begin{proof}
	Let \( \varphi \colon \faktor{R}{I} \to \faktor{R}{J} \) defined by \( r+I \mapsto r+J \).
	We can check that this is a surjective ring homomorphism by considering the third isomorphism theorem for groups.
	Its kernel is \( \qty{r + I \colon r \in J} = \faktor{J}{I} \), which is an ideal in \( \faktor{R}{I} \), and we conclude by use of the first isomorphism theorem.
\end{proof}
% TODO: note: J/I is not a quotient ring structure, it's a set of cosets
\begin{example}
	Consider the surjective ring homomorphism \( \varphi \colon \mathbb R[X] \to \mathbb C \) which is defined by
	\[ f = \sum_n a_n X^n \mapsto f(i) = \sum_n a_n i^n \]
	Its kernel can be found by the Euclidean algorithm, yielding \( \ker \varphi = (X^2 + 1) \).
	Applying the first isomorphism theorem, we immediately find \( \faktor{\mathbb R[X]}{(X^2 + 1)} \cong \mathbb C \).
\end{example}
\begin{example}
	Let \( R \) be a ring.
	Then there exists a unique ring homomorphism \( i \colon \mathbb Z \to R \).
	Indeed, we must have
	\[ 0_{\mathbb Z} \mapsto 0_R;\quad 1_{\mathbb Z} \mapsto 1_R \]
	This inductively defines
	\[ n \mapsto \underbrace{1_R + \dots + 1_R}_{n \text{ times}} \]
	The negative integers are also uniquely defined, since any ring homomorphism is a group homomorphism.
	\[ -n \mapsto -(\underbrace{1_R + \dots + 1_R}_{n \text{ times}}) \]
	We can show that any such construction is a ring homomorphism as required.
	Then, the kernel of the ring homomorphism is an ideal of \( \mathbb Z \), hence it is \( n\mathbb Z \) for some \( n \).
	Hence, by the first isomorphism theorem, any ring contains a copy of \( \faktor{\mathbb Z}{n\mathbb Z} \), since it is isomorphic to the image of \( i \).
	If \( n = 0 \), then the ring contains a copy of \( \mathbb Z \) itself, and if \( n = 1 \), then the ring is trivial since \( 0 = 1 \).
	The number \( n \) is known as the \textit{characteristic} of \( R \).

	For example, \( \mathbb Z, \mathbb Q, \mathbb R, \mathbb C \) have characteristic zero.
	The rings \( \faktor{\mathbb Z}{p \mathbb Z}, \faktor{\mathbb Z}{p \mathbb Z}[X] \) have characteristic \( p \).
\end{example}

\subsection{Integral domains}
\begin{definition}
	An \textit{integral domain} is a ring \( R \) with \( 0 \neq 1 \) such that for all \( a, b \in R \), \( a b = 0 \) implies \( a = 0 \) or \( b = 0 \).
	A \textit{zero divisor} in a ring \( R \) is a nonzero element \( a \in R \) such that there exists \( b \in R \) for some nonzero \( b \in R \).
	A ring is an integral domain if and only if it has no zero divisors.
\end{definition}
\begin{example}
	All fields are integral domains.
	Any subring of an integral domain is an integral domain.
	For instance, \( \mathbb Z[i] \leq \mathbb C \) is an integral domain.
\end{example}
\begin{example}
	The ring \( \mathbb Z \times \mathbb Z \) is not an integral domain.
	Indeed, \( (1,0) \cdot (0,1) = (0,0) \).
\end{example}
\begin{lemma}
	Let \( R \) be an integral domain.
	Then \( R[X] \) is an integral domain.
\end{lemma}
\begin{proof}
	We will show that any two nonzero elements produce a nonzero element.
	In particular, let
	\[ f = \sum_n a_n X^n;\quad g = \sum_n b_n X^n \]
	Since these are nonzero, the leading coefficients \( a_n \) and \( b_m \) are nonzero.
	Here, the leading term of the product \( fg \) has form \( a_n b_m X^{n+m} \).
	Since \( R \) is an integral domain, \( a_n b_m \neq 0 \), so \( fg \) is nonzero.
	Further, the degree of \( fg \) is \( n + m \), the sum of the degrees of \( f \) and \( g \).
\end{proof}
\begin{lemma}
	Let \( R \) be an integral domain, and \( f \neq 0 \) be a nonzero polynomial in \( R[X] \).
	We define \( \mathrm{roots}(f) = \qty{a \in R \colon f(a) = 0} \).
	Then \( \abs{\mathrm{roots}(f)} \leq \mathrm{deg}(f) \).
\end{lemma}
\begin{proof}
	Exercise on the example sheets.
\end{proof}
\begin{theorem}
	Let \( F \) be a field.
	Then any finite subgroup \( G \) of \( (F^\times, \cdot) \) is cyclic.
\end{theorem}
\begin{proof}
	\( G \) is a finite abelian group.
	If \( G \) is not cyclic, we can apply a previous structure theorem for finite abelian groups to show that there exists \( H \leq G \) such that \( H \cong C_{d_1} \times C_{d_1} \) for some integer \( d_1 \geq 2 \).
	The polynomial \( f(X) = X^{d_1} - 1 \in F[X] \) has degree \( d_1 \), but has at least \( d_1^2 \) roots, since any element of \( H \) is a root.
	This contradict the previous lemma.
\end{proof}
\begin{example}
	\( \qty(\faktor{\mathbb Z}{p\mathbb Z})^\times \) is cyclic.
\end{example}
\begin{proposition}
	Any finite integral domain is a field.
\end{proposition}
\begin{proof}
	Let \( 0 \neq a \in R \), where \( R \) is an integral domain.
	Consider the map \( \varphi \colon R \to R \) given by \( x \mapsto ax \).
	If \( \varphi(x) = \varphi(y) \), then \( a(x-y) = 0 \).
	But \( a \neq 0 \), hence \( x - y = 0 \).
	Hence \( \varphi \) is injective.
	Since \( R \) is finite, \( \varphi \) is a bijection, hence it has an inverse \( \varphi^{-1} \), which yields the multiplicative inverse of \( a \) by considering \( \varphi^{-1}(a) \).
	This may be repeated for all \( a \).
\end{proof}
\begin{theorem}
	Any integral domain \( R \) is a subring of a field \( F \), and every element of \( F \) can be written in the form \( ab^{-1} \) where \( a, b \in R \) and \( b \neq 0 \).
	Such a field \( F \) is called the \textit{field of fractions} of \( R \).
\end{theorem}
\begin{proof}
	Consider the set \( S = \qty{(a,b) \in R \colon b \neq 0} \).
	We can define an equivalence relation
	\[ (a,b) \sim (c,d) \iff ad = bc \]
	This is reflexive and commutative.
	We can show directly that it is transitive.
	\begin{align*}
		(a,b) \sim (c,d) \sim (e,f) &\implies ad = bc;\; cf = de \\
		&\implies adf = bcf = bde \\
		&\implies af = be \\
		&\implies (a,b) \sim (e,f)
	\end{align*}
	Hence \( \sim \) is indeed an equivalence relation.
	Now, let \( F = \faktor{S}{\sim} \), and we write \( \frac{a}{b} \) for the class \( [(a,b)] \).
	We define the ring operations
	\[ \frac{a}{b} + \frac{c}{d} = \frac{ad + bc}{bd};\quad \frac{a}{b} + \frac{c}{d} = \frac{ac}{bd} \]
	These can be shown to be well-defined.
	Thus, \( F \) is a ring with identities \( 0_F = \frac{0_R}{1_R} \) and \( 1_F = \frac{1_R}{1_R} \).
	If \( \frac{a}{b} \neq 0_F \), then \( a \neq 0 \).
	Thus, \( \frac{b}{a} \) exists, and \( \frac{a}{b} \cdot \frac{b}{a} = 0 \).
	Hence \( F \) is a field.

	We can identify \( R \) with the subring of \( F \) given by \( \frac{r}{1} \) for all \( r \in R \).
	This is clearly isomorphic to \( R \).
	Further, any element of \( F \) can be written as \( \frac{a}{b} = ab^{-1} \) as required.
\end{proof}
This is analogous to the construction of the rationals using the integers.
\begin{example}
	Consider \( \mathbb C[X] \).
	This has field of fractions \( \mathbb C(X) \), called the field of \textit{rational functions} in \( X \).
\end{example}

\subsection{Maximal ideals}
\begin{definition}
	An ideal \( I \triangleleft R \) is \textit{maximal} if \( I \neq R \) and, if \( I \subseteq J \triangleleft R \), we have \( J = I \) or \( J = R \).
\end{definition}
\begin{lemma}
	A nonzero ring \( R \) is a field if and only if its only ideals are zero or \( R \).
\end{lemma}
\begin{proof}
	Suppose \( R \) is a field.
	If \( 0 \neq I \triangleleft R \), then \( I \) contains a nonzero element, which is a unit since \( R \) is a field.
	Hence \( I = R \).

	Now, suppose a ring \( R \) has ideals that are only zero or \( R \).
	If \( 0 \neq x \in R \), consider \( (x) \).
	This is nonzero since it contains \( x \).
	By assumption, \( (x) = R \).
	Thus, the element 1 lies in \( R \).
	Hence, there exists \( y \in R \) such that \( xy = 1 \), and hence this \( y \) is the multiplicative inverse as required.
\end{proof}
\begin{proposition}
	Let \( I \triangleleft R \).
	Then \( I \) is maximal if and only if \( \faktor{R}{I} \) is a field.
\end{proposition}
\begin{proof}
	\( \faktor{R}{I} \) is a field if and only if its ideals are either zero, denoted \( \faktor{I}{I} \), or \( \faktor{R}{I} \) itself.
	By the correspondence theorem, \( I \) and \( R \) are the only ideals in \( R \) which contain \( I \).
	Equivalently, \( I \triangleleft R \) is maximal.
\end{proof}

\subsection{Prime ideals}
\begin{definition}
	An ideal \( I \triangleleft R \) is \textit{prime} if \( I \neq R \) and, for all \( a,b \in R \) such that \( ab \in I \), we have \( a \in I \) or \( b \in I \).
\end{definition}
\begin{example}
	The ideals in the integers are \( (n) \) for some \( n \geq 0 \).
	\( n\mathbb Z \) is a prime ideal if and only if \( n \) is prime or zero.
	The case for \( n = 0 \) is trivial.
	If \( n \neq 0 \) we can use the property that \( p \mid ab \) implies either \( p \mid a \) or \( p \mid b \).
	Conversely, if \( n \) is composite, we can write \( n = uv \) for \( u, v > 1 \).
	Then \( uv \in n\mathbb Z \) but \( u,v \not\in n\mathbb Z \).
\end{example}
\begin{proposition}
	Let \( I \triangleleft R \).
	Then \( I \) is prime if and only if \( \faktor{R}{I} \) is an integral domain.
\end{proposition}
\begin{proof}
	If \( I \) is prime, then for all \( a,b \in I \) we have \( a \in I \) or \( b \in I \).
	Equivalently, for all \( a + I, b + I \in \faktor{R}{I} \), we have \( (a+I)(b+I) = 0+I \) if \( a+I = 0+I \) or \( b+I = 0+I \).
	This is the definition of an integral domain.
\end{proof}
\begin{remark}
	If \( I \) is a maximal ideal, then \( \faktor{R}{I} \) is a field.
	A field is an integral domain.
	Hence any maximal ideal is prime.
\end{remark}
\begin{remark}
	If the characteristic of a ring is \( n \), then \( \faktor{\mathbb Z}{n\mathbb Z} \leq R \).
	In particular, if \( R \) is an integral domain, then \( \faktor{\mathbb Z}{n\mathbb Z} \) must be an integral domain.
	Equivalently, \( n\mathbb Z \triangleleft \mathbb Z \) is a prime ideal.
	Hence \( n \) is zero or prime.
	Thus, in an integral domain, the characteristic must either be zero or prime.
	A field always has a characteristic, which is either zero (in which case it contains \( \mathbb Z \) and hence \( \mathbb Q \)) or prime (in which case it contains \( \faktor{\mathbb Z}{p\mathbb Z} = \mathbb F_p \) which is already a field).
\end{remark}
