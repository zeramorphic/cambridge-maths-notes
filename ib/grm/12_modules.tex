\subsection{Definitions}
\begin{definition}
	Let \( R \) be a ring.
	A \textit{module over \( R \)} is a triple \( (M, +, \cdot) \) consisting of a set \( M \) and two operations \( + \colon M \times M \to M \) and \( \cdot \colon R \times M \to M \), that satisfy
	\begin{enumerate}[(i)]
		\item \( (M, +) \) is an abelian group with identity \( 0 = 0_M \);
		\item \( (r_1 + r_2) \cdot m = r_1 \cdot m + r_2 \cdot m \);
		\item \( r \cdot (m_1 + m_2) = r \cdot m_1 + r \cdot m_2 \);
		\item \( r_1 \cdot (r_2 \cdot m) = (r_1 \cdot r_2) \cdot m \);
		\item \( 1_R \cdot m = m \);
	\end{enumerate}
\end{definition}
\begin{remark}
	Closure is implicitly required by the types of the \( + \) and \( \cdot \) operations.
\end{remark}
\begin{example}
	A module over a field is precisely a vector space.

	A \( \mathbb Z \)-module is precisely the same as an abelian group, since
	\[ \cdot \colon \mathbb Z \times A \to A;\quad n \cdot a = \begin{cases}
		\underbrace{a + \dots + a}_{n \text{ times}} & \text{if } n > 0 \\
		0 & n = \text{if } 0 \\
		-\qty(\underbrace{a + \dots + a}_{-n \text{ times}}) & \text{if } n < 0
	\end{cases} \]

	Let \( F \) be a field, and \( V \) be a vector space over \( F \).
	Let \( \alpha \colon V \to V \) be an endomorphism.
	We can turn \( V \) into an \( F[X] \)-module by
	\[ \cdot \colon F[X] \times V \to V;\quad f \cdot v = (f(\alpha))(v) \]
	Note that the structure of the \( F[X] \)-module depends on the choice of \( \alpha \).
	We can write \( V = V_\alpha \) to disambiguate.

	For any ring \( R \), we can consider \( R^n \) as an \( R \)-module via
	\[ r \cdot (r_1, \dots, r_n) = (r \cdot r_1, \dots, r \cdot r_n) \]
	In particular, the case \( n = 1 \) shows that any ring \( R \) can be considered an \( R \)-module where the scalar multiplication in the ring and the module agree.
\end{example}
