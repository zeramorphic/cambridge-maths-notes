\chapter[Groups, Rings and Modules \\ \textnormal{\emph{Lectured in Lent \oldstylenums{2022} by \textsc{Dr.\ R.\ Zhou}}}]{Groups, Rings and Modules}
\emph{\Large Lectured in Lent \oldstylenums{2022} by \textsc{Dr.\ R.\ Zhou}}

A ring is a an algebraic structure with an addition and multiplication operation.
Common examples of rings include \( \mathbb Z, \mathbb Q, \mathbb R, \mathbb C \), the Gaussian integers \( \mathbb Z[i] = \qty{a + bi \mid a, b \in \mathbb Z} \), the quotient \( \faktor{\mathbb Z}{n\mathbb Z} \), and the set of polynomials with complex coefficients.
We can study factorisation in a general ring, generalising the idea of factorising integers or polynomials.
Certain rings, called unique factorisation domains, have the property like the integers that every nonzero non-invertible element can be expressed as a unique product of irreducibles (in \( \mathbb Z \), the irreducibles are the prime numbers).
This property, and many others, are studied in this course.

Modules are like vector spaces, but instead of being defined over a field, they are defined over an arbitrary ring.
In particular, every vector space is a module, because every field is a ring.
We use the theory built up over the course to prove that every \( n \)-dimensional complex matrix can be written in Jordan normal form.

\subfile{../../ib/grm/main.tex}
