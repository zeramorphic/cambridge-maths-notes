\subsection{???}
Recall that if \( \Sigma \) is a topological surface, any point lies in an open neighbourhood homeomorphic to a disk.
\begin{definition}
	A pair \( (U, \varphi) \), where \( U \) is an open set in \( \Sigma \) and \( \varphi \colon U \to V \) is a homeomorphism to an open set \( V \subseteq \mathbb R^2 \), is called a \textit{chart} for \( \Sigma \).
	If \( p \in U \), we might say that \( (U, \varphi) \) is a chart for \( \Sigma \) \textit{at \( p \)}.
	A collection of charts whose domains cover \( \Sigma \) is known as an \textit{atlas} for \( \Sigma \).
	The inverse \( \sigma = \varphi^{-1} \colon V \to U \) is known as a \textit{local parametrisation} for the surface.
\end{definition}
\begin{example}
	If \( Z \subseteq \mathbb R^2 \) is closed, \( \mathbb R^2 \setminus Z \) is a topological surface with an atlas containing one chart, \( (\mathbb R^2 \setminus Z, \phi = \id) \).

	For \( S^2 \), there is an atlas with two charts, which are the two stereographic projections from the poles.
	We could consider alternative charts, for instance the projection to the \( yz \) plane, but this would be insufficient for describing the poles.
\end{example}
\begin{definition}
	Let \( (U_i, \varphi_i) \) be charts containing the point \( p \in \Sigma \), for \( i = 1, 2 \).
	Then the map
	\[ \ast \colon \varphi_1(U_1 \cap U_2) \to \varphi_2(U_1 \cap U_2);\quad \ast = \varphi_2 \circ \eval{\varphi_1^{-1}}_{\varphi_1(U_1 \cap U_2)} \]
	converts betwen the corresponding charts, and is called a \textit{transition map}.
	This is a homeomorphism of open sets in \( \mathbb R^2 \).
\end{definition}
Recall from Analysis and Topology that if \( V \subseteq \mathbb R^n \) and \( V' \subseteq \mathbb R^m \) are open, then a continuous map \( f \colon V \to V' \) is called \textit{smooth} if it is infinitely differentiable.
Equivalently, it is smooth if partial derivatives of all orders in all variables exist at all points.
If \( n = m \), then in particular the homeomorphism \( f \colon V \to V' \) is called a \textit{diffeomorphism} if it is smooth and has smooth inverse.
\begin{definition}
	An \textit{abstract smooth surface} is a topological space \( \Sigma \) together with an atlas of charts \( (U_i, \varphi_i) \) such that all transition maps \( \varphi_i \circ \varphi_j^{-1} \colon \varphi_j(U_i \cap U_j) \to \varphi_i(U_i \cap U_j) \) are diffeomorphisms.
\end{definition}
\begin{remark}
	We could not simply consider a smoothness condition for \( \Sigma \) itself without appealing to atlases, since \( \Sigma \) is an arbtrary topological space and could have almost any topology.
\end{remark}
\begin{example}
	The atlas of two charts with stereographic projections gives \( S^2 \) the structure of an abstract smooth surface.

	For the torus \( T^2 = \faktor{\mathbb R^2}{\mathbb Z^2} \), we can find charts of all points by choosing sufficiently small disks in \( \mathbb R^2 \) such that they do not intersect any of their non-trivial integer translates.
	The transition maps for this atlas are all translations of \( \mathbb R^2 \).
	Hence \( T^2 \) inherits the structure of an abstract smooth surface.
\end{example}
