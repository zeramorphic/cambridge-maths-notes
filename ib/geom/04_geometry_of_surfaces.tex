\subsection{Length}
Let \( \gamma \colon (a,b) \to \mathbb R^3 \) be smooth.
The \textit{length} of \( \gamma \) is
\[ L(\gamma) = \int_a^b \norm{\gamma'(t)} \dd{t} \]
This result is independent of the choice of parametrisation.
Let \( s \colon (A,B) \to (a,b) \) be a monotonically increasing function, and let \( \tau(t) = \gamma(s(t)) \).
Then
\[ L(\tau) = \int_A^B \norm{\tau'(t)} \dd{t} = \int_A^B \norm{\gamma(s(t))} \abs{s'(t)} \dd{t} = \int_a^b \norm{\gamma(t')} \dd{t'} = L(\gamma) \]
\begin{lemma}
	If \( \gamma \colon (a,b) \to \mathbb R^3 \) is continuously differentiable and \( \gamma'(t) \neq 0 \), then \( \gamma \) can be parametrised by arc length.
\end{lemma}
The proof is left as an exercise.
Let \( \Sigma \) be a smooth surface in \( \mathbb R^3 \), and let \( \sigma \colon V \to U \subseteq \Sigma \) be an allowable parametrisation.
If \( \gamma \colon (a,b) \to \mathbb R^3 \) is smooth and its image is contained within \( U \), then there exist functions \( (u(t), v(t)) \colon (a,b) \to V \) such that \( \gamma(t) = \sigma(u(t), v(t)) \).
Hence \( \gamma'(t) = \sigma_u u'(t) + \sigma_v v'(t) \), giving
\[ \norm{\gamma'(t)}^2 = E u'(t)^2 + 2F u'(t) v'(t) + G v'(t)^2 \]
for functions
\[ E = \inner{\sigma_u, \sigma_u};\quad F = \inner{\sigma_u, \sigma_v} = \inner{\sigma_v, \sigma_u};\quad G = \inner{\sigma_v, \sigma_v} \]
where \( \inner{\wildcard,\wildcard} \) represents the usual Euclidean inner product.
Note that \( E, F, G \) depend only on \( \sigma \) and not on \( \gamma \).
\begin{definition}
	The \textit{first fundamental form} of \( \Sigma \) in the parametrisation \( \sigma \) is the expression
	\[ E \dd{u}^2 + 2F \dd{u} \dd{v} + G \dd{v}^2 \]
	This notation is illustrative of the fact that if \( \gamma \) has image in the image of \( \sigma(v) \), we find
	\[ L(\gamma) = \int_a^b \sqrt{E (u')^2 + 2F u'v' + G (v')^2} \dd{t} \]
	where \( \gamma(t) = \sigma(u(t),v(t)) \).
\end{definition}
\begin{remark}
	The Euclidean inner product on \( \mathbb R^3 \) provides an inner product on the subspace \( T_p \Sigma \).
	Choosing a parametrisation \( \sigma \), we can say \( T_p \Sigma = \Im D \eval{\sigma}_0 = \vecspan{\qty{\sigma_u, \sigma_v}} \) where \( \sigma(0) = p \).
	The first fundamental form is a symmetric bilinear form on the tangent spaces \( T_p \Sigma \), varying smoothly in \( p \).
	However, we choose to express this in a basis coming from the parametrisation \( \sigma \).
	In particular, we can think about the matrix expression
	\[ \begin{pmatrix}
		E & F \\
		F & G
	\end{pmatrix} \]
\end{remark}
\begin{example}
	The plane \( \mathbb R^2_{xy} \subset \mathbb R^3 \) has the parametrisation \( (u,v) \mapsto (u,v,0) \).
	Hence, \( \sigma_u = e_1 \) and \( \sigma_v = e_2 \), hence the first fundamental form is \( \dd{u}^2 + \dd{v}^2 \).
	We could also use polar coordinates, using \( \sigma(r,\theta) = (r\cos\theta,r\sin\theta,0) \).
	This parametrises the plane without the origin.
	This gives \( \sigma_r = (\cos\theta,\sin\theta,0) \) and \( \sigma_\theta = (-r\sin\theta, r\cos\theta,0) \).
	The first fundamental form is \( \dd{r}^2 + r^2 \dd{\theta}^2 \).
\end{example}
