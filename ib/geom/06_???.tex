\subsection{???}
\begin{definition}
	Let \( V \subseteq \mathbb R^2 \) be an open set.
	An \textit{(abstract) Riemannian metric} is a smooth map from \( V \) to the set of positive definite symmetric bilinear forms, given by
	\[ v \mapsto \begin{pmatrix}
		E(v) & F(v) \\
		F(v) & G(v)
	\end{pmatrix} \]
	such that \( E > 0 \), \( G > 0 \), \( EG - F^2 > 0 \).
	The image of this map can be viewed as an open subset of \( \mathbb R^4 \).
\end{definition}
If \( v \) is a vector at \( p \in V \), we can compute its infinitesimal length by
\[ \norm{v}^2 = v^\transpose \begin{pmatrix}
	E(v) & F(v) \\
	F(v) & G(v)
\end{pmatrix} v \]
Thus, if \( \gamma \colon [a,b] \to V \) is smooth,
\[ \mathrm{length}(\gamma) = \int_a^b \qty( E \dot u^2 + 2F \dot u \dot v + G \dot v^2 )^{\frac{1}{2}} \dd{t} \]
where \( \gamma(t) = (u(t),v(t)) \).
\begin{definition}
	Let \( \Sigma \) be an abstract smooth surface, so \( \Sigma = \bigcup_{i \in I} U_i \) for open sets \( U_i \), with charts \( \varphi_i \colon U_i \to V_i \subseteq \mathbb R^2 \) which are homeomorphisms, and with smooth transition maps \( \varphi_i \varphi_j^{-1} \colon \varphi_j(U_i \cap U_j) \to \varphi_i(U_i \cap U_j) \).
	A \textit{Riemannian metric} on \( \Sigma \), usually called \( g \) or \( \dd{s}^2 \), is a choice of Riemannian metric in the above sense on each \( V_i \), which are compatible in the following sense.
	Let \( \sigma = \varphi_i^{-1} \) and \( \widetilde \sigma = \varphi_j^{-1} \) for some \( i,j \), and define \( f = \widetilde \sigma^{-1} \circ \sigma \).
	Then we require 
	\[ (Df)^\transpose \begin{pmatrix}
		\widetilde E & \widetilde F \\
		\widetilde F & \widetilde G
	\end{pmatrix} (Df) = \begin{pmatrix}
		E & F \\
		F & G
	\end{pmatrix} \]
	So \( \dd{f} \) defines an isometry from an open set in the chart \( (U, \varphi(U) = V) \) to one in \( \qty(\widetilde U, \widetilde \varphi\qty(\widetilde U) = \widetilde V) \).
\end{definition}
This compatibility condition is the transition law for first fundamental forms for smooth surfaces in \( \mathbb R^3 \).
\begin{example}
	Recall the torus \( T^2 = \faktor{\mathbb R^2}{\mathbb Z^2} \).
	\begin{center}
		\tikzfig{torus_polygon}
	\end{center}
	We have an atlas of charts for which the transition maps are the restrictions of translations of open subsets of \( \mathbb R^2 \).
	For each \( V_i \subseteq \mathbb R^2 \), we associate the natural Euclidean metric \( \dd{u}^2 + \dd{v}^2 \).
	If \( f \) is a translation, \( Df \) is the identity, and so
	\[ (Df)^\transpose I (Df) = I \]
	holds trivially.
	So this gives a global Riemannian metric on \( T^2 \).
	This metric is flat, since it is locally isometric to \( \mathbb R^2 \) at all points.

	Conversely, consider the torus of revolution embedded in \( \mathbb R^3 \).
	As a compact smooth surface in \( \mathbb R^3 \), it must contain an elliptic point.
	Hence, the flat Riemannian metric described above is not the same (up to isometry) as the metric obtained by any possible embedding of the torus in \( \mathbb R^3 \).
\end{example}
