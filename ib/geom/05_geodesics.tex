\subsection{Definitions}
Recall that we defined, for a smooth curve \( \gamma \colon [a,b] \to \mathbb R^3 \),
\[ \mathrm{length}(\gamma) = \int_a^b \norm{\gamma'(t)} \dd{t} \]
\begin{definition}
	The \textit{energy} of \( \gamma \) is given by
	\[ E(\gamma) = \int_a^b \norm{\gamma'(t)}^2 \dd{t} \]
\end{definition}
\begin{definition}
	Let \( \gamma \colon [a,b] \to \Sigma \), where \( \Sigma \) is a smooth surface in \( \mathbb R^3 \).
	A \textit{one-parameter variation} (with fixed endpoints) of \( \gamma \) is a smooth map \( \Gamma \colon (-\varepsilon, \varepsilon) \times [a,b] \to \Sigma \), such that if \( \gamma_s = \Gamma(s,\wildcard) \), then
	\( \gamma_0(t) = \gamma(t) \), and \( \gamma_s(a) \) and \( \gamma_s(b) \) are independent of \( s \).
\end{definition}
\begin{definition}
	A smooth curve \( \gamma \colon [a,b] \to \Sigma \) is a \textit{geodesic} if, for every variation \( (\gamma_s) \) of \( \gamma \) with fixed endpoints as above, we have \( \eval{\dv{s}}_{s=0} E(\gamma_s) = 0 \).
	Alternatively, \( \gamma \) is a critical point of the energy functional on curves from \( \gamma(a) \) to \( \gamma(b) \).
\end{definition}

\subsection{The geodesic equations}
Let \( \gamma \) have image contained within the image of an allowable parametrisation \( \sigma \colon V \to U \).
Then, for sufficiently small \( s \), we can write \( \gamma_s(t) = \sigma(u(s,t), v(s,t)) \).
Suppose that the first fundamental form, with respect to \( \sigma \), is
\[ E \dd{u}^2 + 2F \dd{u} \dd{v} + G \dd{v}^2 \]
Let
\[ R = E \dot u^2 + 2F \dot u \dot v + G \dot v^2 \]
By definition,
\[ E(\gamma_s) = \int_a^b R \dd{t} \]
where \( R \) depends on \( s \).
Hence,
\begin{align*}
	\pdv{R}{s} &= \qty(E_u \dot u^2 + 2F_u \dot u \dot v + G_u \dot v^2) \pdv{u}{s} + \qty(E_v \dot v^2 + 2F_v \dot u \dot v + G_v \dot v^2) \pdv{v}{s} \\
	&+ 2 (E \dot u + F \dot v) \pdv{\dot u}{s} + 2(F \dot u + G \dot v) \pdv{\dot v}{s}
\end{align*}
This gives
\[ \dv{s} E(\gamma_s) = \int_a^b \pdv{R}{s} \dd{t} \]
We can integrate by parts.
Note that \( \pdv{u}{s} \) and \( \pdv{v}{s} \) vanish at \( a,b \).
Hence,
\[ \eval{dv{s}}_{s=0} E(\gamma_s) = \int_a^b \qty(A \pdv{u}{s} + B \pdv{v}{s}) \dd{t} \]
where
\begin{align*}
	A &= E_u \dot u^2 + 2F_u \dot u \dot v + G_u \dot v^2 - 2 \pdv{t} \qty(E \dot u + F \dot v) \\
	B &= E_v \dot u^2 + 2F_v \dot u \dot v + G_v \dot v^2 - 2 \pdv{t} \qty(F \dot u + G \dot v)
\end{align*}
\begin{corollary}
	A smooth curve \( \gamma \colon [a,b] \to \Sigma \) with image in \( \Im \sigma \) is a geodesic if and only if it satisfies the \textit{geodesic equations}:
	\begin{align*}
		\dv{t} \qty(E \dot u + F \dot v) &= \frac{1}{2} \qty( E_u \dot u^2 + 2F_u \dot u \dot v + G_u \dot v^2 ) \\
		\dv{t} \qty(F \dot u + G \dot v) &= \frac{1}{2} \qty( E_v \dot u^2 + 2F_v \dot u \dot v + G_v \dot v^2 )
	\end{align*}
	Note that these equations are evaluated at \( s = 0 \), so no choice of variation is required.
\end{corollary}
\begin{remark}
	Solving a differential equation is a local procedure.
	The original definition of the geodesic seems to be a global property.
	However, we can always consider a sub-curve of \( \gamma \) to also be a geodesic, since its variations are variations of \( \gamma \).
	So the definition can be thought of as local.

	Energy is sensitive to reparametrisation.
	If \( f, g \colon [a,b] \to \mathbb R \) are smooth, the Cauchy-Schwarz inequality gives that
	\[ \qty(\int_a^b fg \dd{t})^2 \leq \int_a^b f^2 \dd{t} \cdot \int_a^b g^2 \dd{t} \]
	Let us apply this to \( f = \sqrt{R} \), \( g = 1 \) to find
	\[ \mathrm{length}(\gamma)^2 \leq E(\gamma)(b-a) \]
	Since equality holds only when the two functions are proportional, we must have that \( \norm{\gamma'(t)} \) is constant for the equality to hold.
	In other words, \( \gamma \) must be parametrised proportional to arc length.
\end{remark}
\begin{corollary}
	If \( \gamma \) has constant speed and locally minimises length, then it is a geodesic.
	Further, if \( \gamma \) globally minimises energy, then it must globally minimise length, and is parametrised with constant speed.
\end{corollary}
\begin{remark}
	We would like geodesics to be a local property, but not necessarily global length minimisers.
	For example, all arcs of great circles will be shown to be geodesics, even if large arcs are not global length minimisers between fixed endpoints.
\end{remark}

\subsection{Geodesics on the plane}
The plane \( \mathbb R^2 \) has parametrisation \( \sigma(u,v) = (u,v,0) \) and first fundamental form \( \dd{u}^2 + \dd{v}^2 \).
The geodesic equations here are
\[ \ddot u = 0;\quad \ddot v = 0 \]
In particular, the geodesics on the plane are given by
\[ u(t) = \alpha t + \beta;\quad v(t) = \gamma t + \delta \]
This is a straight line, parametrised at constant speed.

\subsection{Geodesics on the sphere}
Consider the unit sphere with parametrisation
\[ \sigma(u,v) = (\cos u \cos v, \cos u \sin v, \sin u);\quad u \in \qty(-\frac{\pi}{2});\quad v \in \qty(0, 2 \pi) \]
This has first fundamental form
\[ \dd{u}^2 + \cos^2 u \dd{v}^2 \implies E = 1;\quad F = 0;\quad G = \cos^2 u \]
The geodesic equations give
\[ \dv{t} \qty(\dot u) = \frac{1}{2} 2\cos u \sin u \dot v^2;\quad \dv{t} \qty(\cos^2 u \dot v) = 0 \]
This gives
\[ \ddot u + \sin u \cos u \dot v^2 = 0;\quad \ddot v - 2 \tan u \dot u \dot v = 0 \]
Since geodesics are parametrised at constant speed, we can assume that it is parametrised at unit speed without loss of generality.
\[ \norm{\gamma'(t)} = 1 \implies \dot u + \cos^2 u \dot v^2 = 1 \]
Hence,
\[ \frac{\ddot v}{\dot v} = 2 \tan u \dot u \implies \ln \dot v = -2 \ln \cos u + \text{constant} \implies \dot v = \frac{C}{\cos^2 u} \]
Substituting into the unit speed equation,
\[ \dot u^2 = 1 - \frac{C}{\cos^2 u} \implies \dot u = \sqrt{\frac{\cos^2 u - C^2}{\cos^2 u}} \]
Then,
\[ \frac{\dot v}{\dot u} = \dv{v}{u} = \frac{C}{\cos u \sqrt{\cos^2 u - C^2}} \]
Hence,
\[ v = \int \dv{v}{u} \dd{u} = \int \frac{C \sec^2 u}{\sqrt{1-C^2 sec^2 u}} \dd{u} \]
Using the substitution \( w = \frac{C \tan u}{\sqrt{1-C^2}} \), we find
\[ v = \int \frac{w}{1-w^2} \dd{w} = \arcsin w + \text{constant} = \arcsin(\lambda \tan u) + \delta \]
for some constants \( \lambda, \delta \).
