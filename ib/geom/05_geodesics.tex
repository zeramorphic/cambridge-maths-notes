\subsection{Definitions}
Recall that we defined, for a smooth curve \( \gamma \colon [a,b] \to \mathbb R^3 \),
\[ \mathrm{length}(\gamma) = \int_a^b \norm{\gamma'(t)} \dd{t} \]
\begin{definition}
	The \textit{energy} of \( \gamma \) is given by
	\[ E(\gamma) = \int_a^b \norm{\gamma'(t)}^2 \dd{t} \]
\end{definition}
\begin{definition}
	Let \( \gamma \colon [a,b] \to \Sigma \), where \( \Sigma \) is a smooth surface in \( \mathbb R^3 \).
	A \textit{one-parameter variation} (with fixed endpoints) of \( \gamma \) is a smooth map \( \Gamma \colon (-\varepsilon, \varepsilon) \times [a,b] \to \Sigma \), such that if \( \gamma_s = \Gamma(s,\wildcard) \), then
	\( \gamma_0(t) = \gamma(t) \), and \( \gamma_s(a) \) and \( \gamma_s(b) \) are independent of \( s \).
\end{definition}
\begin{definition}
	A smooth curve \( \gamma \colon [a,b] \to \Sigma \) is a \textit{geodesic} if, for every variation \( (\gamma_s) \) of \( \gamma \) with fixed endpoints as above, we have \( \eval{\dv{s}}_{s=0} E(\gamma_s) = 0 \).
	Alternatively, \( \gamma \) is a critical point of the energy functional on curves from \( \gamma(a) \) to \( \gamma(b) \).
\end{definition}

\subsection{The geodesic equations}
Let \( \gamma \) have image contained within the image of an allowable parametrisation \( \sigma \colon V \to U \).
Then, for sufficiently small \( s \), we can write \( \gamma_s(t) = \sigma(u(s,t), v(s,t)) \).
Suppose that the first fundamental form, with respect to \( \sigma \), is
\[ E \dd{u}^2 + 2F \dd{u} \dd{v} + G \dd{v}^2 \]
Let
\[ R = E \dot u^2 + 2F \dot u \dot v + G \dot v^2 \]
By definition,
\[ E(\gamma_s) = \int_a^b R \dd{t} \]
where \( R \) depends on \( s \).
Hence,
\begin{align*}
	\pdv{R}{s} &= \qty(E_u \dot u^2 + 2F_u \dot u \dot v + G_u \dot v^2) \pdv{u}{s} + \qty(E_v \dot v^2 + 2F_v \dot u \dot v + G_v \dot v^2) \pdv{v}{s} \\
	&+ 2 (E \dot u + F \dot v) \pdv{\dot u}{s} + 2(F \dot u + G \dot v) \pdv{\dot v}{s}
\end{align*}
This gives
\[ \dv{s} E(\gamma_s) = \int_a^b \pdv{R}{s} \dd{t} \]
We can integrate by parts.
Note that \( \pdv{u}{s} \) and \( \pdv{v}{s} \) vanish at \( a,b \).
Hence,
\[ \eval{dv{s}}_{s=0} E(\gamma_s) = \int_a^b \qty(A \pdv{u}{s} + B \pdv{v}{s}) \dd{t} \]
where
\begin{align*}
	A &= E_u \dot u^2 + 2F_u \dot u \dot v + G_u \dot v^2 - 2 \pdv{t} \qty(E \dot u + F \dot v) \\
	B &= E_v \dot u^2 + 2F_v \dot u \dot v + G_v \dot v^2 - 2 \pdv{t} \qty(F \dot u + G \dot v)
\end{align*}
\begin{corollary}
	A smooth curve \( \gamma \colon [a,b] \to \Sigma \) with image in \( \Im \sigma \) is a geodesic if and only if it satisfies the \textit{geodesic equations}:
	\begin{align*}
		\dv{t} \qty(E \dot u + F \dot v) &= \frac{1}{2} \qty( E_u \dot u^2 + 2F_u \dot u \dot v + G_u \dot v^2 ) \\
		\dv{t} \qty(F \dot u + G \dot v) &= \frac{1}{2} \qty( E_v \dot u^2 + 2F_v \dot u \dot v + G_v \dot v^2 )
	\end{align*}
	Note that these equations are evaluated at \( s = 0 \), so no choice of variation is required.
\end{corollary}
\begin{remark}
	Solving a differential equation is a local procedure.
	The original definition of the geodesic seems to be a global property.
	However, we can always consider a sub-curve of \( \gamma \) to also be a geodesic, since its variations are variations of \( \gamma \).
	So the definition can be thought of as local.

	Energy is sensitive to reparametrisation.
	If \( f, g \colon [a,b] \to \mathbb R \) are smooth, the Cauchy-Schwarz inequality gives that
	\[ \qty(\int_a^b fg \dd{t})^2 \leq \int_a^b f^2 \dd{t} \cdot \int_a^b g^2 \dd{t} \]
	Let us apply this to \( f = \sqrt{R} \), \( g = 1 \) to find
	\[ \mathrm{length}(\gamma)^2 \leq E(\gamma)(b-a) \]
	Since equality holds only when the two functions are proportional, we must have that \( \norm{\gamma'(t)} \) is constant for the equality to hold.
	In other words, \( \gamma \) must be parametrised proportional to arc length.
\end{remark}
\begin{corollary}
	If \( \gamma \) has constant speed and locally minimises length, then it is a geodesic.
	Further, if \( \gamma \) globally minimises energy, then it must globally minimise length, and is parametrised with constant speed.
\end{corollary}
\begin{remark}
	We would like geodesics to be a local property, but not necessarily global length minimisers.
	For example, all arcs of great circles will be shown to be geodesics, even if large arcs are not global length minimisers between fixed endpoints.
\end{remark}

\subsection{Geodesics on the plane}
The plane \( \mathbb R^2 \) has parametrisation \( \sigma(u,v) = (u,v,0) \) and first fundamental form \( \dd{u}^2 + \dd{v}^2 \).
The geodesic equations here are
\[ \ddot u = 0;\quad \ddot v = 0 \]
In particular, the geodesics on the plane are given by
\[ u(t) = \alpha t + \beta;\quad v(t) = \gamma t + \delta \]
This is a straight line, parametrised at constant speed.

\subsection{Geodesics on the sphere}
Consider the unit sphere with parametrisation
\[ \sigma(u,v) = (\cos u \cos v, \cos u \sin v, \sin u);\quad u \in \qty(-\frac{\pi}{2});\quad v \in \qty(0, 2 \pi) \]
This has first fundamental form
\[ \dd{u}^2 + \cos^2 u \dd{v}^2 \implies E = 1;\; F = 0;\; G = \cos^2 u \]
The geodesic equations give
\[ \dv{t} \qty(\dot u) = \frac{1}{2} 2\cos u \sin u \dot v^2;\quad \dv{t} \qty(\cos^2 u \dot v) = 0 \]
This gives
\[ \ddot u + \sin u \cos u \dot v^2 = 0;\quad \ddot v - 2 \tan u \dot u \dot v = 0 \]
Since geodesics are parametrised at constant speed, we can assume that it is parametrised at unit speed without loss of generality.
\[ \norm{\gamma'(t)} = 1 \implies \dot u + \cos^2 u \dot v^2 = 1 \]
Hence,
\[ \frac{\ddot v}{\dot v} = 2 \tan u \dot u \implies \ln \dot v = -2 \ln \cos u + \text{constant} \implies \dot v = \frac{C}{\cos^2 u} \]
Substituting into the unit speed equation,
\[ \dot u^2 = 1 - \frac{C^2}{\cos^2 u} \implies \dot u = \sqrt{\frac{\cos^2 u - C^2}{\cos^2 u}} \]
Then,
\[ \frac{\dot v}{\dot u} = \dv{v}{u} = \frac{C}{\cos u \sqrt{\cos^2 u - C^2}} \]
Hence,
\[ v = \int \dv{v}{u} \dd{u} = \int \frac{C \sec^2 u}{\sqrt{1-C^2 sec^2 u}} \dd{u} \]
Using the substitution \( w = \frac{C \tan u}{\sqrt{1-C^2}} \), we find
\[ v = \int \frac{w}{1-w^2} \dd{w} = \arcsin w + \text{constant} = \arcsin(\lambda \tan u) + \delta \]
for some constants \( \lambda, \delta \).
Hence,
\[ \sin (v - \delta) = \lambda \tan u \]
Rewriting using the angle addition formula,
\[ \underbrace{(\sin v \cos u)}_{x}\cos \delta - \underbrace{(\cos v \cos u)}_{y}\sin \delta - \lambda \underbrace{\sin u}_z = 0 \]
Hence, the geodesic \( \gamma \) lies on a plane through the origin, since this is a linear equation in \( x, y, z \).
Such planes intersect the sphere in great circles.

\subsection{Geodesics on the torus}
Consider the surface of revolution of a circle in the \( xz \)-plane centred at \( (a,0,0) \) about the \( z \) axis, giving a torus.
An allowable parametrisation for this surface is
\[ \sigma(u,v) = ((a+\cos u)\cos v, (a+\cos u)\sin v, \sin u) \]
The first fundamental form is
\[ \dd{u}^2 + (a+\cos u)^2 \dd{v}^2 \implies E = 1;\;F = 0;\;G = (a+\cos u)^2 \]
Note that if we were to take \( a = 0 \), we would arrive at the unit sphere and its first fundamental form.
We can follow the same procedure as above with the sphere, or formally replace \( \cos u \) with \( a+\cos u \) in the result.
\[ \dv{v}{u} = \frac{C}{(a+\cos u)\sqrt{(a+\cos u)^2 - C^2}} \]
which cannot be integrated using classical functions.
This leads to the study of elliptic functions.

\subsection{Equivalent characterisation of geodesics}
We have so far restricted our analysis to the first fundamental form, without considering its embedding in \( \mathbb R^3 \).
Intuitively, we know that straight lines in \( \mathbb R^2 \) are not just locally shortest but also locally straightest.
We would expect this to hold for other surfaces as well.
We can charaterise this notion via stating that the change in the tangent vector to a curve is as small as it could be, subject to the constraint that it lies on the surface.

\begin{proposition}
	Let \( \Sigma \) be a smooth surface in \( \mathbb R^3 \).
	A smooth curve \( \gamma \colon [a,b] \to \Sigma \) is a geodesic if and only if \( \ddot \gamma(t) \) is everywhere normal to the surface \( \Sigma \).
\end{proposition}
\begin{remark}
	This proposition makes use of the tangent plane, a notion that exists only because we have an embedding in \( \mathbb R^3 \).
	Note that
	\[ \dd{t} \inner{\dot \gamma, \dot \gamma} = 2 \inner{\underbrace{\dot \gamma}_{\text{tangent to } \Sigma}, \underbrace{\ddot \gamma}_{\text{normal to } \Sigma}} = 0 \]
	Hence, \( \inner{\dot \gamma, \dot \gamma} \) is constant, giving that geodesics are parametrised proportional to arc length.
\end{remark}
\begin{proof}
	The property of being a geodesic as we previously defined is a local property, and so is the condition in the proposition.
	Hence, we may work entirely within an allowable parametrisation \( \sigma \colon V \to U \).
	Suppose \( \gamma(t) = \sigma(u(t), v(t)) \).
	Hence,
	\[ \dot \gamma = \sigma_u \dot u + \sigma_v \dot v \]
	\( \ddot \gamma \) is normal to \( \Sigma \) when it is orthogonal to the tangent plane, which is spanned by \( \sigma_u, \sigma_v \).
	This is true if and only if
	\[ \inner{ \dv{t} \qty(\sigma_u \dot u + \sigma_v \dot v), \sigma_u } = 0 = \inner{ \dv{t} \qty(\sigma_u \dot u + \sigma_v \dot v), \sigma_v } \]
	We will prove the first equality.
	This can be rewritten
	\[ \dv{t} \inner{\sigma_u \dot u + \sigma_v \dot v, \sigma_u} - \inner{\sigma_u \dot u + \sigma_v \dot v, \dv{t} \sigma_u} = 0 \]
	Note that \( \inner{\sigma_u, \sigma_u} = E \) and \( \inner{\sigma_u, \sigma_v} = F \).
	\[ \dv{t} (E \dot u + F \dot v) - \inner{\sigma_u \dot u + \sigma_v \dot v, \sigma_{uu} \dot u + \sigma_{uv} \dot v} = 0 \]
	Hence,
	\[ \dv{t} (E \dot u + F \dot v) - \qty[ \dot u^2 \inner{\sigma_u, \sigma_{uu}} + \dot u \dot v \qty(\inner{\sigma_u, \sigma_{uv}} + \inner{\sigma_v, \sigma_{uu}}) + \dot v^2 \inner{\sigma_v \sigma_{uv}} ] = 0 \]
	Note that \( E_u = 2 \inner{\sigma_u, \sigma_{uu}} \), \( F_u = \inner{\sigma_u, \sigma_{uv}} + \inner{\sigma_v, \sigma_{uu}} \), and \( G_u = 2\inner{\sigma_v, \sigma_{uv}} \).
	This gives
	\[ \dv{t} (E \dot u + F \dot v) = \frac{1}{2} \qty(E_u \dot u^2 + 2 F_u \dot u \dot v + G_u \dot v^2) \]
	which is the first of the geodesic equations.
	By symmetry, we find the second geodesic equation similarly.
\end{proof}

\subsection{Planes of symmetry}
Let \( \Sigma \) be a smooth surface in \( \mathbb R^3 \) such that there exists a plane \( \pi \subseteq \mathbb R^3 \) such that \( \pi \cap \Sigma \) is a smooth embedded curve \( C \subseteq \Sigma \), and \( \Sigma \) is setwise preserved by reflection in the plane \( \pi \).
We will show that \( C \) is a geodesic when parametrised at constant speed.
Consider a point \( p \) on \( C \).
We can think of \( \mathbb R^3 = \pi \oplus \pi^\perp \), where we change coordinates such that \( p \) is the origin.
We can also write \( \mathbb R^3 = T_p \Sigma \oplus \mathbb R n_p \), where \( \mathbb R n_p \) is the vector subspace of \( \mathbb R^3 \) generated by \( n_p \).
Clearly, reflection in \( \pi \) acts on \( \pi \) by the identity, and on \( \pi^\perp \) by \( -1 \).
Since reflection in \( \pi \) fixes \( \Sigma \) setwise and fixes \( p \), it must also preserve the subspace \( T_p \Sigma \).
Hence it also preserves \( \mathbb R n_p \), so \( \mathbb R n_p \subseteq \pi \), since \( \pi \) is not the identity on \( T_p \Sigma \).
Now, let us parametrise \( C \) locally near \( p \) using \( t \mapsto \gamma(t) \in C \) at constant speed.
Since \( \gamma(t) \subseteq \pi \), we have \( \dot \gamma(t), \ddot \gamma(t) \in \pi \).
\( \gamma \) has constant speed, so \( \inner{\dot \gamma, \ddot \gamma} = 0 \).
Hence \( \dot \gamma \) lies in \( \pi \cap T_p \Sigma \) and \( \ddot \gamma \) is orthogonal to this and lies in \( \pi \), so lies in \( \mathbb R n_p \subseteq \pi \).
Hence \( \gamma \) is indeed a geodesic.

In particular, arcs of great circles are geodesics, since they lie in planes of symmetry.

\subsection{Surfaces of revolution}
Consider the surface of revolution given by \( \eta(u) = (f(u), 0, g(u)) \) where \( \eta \) is smooth and injective, and \( f(u) > 0 \), rotated about the \( z \) axis.
\begin{definition}
	A circle obtained by rotating a point of \( \eta \) is called a \textit{parallel}.
	A curve optained by rotating \( \eta \) itself by a fixed angle about the \( z \) axis is called a \textit{meridian}.
\end{definition}
A plane in \( \mathbb R^3 \) containing the \( z \) axis is a plane of symmetry, hence meridians are geodesics by the previous discussion.
Not all parallels are geodesics.
\begin{lemma}
	A parallel given by \( u = u_0 \) is a geodesic when parametrised at constant speed if and only if \( f'(u_0) = 0 \).
\end{lemma}
