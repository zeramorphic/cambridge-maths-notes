\chapter[Statistics \\ \textnormal{\emph{Lectured in Lent \oldstylenums{2022} by \textsc{Dr.\ S.\ Bacallado}}}]{Statistics}
\emph{\Large Lectured in Lent \oldstylenums{2022} by \textsc{Dr.\ S.\ Bacallado}}

An estimator is a random variable that approximates a parameter.
For instance, the parameter could be the mean of a normal distribution, and the estimator could be a sample mean.
In this course, we study how estimators behave, what properties they have, and how we can use them to make conclusions about the real parameters.
This is called parametric inference: the study of inferring parameters from statistics of sample data.

Towards the end of the course, we study the normal linear model, which is a useful way to model data that is believed to depend linearly on a vector of inputs, together with some normally distributed noise.
Even nonlinear patterns can be analysed using this model, by letting the inputs to the model be polynomials in the real-world data.

\subfile{../../ib/stats/main.tex}
