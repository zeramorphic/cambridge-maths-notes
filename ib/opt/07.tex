\subsection{Strong Duality of Linear Programs}
\begin{theorem}
	If a linear program is bounded and feasible, then strong duality holds.
\end{theorem}
\begin{proof}
	This is true since the value function is convex.
\end{proof}

\subsection{Duals of Linear Programs in Standard Form}
Consider a linear program in standard form:
\begin{alignat*}{2}
	 & \text{minimise}   & \qquad & \vb c^\transpose \vb x \\
	 & \text{subject to} &        & A \vb x = \vb b        \\
	 &                   &        & \vb x \geq 0
\end{alignat*}
The dual problem is therefore
\begin{alignat*}{2}
	 & \text{maximise}   & \qquad & g(\vb \lambda) = \inf_{\vb x in \mathcal X} \mathcal L(\vb x, \vb \lambda) \\
	 & \text{subject to} &        & \vb \lambda \in \vb \Lambda                                                \\
\end{alignat*}
The function \( g \) is given by
\begin{align*}
	g(\vb \lambda) & = \inf_{\vb x \geq 0} \vb c^\transpose \vb x - \vb \lambda^\transpose (A \vb x - \vb b)                 \\
	               & = \inf_{\vb x \geq 0} (\vb c^\transpose - \vb \lambda^\transpose A)\vb x + \vb \lambda^\transpose \vb b
\end{align*}
This is only bounded below where \( \vb c^\transpose - \vb \lambda^\transpose A \geq 0 \).
Hence
\[
	\vb \Lambda = \qty{\vb \lambda \colon \vb \lambda^\transpose A \leq \vb c^\transpose}
\]
Further, the minimum value of \( g \) for \( \vb \lambda \in \vb \Lambda \) is \( \vb \lambda^\transpose \vb b \).
Therefore, the dual problem is
\begin{alignat*}{2}
	 & \text{maximise}   & \qquad & \vb \lambda^\transpose \vb b                   \\
	 & \text{subject to} &        & \vb \lambda^\transpose A \leq \vb c^\transpose \\
\end{alignat*}
The dual of a linear program in standard form is a linear problem, but no longer in standard form.

\subsection{Duals of Linear Programs in General Form}
Consider a linear program in general form:
\begin{alignat*}{2}
	 & \text{minimise}   & \qquad & \vb c^\transpose \vb x \\
	 & \text{subject to} &        & A \vb x \geq \vb b     \\
\end{alignat*}
We can introduce a slack variable \( \vb s \) and write equivalently
\begin{alignat*}{2}
	 & \text{minimise}   & \qquad & \vb c^\transpose \vb x  \\
	 & \text{subject to} &        & A \vb x - \vb s = \vb b \\
	 &                   &        & \vb s \geq 0
\end{alignat*}
To calculate the dual, we need to calculate \( g(\vb \lambda) \).
\begin{align*}
	g(\vb \lambda) & = \inf_{\vb x, \vb s \geq 0} \vb c^\transpose \vb x - \vb \lambda^\transpose (A \vb x - \vb s - \vb b)                                         \\
	               & = \inf_{\vb x, \vb s \geq 0} (\vb c^\transpose - \vb \lambda^\transpose A) \vb x + \vb \lambda^\transpose \vb s + \vb \lambda^\transpose \vb b \\
\end{align*}
In this case, since \( \vb x \) may be any value, we must have \( \vb c^\transpose - \vb \lambda^\transpose A = 0 \).
Further, since the slack variable can be any positive value, \( \vb \lambda^\transpose \geq 0 \).
The infimum is \( \vb \lambda^\transpose \vb b \) since \( \vb s \) may be set to zero.
Thus, the dual is
\begin{alignat*}{2}
	 & \text{maximise}   & \qquad & \vb \lambda^\transpose \vb b                \\
	 & \text{subject to} &        & \vb \lambda^\transpose A = \vb c^\transpose \\
	 &                   &        & \vb \lambda \geq 0
\end{alignat*}
The dual of a general linear program is a linear program in standard form.

\subsection{Dual of Dual Program}
The dual of a dual problem is the primal problem.
Suppose the primal problem is in standard form:
\begin{alignat*}{2}
	 & \text{minimise}   & \qquad & \vb c^\transpose \vb x \\
	 & \text{subject to} &        & A \vb x = \vb b        \\
	 &                   &        & \vb x \geq 0
\end{alignat*}
We know the dual is
\begin{alignat*}{2}
	 & \text{maximise}   & \qquad & \vb \lambda^\transpose \vb b                   \\
	 & \text{subject to} &        & \vb \lambda^\transpose A \leq \vb c^\transpose \\
\end{alignat*}
Equivalently,
\begin{alignat*}{2}
	 & -\text{ minimise} & \qquad & -\vb \lambda^\transpose \vb b                    \\
	 & \text{subject to} &        & -\vb \lambda^\transpose A \geq -\vb c^\transpose \\
\end{alignat*}
Definining \( \widetilde{\vb \lambda} = -\vb\lambda^\transpose \), we have
\begin{alignat*}{2}
	 & -\text{ minimise} & \qquad & \widetilde{\vb \lambda} \vb b                    \\
	 & \text{subject to} &        & \widetilde{\vb \lambda} A \geq -\vb c^\transpose \\
\end{alignat*}
We can find the dual of this problem using the solution above.
\begin{alignat*}{2}
	 & -\text{ maximise} & \qquad & -\vb \theta^\transpose \vb c                          \\
	 & \text{subject to} &        & \vb \theta^\transpose A^\transpose = \vb b^\transpose \\
	 &                   &        & \vb \theta \geq 0
\end{alignat*}
This is equivalent to the primal problem.

\subsection{Dual of Arbitrary Linear Program}
Consider the problem
\begin{alignat*}{3}
	 & \text{minimise}   & \qquad & \vb c^\transpose \vb x                &       &           \\
	 & \text{subject to} &        & \vb a_i^\transpose \vb x \geq \vb b_i & \quad & i \in M_1 \\
	 &                   &        & \vb a_i^\transpose \vb x \leq \vb b_i &       & i \in M_2 \\
	 &                   &        & \vb a_i^\transpose \vb x = \vb b_i    &       & i \in M_3 \\
	 &                   &        & x_j \geq 0                            &       & j \in N_1 \\
	 &                   &        & x_j \leq 0                            &       & j \in N_2 \\
	 &                   &        & x_j \text{ free}                      &       & j \in N_3 \\
\end{alignat*}
The dual of this problem is
\begin{alignat*}{3}
	 & \text{maximise}   & \qquad & \vb p^\transpose \vb b                &       &           \\
	 & \text{subject to} &        & p_i \geq 0                            & \quad & i \in M_1 \\
	 &                   &        & p_i \leq 0                            &       & i \in M_2 \\
	 &                   &        & p_i \text{ free}                      &       & i \in M_3 \\
	 &                   &        & \vb p^\transpose \vb A_j \leq \vb c_j &       & j \in N_1 \\
	 &                   &        & \vb p^\transpose \vb A_j \geq \vb c_j &       & j \in N_2 \\
	 &                   &        & \vb p^\transpose \vb A_j = \vb c_j    &       & j \in N_3 \\
\end{alignat*}
This will be shown in the example sheets.

\subsection{Optimality Conditions}
If \( \vb x \) is feasible for the primal, \( \vb p \) is feasible for the dual, and complementary slackness holds, then \( \vb x \) is optimal for the primal and \( \vb p \) is optimal for the dual.
\begin{theorem}[Fundamental Theorem of Linear Programming]
	Let \( \vb x, \vb p \) be feasible solutions to the primal and dual problems respectively.
	Then \( \vb x, \vb p \) are optimal for these problems if and only if
	\begin{itemize}
		\item \( p_i (\vb a_i^\transpose \vb x - b_i) = 0 \) for all \( i \), and
		\item  \( (c_j - \vb p^\transpose \vb A_j) x_j = 0 \) for all \( j \).
	\end{itemize}
\end{theorem}
\begin{proof}
	First, let us define \( u_i = p_i (\vb a_i^\transpose \vb x - b_i ) \) and \( v_j = (c_j - \vb p^\transpose \vb A_j) x_j \).
	Observe that if \( \vb x, \vb p \) are feasible, then \( u_i \geq 0 \) for all \( i \), and \( v_j \geq 0 \) for all \( j \).
	This can be seen by the signs of the constraints on the primal and dual problems.
	Now,
	\[
		\sum u_i = \sum p_i (\vb a_i^\transpose \vb x - b_i ) = \vb p^\transpose A \vb x - \vb p^\transpose \vb b
	\]
	Similarly,
	\[
		\sum v_j = \sum (c_j - \vb p^\transpose \vb A_j) x_j = \vb c^\transpose \vb x - \vb p^\transpose A \vb x
	\]
	Then,
	\[
		\sum u_i + \sum v_j = \vb c^\transpose \vb x - \vb p^\transpose \vb b
	\]
	which is the difference between the two objective functions in the primal and the dual.
	Hence,
	\[
		0 \leq \sum u_i + \sum v_j = \vb c^\transpose \vb x - \vb p^\transpose \vb b
	\]
	So if complementary slackness holds, then \( u_i = 0 \) and \( v_j = 0 \) for all \( i, j \).
	This then implies that \( \vb c^\transpose \vb x = \vb p^\transpose \vb b \).
	By weak duality, \( \vb x \) and \( \vb p \) must be optimal.
	Conversely, suppose \( \vb x, \vb p \) are optimal.
	By strong duality, \( \vb c^\transpose \vb x = \vb p^\transpose \vb b \).
	\[
		0 \leq \sum u_i + \sum v_j = \vb c^\transpose \vb x - \vb p^\transpose \vb b = 0
	\]
	Thus \( \sum u_i + \sum v_j = 0 \).
	Since all \( u_i, v_j \) are non-negative, \( u_i = 0 \) and \( v_j = 0 \) for all \( i, j \).
	Equivalently, complementary slackness holds.
\end{proof}

\subsection{Introducing the Simplex Method}
Consider the problem
\begin{alignat*}{2}
	 & \text{minimise}   & \qquad & \vb c^\transpose \vb x \\
	 & \text{subject to} &        & A \vb x = \vb b        \\
	 &                   &        & \vb x \geq 0
\end{alignat*}
The dual problem is
\begin{alignat*}{2}
	 & \text{maximise}   & \qquad & \vb \lambda^\transpose \vb b                   \\
	 & \text{subject to} &        & \vb \lambda^\transpose A \leq \vb c^\transpose \\
\end{alignat*}
The optimality conditions are
\begin{itemize}
	\item (primal feasibility) \( A \vb x = \vb b; \vb x \geq 0 \)
	\item (dual feasibility) \( A^\transpose \vb \lambda \leq \vb c \)
	\item (complementary slackness) \( \vb x^\transpose (\vb c - A^\transpose \vb \lambda) = 0 \)
\end{itemize}
Suppose \( \vb x \) is a basic feasible solution given by
\[
	\vb x_B = (x_{B(1)}, \dots, x_{B(m)})
\]
Substituting this \( \vb x \) into the complementary slackness equation gives
\[
	\vb x_B^\transpose \vb c_B - \vb x_B^\transpose B^\transpose \vb \lambda = 0
\]
This can be solved to find
\[
	\vb x_B^\transpose (\vb c_B - B^\transpose \vb \lambda) = 0
\]
For a basic feasible solution, \( \vb x_B > 0 \).
Hence,
\[
	\vb c_B - B^\transpose \vb \lambda = 0
\]
Hence
\[
	\vb \lambda = (B^\transpose)^{-1} \vb c_B
\]
So for this \( \vb x \) and this calculated \( \vb \lambda \), primal feasibility and complementary slackness both hold.
What remains now is to check if dual feasibility holds.
Equivalently,
\[
	A^\transpose \vb \lambda \leq \vb c \implies A^\transpose (B^\transpose)^{-1} \vb c_B \leq \vb c
\]
If this holds, then the optimality conditions are met.
This means that we do not even need to explicitly find \( \vb \lambda \) in order to check optimality; it suffices to check whether this single inequality holds.
We define
\[
	\overline{\vb c} = \vb c - A^\transpose (B^\transpose)^{-1} \vb c_B
\]
This is called the \textit{vector of reduced costs}.
Then the inequality \( \overline{\vb c} \geq 0 \) implies \( \vb x \) is optimal.
