\subsection{Heine-Borel theorem}
\begin{theorem}
	A subset \( K \) of \( \mathbb R^n \) is compact if and only if \( K \) is closed and bounded.
\end{theorem}
\begin{proof}
	Suppose \( K \) is compact.
	\( \mathbb R^n \) is a metric space and hence Hausdorff.
	Hence, \( K \) is closed in \( \mathbb R^n \).
	The function \( x \mapsto \norm{x} \) is continuous.
	Therefore, it is bounded on \( K \).
	So \( K \) is bounded.

	Conversely, if \( K \) is bounded, there exists \( M \geq 0 \) such that for all \( x \in K \) we have \( \norm{x} \leq M \).
	Hence, \( K \subset [-M, M]^n \).
	Note that \( [-M, M] \) is compact since it is homeomorphic to \( [0,1] \).
	By Tychonov's theorem, \( [-M, M]^n \) is compact in the product topology.
	Since a closed subset of a compact space is compact, \( K \) is compact.
\end{proof}
\begin{example}
	Closed balls \( \mathcal B_r(x) \) in \( \mathbb R^n \) are compact.
	The start of the proof for the Lindel\"of-Picard theorem now makes more sense.
\end{example}

\subsection{Sequential compactness}
\begin{definition}
	A topological space \( X \) is \textit{sequentially compact} if every sequence in \( X \) has a convergent subsequence.
	Given a sequence \( (x_n) \) and an infinite set \( M \subset \mathbb N \), we will write \( (x_m)_{m \in M} \) for the subsequence \( (x_{m_n})_{n=1}^\infty \) where \( m_1 < m_2 < \dots \) are the elements of \( M \).
	Note that if \( L \subset M \subset \mathbb N \), then \( (x_n)_{n \in L} \) is a subsequence of \( (x_n)_{n \in M} \).
\end{definition}
\begin{example}
	Any closed and bounded subset of \( \mathbb R \) is sequentially compact by the Bolzano-Weierstrass theorem.
	Similarly, any closed and bounded subset \( K \) of \( \mathbb R^n \) is sequentially compact.
	Indeed, let \( (x_m) \) be a sequence in \( K \).
	Then, writing \( x_m = (x_{m,1}, \dots, x_{m,n}) \), since \( K \) is bounded we have that \( (x_{m,j}) \) is bounded for all \( j \).
	Applying the Bolzano-Weierstrass theorem to the first coordinate, we find \( M_1 \subset \mathbb N \) such that \( (x_{m,1})_{m \in M_1} \) converges in \( \mathbb R \).
	Now, \( (x_{m,2})_{m \in M_1} \) is bounded in \( \mathbb R \), so again applying the Bolzano-Weierstrass theorem, we can find \( M_2 \subset \mathbb N \) such that \( (x_{m,2})_{m \in M_2} \) converges.
	Note that \( (x_{m,1})_{m \in M_2} \) converges.
	So inductively we can find \( M_1 \supset \dots \supset M_n \) such that \( (x_{m,j})_{m \in M_n} \) converges for all \( j \).
	Hence \( (x_m)_{m \in M_n} \) converges in \( \mathbb R^n \).
	The limit is contained in \( K \) since \( K \) is closed.
\end{example}
\begin{remark}
	In \( \mathbb R^n \), any compact space is sequenctially compact.
	The converse is also true; any sequentially compact subspace must be closed and bounded.
	We aim to show that compactness and sequential compactness are identical in metric spaces.
\end{remark}

\subsection{Compactness and sequential compactness in metric spaces}
Let \( (M, d) \) be a metric space.
\begin{definition}
	For \( \varepsilon > 0 \) and \( F \subset M \), we say that \( F \) is an \textit{\( \varepsilon \)-net for \( M \)} if for all \( x \in M \), there exists \( y \in M \) such that \( d(y,x) \leq \varepsilon \).
	Equivalently, \( M = \bigcup_{y \in M} \mathcal B_\varepsilon(y) \).
	This is called a \textit{finite \( \varepsilon \)-net} if \( F \) is finite.
	We say that \( M \) is \textit{totally bounded} if for all \( \varepsilon > 0 \), there exists a finite \( \varepsilon \)-net for \( M \).
\end{definition}
\begin{example}
	For \( \varepsilon > 0 \), let \( n \) such that \( \frac{1}{n} < \varepsilon \).
	Then \( \qty{\frac{1}{n}, \frac{2}{n}, \dots, \frac{n-1}{n}} \) is an \( \varepsilon \)-net for \( (0,1) \).
\end{example}
\begin{definition}
	For a non-empty \( A \subset M \), the \textit{diameter} of \( A \) is \( \diam A = \sup\qty{d(x,y) \colon x,y \in A} \).
	This is finite if and only if \( A \) is a bounded set.
\end{definition}
\begin{example}
	\( \diam \mathcal B_r(x) \leq 2r \).
\end{example}
\begin{lemma}
	Suppose \( M \) is totally bounded.
	Let \( A \) be a non-empty closed subset of \( M \).
	Let \( \varepsilon > 0 \).
	Then there exists \( K \in \mathbb N \) and non-empty closed sets \( B_1, \dots, B_K \) such that \( A = \bigcup_{k=1}^K B_k \) and \( \diam B_k \leq \varepsilon \) for all \( k \).
\end{lemma}
\begin{proof}
	Let \( F \) be a finite \( \frac{\varepsilon}{2} \)-net for \( M \).
	So \( M = \bigcup_{x \in F} B_{\varepsilon/2}(x) \) and hence \( A = \bigcup_{x \in F} (A \cap B_{\varepsilon/2}(x)) \).
	Let \( G = \qty{ x \in F \colon A \cap B_{\varepsilon / 2}(x) \neq 0 } \).
	Then for \( x \in G \) let \( B_x = A \cap B_{\varepsilon / 2}(x) \).
	So for \( x \in G \), we have \( B_x \neq \varnothing \), \( B_x \subset B_{\varepsilon/2}(x) \) and so \( \diam B_x \leq \varepsilon \), and \( B_x \) is closed.
	Then \( A = \bigcup_{x \in G} B_x \).
\end{proof}
\begin{theorem}
	For a metric space \( (M, d) \), the following are equivalent.
	\begin{enumerate}[(i)]
		\item \( M \) is compact;
		\item \( M \) is sequentially compact;
		\item \( M \) is complete and totally bounded.
	\end{enumerate}
\end{theorem}
\begin{proof}
	We first show (i) implies (ii).
	Let \( (x_n) \) be a sequence in \( M \).
	Then for \( n \in \mathbb N \), let \( T_n = \qty{x_k \colon k > n} \) be the tail of the sequence.
	Note that the limit of any convergent subsequence (if it exists) is in the intersection of \( \bigcap_{n \in \mathbb N} \overline T_n \).
	So first, we prove that this intersection is non-empty.
	Suppose that it is empty.
	Then, \( \bigcup_{n \in \mathbb N} \qty(M \setminus \overline T_n) = M \).
	But the \( M \setminus \overline T_n \) are open, and \( M \) is compact, there is a finite subcover.
	So \( M \setminus \overline T_N = M \) for some \( N \), since the \( T_n \) are a decreasing sequence of sets.
	This is a contradiction since \( T_N \neq \varnothing \).
	Now, let \( x \in \bigcap_{n \in \mathbb N} \overline T_n \), and we want to show the existence of a subsequence converging to \( x \).
	First, \( x \in \overline T_1 \), so \( \mathcal D_1(x) \cap T_1 \neq \varnothing \).
	Hence there exists \( k_1 > 1 \) such that \( d(x_{k_1}, x) < 1 \).
	Now since \( x \in \overline T_{k_1} \), \( \mathcal D_{1/2}(x) \cap T_{k_1} \neq \varnothing \).
	There exists \( k_2 > k_1 \) such that \( d(x_{k_2}, x) < \frac{1}{2} \).
	Inductively, we can find a strictly increasing sequence \( k_1 < k_2 < \dots \) such that \( d(x_{k_n}, x) < \frac{1}{n} \) for all \( n \), so \( x_{k_n} \to x \).

	Now, we show (ii) implies (iii).
	To show \( M \) is complete, let \( (x_n) \) be a Cauchy sequence in \( M \).
	Let \( k_1 < k_2 < \dots \) such that \( x_{k_n} \) converges in \( M \), and let \( x \) be the limit.
	We show \( x_n \to x \).
	Indeed, for \( \varepsilon > 0 \), there exists \( N \in \mathbb N \) such that \( \forall m, n \geq N \), we have \( d(x_m, x_n) < \varepsilon \).
	Then \( \forall m \geq N \), we have \( k_n \geq m \geq N \), so for a fixed \( n \geq N \) and \( \forall m \geq N \), we have \( d(x_n, x) \leq d(x_n, x_{k_m}) + d(x_{k_m}, x) \leq \varepsilon + d(x_{k_m}, x) \).
	Let \( m \to \infty \), so \( d(x_n, x) \leq \varepsilon \).
	So \( x_n \to x \).
	To show \( M \) is totally bounded, suppose it is not.
	There exists \( \varepsilon > 0 \) such that \( M \) has no finite \( \varepsilon \)-net.
	Let \( x_1 \in M \), and suppose we can find \( x_1, \dots, x_{n-1} \) in \( M \).
	Then \( \bigcup_{j=1}^{n-1} \mathcal B_\varepsilon(x_j) \neq M \).
	So we can pick \( x_n \in M \setminus \bigcup_{j-1}^{n-1} \mathcal B_\varepsilon(x_j) \).
	Inductively we obtain \( (x_n) \) such that \( d(x_m, x_n) > \varepsilon \) for all \( n, m \in \mathbb N \).
	So \( (x_n) \) has no Cauchy subsequence.
	There is therefore no convergent subsequence, which is a contradiction.

	Finally, we show (iii) implies (i).
	Let \( \mathcal U \) be an open cover for \( M \).
	We must show there exists a finite subcover.
	Suppose that is not true, so \( \mathcal U \) does not finitely cover \( M \).
	We construct non-empty closed subsets \( A_0 \supset A_1 \supset \dots \) of \( M \) such that for all \( n \geq 0 \), \( \mathcal U \) does not finitely cover \( A_n \), and for all \( n \geq 1 \) we have \( \diam A_n < \frac{1}{n} \).
	Let \( A_0 = M \).
	Suppose that for some \( n \geq 1 \) we have already found \( A_{n-1} \).
	Since \( M \) is totally bounded, we can write \( A_{n-1} = \bigcup_{k=1}^K B_k \) where \( K \in \mathbb N \) and the \( B_k \) are non-empty, closed, and \( \diam B_k < \frac{1}{n} \).
	Since \( \mathcal U \) does not finitely cover \( A_{n-1} \), there exists \( k \leq K \) such that \( \mathcal U \) does not finitely cover \( B_k \).
	Let \( A_n \) be this \( B_k \).
	Now, for all \( n \), pick some \( x_n \in A_n \).
	For all \( N \), \( \forall m,n \geq N \) we have \( x_m, x_n \in A_N \) hence \( d(x_m, x_n) \leq \diam A_N \leq \frac{1}{n} \) so the sequence is Cauchy.
	\( M \) is complete, so \( x_n \to x \) for some \( x \in M \).
	Let \( U \in \mathcal U \) such that \( x \in U \).
	\( U \) is open, so there exists \( r > 0 \) such that \( \mathcal D_r(x) \subset U \).
	But \( x_n \to x \) hence there exists \( n \) such that \( d(x_n, x) < \frac{r}{2} \) and \( \diam A_n < \frac{r}{2} \).
	For every \( y \in A_n \), \( d(y,x) \leq d(y,x_n) + d(x_n, x) \leq \diam A_n + \frac{r}{2} < r \).
	Hence every point in \( A_n \) is contained within \( \mathcal D_r(x) \subset U \).
	But this contradicts the fact that \( \mathcal U \) does not finitely cover \( A_n \), but we have constructed a cover using just one open set.
\end{proof}
\begin{remark}
	We can now deduce the one direction of the Heine-Borel theorem from the Bolzano-Weierstrass theorem; closed and bounded subsets of \( \mathbb R^n \) are compact.
	Similarly, we can check that the product of sequentially compact topological spaces is sequentially compact in the product topology.
	This yields a new proof for Tychonov's theorem for metric spaces.
	In general, there exist topological spaces that are compact but not sequentially compact, and conversely there exist topological spaces which are sequentially compact but not compact.
\end{remark}
