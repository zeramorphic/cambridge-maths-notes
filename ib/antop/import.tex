\chapter[Analysis and Topology \\ \textnormal{\emph{Lectured in Michaelmas \oldstylenums{2021} by \textsc{Dr.\ V.\ Zs\'ak}}}]{Analysis and Topology}
\emph{\Large Lectured in Michaelmas \oldstylenums{2021} by \textsc{Dr.\ V.\ Zs\'ak}}

In the analysis part of the course, we continue the study of convergence from Analysis I.
We define a stronger version of convergence, called uniform convergence, and show that it has some very desirable properties.
For example, if integrable functions \( f_n \) converge uniformly to the integrable function \( f \), then the integrals of the \( f_n \) converge to the integral of \( f \).
The same cannot be said in general about non-uniform convergence.
We also extend our study of differentiation to functions with multiple input and output variables, and rigorously define the derivative in this higher-dimensional context.

In the topology part of the course, we consider familiar spaces such as \( [a,b], \mathbb C, \mathbb R^n \), and generalise their properties.
We arrive at the definition of a metric space, which encapsulates all of the information about how near or far points are from others.
From here, we can define notions such as continuous functions between metric spaces in such a way that does not depend on the underlying space.

We then generalise even further to define topological spaces.
The only information a topological space contains is the neighbourhoods of each point, but it turns out that this is still enough to define continuous functions and similar things.
We study topological spaces in an abstract setting, and prove important facts that are used in many later courses.

\subfile{../../ib/antop/main.tex}
