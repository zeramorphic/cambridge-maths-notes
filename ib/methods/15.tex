\subsection{Explicit form for Green's functions}
We want to solve
\[
	\mathcal L G(x,\xi) = \delta(x-\xi)
\]
on \( a \leq x \leq b \), subject to homogeneous boundary conditions \( G(a,\xi) = G(b,\xi) = 0 \).
The functions \( G_1, G_2 \) satisfy the homogeneous equation, so \( \mathcal L G_i(x,\xi) = 0 \).
Suppose there exist two independent homogeneous solutions \( y_1(x), y_2(x) \) to \( \mathcal L y = 0 \).
Then, \( G_1 = A y_1 + B y_2 \), such that \( A y_1(a) + B y_2(a) = 0 \), which gives a constraint between \( A \) and \( B \).
This defines a complementary function \( y_-(x) \) such that \( y_-(a) = 0 \).
The general homogeneous solution with \( G_1(a) = 0 \) is
\[
	G_1 = C y_-
\]
\( C \) will be found later.
Similarly we can define \( y_+ \) as a linear combination of \( y_1, y_2 \) such that \( y_+(b) = 0 \).
\[
	G_2 = D y_+
\]
We require \( G_1(\xi, \xi) = G_2(\xi, \xi) \) for continuity, hence
\[
	C y_-(\xi) = D y_+(\xi)
\]
Since \( [G']_{\xi_-}^{\xi^+} = \frac{1}{\alpha(\xi)} \), we have
\[
	Dy'_+(\xi) - C Y_-'(\xi) = \frac{1}{\alpha(\xi)}
\]
We can solve these equations for \( C, D \) simultaneously to find
\[
	C(\xi) = \frac{y_+(\xi)}{\alpha(\xi)W(\xi)};\quad D(\xi) = \frac{y_-(\xi)}{\alpha(\xi)W(\xi)}
\]
where \( W(\xi) \) is the Wro\'nskian
\[
	W(\xi) = y_-(\xi) y_+'(\xi) - y_+(\xi) y_-'(\xi)
\]
which is nonzero if \( y_-, y_+ \) are linearly independent.
Hence,
\[
	G(x,\xi) = \begin{cases}
		\frac{y_-(x) y_+(\xi)}{\alpha(\xi)W(\xi)} & a \leq x \leq \xi \\
		\frac{y_-(\xi) y_+(x)}{\alpha(\xi)W(\xi)} & \xi \leq x \leq b
	\end{cases}
\]

\subsection{Solving boundary value problems}
We know that the solution of \( \mathcal L y = f \) is
\[
	y(x) = \int_a^b G(x,\xi) f(\xi) \dd{\xi}
\]
We can split this into two intervals given that \( G = G_1 \) for \( \xi > x \) and \( G = G_2 \) for \( \xi < x \).
\begin{align*}
	y(x) & = \int_a^x G_2(x,\xi) f(\xi) \dd{\xi} + \int_x^b G_1(x,\xi) f(\xi) \dd{\xi}                                                               \\
	     & = y_+(x) \int_a^x \frac{y_-(\xi) f(\xi)}{\alpha(\xi)W(\xi)} \dd{\xi} + y_-(x) \int_a^x \frac{y_+(\xi) f(\xi)}{\alpha(\xi)W(\xi)} \dd{\xi}
\end{align*}
Note that if \( \mathcal L \) is in Sturm-Liouville form, so \( \beta = \alpha' \), then the denominator \( \alpha(\xi)W(\xi) \) is a constant.
Further, \( G \) is symmetric; \( G(x,\xi) = G(\xi,x) \).
Often, by convention, we take \( \alpha = 1 \) (however Sturm-Liouville form typically takes \( \alpha < 0 \)).
\begin{example}
	Consider \( y'' - y = f(x) \) with \( y(0) = y(1) = 0 \).
	Homogeneous solutions are \( y_1 = e^x \), \( y_2 = e^{-x} \).
	Imposing boundary conditions,
	\[
		G = \begin{cases}
			C \sinh x    & 0 \leq x < \xi \\
			D \sinh(1-x) & \xi < x \leq b
		\end{cases}
	\]
	Continuity at \( x = \xi \) implies
	\[
		C \sinh \xi = D \sinh (1 - \xi) \implies C = D \frac{\sinh (1-\xi)}{\sinh \xi}
	\]
	The jump condition is
	\[
		-D \cosh(1-\xi) - C \cosh \xi = 1
	\]
	Hence,
	\begin{align*}
		-D\qty[\cosh(1-\xi)\sinh \xi + \sinh(1-\xi)\cosh \xi] & = \sinh \xi                     \\
		-D\qty[\sinh((1-\xi) + \xi)]                          & = \sinh \xi                     \\
		-D\sinh 1                                             & = \sinh \xi                     \\
		D                                                     & = \frac{\sinh \xi}{\sinh 1}     \\
		\therefore C                                          & = \frac{-\sinh(1-\xi)}{\sinh 1}
	\end{align*}
	Therefore,
	\[
		y(x) = \frac{-\sinh(1-x)}{\sinh 1} \int_0^x \sinh \xi f(\xi) \dd{\xi} - \frac{\sinh x}{\sinh 1} \int_x^1 \sinh (1-\xi) f(\xi) \dd{\xi}
	\]
\end{example}
\noindent Suppose we have inhomogeneous boundary conditions.
In this case, we want to find a homogeneous solution \( y_p \) that solves the inhomogeneous boundary conditions.
That is, \( \mathcal L y_p = 0 \) but \( y_p(a), y_p(b) \) are as required for the boundary conditions.
Then, by subtracting this solution from the original equation, we can solve using a homogeneous set of boundary conditions.
For instance, in the above example, suppose \( y(0) = 0, y(1) = 1 \).
We can find a solution \( y_p = \frac{\sinh x}{\sinh 1} \) which has the inhomogeneous boundary conditions but solves the homogeneous problem.

\subsection{Higher-order ODEs}
Suppose \( \mathcal L y = f(x) \) where \( \mathcal L \) is an \( n \)th order linear differential operator, and \( \alpha(x) \) is the coefficient for the highest degree derivative.
Suppose that homogeneous boundary conditions are satisfied.
Then we can define the Green's function in this case to be the function that solves
\[
	\mathcal L G(x,\xi) = \delta(x-\xi)
\]
which has the properties:
\begin{enumerate}[(i)]
	\item \( G_1, G_2 \) are homogeneous solutions satisfying the homogeneous boundary conditions;
	\item \( G_1^{(k)} = G_2^{(k)} \) for \( k \in \qty{0, \dots, n-2} \);
	\item \( G_2^{(n-1)}(\xi^+) - G_1^{(n-1)}(\xi^-) = \frac{1}{\alpha(\xi)} \).
\end{enumerate}
