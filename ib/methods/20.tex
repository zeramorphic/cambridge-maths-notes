\subsection{Discrete Fourier transform}
Suppose we have a finite number of samples \( h_m = h(t_m) \) for \( t_m = m \Delta \), where \( m = 0,\dots, N-1 \).
We will approximate the Fourier transform for \( N \) frequencies within the Nyquist frequency \( f_c = \frac{1}{2\Delta} \), using equally-spaced frequencies, given by \( \Delta_f = \frac{1}{N\Delta} \) in the range \( -f_c \leq f \leq f_c \).
We could take the convention \( f_n = n \Delta_f = \frac{n}{N\Delta} \) for \( n = -\frac{N}{2}, \dots, -\frac{N}{2} \).
However, this overcounts the Nyquist frequency (which is aliased), giving \( N + 1 \) frequencies instead of the desired \( N \).
Since frequencies above the Nyquist frequency are aliased to below it:
\[
	\qty(\frac{N}{2} + m) \Delta f = f_c + \delta f \mapsto \qty(\frac{N}{2} - m)\Delta f = -(f_c - \delta f)
\]
we can instead use the convention \( f_n = n \Delta_f = \frac{n}{N\Delta} \) for \( n = 0, \dots, N - 1 \).
This counts the Nyquist frequency only once.
The Fourier transform at a frequency \( f_n \) becomes
\begin{align*}
	\widetilde h(f_n) & = \int_{-\infty}^\infty h(t) e^{-2\pi if_n t} \dd{t}    \\
	                  & \approx \Delta \sum_{m=0}^{N-1} h_m e^{-2\pi i f_n t_m} \\
	                  & = \Delta \sum_{m=0}^{N-1} h_m e^{-2\pi i m n / N}       \\
	                  & = \Delta \widetilde h_d(f_n)
\end{align*}
where the function \( \widetilde h_d(f_n) \) is the \textit{discrete Fourier transform}.
The matrix \( [\mathrm{DFT}]_{mn} = e^{-2\pi i m n / N} \) defines the discrete Fourier transform for the vector \( h = \qty{h_m} \).
The discrete Fourier transform is then
\[
	\widetilde h_d = [\mathrm{DFT}] h
\]
By inverting the discrete Fourier transform matrix, we find
\[
	h = [\mathrm{DFT}]^{-1} \widetilde h_d = \frac{1}{N} [\mathrm{DFT}]^\dagger \widetilde h_d
\]
since the inverse of the discrete Fourier transform matrix is its adjoint.
The matrix is built from roots of unity \( \omega = e^{-2\pi i/N} \).
So, for instance, \( n = 4 \) gives \( \omega = e^{-2\pi i/4} = -i \) giving
\[
	[\mathrm{DFT}] = \begin{pmatrix}
		1 & 1  & 1  & 1  \\
		1 & -i & -1 & i  \\
		1 & -1 & 1  & -1 \\
		1 & i  & -1 & -i
	\end{pmatrix}
\]
The inverse discrete Fourier transform is
\begin{align*}
	h_m & = h(t_m)                                                                                  \\
	    & = \frac{1}{2\pi} \int_{-\infty}^\infty \widetilde h(\omega) e^{i \omega t_m} \dd{\omega}  \\
	    & = \int_{-\infty}^\infty \widetilde h(f) e^{2\pi i f t_m} \dd{f}                           \\
	    & \approx \frac{1}{\Delta N} \sum_{n=0}^{N-1} \Delta \widetilde h_d(f_n) e^{2\pi i m n / N} \\
	    & = \frac{1}{N} \sum_{n=0}^{N-1} \widetilde h_n e^{2\pi i m n / N}
\end{align*}
Hence, we can interpolate the initial function from its samples.
\[
	h(t) = \frac{1}{N} \sum_{n=0}^{N-1} \widetilde h_n e^{2\pi i n t / N}
\]
Parseval's theorem becomes
\[
	\sum_{m=0}^{N-1} \abs{h_m}^2 = \frac{1}{N} \sum_{n=0}^{N-1} \abs{\widetilde h_n}^2
\]
and the convolution theorem is
\[
	c_k = \sum_{m=0}^{N-1} g_m h_{k-m} \iff \widetilde c_k = \widetilde g_k \widetilde h_k
\]

\subsection{Fast Fourier transform (non-examinable)}
While the discrete Fourier transform is an order \( O(N^2) \) operation, we can reduce this into an order \( O(n \log N) \) operation.
Such a simplification is called the \textit{fast Fourier transform}.
We can split the discrete Fourier transform into even and odd parts, noting that \( \omega_N = e^{-2\pi i / N} \) implies \( \omega_N^2 = e^{-2 \pi i / (N/2)} = \omega_{N/2} \)
\begin{align*}
	\widetilde h_k & = \sum_{n=0}^{N-1} h_n \omega_N^{nk}                                                                           \\
	               & = \sum_{m=0}^{N/2-1} h_{2m} \omega_N^{2mk} + \sum_{m=0}^{N/2-1} h_{2m + 1} \omega_N^{(2m+1)k}                  \\
	               & = \sum_{m=0}^{N/2-1} h_{2m} (\omega_N^2)^{mk} + \omega_N^k \sum_{m=0}^{N/2-1} h_{2m + 1} (\omega_N^2)^{mk}     \\
	               & = \sum_{m=0}^{N/2-1} h_{2m} (\omega_{N/2})^{mk} + \omega_N^k \sum_{m=0}^{N/2-1} h_{2m + 1} (\omega_{N/2})^{mk} \\
\end{align*}
This algorithm iteratively reduces the Fourier transform's complexity by a factor of two, until the trivial case of finding the discrete Fourier transform of two data points.

\subsection{Well-posed Cauchy problems}
Solving partial differential equations depends on the nature of the equations in combination with the boundary or initial data.
A \textit{Cauchy problem} is the partial differential equation for some function \( \phi \) together with the auxiliary data (in \( \phi \) and its derivatives) specified on a surface (or a curve in two dimensions), which is called \textit{Cauchy data}.
For a Cauchy problem to be \textit{well-posed}, we require that
\begin{enumerate}[(i)]
	\item a solution exists (we do not have excessive auxiliary data);
	\item the solution is unique (we do not have insufficient auxiliary data); and
	\item the solution depends continuously on the auxiliary data.
\end{enumerate}

\subsection{Method of characteristics}
Consider a parametrised curve \( C \) given by Cartesian coordinates \( (x(s), y(s)) \).
The tangent vector is
\[
	v = \qty(\dv{x(s)}{s}, \dv{y(s)}{s})
\]
We then define the directional derivative of a function \( \phi(x,y) \) by
\[
	\eval{\dv{\phi}{s}}_C = \dv{x(s)}{s} \pdv{\phi}{x} + \dv{y(s)}{s}\pdv{\phi}{y} = v \cdot \grad{\phi}
\]
Suppose \( v \cdot \grad{\phi} = 0 \) then \( \dv{\phi}{s} = 0 \) and hence \( \phi \) is constant along the curve.
Suppose there exists a vector field
\[
	u = \qty(\alpha(x,y), \beta(x,y))
\]
with a family of non-intersecting integral curves \( C \) which fill the plane (or domain of the function more generally), such that at a point \( (x,y) \) the integral curve has tangent vector \( u(x,y) \).
Now, define a curve \( B \) by \( (x(t), y(t)) \) such that \( B \) is transverse to \( u \); its tangent is nowhere parallel to \( u \).
\[
	w = \qty(\dv{x(t)}{t}, \dv{y(t)}{t}) \nparallel \qty(\alpha(x,y), \beta(x,y)) = u
\]
This can be used to parametrise the family of curves by labelling each curve \( C \) with the value of \( t \) at the intersection point between it and \( B \).
Along the curve, we use \( s \) such that \( s = 0 \) at the intersection.
The integral curves \( (x(s,t), y(s,t)) \) satisfy
\[
	\dv{x}{s} = \alpha(x,y);\quad \dv{y}{s} = \beta(x,y)
\]
We can solve these equations to find a family of characteristic curves, along which \( t \) remains constant.
This yields a new coordinate system \( (s,t) \) associated with a differential equation we wish to solve.
