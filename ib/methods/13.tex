\subsection{Definition}
\begin{definition}
	We define a generalised function \( \delta(x - \xi) \) such that
	\begin{enumerate}[(i)]
		\item \( \delta(x-\xi) = 0 \) for all \( x \neq \xi \);
		\item \( \int_{-\infty}^\infty \delta(x-\xi) \dd{x} = 1 \).
	\end{enumerate}
	This acts as a linear operator \( \int \dd{x} \delta(x - \xi) \) on some function \( f(x) \) to produce a number \( f(\xi) \).
	\[
		\int_{-\infty}^\infty \dd{x} \delta(x-\xi) f(x) = f(\xi)
	\]
	This relationship holds provided that \( f(x) \) is sufficiently `well-behaved' at \( x=\xi \) and \( x\to\pm \infty \).
\end{definition}
\begin{remark}
	Strictly, the \( \delta \) `function' is classified as a distribution, not as a function.
	For this reason, we will never use \( \delta \) outside an integral, although such an integral may be implied.
	The \( \delta \) function represnts a unit point source or impulse.
\end{remark}
We can approximate the \( \delta \) function using a Gaussian approximation.
\[
	\delta_\varepsilon(x) = \frac{1}{\varepsilon \sqrt{\pi}} \exp[-\frac{x^2}{\varepsilon^2}]
\]
Therefore,
\begin{align*}
	\int_{-\infty}^\infty f(x) \delta(x) \dd{x} & = \lim_{\varepsilon \to 0} \int_{-\infty}^\infty \frac{1}{\varepsilon \sqrt{\pi}} \exp[-\frac{x^2}{\varepsilon^2}] f(x) \dd{x}            \\
	                                            & = \lim_{\varepsilon \to 0} \int_{-\infty}^\infty \frac{1}{\varepsilon \sqrt{\pi}} \exp[-y^2] f(\varepsilon y) \dd{y}                      \\
	                                            & = \lim_{\varepsilon \to 0} \int_{-\infty}^\infty \frac{1}{\varepsilon \sqrt{\pi}} \exp[-y^2] [f(0) + \varepsilon y f'(0) + \cdots] \dd{y} \\
	                                            & = f(0)
\end{align*}
for all well-behaved functions \( f \) at \( 0, \pm \infty \).
We could alternatively use the Dirichlet kernel
\[
	\delta_n(x) = \frac{\sin n x}{\pi x} = \frac{1}{2\pi} \int_{-n}^n e^{ikx} \dd{k}
\]
or even
\[
	\delta_n(x) = \frac{n}{2} \sech^2 nx
\]

\subsection{Integral and derivative of delta function}
We define the Heaviside step function by
\[
	H(x) = \begin{cases}
		1 & x \geq 0 \\
		0 & x < 0
	\end{cases}
\]
For \( x \neq 0 \), we have
\[
	H(x) = \int_{-\infty}^x \delta(t) \dd{t}
\]
Thus,
\[
	\dv{x} H(x) = \delta(x)
\]
where this identification takes place under an implied integral.
We define \( \delta'(x) \) using integration by parts.
\begin{align*}
	\int_{-\infty}^\infty \delta'(x-\xi) f(x) \dd{x} & = \qty[\delta(x-\xi) f(x)]_{-\infty}^\infty - \int_{-\infty}^\infty \delta(x-\xi) f'(x) \dd{x} \\
	                                                 & = - \int_{-\infty}^\infty \delta(x-\xi) f'(x) \dd{x}                                           \\
	                                                 & = - f'(\xi)
\end{align*}
This is valid for all \( f \) that are smooth at \( x = \xi \).
\begin{example}
	Consider the Gaussian approximation:
	\[
		\delta_\varepsilon(x) = \frac{1}{\varepsilon \sqrt{\pi}} \exp[-\frac{x^2}{\varepsilon^2}]
	\]
	Then,
	\[
		\delta_\varepsilon'(x) = \frac{-2x}{\varepsilon^3 \sqrt{\pi}} \exp[-\frac{x^2}{\varepsilon^2}]
	\]
\end{example}

\subsection{Properties of delta function}
Note that
\[
	\int_a^b f(x) \delta(x-\xi) \dd{x} = \begin{cases}
		f(\xi) & a < \xi < b      \\
		0      & \text{otherwise}
	\end{cases}
\]
So the \( \delta \) function only `samples' values within the integral range.
This is known as the sampling property.
Let \( u = -(x-\xi) \), and consider
\begin{align*}
	\int_{-\infty}^\infty f(x) \delta\qty(-(x-\xi)) \dd{x} & = \int_{\infty}^{-\infty} f(\xi - u) \delta(u) (-\dd{u}) \\
	                                                       & = \int_{-\infty}^\infty f(\xi - u) \delta(u) \dd{u}      \\
	                                                       & = f(\xi)
\end{align*}
Hence,
\[
	\int_{-\infty}^\infty f(x) \delta\qty(-(x-\xi)) \dd{x} = \int_{-\infty}^\infty f(x) \delta(x-\xi) \dd{x}
\]
This is called the even property.
Now, consider
\[
	\int_{-\infty}^\infty f(x) \delta(a(x-\xi)) \dd{x} = \frac{1}{\abs{a}}f(\xi)
\]
This is the scaling property.
Let \( g(x) \) be a function with \( n \) isolated roots at \( x_1, \dots, x_n \).
Then, assuming \( g'(x) \) does not vanish at the \( x_i \),
\[
	\delta(g(x)) = \sum_{i=1}^n \frac{\delta(x - x_i)}{\abs{g'(x_i)}}
\]
This is a generalisation of the above, known as the advanced scaling property.
Now, if \( g(x) \) is continuous at \( x = 0 \), then \( g(x) \delta(x) \) equivalent to \( g(0) \delta(x) \) inside an integral.
This is known as the isolation property.
