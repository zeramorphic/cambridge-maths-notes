\subsection{Characteristic coordinates}
Transforming to characteristic coordinates \( u,v \) will set \( a = 0 \) and \( c = 0 \).
Hence, the PDE will take the canonical form
\[
	\pdv{\phi}{u}{v} + \dots + = 0
\]
where the omitted terms are lower order.
\begin{example}
	Consider
	\[
		-y \phi_{xx} + \phi_{yy} = 0
	\]
	Here, \( a = -y, b = 0, c = 1 \) hence \( b^2 - ac = y \).
	For \( y > 0 \), the equation is hyperbolic, for \( y < 0 \) it is elliptic, and for \( y = 0 \) it is parabolic.
	Consider the characteristics for \( y > 0 \).
	\[
		\dv{y}{x} = \frac{b \pm \sqrt{b^2 - ac}}{a} = \pm \frac{1}{\sqrt{y}}
	\]
	Hence,
	\[
		\int \sqrt{y} \dd{y} = \pm \int \dd{x} \implies \frac{2}{3} y^{\frac{3}{2}} \pm x = C_\pm
	\]
	Therefore, the characteristic curves are
	\[
		u = \frac{2}{3} y^{\frac{3}{2}} + x;\quad v = \frac{2}{3} y^{\frac{3}{2}} - x
	\]
	Taking derivatives,
	\[
		u_x = 1;\quad u_y = \sqrt{y};\quad v_x = -1;\quad v_y = \sqrt{y}
	\]
	Hence,
	\begin{align*}
		\phi_x    & = \phi_u u_x + \phi_v v_x = \phi_u - \phi_v                                      \\
		\phi_y    & = \sqrt{y} (\phi_u + \phi_v)                                                     \\
		\phi_{xx} & = \phi_{uu} - 2 \phi_{uv} + \phi_{vv}                                            \\
		\phi_{yy} & = y (\phi_{uu} + 2 \phi_{uv} + \phi_{vv}) + \frac{1}{2\sqrt{y}}(\phi_u + \phi_v)
	\end{align*}
	Substituting into the original PDE,
	\[
		-y \phi_{xx} + \phi_{yy} = y\qty(4 \phi_{uv} + \frac{1}{2y^{\frac{3}{2}}} (\phi_u + \phi_v) )
	\]
	Note, \( u + v = \frac{4}{3} y^{\frac{3}{2}} \), hence we have the canonical form
	\[
		4 \phi_{uv} + \frac{1}{6(u+v)} (\phi_u + \phi_v) = 0
	\]
\end{example}

\subsection{General solution to wave equation}
The wave equation is
\[
	\frac{1}{c^2} \pdv[2]{\phi}{t} - \pdv[2]{\phi}{x} = 0
\]
We wish to solve this with initial conditions \( \phi(x,0) = f(x) \), and \( \phi_t(x,0) = g(x) \).
Here, \( a = \frac{1}{c^2}, b = 0, c = -1 \) hence \( b^2 - ac > 0 \).
The characteristic equation is
\[
	\dv{x}{t} = \frac{0 \pm \sqrt{0 + \frac{1}{c^2}}}{\frac{1}{c^2}} = \pm c
\]
Hence the characteristic coordinates are
\[
	u = x - ct;\quad v = x + ct
\]
This yields the canonical form
\[
	\pdv{\phi}{u}{v} = 0
\]
This may be integrated directly to find
\[
	\pdv{\phi}{v} = F(v) \implies \phi = G(u) + \int^v F(y) \dd{y} = G(u) + H(v)
\]
Imposing the initial conditions at \( t = 0 \), we find
\[
	G(x) + H(x) = f(x);\quad -cG'(x) + cH'(x) = g(x)
\]
Differentiating the first equation, we find
\[
	G'(x) + H'(x) = f'(x)
\]
We can combine this with the second equation to give
\[
	H'(x) = \frac{1}{2} \qty(f'(x) + \frac{1}{c}g(x)) \implies H(x) = \frac{1}{2} \qty(f(x) - f(0)) + \frac{1}{2c}\int_0^x g(y) \dd{y}
\]
Similarly,
\[
	G'(x) = \frac{1}{2} \qty(f'(x) - \frac{1}{c}g(x)) \implies G(x) = \frac{1}{2} \qty(f(x) - f(0)) - \frac{1}{2c}\int_0^x g(y) \dd{y}
\]
The final solution is therefore
\[
	\phi(x,t) = G(x-ct) + H(x+ct) = \frac{1}{2}\qty(f(x-ct) + f(x+ct)) + \frac{1}{2c} \int_{x-ct}^{x+ct} g(y) \dd{y}
\]
Waves propagate at a velocity \( c \), hence \( \phi(x,t) \) is fully determined by values of \( f, g \) in the interval \( [x-ct, x+ct] \).

% new chapter: solving PDEs with Green's functions
\subsection{Diffusion equation and Fourier transform}
Recall the heat equation for a conducting wire given by
\[
	\pdv{\theta}{t}\qty(x,t) - D\pdv[2]{\theta}{x}\qty(x,t) = 0
\]
with initial conditions \( \theta(x,0) = h(x) \) and boundary conditions \( \theta \to 0 \) as \( x \to \pm \infty \).
Taking the Fourier transform with respect to \( x \),
\[
	\pdv{t} \widetilde \theta(k,t) = -D k^2 \widetilde \theta(k,t)
\]
Integrating, we find
\[
	\widetilde \theta(k,t) = C e^{-D k^2 t}
\]
The initial conditions give \( \widetilde \theta(k,0) = \widetilde h(k) \) and therefore
\[
	\widetilde \theta(k,t) = \widetilde h(k) e^{-Dk^2 t}
\]
We take the inverse Fourier transform to find
\[
	\theta(x,t) = \frac{1}{2\pi} \int_{-\infty}^\infty \widetilde h(k) \underbrace{e^{-Dk^2 t}}_{\mathclap{\text{FT of Gaussian}}} e^{ikx} \dd{k}
\]
Hence, by the convolution theorem,
\begin{align*}
	\theta(x,t) & = \frac{1}{\sqrt{4 \pi D t}} \int_{-\infty}^\infty h(u) \exp(-\frac{(x-u)^2}{4Dt}) \dd{u} \\
	            & \equiv \int_{-\infty}^\infty h(u) S_d(x-u,t) \dd{u}
\end{align*}
where the \textit{fundamental solution} is
\[
	S_d(x,t) = \frac{1}{\sqrt{4 \pi D t}} \exp(-\frac{x^2}{4Dt})
\]
which is the Fourier transform of \( \exp(-D k^2 t) \).
Note, with localised initial conditions \( \theta(x,0) = \theta_0 \delta(x) \), the solution is exactly the fundamental solution:
\[
	\theta(x,t) = \theta_0 S_d(x,t) = \frac{\theta_0}{\sqrt{4 \pi D t}} \exp(-\eta^2);\quad \eta = \frac{x}{2\sqrt{Dt}}
\]
where \( \eta \) is the similarity parameter.
