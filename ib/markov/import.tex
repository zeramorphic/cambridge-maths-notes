\chapter[Markov Chains \\ \textnormal{\emph{Lectured in Michaelmas \oldstylenums{2021} by \textsc{Dr.\ P.\ Sousi}}}]{Markov Chains}
\emph{\Large Lectured in Michaelmas \oldstylenums{2021} by \textsc{Dr.\ P.\ Sousi}}

A Markov chain is a common type of random process, where each state in the process depends only on the previous one.
Due to their simplicity, Markov processes show up in many areas of probability theory and have lots of real-world applications, for example in computer science.

One example of a Markov chain is a simple random walk, where a particle moves around an infinite lattice of points, choosing its next direction to move at random.
It turns out that if the lattice is one- or two-dimensional, the particle will return to its starting point infinitely many times, with probability 1.
However, if the lattice is three-dimensional or higher, the particle has probability 0 of ever returning to its starting point.

\subfile{../../ib/markov/main.tex}
