\subsection{Uniqueness of invariant distributions}
\begin{theorem}
	Let \( P \) be irreducible.
	Let \( \lambda \) be an invariant measure (\( \lambda = \lambda P \)) with \( \lambda_k = 1 \).
	Then \( \lambda \geq \nu_k \).
	If \( P \) is recurrent, then \( \lambda = \nu_k \).
\end{theorem}
\begin{proof}
	Let \( \lambda \) be an invariant measure with \( \lambda_k = 1 \).
	Then,
	\begin{align*}
		\lambda_i &= \sum_{j_1} \lambda_{j_1} P(j_1, i) \\
		&= P(k,i) + \sum_{j_1 \neq k} \lambda_{j_1} P(j_1, i) \\
		&= P(k,i) + \sum_{j_1 \neq k} P(k,j_1) P(j_1, i) + \sum_{j_1, j_2 \neq k} P(j_2, j_1) P(j_1, i) \lambda_{j_2} \\
		&= P(k,i) + \sum_{j_1 \neq k} P(k,j_1) P(j_1, i) + \dots + \sum_{j_1, \dots j_{n-1} \neq k} P(k,j_{n-1}) P(j_{n-1}, j_{n-2}) \dots P(j_2, j_1) P(j_1 i) + \underbrace{\sum_{j_1, \dots, j_n \neq k} P(j_n, j_{n-1}) \dots P(j_n, i) \lambda_{j_n}}_{\geq 0} \\
		&\geq \psub{k}{X_1 = i, T_k \geq 1} + \psub{k}{X_2 = i, T_k \geq 2} + \dots + \psub{k}{X_n = i, T_k \geq n} \\
		&\geq \sum_{i=1}^n \psub{k}{X_n = i, T_k \geq n} \\
		&\to \nu_k(i)
	\end{align*}
	as \( n \to \infty \).
	Now, suppose \( P \) is recurrent, so \( \nu_k \) is invariant.
	We define \( \mu = \lambda - \nu_k \).
	Then \( \mu \geq 0 \) is an invariant measure satisfying \( \mu_k = 0 \).
	We need to show \( \mu_i = 0 \) for all \( i \).
	By invariance, for all \( n \),
	\[ \mu_k = \sum_j \mu_j P^n(j,k) \]
	By irreducibility, we can choose \( n \) such that \( P^n(i,k) > 0 \).
	\[ \mu_k \geq P^n(i,k) \mu_i \implies \mu_i = 0 \]
\end{proof}
\begin{remark}
	In the irreducible and recurrent case, all invariant measures are equal up to a scaling factor.
\end{remark}
\noindent Let \( k \) be fixed.
Then, \( \nu_k \) is invariant, and unique in the above sense.
If \( \sum_i \nu_k(i) \) is finite, we can take
\[ \pi_i = \frac{\nu_k(i)}{\sum_j \nu_k(j)} \]
which is an invariant distribution.
The sum as required is
\begin{align*}
	\sum_{i \in I} \nu_k(i) &= \sum_{i\in I} \esub{k}{\sum_{n=0}^{T_k - 1} 1(X_n = i)} \\
	&= \esub{k}{\sum_{n=0}^{T_k - 1} \sum_{i \in I} 1(X_n = i)} \\
	&= \esub{k}{\sum_{n=0}^{T_k - 1} 1} \\
	&= \esub{k}{T_k} \\
\end{align*}
So we require that the expectation of the first return time is finite.
If \( \esub{k}{T_k} \) is finite, we can normalise \( \nu_k \) into a (unique) invariant distribution.
\begin{definition}
	Let \( k \in I \) be a recurrent state (so \( \psub{k}{T_k < \infty} = 1 \)).
	\( k \) is \textit{positive recurrent} if \( \esub{k}{T_k} < \infty \).
	\( k \) is called \textit{null recurrent} otherwise; so if \( \esub{k}{T_k} = \infty \).
\end{definition}
\begin{theorem}
	Let \( P \) be irreducible.
	Then the following are equivalent.
	\begin{enumerate}[(i)]
		\item every state is positive recurrent;
		\item some state is positive recurrent;
		\item \( P \) has an invariant distribution \( \pi \).
	\end{enumerate}
	If any of these conditions hold, we have
	\[ \pi_i = \frac{1}{\esub{i}{T_i}} \]
	for all \( i \).
\end{theorem}
\begin{proof}
	First, (i) clearly implies (ii).
	We now show (ii) implies (iii).
	Let \( k \) be the a positive recurrent state, and consider \( \nu_k \).
	Since \( k \) is recurrent, we know that \( \nu_k \) is an invariant measure.
	Then,
	\[ \sum_{i \in I} \nu_k(i) = \esub{k}{T_k} < \infty \]
	since \( k \) is positive recurrent.
	If we define 
	\[ \pi_i = \frac{\nu_k(i)}{\esub{k}{T_k}} \]
	we have that \( \pi \) is an invariant distribution.

	Now we show that (iii) implies (i).
	Let \( k \) be a state, which we will prove is positive recurrent.
	First, we show that \( \pi_k > 0 \).
	There exists \( i \) such that \( \pi_i > 0 \), and we will choose \( n \) such that \( P^n(i,k) > 0 \) by irreducibility.
	Then,
	\[ \pi_k = \sum_j \pi_j P^n(j,k) \geq \pi_i P^n(i,k) > 0 \]
	Now, we define \( \lambda_i = \frac{\pi_i}{\pi_k} \).
	This is an invariant measure with \( \lambda_k = 1 \).
	So from the above theorem, \( \lambda \geq \nu_k \).
	Now, since \( \pi \) is a distribution,
	\[ \esub{k}{T_k} = \sum_i \nu_k(i) \leq \sum_i \lambda_i = \sum_i \frac{\pi_i}{\pi_k} = \frac{1}{\pi_k} \sum_i \pi_i = \frac{1}{\pi_k} \]
	Hence \( \esub{k}{T_k} < \infty \), so \( k \) is positive recurrent.

	For the last part, we know that \( P \) is recurrent and \( \lambda_i = \frac{\pi_i}{\pi_k} \) is an invariant measure with \( \lambda_k = 1 \).
	From the previous theorem, \( \lambda_i = \nu_k(i) \).
	Hence, \( \frac{\pi_i}{\pi_k} = \nu_k(i) \).
	Taking the sum over all \( i \),
	\[ \frac{1}{\pi_k} = \esub{k}{T_k} \]
	which proves the last part.
\end{proof}
\begin{corollary}
	If \( P \) is irreducible and \( \pi \) is an invariant distribution, then for all \( x,y \), the expected number of visits to \( y \) starting from \( x \) is given by
	\[ \nu_x(y) = \frac{\pi(y)}{\pi(x)} \]
\end{corollary}
\begin{example}
	Consider the simple symmetric random walk on \( \mathbb Z \).
	We have proven that this is recurrent.
	Suppose \( \pi \) is an invariant measure.
	So \( \pi = \pi P \), giving
	\[ \pi_i = \frac{1}{2} \pi_{i-1} + \frac{1}{2} \pi_{i+1} \]
	So \( \pi_i = 1 \) is an invariant measure.
	So all invariant measures are multiples of this.
	But since this is not normalisable, there exists no invariant distribution.
	So this walk is null recurrent.
\end{example}
\begin{remark}
	If \( I \) is finite, we can always normalise the distribution, since we have only a finite sum.
\end{remark}
