\subsection{Trees}
Recall that the language \( \qty{0^k1^k \mid k > 0} \) is context-free but not regular, so context-free languages are indeed a proper superset of regular languages.
The structure of regular derivations was very simple; each intermediate step was of the form \( wA \) for a word \( w \) and a variable \( A \in V \).
However, the structure of context-free derivations is more complicated: we use a parse tree instead of a linear derivation.
\begin{definition}
	A set \( T \subseteq \mathbb N^\star \) is called a \emph{(finitely-branching) tree} if it is closed under initial segments, and for every \( t \in T \), there is a \emph{branching number} \( n \in \mathbb N \) such that for all \( k \), the sequence \( tk \) lies in \( T \) if and only if \( k < n \).
	A node \( t \in T \) with no sucessors is called a \emph{leaf}.
	The empty sequence, which is an element of every tree, is called the \emph{root}.
	A node \( t \in T \) has \emph{level} \( k \) if the length of the sequence is \( k \), so \( \abs{t} = k \).
	If \( T \) is finite, there is a maximum level, called the \emph{height} of the tree.
	For a node \( t \in T \), the sequence \( \eval{t}_0, \eval{t}_1, \dots, \eval{t}_{\abs{t}} = t \) is called the \emph{branch} leading to \( t \).
\end{definition}
\begin{remark}
	The first requirement is that if \( T \) is a tree, \( t \in T \) and \( s \subseteq t \) implies \( s \in T \).
\end{remark}
\begin{example}
	% TODO: visual example
\end{example}
\begin{definition}
	Let \( T \) be a tree and \( t \in T \).
	Then \( T_t = \qty{s \mid ts \in T} \) is the \emph{subtree starting from \( t \)}.
\end{definition}
\begin{definition}
	We define a partial order on \( T \) by \( t < s \) if \( t \neq s \) and if there exists \( k \) such that \( t(k) \neq s(k) \) and \( k_0 \) is minimal with this property, then \( t(k_0) < s(k_0) \).
	This is called the \emph{left-to-right order}.
\end{definition}
\begin{remark}
	This order is only a partial order since it does not order two distinct nodes that lie on the same branch, for example, \( 0 \) and \( 00 \).
	For each level \( k \), the nodes of length \( k \) are totally ordered.
	The leaves are totally ordered.
\end{remark}

\subsection{Parse trees}
\begin{definition}
	Let \( G \) be a context-free grammar.
	A pair \( \mathbb T = (T, \ell) \) is a \emph{\( G \)-parse tree} if \( T \) is a finitely-branching tree and \( \ell \colon T \to \Omega \) is a \emph{labelling function} such that:
	\begin{enumerate}
		\item \( \ell(\varepsilon) \in V \), we say \( T \) \emph{starts with} \( \ell(\varepsilon) \);
		\item if \( \ell(t) \in \Sigma \), \( t \) has no successors;
		\item if \( t \) has \( n + 1 \) successors and \( \ell(t) = A \in V \), then \( A \to \ell(t0) \ell(t1) \dots \ell(tn) \in P \).
	\end{enumerate}
	If \( \mathbb T = (T,\ell) \) is a \( G \)-parse tree, and \( t_0, \dots, t_m \) are its leaves written in the left-to-right order, then the \emph{string parsed by \( \mathbb T \)} is \( \sigma_{\mathbb T} = \ell(t_0) \dots \ell(t_m) \).
\end{definition}
\begin{remark}
	If \( t \in T \), \( \sigma_{\mathbb T} = \alpha \sigma_{\mathbb T_t} \beta \) where \( \mathbb T_t = (T_t, \ell_t) \), \( \ell_t(s) = \ell(ts) \).
\end{remark}
\begin{proposition}
	Let \( G \) be a context-free grammar.
	Then \( w \in \mathcal L(G) \) if and only if there is a \( G \)-parse tree \( \mathbb T \) starting from \( S \) such that \( \sigma_{\mathbb T} = w \).
\end{proposition}
\begin{proof}
	Observe that certain sequences of parse trees correspond to derivations.
	In particular, a sequence \( \mathbb T_0, \dots, \mathbb T_n \) of \( G \)-parse trees is \emph{derivative} if \( \mathbb T_0 = (\qty{\varepsilon}, \ell_0) \) with \( \ell_0(\varepsilon) = S \), and \( T_{i+1} \supseteq T_i \) constructed by considering a leaf \( t \in T_i \) such that \( \ell_i(t) = A \in V \) and \( A \to x_0 \dots x_n \in P \), and giving it \( n + 1 \) successors with \( \ell_{i+1}(tk) = x_k \).
	There is a one-to-one correspondence between \( G \)-derivations starting from \( S \) and such derivative sequences of parse trees.
	In particular, any derivation yields a derivative sequence of parse trees, and hence the last parse tree in the sequence has \( \sigma_{\mathbb T_n} = w \).

	Conversely, given a parse tree \( \mathbb T \), it suffices to construct such a derivative sequence of parse trees, because then the correspondence yields a derivation as required.
	We start with the trivial tree \( \mathbb T_0 = \qty(\qty{\varepsilon}, \eval{\ell}_{\qty{\varepsilon}}) \).
	In each step, suppose \( \mathbb T_0, \dots, \mathbb T_i \) already form a derivative sequence, and \( T_i \neq T \).
	Let \( t \in T \setminus T_i \).
	Then there is a terminal node in \( T_i \) on the branch containing \( t \) in \( T \), which is not a terminal node in \( T \).
	We can then create \( T_{i+1} \) by adding the \( T \)-successors of \( t \) to \( T_i \).
	Since \( T \) is finite, after finitely many steps we are done.
	In particular, \( \mathbb T_0, \dots, \mathbb T_n \) is a derivative sequence, and thus \( S \xrightarrow G \sigma_{\mathbb T_n} = \sigma_{\mathbb T} = w \) as required.
\end{proof}
Suppose \( \mathbb T \) is a parse tree and \( t \in T \) such that \( \ell(t) = A \), and \( \mathbb T' \) is a parse tree starting from \( A \).
Then, we can remove the subtree \( T_t \), and replace it with \( T' \), which also yields a parse tree.
This technique is known as \emph{grafting}.
\begin{definition}
	We define \( \graft(\mathbb T, t, \mathbb T') = (S, \ell^\star) \) where
	\[ S = \qty{s \in T \mid t \not\subseteq s} \cup \qty{tu \mid u \in T'} \]
	and
	\[ \ell^\star(s) = \begin{cases}
		\ell(s) & t \not\subseteq s \\
		\ell'(u) & \exists u \in T',\, s = tu
	\end{cases} \]
\end{definition}
Then we have
\[ \sigma_{\graft(\mathbb T, t, \mathbb T')} = \alpha \sigma_{\mathbb T'} \beta;\quad \sigma_{\mathbb T} = \alpha \sigma_{\mathbb T_t} \beta \]

\subsection{Chomsky normal form}
\begin{definition}
	A grammar is in \emph{Chomsky normal form} if all of its rules are of the form \( A \to BC \) or \( A \to a \).
	Rules of the form \( A \to BC \) are called \emph{binary}; rules of the form \( A \to a \) are called \emph{unary}.
\end{definition}
Every grammar in Chomsky normal form is context-free.
\begin{lemma}
	Let \( G \) be a grammar in Chomsky normal form, and \( w \in \mathcal L(G) \) with \( \abs{w} = n \).
	Then every \( G \)-derivation of \( w \) has length \( 2n - 1 \).
\end{lemma}
\begin{proof}
	Binary rules increment the length, and increment the variable count.
	Unary rules preserve the length, and decrement the variable count.
	Since \( w \) is comprised only of letters, exactly \( n - 1 \) binary rules and \( n \) unary rules were used.
\end{proof}
We will show that every context-free grammar is equivalent to a Chomsky normal form grammar, and there is an algorithm to produce such a grammar.
There are three types of rules that are obstructions to a context-free grammar being in Chomsky normal form:
\begin{enumerate}
	\item rules \( A \to x_0 \dots x_n \) with \( n \geq 2 \);
	\item rules \( A \to \alpha \) where \( \alpha \) contains both letters and variables, called \emph{mixed rules};
	\item rules of the form \( A \to B \), called \emph{unit rules}.
\end{enumerate}
