\subsection{Regular derivations}
\begin{definition}
	A rule of the form \( A \to a \) is called a \emph{terminal rule}.
	A rule of the form \( A \to aB \) is called a \emph{nonterminal rule}.
\end{definition}
\begin{lemma}
	Let \( G \) be a regular grammar.
	If \( S \xrightarrow G \alpha \), then \( \alpha \in \mathbb W \cup \mathbb W V \).
\end{lemma}
\begin{proof}
	This is shown by induction on the length of the derivation.
	The length-zero derivation gives \( \alpha = S = \varepsilon S \in \mathbb W V \).
	Suppose \( S \xrightarrow G \beta \xrightarrow G_1 \alpha \) where \( \beta \in \mathbb W \cup \mathbb W V \).
	If \( \beta \in \mathbb W \), \( \beta \) contains no variables, but all rules rewrite a variable.
	This contradicts that \( \beta \xrightarrow G_1 \alpha \).
	So suppose \( \beta = wA \) for \( w \in \mathbb W \), \( A \in V \).
	Then the rule must be of the form \( A \to a \) or \( A \to aB \).
	Hence \( \beta = wa \) or \( \beta = waB \).
	In either case, the required invariant holds.
\end{proof}
\begin{lemma}
	If \( S \xrightarrow G w \) for \( w \in \mathbb W \), then the derivation has length \( \abs{w} \) and consists of precisely \( \abs{w}-1 \) nonterminal rules and one final terminal rule.
\end{lemma}
\begin{proof}
	Terminal rules preserve the length of a string, and decrement the amount of variables.
	Nonterminal rules increment the length of a string, and preserve the amount of variables.
	Given that \( S \) is a string of length one with one variable, we must apply \( \abs{w}-1 \) nonterminal rules to increment the length of the string \( \abs{w}-1 \) times.
	By the previous lemma, the number of variables in each derived string is always 0 or 1.
	If the number ever reaches zero, nothing can be rewritten.
	Given \( w \in \mathbb W \), the number must reach zero, so a single terminal rule must be applied at the end.
\end{proof}
Note that the derivation is not uniquely determined.
\begin{lemma}
	Regular languages are closed under concatenation.
\end{lemma}
\begin{proof}
	Let \( G = (\Sigma, V, P, S), G' = (\Sigma, V', P', S') \), where without loss of generality \( V \cap V' = \varnothing \).
	Let \( P^\star \) be the set of production rules given by \( P \), but for each terminal rule \( A \to a \) in \( P \), replace it with a nonterminal rule \( A \to aS' \).
	Then let \( H = (\Sigma, V \cup V', P^\star \cup P', S) \).
	We claim \( \mathcal L(H) = \mathcal L(G)\mathcal L(G') \).

	Suppose \( S \xrightarrow G v \) and \( S' \xrightarrow {G'} w \).
	Then \( S \xrightarrow H vS' \), and so \( S \xrightarrow H vw \) as required.

	Conversely, suppose \( S \xrightarrow u \) for \( u \in \mathbb W \).
	By the above lemma, the derivation is of the form
	\[ S = \sigma_0 \xrightarrow H_1 \dots \xrightarrow H_1 \sigma_n = u \]
	where \( \sigma_i = w_i X_i \) for some \( w_i \in \mathbb W, X_i \in V \).
	An initial segment of the \( X_i \) belongs to \( V \), until rewritten as \( S' \) by a rule added into \( P^\star \).
	Then, the rest of the \( X_i \) belong to \( V' \), because only the new rules in \( P^\star \) map variables between \( V \) and \( V' \).
	Hence the derivation splits into two halves, \( u = vw \) where \( S \xrightarrow G v, S' \xrightarrow {G'} w \), giving the concatenation as required.
\end{proof}

\subsection{Deterministic automata}
\begin{definition}
	Let \( \Sigma \) be an alphabet.
	Then a \emph{deterministic automaton} is a tuple of the form \( D = (\Sigma, Q, \delta, q_0, F) \) where \( Q \) is a finite set of \emph{states}, \( q_0 \in Q \) is the \emph{start state}, \( F \subseteq Q \setminus \qty{q_0} \) is the \emph{accept states}, and \( \delta \colon Q \times \Sigma \to Q \) is the \emph{transition function}.
\end{definition}
We graphically represent deterministic automata using labelled directed graphs.
The nodes are elements of \( Q \), circled twice for accept states and circled once for other states.
Each node has \( \abs{\Sigma} \)-many outgoing arrows labelled with the corresponding letter.

\begin{center}
	\begin{tikzpicture}[shorten >=1pt, node distance=2cm, on grid, auto]
		\node[state, initial] (q_0) {\( q_0 \)};
		\node[state, accepting] (q_1) [above right=of q_0] {\( q_1 \)};
		\node[state] (q_2) [right=of q_0] {\( q_2 \)};

		\path[->]
			(q_0) edge node {0} (q_1)
			(q_0) edge node [swap] {1} (q_2)
			(q_2) edge node [swap] {0} (q_1)
			(q_2) edge [loop below] node {1} ()
			(q_1) edge [loop above] node {0, 1} ()
		;
	\end{tikzpicture}
\end{center}

We intuitively interpret a deterministic automaton as a machine that starts at \( q_0 \) and reads a word \( w \in \mathbb W \) symbol-by-symbol, transitioning to a new state according to \( \delta \) at each step.
After all symbols in the word are parsed, we check whether the machine lies in an accept state or not.
We say the automaton \emph{accepts} \( w \) if the final state is an accept state; otherwise, it \emph{rejects} \( w \).
\begin{definition}
	We define by recursion a function \( \hat\delta \colon Q \times \mathbb W \to Q \) by \( \hat\delta(q,\varepsilon) = q \) and \( \hat\delta(q,aw) = \hat\delta(\delta(q,a),w) \).
	The \emph{language accepted by \( D \)} is
	\[ \mathcal L(D) = \qty{w \mid \hat \delta(q_0, w) \in F} \]
	The sequence of states produced from \( q_0 \) and reading \( w \) is uniquely determined and of length \( \abs{w} + 1 \), known as the \emph{state sequence} of the computation.
\end{definition}
We claim that in the example above, \( \mathcal L(D) = \qty{w \mid w \text{ contains at least one } 0} \).
Note that \( \hat\delta(q_0,w) = q_0 \) if and only if \( w = \varepsilon \).
There are three transitions in the diagram for the letter 0, but all such 0-transitions lead to \( q_1 \) hence every string with a zero goes to \( q_1 \).
All transitions from \( q_1 \) go back to \( q_1 \), so any string containing a zero must end at \( q_1 \).
All other strings are of the form \( 1111\dots 1 \), which ends at \( q_2 \).
\begin{definition}
	Let \( D = (\Sigma, Q, \delta, q_0, F), D' = (\Sigma, Q', \delta', q_0', F') \) be deterministic automata.
	Then a map \( f \colon Q \to Q' \) is called a \emph{homomorphism} from \( D \) to \( D' \) if
	\begin{enumerate}
		\item for all \( q \in Q \) and \( a \in \Sigma \), we have \( \delta'(f(q),a) = f(\delta(q,a)) \);
		\item \( f(q_0) = q_0' \);
		\item \( q \in F \) implies \( f(q) \in F' \).
	\end{enumerate}
\end{definition}
\begin{proposition}
	Let \( f \) be a homomorphism from \( D \) to \( D' \).
	Then \( \mathcal L(D) \subseteq \mathcal L(D') \).
	In particular, if \( f \) is bijective, so is an \emph{isomorphism}, \( \mathcal L(D) = \mathcal L(D') \).
\end{proposition}
