\subsection{Definitions}
We have shown that \( \pi_1(S_1,x_0) \simeq \mathbb Z \), and \( \pi_1(S^n,x_0) \simeq 1 \) for \( n > 1 \), so \( S_1 \not\sim S^n \).
We would like to show that \( S^n \sim S^m \) only holds if \( n = m \).
One proof of this fact is that any \( f \colon S^n \to S^m \) with \( n < m \) is null-homotopic, but the identity on \( S^m \) is not.
Both of these claims require proof: simplicial complexes will allow us to prove the first, and homology will allow us to prove the second.
\begin{definition}
	The \emph{\( n \)-simplex} is the topological space
	\[ \Delta^n  = \qty{(x_0, \dots, x_n) \in \mathbb R^{n+1} \midd x_i \geq 0, \sum_{i=0}^n x_i = 1 } \]
	with the subspace topology.
\end{definition}
\begin{remark}
	\( \Delta^1 \) is homeomorphic to \( I \).
	\( \Delta^2 \) is an equilateral triangle, and \( \Delta^3 \) is a regular tetrahedron.
	For all \( n \), \( \Delta^n \) is closed and bounded in \( \mathbb R^{n+1} \), and hence compact and Hausdorff.
	The standard basis vectors \( e_0, \dots, e_n \) are the vertices of \( \Delta^n \).
\end{remark}
\begin{definition}
	If \( I \subseteq \qty{0, \dots, n} \), the \emph{\( I \)th face of \( \Delta^n \)} is
	\[ e_I = \qty{x \in \Delta^n \mid x_i = 0 \text{ for } i \not\in I} \]
	We define \( F(\Delta^n) = \qty{e_I \mid I \subseteq \qty{0,\dots, n}} \) to be the set of faces of \( \Delta^n \).
\end{definition}
If \( I = \qty{i_0, \dots, i_k} \) with \( i_0 < \dots < i_k \), we write \( I = i_0i_1\dots i_k \).
\begin{remark}
	Note that \( e_{\qty{i}} = e_i \), and \( \Delta^n = e_{\qty{0,1, \dots, n}} \).
	\( e_I \) is a closed subset of \( \Delta^n \), and is homeomorphic to \( \Delta^{\abs{I} - 1} \).
	\( e_I \subseteq e_J \) if and only if \( I \subseteq J \).
	\( e_I \cap e_J = e_{I \cap J} \).
\end{remark}
\begin{definition}
	A map \( \abs{f} \colon \Delta^n \to \mathbb R^N \) is \emph{affine linear} if it is the restriction of a linear map \( \mathbb R^{n+1} \to \mathbb R^n \).
	Equivalently, \( \abs{f}\qty(\sum_{i=0}^n x_i e_i) = \sum_{i=0}^n x_i \abs{f}(e_i) \).
	We say an affine linear map \( \abs{f} \colon \Delta^n \to \Delta^m \) is \emph{simplicial} if it maps vertices in \( \Delta^n \) to vertices in \( \Delta^m \), so there is a map of sets \( \hat f \colon \qty{0,\dots, n} \to \qty{0, \dots, m} \) where \( \abs{f}(e_i) = e_{\hat f(i)} \).
\end{definition}
\begin{remark}
	Affine linear maps are continuous, and are determined entirely by their action on \( e_i \).
	In particular, simplicial maps \( \abs{f} \) are determined by \( \hat f \).
	For \( I \subseteq \qty{0, \dots, n} \), we have \( \abs{f}(e_I) = e_{\hat f(I)} \).
\end{remark}
\begin{definition}
	Vectors \( v_0, \dots, v_n \in \mathbb R^N \) are \emph{affine linearly independent} if whenever \( \sum t_i v_i = 0 \) and \( \sum t_i = 0 \), we have \( t_i = 0 \) for all \( i \).
	Equivalently,
	\begin{enumerate}
		\item If \( \sum t_i v_i = \sum t_i' v_i' \) and \( \sum t_i = \sum t_i' \), then for each \( i \), \( t_i = t_i' \).
		\item The vectors \( v_1 - v_0, v_2 - v_0, \dots, v_n - v_0 \) are linearly independent.
		\item The unique affine linear map \( \abs{f} \colon \Delta^n \to \mathbb R^N \) given by \( \abs{f}(e_i) = v_i \) is injective.
	\end{enumerate}
	If \( v_0, \dots, v_n \) are affine linearly independent, we write \( [v_0, \dots, v_n] = \Im \abs{f} = \qty{\sum x_i v_i \mid \sum x_i = 1, x_i \geq 0} \), and we say \( [v_0, \dots, v_n] \) is a \emph{Euclidean simplex}.
\end{definition}
\begin{remark}
	\( \Delta^n \) is compact and Hausdorff, so \( \abs{f} \colon \Delta^n \to [v_0, \dots, v_n] \) is a homeomorphism if the \( v_i \) are affine linearly independent.
\end{remark}
\begin{lemma}
	If \( X \subseteq \mathbb R^N \), let \( Z(X) \) be the set of \( x \in X \) such that if \( x = \sum t_i x_i \) for \( t_i > 0, \sum t_i = 1 \) and all \( x_i \in X \), then \( x_i = x \).
	Then \( Z([v_0, \dots, v_n]) = \qty{v_0, \dots, v_n} \).
\end{lemma}
\begin{proof}
	We show that \( v_k \in Z([v_0, \dots, v_n]) \); the converse is clear from the definition of the simplex.
	Suppose \( v_k = \sum t_i x_i \) for \( t_i > 0 \) and \( \sum t_i = 1 \).
	Then \( x_i = \sum_{j=0}^n s_{ij} v_j \), since \( x_i \in [v_0, \dots, v_n] \).
	So \( v_k = \sum_j \qty(\sum_i t_i s_{ij}) v_j \).
	Since the \( v_i \) are affine linearly independent, and \( \sum_j \qty(\sum_i t_i s_{ij}) = 1 \), we must have \( \sum t_i s_{ij} = 0 \) for \( j \neq k \).
	But \( t_i > 0 \) and \( s_{ij} \geq 0 \), so the only case is when all \( s_{ij} \) are exactly zero for \( j \neq k \), so \( x_j = v_k \).
\end{proof}
\begin{corollary}
	If \( [v_0, \dots, v_n] = [v_0', \dots, v_n'] \) as subsets of \( \mathbb R^N \), then \( \qty{v_0, \dots, v_n} = \qty{v_0', \dots, v_n'} \) as sets.
\end{corollary}
Therefore, a simplex determines its set of vertices.
\begin{proof}
	\( \qty{v_0, \dots, v_n} = Z([v_0, \dots, v_n]) = Z([v_0', \dots, v_n']) = \qty{v_0', \dots, v_n'} \).
\end{proof}
\begin{definition}
	\( \mathcal S(\mathbb R^n) \) is the set of Euclidean simplices \( \sigma \subseteq \mathbb R^n \).
	Hence, \( \mathcal S(\mathbb R^n) \) is in bijection with the set \( \qty{\qty{v_0, \dots, v_k} \mid v_i \in \mathbb R^N, k \geq -1, v_i \text{ affine linearly independent}} \).
\end{definition}

\subsection{Abstract simplicial complexes}
\begin{definition}
	An \emph{abstract simplicial complex} in \( \Delta^n \) is a subset \( K \) of the faces \( F(\Delta^n) \) such that \( e_I \in K \) whenever \( e_J \) is in \( K \) and \( I \subseteq J \).
\end{definition}
\begin{remark}
	Abstract simplicial complexes are downward-closed sets of faces.
	They have no intrinsic topology.
\end{remark}
\begin{definition}
	If \( K \) is an abstract simplicial complex, its \emph{polyhedron} is \( \abs{K} = \bigcup_{e_I \in K} e_i \subseteq \Delta^n \).
\end{definition}
