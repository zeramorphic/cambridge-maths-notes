\subsection{Definitions}
We will study the possible sizes of sets in \( \mathsf{ZFC} \).
Write \( x \leftrightarrow y \) if there exists a bijection from \( x \) to \( y \); we wish to define \( \mathrm{card}(x) = \abs{x} \) such that \( x \leftrightarrow y \) if and only if \( \mathrm{card}(x) = \mathrm{card}(y) \).
This cannot be formulated as an equivalence class, due to Russell's paradox.
However, for any \( x \), there exists an ordinal \( \alpha \) such that \( x \leftrightarrow \alpha \) by the well-ordering theorem.
Hence, we can define \( \mathrm{card}(x) \) to be the least ordinal that \( x \) bijects with.
We say that a set \( m \) is a \emph{cardinality} or a \emph{cardinal} if \( m = \mathrm{card}(x) \) for some set \( x \).

If we were studying sets in \( \mathsf{ZF} \) and not \( \mathsf{ZFC} \), there may not be an ordinal that bijects with a given set \( x \).
However, we can apply \emph{Scott's trick}, which is as follows.
We can consider the least \( \alpha \) such that there exists \( y \leftrightarrow x \) with \( \mathrm{rank}(y) = \alpha \).
This is often called the \emph{essential rank} of \( x \).
In this case, we let \( \mathrm{card}(x) \) be the set \( \qty{y \subseteq V_\alpha \mid y \leftrightarrow x} \).

\subsection{The hierarchy of alephs}
