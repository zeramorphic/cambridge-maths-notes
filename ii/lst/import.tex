\chapter[Logic and Set Theory \\ \textnormal{\emph{Lectured in Lent \oldstylenums{2023} by \textsc{Prof.\ I.\ B.\ Leader}}}]{Logic and Set Theory}
\emph{\Large Lectured in Lent \oldstylenums{2023} by \textsc{Prof.\ I.\ B.\ Leader}}

Mathematics is the study of logical systems, and proving true statements about them.
In this course, we make precise the notion of a proof, and what it means for a logical sentence to be true.
This allows us to reason about truth mathematically rather then philosophically.
One important result, the completeness theorem, states that a sentence is true exactly when it has a proof.
This assures us that proofs are a sensible way of showing that a statement is true, and shows us that if a statement is false there must be a counterexample.

A major application of our theory of logic is set theory.
With it, we can formalise the intuitive notion of a set into a concrete mathematical object that can be studied in its own right.
We can prove results about sets and set theory itself without worrying about circular logic.

To learn about the structure of the universe of sets, we will study ordinals and cardinals, which are different kinds of transfinite number.
Ordinals measure discrete processes that are allowed to continue past infinity.
They have rich structure, and are used to prove important and far-reaching results, such as Zorn's lemma.
Cardinals measure the sizes of sets.
Both ordinals and cardinals have their own arithmetic, which allow us to reason about various kinds of composition of sets and orders.

\subfile{../../ii/lst/main.tex}
