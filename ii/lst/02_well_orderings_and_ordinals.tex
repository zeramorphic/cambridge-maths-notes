\subsection{Definition}
\begin{definition}
    A \emph{total order} or \emph{linear order} is a pair \( (X, <) \) where \( X \) is a set, and \( < \) is a relation on \( X \) such that
    \begin{itemize}
        \item (irreflexivity) for all \( x \in X \), \( x \not < x \);
        \item (transitivity) for all \( x, y, z \in X \), \( x < y \) and \( y < z \) implies \( x < z \);
        \item (trichotomy) for all \( x, y \in X \), either \( x < y \), \( y < x \), or \( x = y \).
    \end{itemize}
\end{definition}
We use the obvious notation \( x > y \) to denote \( y < x \).
In terms of the \( \leq \) relation, we can equivalently write the axioms of a total order as
\begin{itemize}
    \item (reflexivity) for all \( x \in X \), \( x \leq x \);
    \item (transitivity) for all \( x, y, z \in X \), \( x \leq y \) and \( y \leq z \) implies \( x \leq z \);
    \item (antisymmetry) for all \( x, y \in X \), if \( x \leq y \) and \( y \leq x \) then \( x = y \).
    \item (trichotomy, or totality) for all \( x, y \in X \), either \( x \leq y \) or \( y \leq x \).
\end{itemize}
\begin{example}
    \begin{enumerate}
        \item \( (\mathbb N, \leq) \) is a total order.
        \item \( (\mathbb Q, \leq) \) is a total order.
        \item \( (\mathbb R, \leq) \) is a total order.
        \item \( (\mathbb N^+, |) \) is not a total order, where \( | \) is the divides relation, since \( 2 \) and \( 3 \) are not related.
        \item \( (\mathcal P(S), \subseteq) \) is not a total order if \( \abs{S} > 1 \), since it fails trichotomy.
    \end{enumerate}
\end{example}
\begin{definition}
    A total order \( (X, <) \) is a \emph{well-ordering} if every nonempty subset \( S \subseteq X \) has a least element.
    \[ \forall S \subseteq X,\, S \neq 0 \implies \exists x \in S,\, \forall y \in S,\, x \leq y \]
\end{definition}
\begin{example}
    \begin{enumerate}
        \item \( (\mathbb N, <) \) is a well-ordering.
        \item \( (\mathbb Z, <) \) is not a well-ordering, since \( \mathbb Z \) has no least element.
        \item \( (\mathbb Q, <) \) is not a well-ordering.
        \item \( (\mathbb R, <) \) is not a well-ordering.
        \item \( [0,1] \subset \mathbb R \) with the usual order is not a well-ordering, since \( (0,1] \) has no least element.
        \item \( \qty{\frac{1}{2}, \frac{2}{3}, \frac{3}{4}, \dots} \subset \mathbb R \) with the usual order is a well-ordering.
        \item \( \qty{\frac{1}{2}, \frac{2}{3}, \frac{3}{4}, \dots} \cup \qty{1} \) with the usual order is also a well-ordering.
        \item \( \qty{\frac{1}{2}, \frac{2}{3}, \frac{3}{4}, \dots} \cup \qty{2} \) with the usual order is another example.
        \item \( \qty{\frac{1}{2}, \frac{2}{3}, \frac{3}{4}, \dots} \cup \qty{1 + \frac{1}{2}, 1 + \frac{2}{3}, 1 + \frac{3}{4}, \dots} \) is another example.
    \end{enumerate}
\end{example}
\begin{remark}
    Let \( (X, <) \) be a total order.
    \( (X, <) \) is a well-ordering if and only if there is no infinite decreasing sequence \( x_1 > x_2 > \dots \).
    Indeed, if \( (X, <) \) is a well-ordering, then the set \( \qty{x_1, x_2, \dots} \) has no minimal element, contradicting the assumption.
    Conversely, if \( S \subseteq X \) has no minimal element, then we can construct an infinite decreasing sequence by arbitrarily choosing points \( x_1 > x_2 > \dots \) in \( S \), which exists as \( S \) has no minimal element.
\end{remark}
\begin{definition}
    Total orders \( X, Y \) are \emph{isomorphic} if there is a bijection \( f \) between \( X \) and \( Y \) that preserves \( < \): \( x < y \) if and only if \( f(x) < f(y) \).
\end{definition}
Examples (i) and (vi) are isomorphic, and (vii) and (viii) are isomorphic.
Examples (i) and (vii) are not isomorphic, since example (vii) has a greatest element and (i) does not.
\begin{proposition}[proof by induction]
    Let \( X \) be a well-ordered set, and let \( S \subseteq X \) such that
    \[ \forall x \in S,\,(\forall y < x,\, y \in S) \implies x \in S \]
    Then \( S = X \). 
\end{proposition}
\begin{remark}
    Equivalently, if \( p(x) \) is a property such that if \( p(y) \) is true for all \( y < x \) then \( p(x) \), then \( p(x) \) holds for all \( x \).
\end{remark}
\begin{proof}
    Suppose \( S \neq X \).
    Then \( X \setminus S \) is nonempty, and therefore has a least element \( x \).
    But all elements \( y < x \) lie in \( S \), and so by the property of \( S \), we must have \( x \in S \), contradicting the assumption.
\end{proof}
\begin{proposition}
    Let \( X, Y \) be isomorphic well-orderings.
    Then there is exactly one isomorphism between \( X \) and \( Y \).
\end{proposition}
Note that this does not hold for general total orderings, such as \( \mathbb Q \) to itself or \( [0,1] \) to itself.
\begin{proof}
    Let \( f, g \colon X \to Y \) be isomorphisms.
    We show that \( f(x) = g(x) \) for all \( x \) by induction on \( x \).
    Suppose \( f(y) = g(y) \) for all \( y < x \).
    We must have that \( f(x) = a \), where \( a \) is the least element of \( Y \setminus \qty{f(y) \mid y < x} \).
    Indeed, if not, we have \( f(x') = a \) for some \( x' > x \) by bijectivity, contradicting the order-preserving property.
    Note that the set \( Y \setminus \qty{f(x) \mid y < x} \) is nonempty as it contains \( f(x) \).
    So \( f(x) = a = g(x) \), as required.
\end{proof}

\subsection{Initial segments}
\begin{definition}
    A subset \( I \) of a totally ordered set \( X \) is an \emph{initial segment} if \( x \in I \) implies \( y \in I \) for all \( y < x \).
\end{definition}
\begin{example}
    In any total ordering \( X \) and element \( x \in X \), the set \( \qty{y \mid y < x} \) is an initial segment.
    Not every initial segment is of this form, for instance \( \qty{x \mid x \leq 3} \) in \( \mathbb R \), or \( \qty{x \mid x > 0, x^2 < 2} \) in \( \mathbb Q \).

    In a well-ordering, every proper initial segment \( I \neq X \) is of this form.
    Indeed, \( I = \qty{y \mid y < x} \) where \( x \) is the least element of \( X \setminus I \): \( y \in I \) implies \( y < x \), otherwise \( y = x \) or \( x < y \), giving the contradiction \( x \in I \); and conversely, \( y < x \) implies \( y \in I \), otherwise \( y \) is a smaller element of \( X \setminus I \).
\end{example}
\begin{theorem}[definition by recursion]
    Let \( X \) be a well-ordering and \( Y \) be any set.
    Let \( G \colon \mathcal P(X \times Y) \to Y \) be a rule that assigns a point in \( Y \) given a definition of the function `so far', represented as a set of ordered pairs.
    Then there exists a function \( f \colon X \to Y \) such that \( f(x) = G\qty(\eval{f}_{I_x}) \), and such a function is unique.
\end{theorem}
\begin{remark}
    In defining \( f(x) \), we may use the value of \( f(y) \) for all \( y < x \).
\end{remark}
\begin{proof}
    We say that \( h \) is an \emph{attempt} to mean that \( h \colon I \to Y \) where \( I \) is some initial segment of \( X \), and for all \( x \in I \) we have that \( h(x) = G\qty(\eval{h}_{I_x}) \).
    Note that if \( h, h' \) are attempts both defined at \( x \), then \( h(x) = h'(x) \) by induction on \( x \).
    
    Also, for all \( x \), there exists an attempt defined at \( x \), by induction on \( x \).
    Indeed, by induction we can assume there exists an attempt \( h_y \) defined at \( y \) for all \( y < x \), and then we can define \( h \) to be the union of the \( h_y \).
    This is an attempt with domain \( I_x \), so the attempt \( h' = h \cup \qty{x, G(h)} \) is an attempt defined at \( x \).
    Therefore, there is an attempt defined at each \( x \), so we can define the function \( f \colon X \to Y \) by setting \( f(x) \) to be the value of \( h(x) \) where \( h \) is some attempt defined at \( x \).

    For uniqueness, we apply induction on \( x \).
    If \( f, f' \) agree below \( x \), then they must agree at \( x \) since \( f(x) = G\qty(\eval{f}_{I_x}) = G\qty(\eval{f'}_{I_x}) = f'(x) \).
\end{proof}
\begin{proposition}[subset collapse]
    Any subset \( Y \) of a well-ordering \( X \) is isomorphic to a unique initial segment of \( X \).
\end{proposition}
This is not true for general total orderings, such as \( \qty{1, 2, 3} \subset \mathbb Z \), or \( \mathbb Q \) in \( \mathbb R \).
\begin{proof}
    If \( f \) is some such isomorphism, we must have that \( f(x) \) is the least element of \( X \) not of the form \( f(y) \) for \( y < x \).
    We define \( f \) in this way by recursion, and this is an isomorphism as required.
    Note that this is always well-defined as \( f(y) \leq y \), so there is always some element of \( X \) (namely, \( x \)) not of the form \( f(y) \) for \( y < x \).
    Uniqueness follows by induction.
\end{proof}
\begin{remark}
    \( X \) itself cannot be isomorphic to a proper initial segment by uniqueness as it is isomorphic to itself.
\end{remark}

\subsection{Relating well-orderings}
\begin{definition}
    For well-orderings \( X, Y \), we will write \( X \leq Y \) if \( X \) is isomorphic to an initial segment of \( Y \).    
\end{definition}
\( X \leq Y \) if and only if \( X \) is isomorphic to some subset of \( Y \).
\begin{example}
    \( \mathbb N \leq \qty{\frac{1}{2}, \frac{2}{3}, \dots} \).
\end{example}
\begin{proposition}
    Let \( X, Y \) be well-orderings.
    Then either \( X \leq Y \) or \( Y \leq X \).
\end{proposition}
\begin{proof}
    Suppose \( Y \not\leq X \), and we will show \( X \leq Y \).
    By recursion we define the function \( f \colon X \to Y \) by letting \( f(x) \) be the least element of \( Y \) not of the form \( f(y) \) for all \( y < x \).
    If a least element of this form always exists, this is a well-defined isomorphism from \( X \) to an initial segment of \( Y \) as required.
    Suppose that \( Y \setminus \qty{f(y) \mid y < x} \) is empty, so \( \qty{f(y) \mid y < x} \).
    Then \( Y \) is isomorphic to \( I_x \subseteq X \), contradicting the assumption that \( Y \not\leq X \).
\end{proof}
\begin{proposition}
    Let \( X, Y \) be well-orderings, and suppose \( X \leq Y \) and \( Y \leq X \).
    Then \( X \) is isomorphic to \( Y \).
\end{proposition}
\begin{proof}
    Let \( f \colon X \to Y \) and \( g \colon Y \to X \) be isomorphisms to initial segments.
    Then \( g \circ f \) is an isomorphism from \( X \) to some initial segment of \( X \), as an initial segment of an initial segment is an initial segment.
    So by uniqueness, \( g \circ f \) is the identity map on \( X \).
    Similarly, \( f \circ g \) is the identity on \( Y \), so \( f \) and \( g \) are inverses.
\end{proof}
