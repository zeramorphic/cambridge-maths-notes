In this section, we will attempt to understand the structure of the universe of sets.
In order to do this, we will treat set theory as a first-order theory like any other, and can therefore study it with our usual tools.
In particular, we will study a particular theory called \emph{Zermelo--Fraenkel set theory}, denoted \( \mathsf{ZF} \).
The language has \( \Omega = \varnothing, \Pi = \qty{\in}, \alpha(\in) = 2 \).
A `universe of sets' is simply a model \( (V, \in_V) = (V, \in) \) for the axioms of \( \mathsf{ZF} \).
% This theory has nine axioms: two to begin the theory, four to construct sets, and three that might be unintuitive at first.
We can view this section as a worked example of the concepts of predicate logic, but every model of \( \mathsf{ZF} \) will contain a copy of (most of) mathematics, so they will be very complicated.

\subsection{Axioms of \texorpdfstring{\(\mathsf{ZF}\)}{ZF}}
\begin{enumerate}
    \item \emph{Axiom of extension}.
    \[ (\forall x)(\forall y)((\forall z)(z \in x \Leftrightarrow z \in y) \Rightarrow x = y) \]
    Note that the converse follows from the definition of equality.
    This implies that sets have no duplicate elements, and have no ordering.
    \item \emph{Axiom of separation} or \emph{comprehension}.
    For a set \( x \) and a property \( p \), we can form the set of \( z \in x \) such that \( p(z) \) holds.
    \[ (\forall t_1)\dots(\forall t_n)(\forall x)(\exists y)(\forall z)(z \in y \Leftrightarrow z \in x \wedge p) \]
    where the \( t_i \) are the parameters, and \( p \) is a formula with free variables \( t_1, \dots, t_n, z \).
    Note that we need the parameters as we may wish to form the set \( \qty{z \in x \mid z \in t} \) for some variable \( t \).
    We write \( \qty{z \in x \mid p(z)} \) for the set guaranteed by this axiom; this is an abbreviation and does not change the language.
    \item \emph{Empty-set axiom}.
    \[ (\exists x)(\forall y)(\neg y \in x) \]
    This empty set is unique by extensionality.
    We write \( \varnothing \) for the set guaranteed by this axiom.
    For instance, \( p(\varnothing) \) is the sentence \( (\exists x)((\forall y)(\neg y \in x) \wedge p(x)) \).
    \item \emph{Pair-set axiom}.
    \[ (\forall x)(\forall y)(\exists z)(\forall t)(t \in z \Leftrightarrow t = x \vee t = y) \]
    We write \( \qty{x, y} \) for this set \( z \), which is unique by extensionality.
    Some basic set-theoretic principles can now be defined.
    \begin{itemize}
        \item We write \( \qty{x} = \qty{x, x} \) for the singleton set containing \( x \).
        \item We can now define the ordered pair \( (x, y) = \qty{\qty{x}, \qty{x, y}} \); from the axioms so far we can prove that \( (x, y) = (z, t) \) if and only if \( x = z \) and \( y = t \).
        \item We say that \( x \) is an ordered pair if \( (\exists y)(\exists z)(x = (y,z)) \), and \( f \) is a function if \( (\forall x)(x \in f \Rightarrow x \text{ is an ordered pair}) \) and \( (\forall x)(\forall y)(\forall z)((x,y) \in f \wedge (x,z) \in f \Rightarrow y = z) \).
        \item We call a set \( x \) the domain of \( f \), written \( x = \dom f \), if \( f \) is a function and \( (\forall y)(y \in x \Leftrightarrow (\exists z)((y,z) \in f)) \).
        \item The notation \( f \colon x \to y \) means that \( f \) is a function, \( x = \dom f \), and \( (\forall z)(\forall t)((z, t) \in f \Rightarrow t \in y) \).
    \end{itemize}
    \item \emph{Union axiom}.
    For each family of sets \( x \), we can form its union \( \bigcup_{t \in x} t \).
    \[ (\forall x)(\exists y)(\forall z)(z \in y \Leftrightarrow (\exists t)(z \in t \wedge t \in x)) \]
    The set guaranteed by this axiom can be written \( \bigcup x \), and we can write \( x \cup y \) for \( \bigcup \qty{x, y} \).
    We need no intersection axiom, as such intersections already exist by the axiom of separation.
    This cannot be used to create empty intersections, as the axiom of separation can only create subsets of a set that already exists.
    \item \emph{Power-set axiom}.
    \[ (\forall x)(\exists y)(\forall z)(z \in y \Leftrightarrow z \subseteq x) \]
    where \( z \subseteq x \) means \( (\forall t)(t \in z \Rightarrow t \in x) \).
    We write \( \mathcal P(x) \) for the power set of \( x \).
    We can form the Cartesian product \( x \times y \) as a suitable subset of \( \mathcal P(\mathcal P(x \cup y)) \), as if \( z \in x, t \in y \), we have \( (z, t) = \qty{\qty{z}, \qty{z, t}} \in \mathcal P(\mathcal P(x \cup y)) \).
    The set of all functions \( x \to y \) can be defined as a subset of \( \mathbb P(x \times y) \).
    \item \emph{Axiom of infinity}.
    Using our currently defined axioms, any model \( V \) must be infinite.
    For example, writing \( x^+ \) for the \emph{successor} of \( x \) defined as \( x \cup \qty{x} \), the sets \( \varnothing, \varnothing^+, \varnothing^{++}, \dots \) are distinct.
    \[ \varnothing^+ = \qty{\varnothing};\quad \varnothing^{++} = \qty{\varnothing, \qty{\varnothing}};\quad \varnothing^{+++} = \qty{\varnothing, \qty{\varnothing}, \qty{\varnothing, \qty{\varnothing}}};\quad \dots \]
    We write \( 0 = \varnothing, 1 = \varnothing^+, 2 = \varnothing^{++}, \dots \) for the successors created in this way.
    For instance, \( 3 = \qty{0, 1, 2} \).
    \( V \) may not have an infinite element, even though \( V \) itself is infinite, because no \( x \in V \) has all \( y \in V \) as elements: \( V \) does not think of itself as a set, because Russell's paradox follows from the axioms defined so far.

    We say that \( x \) is a successor set if \( \varnothing \in x \) and \( (\forall y)(y \in x \Rightarrow y^+ \in x) \).
    Note that this is a finite-length formula that characterises an infinite set.
    The axiom of infinity is that there exists a successor set.
    \[ (\exists x)(\varnothing \in x \wedge (\forall y)(y \in x \Rightarrow y^+ \in x)) \]
    Note that this set is not uniquely defined, but any intersection of successor sets is a successor set.
    We can therefore take the intersection of all successor sets by the axiom of separation, giving a least successor set denoted \( \omega \).
    Thus, \( (\forall x)(x \in \omega \Leftrightarrow (\forall y)(y \text{ is a successor set} \Rightarrow x \in y)) \).
    For example, we can prove that \( 3 \in \omega \).
\end{enumerate}
