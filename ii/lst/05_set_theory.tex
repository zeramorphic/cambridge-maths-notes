In this section, we will attempt to understand the structure of the universe of sets.
In order to do this, we will treat set theory as a first-order theory like any other, and can therefore study it with our usual tools.
In particular, we will study a particular theory called \emph{Zermelo--Fraenkel set theory}, denoted \( \mathsf{ZF} \).
The language has \( \Omega = \varnothing, \Pi = \qty{\in}, \alpha(\in) = 2 \).
A `universe of sets' is simply a model \( (V, \in_V) = (V, \in) \) for the axioms of \( \mathsf{ZF} \).
We can view this section as a worked example of the concepts of predicate logic, but every model of \( \mathsf{ZF} \) will contain a copy of (most of) mathematics, so they will be very complicated.

\subsection{Axioms of \texorpdfstring{\(\mathsf{ZF}\)}{ZF}}
\begin{enumerate}
    \item \emph{Axiom of extension}.
    \[ (\forall x)(\forall y)((\forall z)(z \in x \Leftrightarrow z \in y) \Rightarrow x = y) \]
    Note that the converse follows from the definition of equality.
    This implies that sets have no duplicate elements, and have no ordering.
    \item \emph{Axiom of separation} or \emph{comprehension}.
    For a set \( x \) and a property \( p \), we can form the set of \( z \in x \) such that \( p(z) \) holds.
    \[ (\forall t_1)\dots(\forall t_n)(\forall x)(\exists y)(\forall z)(z \in y \Leftrightarrow z \in x \wedge p) \]
    where the \( t_i \) are the parameters, and \( p \) is a formula with free variables \( t_1, \dots, t_n, z \).
    Note that we need the parameters as we may wish to form the set \( \qty{z \in x \mid z \in t} \) for some variable \( t \).
    We write \( \qty{z \in x \mid p(z)} \) for the set guaranteed by this axiom; this is an abbreviation and does not change the language.
    \item \emph{Empty-set axiom}.
    \[ (\exists x)(\forall y)(\neg y \in x) \]
    This empty set is unique by extensionality.
    We write \( \varnothing \) for the set guaranteed by this axiom.
    For instance, \( p(\varnothing) \) is the sentence \( (\exists x)((\forall y)(\neg y \in x) \wedge p(x)) \).
    \item \emph{Pair-set axiom}.
    \[ (\forall x)(\forall y)(\exists z)(\forall t)(t \in z \Leftrightarrow t = x \vee t = y) \]
    We write \( \qty{x, y} \) for this set \( z \), which is unique by extensionality.
    Some basic set-theoretic principles can now be defined.
    \begin{itemize}
        \item We write \( \qty{x} = \qty{x, x} \) for the singleton set containing \( x \).
        \item We can now define the ordered pair \( (x, y) = \qty{\qty{x}, \qty{x, y}} \); from the axioms so far we can prove that \( (x, y) = (z, t) \) if and only if \( x = z \) and \( y = t \).
        \item We say that \( x \) is an ordered pair if \( (\exists y)(\exists z)(x = (y,z)) \), and \( f \) is a function if \( (\forall x)(x \in f \Rightarrow x \text{ is an ordered pair}) \) and \( (\forall x)(\forall y)(\forall z)((x,y) \in f \wedge (x,z) \in f \Rightarrow y = z) \).
        \item We call a set \( x \) the domain of \( f \), written \( x = \dom f \), if \( f \) is a function and \( (\forall y)(y \in x \Leftrightarrow (\exists z)((y,z) \in f)) \).
        \item The notation \( f \colon x \to y \) means that \( f \) is a function, \( x = \dom f \), and \( (\forall z)(\forall t)((z, t) \in f \Rightarrow t \in y) \).
    \end{itemize}
    \item \emph{Union axiom}.
    For each family of sets \( x \), we can form its union \( \bigcup_{t \in x} t \).
    \[ (\forall x)(\exists y)(\forall z)(z \in y \Leftrightarrow (\exists t)(z \in t \wedge t \in x)) \]
    The set guaranteed by this axiom can be written \( \bigcup x \), and we can write \( x \cup y \) for \( \bigcup \qty{x, y} \).
    We need no intersection axiom, as such intersections already exist by the axiom of separation.
    This cannot be used to create empty intersections, as the axiom of separation can only create subsets of a set that already exists.
    \item \emph{Power-set axiom}.
    \[ (\forall x)(\exists y)(\forall z)(z \in y \Leftrightarrow z \subseteq x) \]
    where \( z \subseteq x \) means \( (\forall t)(t \in z \Rightarrow t \in x) \).
    We write \( \mathcal P(x) \) for the power set of \( x \).
    We can form the Cartesian product \( x \times y \) as a suitable subset of \( \mathcal P(\mathcal P(x \cup y)) \), as if \( z \in x, t \in y \), we have \( (z, t) = \qty{\qty{z}, \qty{z, t}} \in \mathcal P(\mathcal P(x \cup y)) \).
    The set of all functions \( x \to y \) can be defined as a subset of \( \mathbb P(x \times y) \).
    \item \emph{Axiom of infinity}.
    Using our currently defined axioms, any model \( V \) must be infinite.
    For example, writing \( x^+ \) for the \emph{successor} of \( x \) defined as \( x \cup \qty{x} \), the sets \( \varnothing, \varnothing^+, \varnothing^{++}, \dots \) are distinct.
    \[ \varnothing^+ = \qty{\varnothing};\quad \varnothing^{++} = \qty{\varnothing, \qty{\varnothing}};\quad \varnothing^{+++} = \qty{\varnothing, \qty{\varnothing}, \qty{\varnothing, \qty{\varnothing}}};\quad \dots \]
    We write \( 0 = \varnothing, 1 = \varnothing^+, 2 = \varnothing^{++}, \dots \) for the successors created in this way.
    For instance, \( 3 = \qty{0, 1, 2} \).
    \( V \) may not have an infinite element, even though \( V \) itself is infinite, because no \( x \in V \) has all \( y \in V \) as elements: \( V \) does not think of itself as a set, because Russell's paradox follows from the axioms defined so far.

    We say that \( x \) is a successor set if \( \varnothing \in x \) and \( (\forall y)(y \in x \Rightarrow y^+ \in x) \).
    Note that this is a finite-length formula that characterises an infinite set.
    The axiom of infinity is that there exists a successor set.
    \[ (\exists x)(\varnothing \in x \wedge (\forall y)(y \in x \Rightarrow y^+ \in x)) \]
    Note that this set is not uniquely defined, but any intersection of successor sets is a successor set.
    We can therefore take the intersection of all successor sets by the axiom of separation, giving a least successor set denoted \( \omega \).
    Thus, \( (\forall x)(x \in \omega \Leftrightarrow (\forall y)(y \text{ is a successor set} \Rightarrow x \in y)) \).
    For example, we can prove that \( 3 \in \omega \).

    In particular, \( x \subseteq \omega \) is a successor set, \( x = \omega \).
    Hence, \( (\forall x)(x \subseteq \omega \wedge \varnothing in x \wedge (\forall y)(y \in x \Rightarrow y^+ \in x) \Rightarrow x = \omega) \).
    This is `proper' induction over all subsets of \( \omega \), unlike the weaker first-order induction defined in the Peano axioms.
    It is easy to check that \( (\forall x)(x \in \omega \Rightarrow x^+ \neq \varnothing) \) and \( (\forall x)(\forall y)(x \in \omega \wedge y \in \omega \wedge x^+ = y^+ \Rightarrow x = y) \), so \( \omega \) satisfies (in \( V \)) the usual axioms for the natural numbers.
    We can now define `\( x \) is finite' to mean \( (\exists y)(y \in \omega \wedge x \text{ bijects with } y) \), and define `\( x \) is countable' to mean that \( x \) is finite or bijects with \( \omega \).
    \item \emph{Axiom of foundation} or \emph{regularity}.
    We require that sets are built out of simpler sets.
    For example, we want to disallow a set from being a member of itself, and similarly forbid \( x \in y \) and \( y \in x \).
    In general, we want to forbid sets \( x_i \) such that \( x_{i+1} \in x_i \) for each \( i \in \mathbb N \).
    
    Note that if \( x \in x \), \( \qty{x} \) has no \( \in \)-minimal element.
    If \( x \in y, y \in x \), \( \qty{x, y} \) has no \( \in \)-minimal element.
    In the last example, \( \qty{x_0, x_1, \dots} \) has no \( \in \)-minimal element.
    We now define the axiom of foundation: every nonempty set has an \( \in \)-minimal element.
    \[ (\forall x)(x \neq \varnothing \Rightarrow (\exists y)(y \in x \wedge (\forall z)(z \in x \Rightarrow z \not\in y))) \]
    Any model of \( \mathsf{ZF} \) without this axiom has a submodel of all of \( \mathsf{ZF} \).
    \item \emph{Axiom of replacement}.
    Often, we are given an index set \( I \) and construct a set \( A_i \) for each \( i \in I \), then take the collection \( \qty{A_i \mid i \in I} \).
    In order to write this down, the mapping \( i \mapsto A_i \) must be a function, or equivalently, there must be a set \( \qty{(i, A_i) \mid i \in I} \).
    This is not clear from the other axioms.
    We would like to say that the image of a set under something that looks like a function (since we do not yet have such a set-theoretic function) is a set.

    Let \( (V, \in) \) be an \( L \)-structure.
    A \emph{class} is a set \( C \subseteq V \) such that for some formula \( p \) with free variables \( x \) and some parameters, we have \( x \in C \) if and only if \( p \) holds in \( V \).
    \( C \) is a set outside of our model; it may not correspond to a set \( x \in V \) inside the model.
    For instance, \( V \) is a class, taking \( p \) to be \( x = x \).
    There is a class of infinite sets, taking \( p \) to be `\( x \) is not finite'.
    For any \( t \in V \), the collection of \( x \) with \( t \in x \) is a class; here, \( t \) is a parameter to the class.
    Every set \( y \in V \) is a class by setting \( p \) to be \( x \in y \).
    A \emph{proper class} is a class that does not correspond to a set \( x \in V \): \( \neg(\exists y)(\forall x)(x \in y \Leftrightarrow p) \).
    When writing about classes inside \( \mathsf{ZF} \), we instead write about their defining formulae, as classes have no direct representation in the language.

    Similarly, a \emph{function-class} is a set \( F \subseteq V \) of ordered pairs from \( V \) such that for some formula \( p \) with free variables \( x, y \) and parameters, we have \( (x, y) \) belongs to \( F \) if and only if \( p \), and if \( (x, y), (x, z) \) belong to \( F \), \( y = z \).
    This is intuitively a function whose domain may not be a set.
    For example, the mapping \( x \mapsto \qty{x} \) is a function-class, taking \( p \) to be \( y = \qty{x} \).
    This is not a function, for example, every \( f \) has a domain which is a set in \( V \), and this function has domain \( V \) which is not a set.

    We can now define the axiom of replacement: the image of a set under a function-class is a set.
    \[ (\forall t_1)\dots(\forall t_n)\qty[(\forall x)(\forall y)(\forall z)(p \wedge p[z/y] \Rightarrow y = z) \Rightarrow (\forall x)(\exists y)(\forall z)(z \in y \Leftrightarrow (\exists t)(t \in x \wedge p[t/x,z/y]))] \]
    For example, for any set \( x \), we can form the set \( \qty{\qty{t} \mid t \in x} \), which is the image of \( x \) under the function class \( t \mapsto \qty{t} \).
    This set could alternatively have been formed using the power-set and separation axioms; we will later present some examples of sets built with this axiom that cannot be constructed from the other axioms.
\end{enumerate}
