\newcommand{\yearnumber}{II}
\documentclass{book}

\usepackage[book]{../../util}
\usepackage{subfiles}

\addto\captionsUKenglish{\renewcommand{\chaptername}{Course}}

% increase spacing between toc number and title
% https://stackoverflow.com/a/33093889
\makeatletter
\renewcommand{\l@section}{\@dottedtocline{1}{1.5em}{2.6em}}
\renewcommand{\l@subsection}{\@dottedtocline{2}{4.0em}{3.6em}}
\renewcommand{\l@subsubsection}{\@dottedtocline{3}{7.4em}{4.5em}}
\makeatother

\begin{document}

\begin{titlepage}
	\begin{center}
		\vspace*{1cm}

		\Huge
		\textbf{Notes on the Cambridge University Mathematical Tripos}

		\vspace{0.5cm}
		\LARGE
		Sky Wilshaw

		\vfill

		\Huge
		\textsc{Part \yearnumber}

		\vfill

		\Large
		University of Cambridge\\
		2020--\the\year{}

	\end{center}
\end{titlepage}

\dominitoc{}

\setcounter{tocdepth}{0}
\tableofcontents
\newpage
\setcounter{tocdepth}{3}

\chapter*{Introduction}
This book contains notes for the maths courses at Cambridge University.
Please note that while efforts have been made to ensure completeness and correctness, no guarantees can be made; this is simply a reasonably complete way of collating information about the courses.

This book can be downloaded in PDF form for free at \url{https://thirdsgames.co.uk/maths.html}, and the source code (for the book itself and for the individual courses) can be accessed at \url{https://github.com/zeramorphic/cambridge-maths-notes}.

You are given the right to download the PDF of the book (or its component parts) for private use.
You are permitted to download and modify the source code of the repository (the book and the course notes it contains), but may not distribute these modifications (including object files such as PDFs generated from these modifications) to others.
However, you are permitted to make public forks of the repository in order to create pull requests, but this does not grant you permission to distribute object files created from these forked repositories.
It must be clear when visiting it that such a repository is a fork of \url{https://github.com/zeramorphic/cambridge-maths-notes}, and must include a link to the original repository.
(Forks created on GitHub satisfy this requirement, as the title contains the words `forked from' and then a link to the original repository.)

This project makes use of free software in accordance with their license terms, including the \texttt{cobra} CLI interface generator \\ \url{https://pkg.go.dev/mod/github.com/spf13/cobra@v1.1.3}.
For more information, read the licenses of each dependency.

\let\maketitle\ignorespaces{}
\renewcommand{\tableofcontentsnewpage}{\minitoc\newpage}


\chapter{Algebraic Topology}
This course is an introduction to the basic ideas of algebraic topology. In the first half of the course, we will study an invariant of based topological spaces called the fundamental group. This invariant associates a group to a topological space (with a basepoint). It has the important property that a continuous map between topological spaces induces a homomorphism between their fundamental groups, and that the composition of two maps goes to the composition of the corresponding homomorphisms. In slightly fancier language, the fundamental group determines a functor from the category of based topological spaces to the category of groups. The phenomena that the fundamental group detects are essentially 1-dimensional; it measures the failure of closed loops in the space to bound two-dimensional disks.

In the second half of the course, we will study another functor from spaces to groups, called homology, which enables us to get a handle on higher-dimensional `holes' in the space. There are many different ways to define homology; we will use a relatively concrete one called simplicial homology, which makes sense for a somewhat restricted class of spaces. The notion of homology plays a central role in modern geometry and topology as well as in many branches of algebra and number theory.

Using these invariants we will be able to distinguish various spaces from each other; for example, to prove that \( \mathbb R^n \) is not homeomorphic to \( \mathbb R^m \) when \( n \) is not equal to \( m \). We will also be able to prove the fundamental theorem of algebra, and to show that certain maps from a space to itself (for example, any continous map from the closed \( n \)-dimensional disk to itself) must have fixed points.

\subfile{../algtop/main.tex}
\chapter{Probability and Measure}
\subfile{../pm/main.tex}
\chapter{Graph Theory}
\subfile{../graph/main.tex}
\chapter{Automata and Formal Languages}
\subfile{../afl/main.tex}
\chapter{Galois Theory}
\subfile{../galois/main.tex}
\chapter{Coding and Cryptography}
\subfile{../cc/main.tex}
\chapter{Quantum Information and Computation}
\subfile{../qic/main.tex}
\chapter{Number Fields}
\subfile{../nf/main.tex}
\chapter{Algebraic Geometry}
\subfile{../alggeom/main.tex}
\chapter{Logic and Set Theory}
\subfile{../lst/main.tex}

\end{document}
