\subsection{Number Fields}
Recall that if \( K \) and \( L \) are fields and \( \dim_K L < \infty \), we write \( [L : K] \) for this dimension and say that \( L / K \) is a finite extension.
If \( L / K \) is a finite extension, every element \( x \in L \) is algebraic over \( K \).
\begin{definition}
    A \emph{number field} is a finite extension of \( \mathbb Q \).
\end{definition}
\begin{definition}
    Let \( L \) be a number field.
    \( \alpha \in L \) is an \emph{algebraic integer} if there exists \( f \in \mathbb Z[x] \) monic such that \( f(\alpha) = 0 \).
    We write \( O_L = \qty{\alpha \in L \mid \alpha \text{ is an algebraic integer}} \) for the set of \emph{integers of \( L \)}.
\end{definition}
% https://q.uiver.app/?q=WzAsNCxbMCwwLCJcXG1hdGhiYiBaIl0sWzEsMCwiXFxtYXRoYmIgUSJdLFsxLDEsIkwiXSxbMCwxLCJPX0wiXSxbMCwxXSxbMSwyXSxbMCwzXSxbMywyXV0=
\[\begin{tikzcd}
	{\mathbb Z} & {\mathbb Q} \\
	{O_L} & L
	\arrow[from=1-1, to=1-2]
	\arrow[from=1-2, to=2-2]
	\arrow[from=1-1, to=2-1]
	\arrow[from=2-1, to=2-2]
\end{tikzcd}\]
\begin{lemma}
    \( O_{\mathbb Q} = \mathbb Z \).
\end{lemma}
\begin{proof}
    Clearly if \( \alpha \) is an integer, then \( f(x) = x - \alpha \) is a monic polynomial such that \( f(\alpha) = 0 \).
    Conversely, if \( \alpha \) is a rational number, we can let \( \alpha = \frac{r}{s} \) where \( r \) and \( s \) are coprime.
    Let \( f(x) = x^n + a_{n-1} x^{n-1} + \dots + a_0 \in \mathbb Z[x] \) such that \( f(\alpha) = 0 \).
    Clearing denominators, \( r^n + a_{n-1} r^{n-1} s + \dots + a_0 s^n = 0 \).
    Hence \( s \mid r^n \).
    If \( s \neq 1 \), let \( p \mid s \) be a prime, then \( p \mid r \), so \( r \) and \( s \) were not coprime.
\end{proof}
We will soon show the following theorem.
\begin{theorem}
    \( O_L \) is a ring.
    In other words, \( \alpha, \beta \in L \) implies \( \alpha \pm \beta, \alpha \beta \in L \).
\end{theorem}
Note that \( \alpha \in L \) does not in general imply \( \frac{1}{\alpha} \in L \).
Recall from Galois Theory that if \( \alpha, \beta \in L \), and \( \alpha, \beta \) are algebraic over \( K \), then so is \( \alpha \pm \beta, \alpha \beta \).
The proof from Galois Theory will not work in this case, since that proof does not provide for monic polynomials.
\begin{definition}
    Let \( R \subseteq S \) be commutative rings with a 1.
    \begin{enumerate}
        \item \( \alpha \in S \) is \emph{integral over \( R \)} if there exists a monic polynomial \( f \in R[x] \) such that \( f(\alpha) = 0 \).
        \item \( S \) is \emph{integral over \( R \)} if all \( \alpha \in S \) are integral over \( R \).
        \item \( S \) is \emph{finitely generated over \( R \)} if there exist elements \( \alpha_1, \dots, \alpha_n \in S \) such that any element of \( S \) can be written as an \( R \)-linear combination of the \( \alpha_i \).
        Equivalently, the map \( R^n \to S \) given by \( (r_1, \dots, r_n) \mapsto \sum_{i=1}^n r_i \alpha_i \) is surjective.
    \end{enumerate}
\end{definition}
\begin{example}
    Let \( \mathbb Q \subseteq L \) be a number field.
    Then \( \alpha \in L \) is an algebraic integer if and only if \( \alpha \) is integral over \( \mathbb Z \).
    \( O_L \) is integral over \( \mathbb Z \) (once we have proven it is a ring).
\end{example}
If \( \alpha_1, \dots, \alpha_r \in S \), we write \( R[\alpha_1, \dots, \alpha_r] \) for the subring of \( S \) generated by \( R \) and the \( \alpha_i \).
This is equivalently the image of the polynomial ring \( R[x_1, \dots, x_r] \to S \) mapping \( x_i \) to \( \alpha_i \).
\begin{proposition}
    Let \( S = R[s] \), where \( s \) is integral over \( R \).
    Then \( S \) is finitely generated over \( R \).
    Further, if \( S = R[s_1, \dots, s_n] \) with each \( s_i \) integral over \( R \), then \( S \) is finitely generated over \( R \).
\end{proposition}
\begin{proof}
    \( S \) is spanned by \( 1, s, s^2, \dots \) over \( R \).
    By assumption, there exists \( a_0, \dots, a_{n-1} \in R \) such that \( s^n = \sum_{i=0}^{n-1} a_i s^i \).
    So the \( R \)-module spanned by \( 1, \dots, s^{n-1} \) is stable under multiplication by \( s \), so contains \( s^n, s^{n+1}, \dots \) and hence is all of \( S \).

    Let \( S_i = R[s_1, \dots, s_{i-1}] \).
    Then \( S_{i+1} = S_i[s_{i+1}] \), and \( s_{i+1} \) is integral over \( R \), hence is integral over \( S_i \).
    So \( S_{i+1} \) is finitely generated over \( S_i \).
    Note that if \( A \subseteq B \subseteq C \) where \( B \) is finitely generated over \( A \) and \( C \) is finitely generated over \( B \), then \( C \) is finitely generated over \( A \).
    Indeed, if \( b_i \) generate \( B \) over \( A \) and \( c_j \) generate \( C \) over \( B \), the \( b_i c_j \) generate \( C \) over \( A \).
\end{proof}
\begin{theorem}
    If \( S \) is finitely generated over \( R \), \( S \) is integral over \( R \).
\end{theorem}
\begin{proof}
    Let \( alpha_1, \dots, \alpha_n \) generate \( S \) as an \( R \)-module.
    Without loss of generality, we can assume \( \alpha_1 = 1 \).
    Let \( s \in S \), and consider the function \( m_s \colon S \to S \) given by \( m_s(x) = sx \).
    Then, \( m_s(\alpha_i) = s\alpha_i = \sum b_{ij} \alpha_j \) for some choice of \( b_{ij} \).
    Let \( B = (b_{ij}) \).
    By definition, \( (sI - B) (\alpha_1, \dots, \alpha_n)^\transpose = 0 \).
    
    Recall that for any matrix \( X \), the adjugate has the property that \( \adj(X) X = \det X \cdot I \).
    Hence, \( \det(sI - B) (\alpha_1, \dots, \alpha_n)^\transpose = 0 \).
    In particular, \( \det(sI - B) \alpha_1 = \det(sI - B) = 0 \).
    Let \( f(t) = \det(tI - B) \), which is a monic polynomial in \( R \).
    As \( f(s) = 0 \), \( s \) is integral over \( R \).
\end{proof}
Note the similarity to a proof of the Cayley-Hamilton theorem.
Note further that this proof is constructive.
\begin{corollary}
    Let \( \mathbb Q \subseteq L \) be a number field.
    Then \( O_L \) is a ring.
\end{corollary}
\begin{proof}
    If \( \alpha, \beta \in O_L \), then \( \mathbb Z[\alpha, \beta] \) is finitely generated over \( \mathbb Z \).
    So this ring is integral.
\end{proof}
\begin{corollary}
    Let \( A \subseteq B \subseteq C \) be ring extensions, where \( B / A \) is integral and \( C / B \) is integral.
    Then \( C / A \) is integral.
\end{corollary}
\begin{proof}
    If \( c \in C \), let \( f(x) = \sum_{i=0}^n b_i x^i \) be the monic polynomial in \( B[x] \) it satisfies, and set \( B_0 = A[b_0, \dots, b_{n-1}] \), \( C_0 = B[c] \).
    Then \( B_0 \) is finitely generated over \( A \) as \( b_0, \dots, b_{n-1} \) are integral over \( A \), and \( C_0 \) is finitely generated over \( B_0 \) as \( c \) is integral over \( B_0 \).
    \( C_0 \) is therefore finitely generated over \( A \).
    Then the theorem implies that \( c \) is integral over \( A \).
\end{proof}
\begin{remark}
    \( C \) could have had infinitely many generators, for instance, \( C = \qty{\alpha \in \mathbb C \mid \alpha \text{ is an algebraic integer}} \), which is why we passed to \( C_0 \).
    This kind of proof is common in commutative algebra, applying a powerful theorem such as the Cayley-Hamilton theorem carefully to find its consequences.
\end{remark}
\begin{example}
    \( O_{\mathbb Q[i]} = \mathbb Z[i] \).
\end{example}

\subsection{?}
Let \( K \subseteq L \) be fields.
Recall that the minimal polynomial of \( \alpha \in L \) is the monic polynomial \( p_\alpha(x) \in K[x] \) of minimum degree such that \( p_\alpha(\alpha) = 0 \).
\begin{lemma}
    Let \( f(x) \in K[x] \) satisfy \( f(\alpha) = 0 \).
    Then \( p\alpha \mid f \).
\end{lemma}
\begin{proof}
    By Euclid, \( f = p_\alpha h + r \) where \( r \in K[x] \) has degree less than that of \( p \).
    Then \( 0 = f(\alpha) = p_\alpha(\alpha) h(\alpha) + r(\alpha) \).
    If \( r \neq 0 \), this contradicts minimality of \( \deg p_\alpha \).
\end{proof}
The converse is obvious, so the lemma implies the uniqueness of \( p_\alpha \)
\begin{proposition}
    Let \( L \) be a number field.
    Then \( \alpha \in O_L \) if and only if \( p_\alpha(x) \in \mathbb Q[x] \) is in \( \mathbb Z[x] \).
\end{proposition}
\begin{proof}
    If \( p_\alpha \) has integer coefficients, this holds by definition.
    Conversely, suppose \( \alpha \in O_L \), where \( p_\alpha \) is the minimal polynomial.
    Let \( M \supseteq L \) be a splitting field for \( p_\alpha \), i.e.\, a field in which \( p_\alpha \) splits into linear factors.
    Let \( h(x) \) be a monic polynomial which \( \alpha \) satisfies.
    By the lemma, \( p_\alpha \mid h \), so each root \( \alpha_i \) of \( p_\alpha \) in \( M \) is an algebraic integer.
    By the previous theorem, sums and products of algebraic integers are algebraic.
    So the coefficients of \( p_\alpha \) are algebraic integers.
    But \( p_\alpha \in \mathbb Q[x] \), so the coefficients are in \( \mathbb Z \).
\end{proof}
\begin{remark}
    One can also show this from the previous result and Gauss' lemma.
\end{remark}
\begin{lemma}
    The field of fractions of \( O_L \) is \( L \).
    In fact, if \( \alpha \in L \), there exists \( n \in \mathbb Z \) such that \( n\alpha \in O_L \).
\end{lemma}
\begin{proof}
    Let \( \alpha \in L \), and \( g \) be the minimal polynomial of \( \alpha \).
    Then \( g \) is monic, and there exists an integer \( n \in \mathbb Z, n \neq 0 \) such that \( ng \in \mathbb Z[x] \).
    So \( h(x) = n^{\deg g} g\qty(\frac{x}{n}) \) is an integer polynomial which is monic, and this is the minimal polynomial of \( n\alpha \), so \( n\alpha \in O_L \).
\end{proof}
