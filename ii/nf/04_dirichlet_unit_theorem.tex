\subsection{Real quadratic fields}
Recall that \( \alpha \in \mathcal O_L \) is a unit if and only if \( N(\alpha) = \pm 1 \).
We aim to show that \( \mathcal O_L^\star \simeq \bm \mu_L \times \mathbb Z^{r+s-1} \) where \( \bm \mu_L = \qty{\alpha \in L \mid \alpha^a = 1 \text{ for some } a > 0} \) is the set of roots of unity in \( L \), a finite cyclic group.
\begin{example}
    Let \( L = \mathbb Q(\sqrt{d}) \) where \( d > 0 \) is square-free.
    Here, \( r = 2, s = 0, n = 2 \).
    \( L \subseteq \mathbb R \) gives \( \bm \mu_L \subseteq \qty{\pm 1} \) so \( \bm \mu_L = \qty{\pm 1} \).
    Note that \( N(x+y\sqrt{d}) = x^2 - dy^2 \), so Dirichlet's theorem implies the following statement, which we will now prove directly.
\end{example}
\begin{theorem}[Pell's equation]
    There exist infinitely many \( x + y \sqrt{d} \in \mathcal O_L \) with \( x^2 - dy^2 = \pm 1 \).
\end{theorem}
\begin{proof}
    Recall that we have \( \sigma \colon \mathcal O_L \to \mathbb R^2 \) given by \( x + y\sqrt{d} \mapsto (x + y \sqrt{d}, x - y \sqrt{d}) \).
    For example, if \( d = 2 \), the image is a lattice with basis \( (1,1), (-\sqrt{2}, \sqrt{2}) \), note also that no point lies in the coordinate axes apart from 0.
    The covolume of \( \sigma(\mathcal O_L) \) is \( \abs{D_L}^{\frac{1}{2}} \).

    Consider
    \[ S_t = \qty{(y_1, y_2) \in \mathbb R^2 \midd \abs{y_1} \leq t, \abs{y_2} \leq \frac{\abs{D_L}^{\frac{1}{2}}}{t}} \]
    The volume of \( S_t \) is \( 4\abs{D_L}^{\frac{1}{2}} = 2^n \operatorname{covol}(\sigma(\mathcal O_L)) \) as \( n = 2 \).
    Minkowski's lemma implies that there exists a nonzero \( \alpha \in \mathcal O_L \) with \( \sigma(\alpha) \in S_t \).
    But \( \sigma(\alpha) = (y_1, y_2) \) gives \( N(\alpha) = y_1 y_2 \).

    We have therefore found an element \( \alpha \in \mathcal O_L \) with \( \sigma(\alpha) \in S_t \) that has norm satisfying \( 1 \leq n(\alpha) \leq \abs{D_L}^{\frac{1}{2}} \).
    We show that there exist infinitely many such \( \alpha \) for \( 0 < t < 1 \), so there are infinitely many \( \alpha \in \mathcal O_L \) with \( \abs{N(\alpha)} = N((\alpha)) < \abs{D_L}^{\frac{1}{2}} \).
    For fixed \( t \), \( S_t \cap \sigma(\mathcal O_L) \) is finite as \( S_t \) is compact.
    Given \( t_1 > t_2 > \dots > t_n \), choose \( t_{n+1} \) less than all \( y_1 \) where \( \sigma(\alpha) = (y_1, y_2) \in S_{t_n} \cap \sigma(\mathcal O_L) \).
    Note that \( \alpha \neq 0 \) so \( \sigma_1(\alpha) \neq 0 \), so \( t_{n+1} > 0 \).

    Hence, there exists \( m \in \mathbb Z \) with \( 1 \leq \abs{m} \leq \abs{D_L}^{\frac{1}{2}} \) for which there are infinitely many \( \alpha \) with \( N(\alpha) = m \), by the pigeonhole principle.
    But ideals \( \mathfrak a \trianglelefteq \mathcal O_L \) with \( m \in \mathfrak a \) biject with ideals in \( \faktor{\mathcal O_L}{m} = \qty(\faktor{\mathbb Z}{m\mathbb Z})^2 \), and hence there are finitely many of them.
    Again by the pigeonhole principle, there exists \( \beta \in \mathcal O_L \) and infinitely many \( \alpha \in \mathcal O_L \) with \( N(\beta) = N(\alpha) = m \), where \( (\beta) = (\alpha) \).
    But \( \frac{\beta}{\alpha} \) is a unit, so there are infinitely many units.
\end{proof}
We can prove Dirichlet's unit theorem for real quadratic fields from this result.
\begin{corollary}
    \( \mathcal O_L^\star = \qty{\pm \varepsilon_0^n \mid n \in \mathbb Z} \) for \( \varepsilon_0 \in \mathcal O_L^\star \).
\end{corollary}
Such an \( \varepsilon_0 \) is called a \emph{fundamental unit}.
