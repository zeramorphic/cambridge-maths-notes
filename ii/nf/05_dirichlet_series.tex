\subsection{?}
\begin{theorem}[Euclid]
    There exist infinitely many primes.
\end{theorem}
The following proof is due to Euler in 1748.
\begin{proof}
    Consider
    \[ \prod_{p \text{ prime}} \qty(1 - \frac{1}{p})^{-1} = \prod_{p \text{ prime}} \qty(1 + \frac{1}{p} + \frac{1}{p^2} + \dots) = \sum_{n=1}^\infty \frac{1}{n} \]
    as every \( n > 0 \) factors uniquely as a product of primes so occurs exactly once when we expand the product.
    If there are finitely many primes, the product is finite.
    As \( \sum_{i=1}^\infty p^{-i} \) converges to \( \qty(1 - \frac{1}{p})^{-1} \), \( \sum_{i=1}^\infty \frac{1}{n} \) must converge.
\end{proof}
We aim to prove that for all \( a, q \in \mathbb Z \) coprime, there are infinitely many primes of the form \( a + kq \), \( k \in \mathbb N \).
Note that there is no nice series expansion for \( \prod_{p \equiv a \text{ mod } q, p \text{ prime}} \qty(1 - \frac{1}{p})^{-1} \), so Euler's proof does not generalise.
\begin{definition}
    The \emph{Riemann zeta function} is \( \zeta(s) = \sum_{n \geq 1} n^{-s} \) for \( s \in \mathbb C \).
\end{definition}
\begin{proposition}
    \begin{enumerate}
        \item \( \zeta(s) \) converges for \( \Re(s) > 1 \).
        \item \( \zeta(s) = \prod_{p \text{ prime}} \qty(1 - \frac{1}{p^s})^{-1} \) in this region; this result is known as the \emph{Euler product}.
        This product converges absolutely.
        \item \( \zeta(s) - \frac{1}{s - 1} \) extends to a holomorphic function for \( \Re(s) > 0 \), so the zeta function has a simple pole with residue 1 at \( s = 1 \).
    \end{enumerate}
\end{proposition}
If \( \sum \log(1 - a_n) \) converges, \( \prod (1 - a_n) \) converges.
\( \prod (1 - a_n) \) absolutely converges if \( \sum \abs{\log(1-a_n)} \) converges.

If \( a_n \) is a sequence of complex numbers, call the function \( \sum_{n \geq 1} a_n n^{-s} \) a \emph{Dirichlet series}.
Instead of part (i), we will prove the following more general lemma.
\begin{lemma}
    If there exists \( r \in \mathbb R \) with \( a_1 + \dots + a_N = O(N^r) \), then \( \sum_{n \geq 1} a_n n^{-s} \) converges for \( \Re(s) > r \), and it is holomorphic in this region.
\end{lemma}
\begin{proof}[Proof of lemma]
    \[ \sum_{n=1}^N a_n n^{-s} = a_1 (1^{-s} - 2^{-s}) + (a_1 + a_2) (2^{-s} - 3^{-s}) + \dots + (a_1 + a_{N-1})((N-1)^{-s} - N^{-s}) + R_n \]
    where \( R_n = \frac{T(N)}{N^s} \) with \( T(N) = a_1 + \dots + a_N = O(N^r) \).
    By assumption, if \( \Re(s) > r \),
    \[ \abs{\frac{T(N)}{N^s}} = \abs{\frac{T(N)}{N^r}} \cdot \frac{1}{\abs{N^{s-r}}} = \abs{\frac{T(N)}{N^r}} \cdot \frac{1}{N^{\Re(s) - r}} \to 0 \]
    as \( x^s = e^{s \log x} \) so \( \abs{x^s} = \abs{x^{\Re s}} \).
    So if \( \Re(s) > r \), \( \sum a_n n^{-s} = \sum T(N) (N^{-s} - (N+1)^{-s}) \).
    But \( T(N) \leq BN^r \) for some constant \( B \) by assumption, so it suffices to show \( \sum N^r (N^{-s} - (N+1)^{-s}) \) converges.
    Note that
    \[ N^{-s} - (N+1)^{-s} = \int_N^{N+1} s \frac{\dd{x}}{x^{s+1}} \]
    and \( N^r \leq x^r \) if \( x \in [N, N+1] \).
    Hence
    \[ N^r (N^{-s} - (N+1)^{-s}) \leq \int_N^{N+1} x^r s \frac{\dd{x}}{x^{s+1}} \leq s \int_N^{N+1} \frac{\dd{x}}{x^{s+1-r}} \]
    It is enough to show that \( s\int_1^N \frac{\dd{x}}{x^{s+1-r}} \) converges, which it does to \( \frac{s}{s-r} \).
\end{proof}
\begin{proof}[Proof of proposition]
    \emph{Part (ii).}
    Let \( p_1, \dots, p_r \) be the first \( r \) primes.
    Then, \( \prod_{i=1}^r (1 - p_r^{-s})^{-1} = \sum_{n \in X} n^{-s} \) where \( X \) is the set of positive integers whose prime divisors are only in \( p_1, \dots, p_r \).
    So
    \[ \abs{\zeta(s) - \prod_{i=1}^r (1 - p_r^{-s})^{-1}} = \abs{\sum_{n \not\in X} n^{-s}} \leq \sum_{n \not\in X} \abs{n^{-s}} = \sum_{n \not\in X} n^{-\Re(s)} \leq \sum_{n > r} n^{-\Re(s)} \]
    as \( 1, \dots, r \in X \).
    Hence the infinite product converges to \( \zeta(s) \).
    The proof of absolute convergence is omitted. 

    \emph{Part (iii).}
    Left as an exercise, noting that
    \[ \frac{1}{s-1} = \sum_{i=1}^\infty \int_n^{n+1} \frac{\dd{t}}{t^s} \]
\end{proof}

\subsection{Zeta functions in number fields}
The remaining new content in this course is nonexaminable.
\begin{definition}
    Let \( L \) be a number field.
    The \emph{zeta function of \( L \)} is
    \[ \zeta_L(s) = \sum_{\mathfrak a \trianglelefteq \mathcal O_L} N(\mathfrak a)^{-s} = \sum_{n \geq 1} \#\qty{\mathfrak a \trianglelefteq \mathcal O_L \mid N(\mathfrak a) = n} n^{-s} \]
\end{definition}
\begin{proposition}
    \begin{enumerate}
        \item \( \zeta_L(s) \) converges to a holomorphic function for \( \Re(s) > 1 \).
        \item \( \zeta_L(s) = \prod_{\mathfrak p \text{ prime ideal}} (1 - N(\mathfrak p)^{-s})^{-1} \) in this region.
        \item \( \zeta_L(s) \) is a meromorphic function for \( \Re(s) > 1 - \frac{1}{[L:\mathbb Q]} \), with a simple pole at \( s = 1 \) with residue
        \[ \frac{\abs{\mathrm{Cl}_L} 2^{r+s} \pi^s R_L}{\abs{D_L}^{\frac{1}{2}} \abs{\bm\mu_L}} \]
        This is called the \emph{analytic class number formula}.
    \end{enumerate}
\end{proposition}
\begin{proof}
    Part (ii) is clear.
    Parts (i) and (iii) follow from the following estimate.
    Writing \( \zeta_L(s) = \sum \frac{a_n}{n^s} \) where \( a_n \) is the number of ideals of norm \( n \), one can show
    \[ a_1 + \dots + a_N = \frac{\abs{\mathrm{Cl}_L} 2^{r+s} \pi^s R_L}{\abs{D_L}^{\frac{1}{2}} \abs{\bm\mu_L}} \cdot N + O\qty(N^{1 - \frac{1}{[L:\mathbb Q]}}) \]
\end{proof}
If \( L \neq \mathbb Q \), it turns out that \( \zeta_L(s) \) factors into \( \zeta_{\mathbb Q}(s) = \zeta(s) \) and some other factors.
Suppose \( L = \mathbb Q(\sqrt{d}) \) and \( d \neq 0, 1 \) is square-free.
\[ \zeta_L - \prod_{\mathfrak p \text{ prime ideal}} (1 - N(\mathfrak p)^{-s})^{-1} = \prod_{p \text{ prime}} \prod_{\mathfrak p \mid (p)} (1 - N(\mathfrak p)^{-s})^{-1} \]
If \( p \mid D_L \), then \( (p) = \mathfrak p^2 \) ramifies.
In this case, \( N(\mathfrak p) = p \) and we have a term \( (1 - p^{-s}) \) in the product.
If \( (p) \) remains prime in \( L \), then \( N(\mathfrak p) = p^2 \) giving the term \( (1 - p^{-2s}) = (1 - p^{-s})(1 - p^s) \).
If \( (p) = \mathfrak p_1 \mathfrak p_2 \) splits, then \( N(\mathfrak p_i) = p \) and we have a term \( (1 - p^{-s})^2 \).
Let
\[ \chi_{D_L}(p) = \chi(p) = \begin{cases}
    0 & p \text{ ramifies} \\
    -1 & p \text{ inert} \\
    1 & p \text{ splits}
\end{cases} = \underbrace{\qty(\frac{D_L}{p})}_{\mathclap{\text{if } p \text{ odd}}} \]
Then, defining \( L(\chi,s) = \prod_{p \text{ prime}} {1 - \chi(p) p^{-s}}^{-1} \), we have \( \zeta_L(s) = \zeta_{\mathbb Q}(s) L(\chi,s) \).
The function \( L \) is called a \emph{Dirichlet \( L \)-function}.
When expanding the infinite product defining \( L(\chi_D, s) \) the coefficient of \( n^{-s} \), if \( n = p_1^{e_1} \dots p_r^{e_r} \) is \( \chi_D(p_1)^{e_1} \dots \chi_D(p_r)^{e_r} \).
We can extend the definition of \( \chi \) to make it multiplicative: \( \chi_D(p_1^{e_1} \dots p_r^{e_r}) = \chi_D(p_1)^{e_1} \dots \chi_D(p_r)^{e_r} \).
\begin{example}
    Let \( L = \mathbb Q(\sqrt{-1}) \), so \( D_L = 4 \).
    We have \( \qty(\frac{-4}{p}) = \qty(\frac{-1}{p}) = (-1)^{\frac{p-1}{2}} \) for \( p \neq 2 \).
    2 ramifies, so \( \chi_D(2) = 0 \).
    We claim that
    \[ \chi_{-4}(m) = \begin{cases}
        (-1)^{\frac{m-1}{2}} & m \text{ odd} \\
        0 & m \text{ even}
    \end{cases} \]
    Indeed, if \( n \) is even, this is clear; otherwise, this claim is that \( (-1)^{\frac{mn-1}{2}} = (-1)^{\frac{m-1}{2}} (-1)^{\frac{n-1}{2}} \), which is easy to verify.
    Hence,
    \[ L(\chi_{-4}, s) = 1 - \frac{1}{3^s} + \frac{1}{5^s} - \frac{1}{7^s} + \dots \]
    In this example, the coefficients are periodic mod 4; this is true for general \( L(\chi_D, s) \).
    Since \( \zeta_L(s) = \zeta_{\mathbb Q}(s) L(\chi_{-4}, s) \), the fact that \( \zeta_{\mathbb Q}(s) \) has a simple pole at \( s = 1 \) with residue 1, together with the analytic class number formula, gives \( L(\chi_{-4}, 1) = \frac{\pi}{4} \).
\end{example}
\begin{definition}
    \( \chi \colon \mathbb Z \to \mathbb C \) is a \emph{Dirichlet character} of modulus \( D \) if there exists a group homomorphism \( \omega \colon \qty(\faktor{\mathbb Z}{D\mathbb Z})^\star \to \mathbb C \) such that
    \[ \chi(n) = \begin{cases}
        \omega(w \text{ mod } D) & n \text{ invertible mod } D \\
        0 & \text{otherwise}
    \end{cases} \]
\end{definition}
For such a \( \chi \), we have \( \chi(n)\chi(m) = \chi(nm) \), and we can define
\[ L(\chi,s) = \prod_{p \text{ prime}}(1 - \chi(p)p^{-s})^{-1} = \sum_{n \geq 1}\chi(n)n^{-s} \]
The previous example shows that \( \chi_{-4} \) is a Dirichlet character of modulus 4.
\begin{theorem}
    For any \( d \neq 0, 1 \) square-free, defining \( L = \mathbb Q(\sqrt{d}), D = D_L \), we have that \( \chi_D \) is a Dirichlet character of modulus \( D \).
\end{theorem}
\begin{proof}
    We must show \( \chi_D(n+D) = \chi_D(n) \) for \( n \in \mathbb N \).
    Suppose first that \( d \equiv 3 \) mod 4.
    Here, \( D = 4d \), so \( \chi_D(2) = 0 \) as 2 ramifies, so \( \chi_D(n) = 0 \) if \( n \) is even as required.
    For \( p > 2 \), \( \chi_D(p) = \qty(\frac{D}{p}) = \qty(\frac{d}{p}) \) by definition, but this is equal to \( \qty(\frac{p}{d}) (-1)^{\frac{p-1}{2}} \) by quadratic reciprocity as \( p, d \) are odd, and as \( d \equiv 3 \) mod 4, \( \frac{d-1}{2} \equiv 1 \) mod 4.
    \( n \mapsto (-1)^{\frac{n-1}{2}} \) is multiplicative, so \( \chi_D(n+D) = \qty(\frac{n+D}{d})(-1)^{\frac{n-1}{2}}(-1)^{4d}{2} = \chi_D(n) \).
    The other cases are omitted.
\end{proof}
This theorem can be seen as equivalent to the law of quadratic reciprocity.
