\subsection{?}
\begin{theorem}[Euclid]
    There exist infinitely many primes.
\end{theorem}
The following proof is due to Euler in 1748.
\begin{proof}
    Consider
    \[ \prod_{p \text{ prime}} \qty(1 - \frac{1}{p})^{-1} = \prod_{p \text{ prime}} \qty(1 + \frac{1}{p} + \frac{1}{p^2} + \dots) = \sum_{n=1}^\infty \frac{1}{n} \]
    as every \( n > 0 \) factors uniquely as a product of primes so occurs exactly once when we expand the product.
    If there are finitely many primes, the product is finite.
    As \( \sum_{i=1}^\infty p^{-i} \) converges to \( \qty(1 - \frac{1}{p})^{-1} \), \( \sum_{i=1}^\infty \frac{1}{n} \) must converge.
\end{proof}
We aim to prove that for all \( a, q \in \mathbb Z \) coprime, there are infinitely many primes of the form \( a + kq \), \( k \in \mathbb N \).
Note that there is no nice series expansion for \( \prod_{p \equiv a \mod q, p \text{ prime}} \qty(1 - \frac{1}{p})^{-1} \), so Euler's proof does not generalise.
\begin{definition}
    The \emph{Riemann zeta function} is \( \zeta(s) = \sum_{n \geq 1} n^{-s} \) for \( s \in \mathbb C \).
\end{definition}
\begin{proposition}
    \begin{enumerate}
        \item \( \zeta(s) \) converges for \( \Re(s) > 1 \).
        \item \( \zeta(s) = \prod_{p \text{ prime}} \qty(1 - \frac{1}{p^s})^{-1} \) in this region; this result is known as the \emph{Euler product}.
        This product converges absolutely.
        \item \( \zeta(s) - \frac{1}{s - 1} \) extends to a holomorphic function for \( \Re(s) > 0 \), so the zeta function has a simple pole with residue 1 at \( s = 1 \).
    \end{enumerate}
\end{proposition}
If \( a_n \) is a sequence of complex numbers, call the function \( \sum_{n \geq 1} a_n n^{-s} \) a \emph{Dirichlet series}.
Instead of part (i), we will prove the following more general lemma.
\begin{lemma}
    If there exists \( r \in \mathbb R \) with \( a_1 + \dots + a_N = O(N^r) \), then \( \sum_{n \geq 1} a_n n^{-s} \) converges for \( \Re(s) > r \), and it is holomorphic in this region.
\end{lemma}
\begin{proof}[Proof of lemma]
    \[ \sum_{n=1}^N a_n n^{-s} = a_1 (1^{-s} - 2^{-s}) + (a_1 + a_2) (2^{-s} - 3^{-s}) + \dots + (a_1 + a_{N-1})((N-1)^{-s} - N^{-s}) + R_n \]
    where \( R_n = \frac{T(N)}{N^s} \) with \( T(N) = a_1 + \dots + a_N = O(N^r) \).
    By assumption, if \( \Re(s) > r \),
    \[ \abs{\frac{T(N)}{N^s}} = \abs{\frac{T(N)}{N^r}} \cdot \frac{1}{\abs{N^{s-r}}} = \abs{\frac{T(N)}{N^r}} \cdot \frac{1}{N^{\Re(s) - r}} \to 0 \]
    as \( x^s = e^{s \log x} \) so \( \abs{x^s} = \abs{x^{\Re s}} \).
    So if \( \Re(s) > r \), \( \sum a_n n^{-s} = \sum T(N) (N^{-s} - (N+1)^{-s}) \).
    But \( T(N) \leq BN^r \) for some constant \( B \) by assumption, so it suffices to show \( \sum N^r (N^{-s} - (N+1)^{-s}) \) converges.
    Note that
    \[ N^{-s} - (N+1)^{-s} = \int_N^{N+1} s \frac{\dd{x}}{x^{s+1}} \]
    and \( N^r \leq x^r \) if \( x \in [N, N+1] \).
    Hence
    \[ N^r (N^{-s} - (N+1)^{-s}) \leq \int_N^{N+1} x^r s \frac{\dd{x}}{x^{s+1}} \leq s \int_N^{N+1} \frac{\dd{x}}{x^{s+1-r}} \]
    It is enough to show that \( s\int_1^N \frac{\dd{x}}{x^{s+1-r}} \) converges, which it does to \( \frac{s}{s-r} \).
\end{proof}
\begin{proof}[Proof of proposition]
    \emph{Part (ii).}

    \emph{Part (iii).}
    Left as an exercise, noting that
    \[ \frac{1}{s-1} = \sum_{i=1}^\infty \int_n^{n+1} \frac{\dd{t}}{t^s} \]
\end{proof}
