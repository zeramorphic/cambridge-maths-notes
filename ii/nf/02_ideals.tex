\subsection{?}
\begin{lemma}
    Let \( x \in \mathcal O_L \), where \( L \) is a number field.
    Then \( x \) is a unit in \( \mathcal O_L \) if and only if \( N_{L/\mathbb Q}(x) = \pm 1 \).
    We write \( \mathcal O_L^\star \) for the set of units of \( \mathcal O_L \).
\end{lemma}
\begin{proof}
    If \( x \) is a unit, then as the norm is multiplicative, \( N(x x^{-1}) = 1 \) so \( N(x)N(x^{-1}) = 1 \).
    So \( N(x) = \pm 1 \).
    Conversely, let \( \sigma_1, \dots, \sigma_n \colon L \to \mathbb C \) be the distinct field embeddings.
    Let \( L \subseteq \mathbb C \) be the containment given by \( \sigma_1 \).
    If \( x \in \mathcal O_L \), then \( N(x) = x\sigma_2(x) \dots \sigma_n(x) \).
    So if \( N(x) = \pm 1 \), we have \( \frac{1}{x} = \pm \prod_{i=2}^n \sigma_i(x) \).
    This is a product of algebraic integers, hence an algebraic integer.
    So \( x^{-1} \in \mathcal O_L \).
\end{proof}
Recall that if \( x \in \mathcal O_L \), it is irreducible if it does not factorise as \( ab \) where \( a, b \in \mathcal O_L \) not units.
If \( x = uy \) where \( u \) is a unit, we say \( x \) and \( y \) are associate.
Many number fields have rings of algebraic integers which are not unique factorisation domains.
\begin{example}
    Let \( L = \mathbb Q\qty(\sqrt{-5}) \).
    Here, \( \mathcal O_L = \mathbb Z\qty[\sqrt{-5}] \).
    Note that \( 3 \cdot 7 = \qty(1 + 2\sqrt{-5})\qty(1 - 2\sqrt{-5}) \), and \( N(3) = 9, N(7) = 49, N\qty(1 \pm \sqrt{-5}) = 21 \).
    These are not associates.
    We claim that \( 3, 7, 1 \pm 2 \sqrt{-5} \) are irreducible, so \( \mathcal O_L \) is not a unique factorisation domain.
    If this were not the case, \( 3 = \alpha \overline \alpha \), where \( \alpha = x + y\sqrt{-5} \), but \( N(3) = 9 = N(\alpha) N(\overline \alpha) = N(\alpha)^2 \) so \( N(\alpha) = x^2 + 5y^2 = \pm 3 \), but there are no integer solutions to this equation.
    All of the other factors are similarly irreducible.
\end{example}
\begin{remark}
    In any number field, one can factorise any \( \alpha \in \mathcal O_L \) into a product of irreducibles by induction on \( \abs{N(\alpha)} \), but this factorisation is not in general unique.
    An idea due to Kummer is to measure the failure of unique factorisation by studying ideals \( \mathfrak a \triangleleft \mathcal O_L \).
\end{remark}
If \( x_1, \dots, x_n \in \mathcal O_L \), we write \( \genset{x_1, \dots, x_n} \) for the ideal \( \sum x_i \mathcal O_L \) generated by the \( x_i \).
We will consider products of ideals, rather than products of elements.
\begin{definition}
    If \( \mathfrak a, \mathfrak b \triangleleft \mathcal O_L \), define \( \mathfrak a \mathfrak b = \qty{\sum a_i b_i \mid a_i \in \mathfrak a, b_i \in \mathfrak b} \).
\end{definition}
One can check that this is an ideal, and that products are associative.
\begin{example}
    \( \genset{x_1, \dots, x_n} \genset{y_1, \dots, y_m} = \genset{\qty{x_i y_j \mid 1 \leq i \leq n, 1 \leq j \leq n}} \).
    For instance, \( \genset{x}\genset{y} = \genset{xy} \), so the product of principal ideals is principal.
\end{example}
