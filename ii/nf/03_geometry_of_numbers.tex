\subsection{Imaginary quadratic fields}
Let \( L = \mathbb Q(\sqrt{-d}) \) where \( d \) is square-free and \( d < 0 \).
\( \mathcal O_L = \mathbb Z[\alpha] \) where \( \alpha = \frac{1}{2}(1+\sqrt{d}) \) or \( \alpha = \sqrt{d} \).
Choose a square root of \( d \) in \( \mathbb C \) to construct an embedding of \( \mathcal O_L \) into \( \mathbb C \).

Suppose \( \Lambda = \mathbb Z v_1 + \mathbb Z v_2 \subseteq \mathbb R^2 \) where \( \mathbb R^2 \) is equipped with the Euclidean norm, and \( v_1, v_2 \) are linearly independent over \( \mathbb R \).
Let \( A(\Lambda) \) be the area of the parallelogram generated by \( v_1 \) and \( v_2 \).
If \( v_i = a_i e_1 + b_i e_2 \), we have
\[ A(\Lambda) = \abs{\det \begin{pmatrix}
    a_1 & a_2 \\
    b_1 & b_2
\end{pmatrix}} \]
Minkowski's lemma is that a closed disk \( S \) around zero contains a nonzero point of \( \Lambda \) whenever the area of \( S \) is at least \( 4A(\Lambda) \).
More precisely, there exists \( \alpha \in \Lambda \) such that \( 0 < \abs{\alpha}^2 < \frac{4A(\Lambda)}{\pi} \).
Note that this condition depends only on the area of the parallelogram, not its shape.
This will be proven shortly.

We will apply this to \( \Lambda = \mathfrak \trianglelefteq \mathcal O_L \) for \( L = \mathbb Q(\sqrt{-d}) \), \( d < 0 \) square-free.
Let \( \sqrt{d} \in \mathbb C \) be chosen with positive imaginary part to embed \( \mathcal O_L \) in \( \mathbb C \).
\begin{lemma}
    \begin{enumerate}
        \item if \( \alpha = a + b \sqrt{d} \in \mathcal O_L \), then \( \abs{\alpha}^2 = (a+b\sqrt{d})(a-b\sqrt{d}) = N(\alpha) \);
        \item \( A(\mathcal O_L) = \frac{1}{2}\sqrt{\abs{D_L}} \);
        \item \( A(\mathfrak a) = N(\mathfrak a) A(\mathcal O_L) \);
        \item \( A(\mathfrak a) = \frac{1}{2} \abs{\Delta(\alpha_1, \alpha_2)}^{\frac{1}{2}} \) where \( \alpha_1, \alpha_2 \) are an integral basis for \( \mathfrak a \).
    \end{enumerate}
\end{lemma}
\begin{proof}
    Part (i) is clear.
    (iv) implies (ii) and (iii).
    We will prove (iv) later in a more general setting, giving the justification for the coefficient \( \frac{1}{2} \).
    
    We now prove (ii) and (iii) manually, without appealing to (iv).
    For part (ii), \( \mathcal O_L \) has basis \( 1, \alpha \).
    Therefore, \( A(\mathcal O_L) = \frac{1}{2}\sqrt{d} \) or \( \sqrt{d} \), which is exactly \( \frac{1}{2} \sqrt{\abs{D_L}} \).
    Part (iii) is a variant of the fact that \( \Delta(\alpha_1, \dots, \alpha_n) = N(\mathfrak a)^2 D_L \).
\end{proof}
Minkowski's lemma implies that there exists \( \alpha \in \mathfrak a \) with \( N(\alpha) \leq \frac{4A(\mathfrak a)}{\pi} = N(\mathfrak a) c_L \) where \( c_L = \frac{2\sqrt{\abs{D_L}}}{\pi} \) is Minkowski's constant.
Since \( \alpha \in \mathfrak a \), \( (\alpha) \subseteq \mathfrak a \).
Hence \( (\alpha) = \mathfrak a \mathfrak b \) for some \( \mathfrak b \trianglelefteq \mathcal O_L \).
So \( N(\alpha) = N((\alpha)) = N(\mathfrak a) N(\mathfrak b) \), so \( N(\mathfrak b) \leq c_L \).

Recall that the class group of \( L \) is \( \faktor{I_L}{P_L} \), the quotient of fractional ideals over principal ideals.
Then, \( [\mathfrak b] = [\mathfrak a^{-1}] \in \mathrm{Cl}_L \).
Replacing \( \mathfrak a \) with \( \mathfrak a^{-1} \), we have shown that for all \( [\mathfrak a] \in \mathrm{Cl}_L \), there exists a representative \( \mathfrak b \) of \( [\mathfrak a] \) which is an ideal with \( N(\mathfrak b) \leq \frac{2\sqrt{\abs{D_L}}}{\pi} = c_L \).
But for all \( m \in \mathbb Z \), the number of ideals \( \mathfrak a \trianglelefteq \mathcal O_L \) with \( N(\mathfrak a) = m \) is finite; indeed, if \( N(\mathfrak a) = m \), then \( m \in \mathfrak a \) so \( \mathfrak a \mid (m) \), but there are only finitely many ideals dividing \( (m) \), as they biject with ideals in \( \faktor{\mathcal O_L}{m\mathcal O_L} \simeq \qty(\faktor{\mathbb Z}{m\mathbb Z})^n \).

Therefore, we have shown that \( \mathrm{Cl}_L \) is finite, and generated by the class of prime ieals dividing \( (p) \), for \( p \) a prime integer less than \( \frac{2\sqrt{\abs{D_L}}}{\pi} = c_L \).
Indeed, if \( \mathfrak a = \mathfrak p_1^{e_1} \dots \mathfrak p_r^{e_r} \) with \( N(\mathfrak a) < c_L \), then \( N(\mathfrak p_i) < c_L \).
\begin{example}
    Let \( d = -7 \).
    Then \( D_L = -7 \), and \( \frac{2\sqrt{7}}{\pi} < 2 \).
    So there are no primes \( p < c_L \), giving \( \mathrm{Cl}_L = \qty{1} \).
    In particular, \( \mathcal O_L \) is a unique factorisation domain.
    Similarly, \( d = -1, -2, -3 \) give unique factorisation domains.
\end{example}
\begin{example}
    Let \( d = -5 \).
    Here, \( D_L = -20 \), and \( 2 < \frac{4\sqrt{5}}{\pi} < 3 \).
    Hence, \( \mathrm{Cl}_L \) is generated by prime ideals dividing \( (2) \).
    Note that \( (2) = (2, 1 + \sqrt{-5})^2 \) by Dedekind's theorem.

    We now must check if \( (2, 1 + \sqrt{-5}) \) is principal.
    If \( (2, 1 + \sqrt{-5}) = (\beta) \), then \( N(\beta) = 2 \).
    But \( \beta = a + b\sqrt{-5} \), so \( N(\beta) = a^2 + 5 b^2 \), which is not satisfiable by integers.
    So \( (2, 1 + \sqrt{-5}) \) is principal but its square is, so \( \mathrm{Cl}_L = \faktor{\mathbb Z}{2\mathbb Z} \).
\end{example}
\begin{example}
    Let \( d = -17 \), then \( 5 < c_L < 6 \).
    \( \mathrm{Cl}_L \) is generated by prime ideals dividing \( (2), (3), (5) \).
    Modulo 2, \( x^2 + 17 = x^2 + 1 = (x + 1)^2 \), so \( (2) = \mathfrak p^2 \) where \( \mathfrak p = (2, 1 + \sqrt{-17}) \).
    Modulo 3, \( x^2 + 17 = x^2 - 1 = (x + 1)(x - 1) \), giving \( (3) = \mathfrak q \overline{\mathfrak q} \) where \( \mathfrak q = (3, 1 + \sqrt{-17}), \overline{\mathfrak q} = (3, 1 - \sqrt{-17}) \).
    Modulo 5, \( x^2 + 17 = x^2 + 2 \) which is irreducible, so \( (5) \) is inert, so is trivial in the class group.

    Hence \( \mathrm{Cl}_L = (\mathfrak p, \mathfrak q, \overline{\mathfrak q}) = (\mathfrak p, \mathfrak q) \).
    We could compute powers of \( \mathfrak p \) and \( \mathfrak q \) until we obtain all nontrivial relations between them.
    A more efficient way to compute \( \mathrm{Cl}_L \) in this case is to find principal ideals of small norm which are multiples of 2 and 3 to find the relations.
    Consider \( (1 + \sqrt{-17}) \), which has norm \( N(1 + \sqrt{-17}) = 18 = 2 \cdot 3^2 \).
    Note that \( 1 + \sqrt{-17} \in \mathfrak p \cap \mathfrak q \) so \( (1 + \sqrt{-17}) = \mathfrak p \mathfrak q \mathfrak r \) where \( \mathfrak r \in (\mathfrak p, \mathfrak q) \).
    We can show that \( \mathfrak r = \mathfrak q \).
    This shows that \( [\mathfrak p] = [\mathfrak q]^{-2} \) in \( \mathrm{Cl}_L \).
    So \( \mathrm{Cl}_L \) is generated by \( [\mathfrak q] \).
    So it is cyclic, and we can show \( [\mathfrak q]^2 \neq 1 \), as \( \mathfrak p \) is not principal, but \( [\mathfrak q]^4 = [\mathfrak p^2]^{-1} = 1 \).
    So \( \mathrm{Cl}_L = \faktor{\mathbb Z}{4\mathbb Z} \).
\end{example}
\begin{theorem}
    Let \( L = \mathbb Q(\sqrt{-d}) \) with \( d > 0 \).
    \begin{enumerate}
        \item \( \mathcal O_L \) is a unique factorisation domain if \( d \in \qty{1, 2, 3, 7, 11, 19, 43, 67, 163} \);
        \item there are no others.
    \end{enumerate}
\end{theorem}
