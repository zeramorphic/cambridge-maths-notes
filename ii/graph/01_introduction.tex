\subsection{Definitions}
We use the notation \( [n] \) for \( \qty{1, \dots,n} \).
For a set \( X \) and \( k \in \mathbb N \), we define \( X^{(k)} = \qty{Y \subseteq X \mid \abs{Y} = k} \).
\begin{definition}
	A \emph{graph} is a pair \( (V, E) \), where \( V \) is a set of \emph{vertices} and \( E \) is a set of \emph{edges} where \( E \subseteq V^{(2)} \).
	We use the notation \( V(G) \) to denote the set of vertices and \( E(G) \) to denote the set of edges, where \( G = (V, E) \) is a graph.
	We define \( \abs{G} = \abs{V(G)} \), and \( e(G) = \abs{E(G)} \).
\end{definition}
\begin{example}
	The complete graph on \( n \) vertices, denoted \( K_n \), is the graph with \( V = [n] \) and \( E = V^{(2)} \).
\end{example}
Note that we sometimes use juxtaposition of names of vertices to denote an edge between them, so \( 13 \) represents the edge \( \qty{1, 3} \).
\begin{remark}
	Edges are undirected. There are no edges from a vertex to itself. Edges between vertices are unique if they exist.
	Most of the graphs covered in this course are finite.
\end{remark}
\begin{example}
	The empty graph on \( n \) vertices, denoted \( \overline K_n \), is the graph with vertex set \( V = [n] \) and \( E = \varnothing \).
\end{example}
\begin{example}
	The path of length \( n \), denoted \( P_n \), is the graph with \( V = [n+1] \) and \( E = \qty{\qty{1,2},\dots,\qty{n,n+1}} \).
\end{example}
\begin{example}
	The cycle of length \( n \), denoted \( C_n \), is the graph with \( V = [n] \) and \( E = \qty{\qty{1,2}, \dots, \qty{n-1,n}, \qty{n,1}} \).
\end{example}
\begin{definition}
	Let \( G \) be a graph, \( x \in V(G) \).
	The \emph{neighbourhood} of \( x \) in \( G \) is
	\[ N_G(x) = \qty{y \in V(G) \mid \qty{x,y} \in E(G)} \]
	If \( y \) is a neighbour of \( x \), we write \( x \sim y \).
\end{definition}
Note that \( \sim \) is irreflexive and not transitive in general.
\begin{definition}
	The \emph{degree} of a vertex \( x \in V(G) \) is defined as \( \deg x = \abs{N(x)} \).
\end{definition}
\begin{definition}
	Let \( G, H \) be graphs.
	A \emph{graph isomorphism} is a bijection \( \varphi \colon V(G) \to V(H) \) such that \( \qty{u,v} \in E(G) \iff \qty{\varphi(u),\varphi(v)} \in E(H) \).
\end{definition}
\begin{definition}
	We say \( H \) is a \emph{subgraph} of \( G \) if \( V(H) \subseteq V(G) \) and \( E(H) \subseteq E(G) \).
\end{definition}
If \( G \) is a graph, and \( xy \in E(G) \), we define \( G - xy \) to be the graph \( (V(G), E(G)\setminus \qty{xy}) \).
Similarly, for \( x, y \in V(G) \), we define \( G + xy \) to be the graph \( (V(G), E(G) \cup \qty{xy}) \).
\begin{definition}
	Let \( x, y \in V(G) \).
	A \emph{walk} from \( x \) to \( y \) in \( G \) is a sequence of vertices \( (x, \dots, y) \) such that each consecutive pair of elements of the sequence is connected by an edge in \( G \).
	A \emph{path} from \( x \) to \( y \) in \( G \) is a walk where all the vertices are disjoint.
\end{definition}
\begin{definition}
	A graph is \emph{connected} if every pair of vertices is connected with a path.
\end{definition}
The concatenation of two paths or walks \( P \) and \( P' \) is written \( PP' \).
\begin{remark}
	The concatenation of two walks is a walk.
	The concatenation of two paths is not necessarily a path, if the two paths share a vertex.
\end{remark}
\begin{proposition}
	If \( W \) is a \( x \)--\( y \) walk for \( x \neq y \), \( W \) contains a \( x \)--\( y \) path, where `contains' denotes a subsequence.
\end{proposition}
\begin{proof}
	Let \( W' \) be the minimal \( x \)--\( y \) walk in \( W \).
	This is a path, because if there were a repeated vertex, we could find a shorter path by eliminating the detour.
\end{proof}
\begin{definition}
	We define the \emph{distance} between two vertices, denoted \( d(x,y) \), to be the shortest length of a path between \( x \) and \( y \).
	If \( G \) is connected, this turns \( G \) into a metric space on its vertices.
\end{definition}

\subsection{Trees}
\begin{definition}
	A graph \( G \) is a \emph{acyclic} if it does not contain a cycle \( C_k \) as a subgraph.
	A graph \( G \) is a \emph{tree} if it is acyclic and connected.
\end{definition}
\begin{proposition}
	The following are equivalent.
	\begin{enumerate}
		\item \( G \) is a tree (acyclic and connected).
		\item \( G \) is \emph{minimally connected}: \( G \) is connected and for all \( xy \in E(G) \), \( G - xy \) is not connected.
		\item \( G \) is maximally acyclic: \( G \) is acyclic and for all \( xy \not\in E(G) \), \( G + xy \) contains a cycle.
	\end{enumerate}
\end{proposition}
\begin{proof}
	\emph{(i) implies (ii).}
	Let \( xy \in E(G) \).
	Suppose \( G - xy \) were connected.
	Then there exists an \( x \)--\( y \) path \( P \) in \( G - xy \).
	We can then close up the path \( P \) into a cycle in \( G \) by adding the edge \( xy \).
	This contradicts the fact that \( G \) is acyclic.

	\emph{(ii) implies (i).}
	Suppose \( G \) has a cycle \( C \).
	Let \( xy \in E(C) \) be an edge in the cycle.
	We claim that \( G - xy \) is connected.
	Let \( P \) be a \( u \)--\( v \) path in \( G \).
	If \( P \) contains the edge \( xy \), replace the use of this edge with the remainder of the cycle, traversed in the opposite direction.
	This yields a \( u \)--\( v \) walk in \( G - xy \) which contains a \( u \)--\( v \) path.

	\emph{(i) implies (iii).}
	Let \( xy \not\in E(G) \).
	By connectedness, there exists an \( x \)--\( y \) path \( P \) in \( G \).
	Hence, adding \( xy \) to \( E(G) \), we obtain a cycle by concatenating \( P \) with \( xy \).

	\emph{(iii) implies (i).}
	Suppose \( G \) is not connected.
	Then there exist \( x \neq y \) such that there is no \( x \)--\( y \) path in \( G \).
	Hence, adding \( xy \) to \( E(G) \) cannot yield a cycle.
\end{proof}
\begin{definition}
	Let \( T \) be a tree.
	A \emph{leaf} of \( T \) is a vertex \( v \in V(T) \) where \( \mathrm{deg}(v) = 1 \).
\end{definition}
\begin{definition}
	Let \( G \) be a graph, and \( X \subseteq V(G) \).
	Then the \emph{graph induced on \( X \)}, denoted \( G[X] \) is the graph \( (X, \qty{xy \in E(G) \mid x \in X, y \in X}) \).
	If \( x \in G \), we define \( G - x \) to be the graph \( G[V(G) \setminus \qty{x}] \).
\end{definition}
\begin{proposition}
	Let \( T \) be a tree where \( \abs{T} \geq 2 \).
	Then \( T \) has a leaf.
\end{proposition}
\begin{proof}
	Let \( P = x_1, \dots, x_k \) be a longest possible path in \( T \).
	\( N(x_k) \subseteq \qty{x_1, \dots, x_{k-1}} \) by maximality of \( P \).
	If \( x_i \sim x_k \) for any \( 1 \leq i \leq k - 2 \), we have a cycle, which is a contradiction.
	Hence \( N(x_k) = \qty{x_{k-1}} \), so \( x_k \) is a leaf.
\end{proof}
\begin{remark}
	This proof actually demonstrates that any tree has at least two leaves, by considering \( x_1 \).
	We could alternatively have proven the lemma by taking a non-backtracking walk in \( G \), which exists assuming no leaf exists; then, since \( V(G) \) is finite, we must return to a point somewhere on the graph.
\end{remark}
\begin{proposition}
	Let \( T \) be a tree with \( n \geq 1 \) vertices.
	Then \( \abs{E(T)} = e(t) = n - 1 \).
\end{proposition}
\begin{proof}
	We prove this by induction on \( n \).
	The \( n = 1 \) case is trivial.
	Now, assume that all trees with \( n \) vertices have \( n - 1 \) edges, and suppose \( T \) has \( n + 1 \) vertices.
	\( T \) has a leaf \( x \).
	Then \( T - x \) is a tree with \( n \) vertices since it is still connected, and hence has \( n - 1 \) edges.
	Since \( T \) has one more edge than \( T - x \), namely the edge connecting the leaf \( x \) to \( T - x \), \( T \) has \( n \) edges as required.
\end{proof}
\begin{definition}
	Let \( G \) be a connected graph.
	Then a subgraph \( T \) of \( G \) is a \emph{spanning tree} if \( V(T) = V(G) \) and \( T \) is a tree.
\end{definition}
\begin{proposition}
	Every connected graph has a spanning tree.
\end{proposition}
\begin{proof}
	Begin with \( G \) and remove edges of \( E(G) \) such that the graph stays connected.
	When we can no longer remove edges, we must have a minimally connected subgraph of \( G \), and hence a tree.
\end{proof}

\subsection{Bipartite graphs}
\begin{definition}
	Let \( G = (V, E) \) be a graph.
	\( G \) is \emph{bipartite} if \( V = A \cup B \) where \( A \cap B = \varnothing \), such that all edges \( (x,y) \in E \) satisfy \( x \in A, y \in B \) or \( x \in B, y \in A \).

	The \emph{complete bipartite graph} on \( n \) and \( m \) vertices, denoted \( K_{n,m} \), is the bipartite graph with \( \abs{A} = n \), \( \abs{B} = m \) and with all possible edges.
\end{definition}
\begin{remark}
	Even cycles \( C_{2n} \) are bipartite, and odd cycles \( C_{2n+1} \) are not bipartite.
\end{remark}
\begin{definition}
	A \emph{circuit} is a sequence \( x_1, x_2, \dots, x_\ell, x_{\ell + 1} \) where \( x_i x_{i+1} \in E \) and \( x_{\ell + 1} = x_1 \).
	In other words, a circuit is a closed walk.
	The \emph{length} of this circuit is \( \ell \).
	A circuit is odd if its length is odd; a circuit is even if its length is even.
\end{definition}
\begin{proposition}
	Let \( C \) be an odd circuit in a graph \( G \).
	Then \( C \) contains an odd cycle.
\end{proposition}
\begin{theorem}
	Let \( G \) be a graph.
	Then \( G \) is bipartite if and only if \( G \) does not contain an odd cycle.
\end{theorem}
\begin{proof}
	If \( G \) contains an odd cycle, \( G \) is not bipartite because there exists a subgraph that is not bipartite.
	Suppose now that \( G \) contains no odd cycles.
\end{proof}
