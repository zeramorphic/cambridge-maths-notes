\subsection{?}
\begin{definition}
    A \emph{curve} is a variety of dimension 1.
\end{definition}
For our purposes, a curve is taken to mean a smooth irreducible projective variety of dimension 1.
By convention, a curve \( C \) implicitly has an expression as \( \mathbb V(I) \subseteq \mathbb P^n \), but this ambient space will not play an important role.
\begin{example}
    Let \( f_d \in \mathbb C[X, Y, Z] \) be homogeneous of degree \( d \).
    For almost all choices of coefficients, \( \mathbb V(f_d) \) is a (smooth irreducible projective) curve.
    We will show that for \( d, d' \geq 2 \), \( \mathbb V(f_d) \) and \( \mathbb V(f_{d'}) \) are never isomorphic.
\end{example}
\begin{proposition}
    Let \( C \) be a curve, and let \( D \subsetneq C \) be a proper Zariski closed subset.
    Then \( D \) is a finite union of points.
\end{proposition}
\begin{proof}
    It suffices to prove this for irreducible affine curves \( V \subseteq \mathbb A^n \).
    Let \( W \subsetneq V \) be a proper irreducible closed subset; we will show this is a single point.
    By the Nullstellensatz, there is a strict containment \( I(V) \subsetneq I(W) \).
    
    If \( t \in \mathbb C[W] \setminus \mathbb C \), we can use this to produce an element \( y \in \mathbb C[V] \) as follows.
    \( \varphi \colon W \hookrightarrow V \) gives the pullback map \( \varphi^\star \colon C[V] \to C[W] \) which is a surjection.
    Take any \( y \) such that \( \varphi^\star(y) = t \).

    We can also take \( x \in \mathbb C[V] \) such that \( \varphi^\star(x) = 0 \), so \( x \not\in \mathbb C \).
    One can show that \( x, y \) are algebraically independent in \( \mathbb C(V) \), as \( t \) is transcendental.
    This gives two algebraically independent elements of \( C(V) \), which has transcendence degree 1.
    So no such \( t \) can exist, so \( \mathbb C[W] = \mathbb C \).
    Therefore \( W \) is a point.
\end{proof}
Recall that if \( V \) is an irreducible variety, it has a coordinate ring (if it is affine), a function field, a local ring at each point, and the maximal ideal of functions vanishing at the given point in the local ring.
These can be specialised in the case of curves.
Note that if \( C \) is a smooth irreducible projective curve, there exists \( t \in \mathbb C(V) \) such that \( \faktor{\mathbb C(V)}{\mathbb C(t)} \) is finite.
\begin{theorem}
    Let \( P \) be a smooth point of an irreducible curve \( V \).
    Then, the ideal \( \mathfrak m_P \trianglelefteq \mathcal O_{V,P} \) is principal.
\end{theorem}
A generator \( \pi_P \) of \( \mathfrak m_P \) is called a \emph{local parameter}, a \emph{coordinate}, or a \emph{uniformiser}.
\begin{proof}
    We assume \( P \) lies in the affine patch \( V_0 \) of \( V \).
    By changing coordinates, we can set \( P = 0 \in \mathbb A^n \).
    \begin{align*}
        \mathbb C[V_0] &= \faktor{\mathbb C[X_1, \dots, X_n]}{I(V_0)} = \mathbb C[x_1, \dots, x_n];\\
        \mathcal O_P &= \mathcal O_{V_0,P} = \qty{\frac{f}{g} \mid f, g \in \mathbb C[V_0], g \not\in (x_1, \dots, x_n)}\\
        \mathfrak m_P &= \qty{\frac{f}{g} \mid f \in (x_1, \dots, x_n), g \not\in (x_1, \dots, x_n)} = x_1 \mathcal O_P + \dots + x_n \mathcal O_P \subseteq \mathcal O_P
    \end{align*}
    where \( x_i \) is the image of \( X_i \) under the quotient map.
    More generally, if \( J \trianglelefteq \mathcal O_P \) is any ideal, \( \frac{f}{g} \in J \) if and only if \( f \in J \).
    Therefore,
    \[ J = \qty{\frac{f}{g} \mid f \in J \cap \mathbb C[V_0], g \in \mathbb C[V_0], g(P) \neq 0} \]
    In particular, \( J \) is finitely generated.
    
    Since \( P \) is smooth, \( T_{V_0,P}^{\text{aff}} \) is a line, and by changing coordinates, \( T_{V,P} = \qty{X_2 = X_3 = \dots = X_n = 0} \).
    We claim that \( x_1 \) generates \( \mathfrak m_P \).
    Since \( T_{V,P} \) is cut out by linearisations at \( P = 0 \) of elements in \( I(V_0) \), there exist functions \( f_2, \dots, f_n \in I(V_0) \) such that \( f_j = X_j - h_j \) where \( h_j \) has no terms of degree less than 2.
    In \( \mathcal O_P \),
    \[ x_j = h_j(x_1, \dots, x_n) \in (x_1^2, x_1 x_2, \dots, x_n^2) = \mathfrak m_P^2 \]
    Thus, \( \mathfrak m_P = \sum_{j=1}^n x_i \mathcal O_P = x_1 \mathcal O_P + \mathfrak m_P^2 \).
    The result that \( \mathfrak m_P \) is generated by \( x_1 \) follows from Nakayama's lemma.
    \begin{lemma}[Nakayama]
        Let \( R \) be a ring, let \( M \) be a finitely generated \( R \)-module, and let \( J \trianglelefteq R \) be an ideal.
        Then,
        \begin{enumerate}
            \item if \( JM = M \), there exists \( r \in J \) such that \( (1+r)M = 0 \); and
            \item if \( N \leq M \) is a submodule such that \( JM + N = M \), then there exists \( r \in J \) such that \( (1+r)M \subseteq N \).
        \end{enumerate}
    \end{lemma}
    % proof in notes, not examinable or important
    % acts like the inverse/implicit function theorem
    Let
    \[ R = \mathcal O_L \supseteq J = \mathfrak m_P = M \supseteq N = (x_1) \]
    and apply part (ii) of Nakayama's lemma to conclude.
\end{proof}
\begin{corollary}
    Let \( V = \mathbb V(f) \subseteq \mathbb A^2 \) be an irreducible affine curve.
    Then, if \( P \in V \) is a smooth point, the function \( V \to \mathbb C \) defined by \( Q \mapsto X(Q) - X(P) \) is a local parameter if and only if \( \pdv{f}{Y}\qty(P) \neq 0 \).
\end{corollary}
\begin{proof}
    Use the proof of the above theorem.
\end{proof}
