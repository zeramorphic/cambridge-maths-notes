\chapter[Algebraic Geometry \\ \textnormal{\emph{Lectured in Lent \oldstylenums{2023} by \textsc{Dr.\ D.\ Ranganathan}}}]{Algebraic Geometry}
\emph{\Large Lectured in Lent \oldstylenums{2023} by \textsc{Dr.\ D.\ Ranganathan}}

In this course, we study the duality between systems of polynomial equations and the geometry or topology of their solution sets.
Sets of points that arise as solution sets of polynomials are called algebraic varieties.
We therefore study the correspondence between sets of polynomials and the varieties they define.

One can show that the set of varieties satisfy the axioms of the closed sets of a topological space.
We can thus define a topology where the closed sets are precisely the algebraic varieties; this is called the Zariski topology.
This very explicit description of the closed sets allows us to study this topology in depth.
There are some drawbacks; the Zariski topology is not even Hausdorff.

Some geometric properties of algebraic varieties can be studied algebraically.
For example, the dimension of a variety is the amount of algebraically independent transcendental elements of a field associated to the variety.
Perhaps the simplest varieties are the curves, those varieties with dimension 1.
They have comparatively simple field structure, and are studied in depth.

\subfile{../../ii/alggeom/main.tex}
