\subsection{Coordinate rings}
Consider a polynomial \( f \in \mathbb C[\vb X] \).
We obtain a function \( f \colon \mathbb A^n \to \mathbb A^1 \),
If \( V \subseteq \mathbb A^n \) and \( f, g \in \mathbb C[\vb X] \), we are interested in when \( f, g \) induce the same set-theoretic function on \( V \).
We intend to show that \( f, g \) induce the same function if and only if \( f - g \in I(V) \).
Therefore, we can study polynomials modulo this relation by taking the quotient with respect to this ideal.
\begin{definition}
    Let \( V \subseteq \mathbb A^n \) be a variety.
    The \emph{coordinate ring} of \( V \), or the \emph{ring of regular functions} of \( V \), is defined as \( \faktor{\mathbb C[\vb X]}{I(V)} \), denoted \( \mathbb C[V] \) or \( \mathcal O(V) \).
\end{definition}
\begin{corollary}
    Let \( V \) be a variety.
    Then \( V \) is irreducible if and only if \( \mathbb C[V] \) is an integral domain.
\end{corollary}
\begin{remark}
    \( \mathbb C[V] \) does not precisely determine \( V \) or \( I(V) \).
    For instance, consider a surjective homomorphism \( \theta \colon \mathbb C[\vb X] \to \mathbb C[V] \), then \( \ker \theta = I \) is an ideal, and \( \mathbb V(I) \) is a variety with coordinate ring \( \mathbb C[V] \).
    However, there is not a unique such homomorphism in general.
    For instance, \( \mathbb C[X] \simeq \faktor{\mathbb C[X,Y]}{(Y)} \).
\end{remark}
\begin{definition}
    Let \( I \vartriangleleft \mathbb C[\vb X] \).
    We define the \emph{radical ideal} of \( I \) to be
    \[ \sqrt{I} = \qty{f \in \mathbb C[\vb X] \mid \exists m > 0,\, f^m \in I} \]
\end{definition}
This is an ideal.
\( \sqrt{\sqrt{I}} = \sqrt{I} \).
Note that \( \mathbb V(I) = \mathbb V\qty(\sqrt{I}) \).
\begin{theorem}[strong form of Hilbert's Nullstellensatz]
    Let \( I \vartriangleleft \mathbb C[\vb X] \) be an ideal, and \( V = \mathbb V(I) \).
    Then \( I(V) = \sqrt{I} \).
\end{theorem}
Therefore, the map \( V \mapsto I(V) \) maps precisely onto the space of radical ideals, ideals which are equal to their radicals.
\begin{example}
    Let \( V = \qty{0} \in \mathbb A^1 \).
    We can write \( V = \mathbb V(X^2) \), so its coordinate ring is
    \[ \faktor{\mathbb C[X]}{I(\mathbb V(X^2))} = \faktor{\mathbb C[X]}{\sqrt{(X^2)}} = \faktor{\mathbb C[X]}{(X)} \simeq \mathbb C \]
    In building the coordinate ring, we forget the structure of \( X^2 \).
    If we had instead considered \( \faktor{\mathbb C[X]}{(X^2)} \), we would have certain nonzero elements whose squares are zero.
\end{example}

\subsection{Morphisms}
Let \( V \subseteq \mathbb A^n \) and \( W \subseteq \mathbb A^m \) be affine varieties.
\begin{definition}
    A \emph{regular map} or \emph{morphism} from \( V \) to \( W \) is a function \( \varphi \colon V \to W \) such that there exist elements \( f_1, \dots, f_m \in \mathbb C[V] \) such that
    \[ \varphi(P) = (f_1(P), \dots, f_m(P)) \]
    for all \( P \in V \).
\end{definition}
The set of all morphisms from \( V \) to \( W \) is denoted \( \mathrm{Mor}(V,W) \).
\begin{example}
    The morphisms \( V \) to \( \mathbb A^1 \) are precisely the functions in the coordinate ring \( \mathbb C[V] \).
\end{example}
\begin{example}
    Linear projections \( \mathbb A^n \to \mathbb A^m \) are morphisms.
    More generally, linear transformations and affine translations are also morphisms.
\end{example}
\begin{example}
    If \( V \subseteq W \subseteq \mathbb A^n \) where \( V, W \) are varieties, then the inclusion map \( V \hookrightarrow W \) is a morphism.
\end{example}
\begin{proposition}
    Let \( \varphi \colon V \to W, \psi \colon W \to Z \) be morphisms.
    Then the composite map \( \psi \circ \varphi \) is a morphism \( V \to Z \).
\end{proposition}
\begin{proof}
    The composition of polynomials is a polynomial.
\end{proof}

\subsection{Pullbacks}
\begin{definition}
    Let \( \varphi \colon V \to W \) be a morphism, and let \( g \in \mathbb C[W] \).
    Then, the \emph{pullback} is \( \varphi^\star(g) = g \circ \varphi \colon V \to \mathbb C \).
    Note that \( \varphi^\star(g) \in \mathbb C[V] \), so \( \varphi^\star \) gives a map \( \mathbb C[W] \to \mathbb C[V] \).
\end{definition}
\begin{remark}
    This map \( \varphi^\star \) is a ring homomorphism, and restricts to the identity on \( \mathbb C \).
\end{remark}
\begin{definition}
    A ring homomorphism \( \mathbb C[X] \to \mathbb C[Y] \) that restricts to the identity on \( \mathbb C \) is called a \emph{\( \mathbb C \)-algebra homomorphism}.
\end{definition}
\begin{theorem}
    Let \( V \subseteq \mathbb A^n, W \subseteq \mathbb A^m \) be affine varieties.
    The map \( \alpha \colon \varphi \mapsto \varphi^\star \) defines a bijection from \( \mathrm{Mor}(V, W) \) to the space of \( \mathbb C \)-algebra homomorphisms \( \mathbb C[W] \to \mathbb C[V] \).
\end{theorem}
\begin{proof}
    Let \( y_1, \dots, y_n \in \mathbb C[W] \) be the coordinate functions on \( W \), which are the restrictions of the standard linear coordinate functions on \( \mathbb A^n \).

    First, we show injectivity of \( \alpha \).
    Let \( \varphi \colon V \to W \) be a morphism.
    For any point \( P \in V \),
    \[ \varphi(P) = (y_1(\varphi(P)), \dots, y_m(\varphi(P))) = (\varphi^\star(y_1)(P), \dots, \varphi^\star(y_n)(P)) \]
    So \( \varphi \) is determined by the values of \( \varphi^\star(y_1), \dots, \varphi^\star(y_n) \).

    Now we show its surjectivity.
    Let \( \lambda \colon \mathbb C[W] \to \mathbb C[V] \) be a \( \mathbb C \)-algebra homomorphism, and let \( f_i = \lambda(y_i) \in \mathbb C[V] \).
    We can now define the map \( \varphi = (f_1, \dots, f_m) \colon V \to \mathbb A^m \).
    We will show that \( \varphi \) has image contained in \( W \), so that we have \( \varphi \colon V \to W \), which then shows that \( \varphi \) is a morphism \( V \to W \).
    For \( P \in V \), we must show \( g(\varphi(P)) = 0 \) for all \( g \in I(W) \).
    We obtain \( g(f_1(P), \dots, f_m(P)) = \lambda(g)(P) \).
    But \( g = 0 \) in \( \mathbb C[W] \), so \( g(\varphi(P)) = 0 \) as required.
    Hence \( \varphi \colon V \to W \) is a morphism, and \( \lambda = \varphi^\star \) since \( \varphi^\star(y_i) = f_i = \lambda(y_i) \).
\end{proof}
\begin{definition}
    Two affine varieties \( V, W \) are \emph{isomorphic} if we have \( \varphi \colon V \to W, \psi \colon W \to V \) where \( \varphi \circ \psi = \mathrm{id}_W \) and \( \psi \circ \varphi = \mathrm{id}_V \).
\end{definition}
\begin{theorem}
    \( V \) is isomorphic to \( W \) if and only if \( \mathbb C[V] \) is isomorphic to \( \mathbb C[W] \) as \( \mathbb C \)-algebras.
\end{theorem}
\begin{proof}
    Use the previous theorem.
\end{proof}
\begin{example}
    The affine line \( \mathbb A^1 \) is isomorphic to the twisted cubic \( \qty{(t, t^2, t^3) \mid t \in \mathbb C} \).
    This can be easily shown by calculating the coordinate rings explicitly.
\end{example}
\begin{example}
    Let \( V \subseteq \mathbb A^2 \) be given by \( X_1 X_2 (X_1 - X_2) = 0 \).
    This is the union of three lines, intersecting at the origin.
    Let \( W \subseteq \mathbb A^3 \) be given by \( X_1 X_2 = X_2 X_3 = X_3 X_1 = 0 \), which is also a union of three lines, which in this case are the coordinate axes.
    These are not isomorphic as varieties, because their coordinate rings are not isomorphic, which can be easily shown using tangent spaces, defined in later sections.
    Note, however, that \( V \) and \( W \) are homeomorphic in the Euclidean topology.
\end{example}

\subsection{Rational functions}
\begin{definition}
    Let \( V \subseteq \mathbb A^n \) be an irreducible variety.
    Its \emph{function field}, \emph{field of rational functions}, or \emph{field of meromorphic functions} is the field of fractions \( \mathbb C(V) = FF(\mathbb C[V]) \) of \( \mathbb C[V] \).
\end{definition}
\begin{remark}
    Since \( V \) is irreducible, \( I(V) \) is prime, so \( \mathbb C[V] \) is an integral domain.
    This allows us to construct the field of fractions.
\end{remark}
\begin{definition}
    Let \( \varphi \) be a rational function.
    A point \( P \in V \) is called \emph{regular} if \( \varphi \) can be expressed as a ratio \( \frac{f}{g} \) with \( g(P) \neq 0 \).
\end{definition}
\begin{remark}
    If \( \varphi = \frac{f}{g} \), we obtain a well-defined function \( \varphi \colon V \setminus \mathbb V(g) \to \mathbb C \).
    The domain is an open set in \( V \), since \( \mathbb V(g) \) is Zariski closed.
\end{remark}
\begin{example}
    Consider the rational function \( X_1^2 / X_2 \in \mathbb C(\mathbb A^2) \).
    This defines a map on the complement of the \( X_2 \)-axis.
    Note that \( X^3 / X_1 X_2 \) defines the same function, but only on points other than \( \mathbb V(X_1 X_2) \).
    Note that \( X^3 / X_1 X_2 = X_1^2 / X_2 \in \mathbb C(\mathbb A^2) \), so we cannot quite think of elements of \( \mathbb C(\mathbb A^2) \) as functions.
\end{example}
\begin{remark}
    A rational function on \( V \) can be thought of as a pair \( (U, f) \) with \( U \subseteq V \) Zariski open, such that \( f \) is a function \( U \to \mathbb C \).
    We define the equivalence relation \( (U, f) \sim (U', f') \) if \( f, f' \) agree on some nonempty Zariski open set contained in \( U \) and \( U' \).
    Note that if \( V \) is irreducible, every nonempty open subset is dense in the Zariski topology.
\end{remark}
\begin{definition}
    A \emph{local ring} is a ring \( R \) that contains a unique maximal ideal.
\end{definition}
\begin{definition}
    Let \( V \) be an irreducible variety, and let \( P \).
    The \emph{local ring of \( V \) at \( P \)} is \( \mathcal O_{V,P} = \qty{f \in \mathbb C(V) \mid f \text{ is regular at } P} \).
\end{definition}
\begin{proposition}
    The local ring of an irreducible variety \( V \) at a point \( P \) is a local ring.
    Its unique maximal ideal is
    \[ \mathfrak m_{V,P} = \qty{f \in \mathcal O_{V,P} \mid f(P) = 0} = \ker (f \mapsto f(P)) \]
    Further, the invertible elements of \( \mathcal O_{V,P} \) are precisely those \( f \) such that \( f(P) \neq 0 \).
\end{proposition}
The proof follows from the following more general lemma.
\begin{lemma}
    A ring \( R \) is a local ring if and only if \( R \setminus R^\star \) is an ideal.
    If so, the unique maximal ideal is \( R \setminus R^\star \).
\end{lemma}
\begin{proof}
    If \( A \trianglelefteq R \) is an ideal, then \( A \neq R \) if and only if \( A \subseteq R \setminus R^\star \), because if any unit lies in \( A \), it must be all of \( R \).
    Hence, if \( R \setminus R^\star \) is an ideal, it is automatically the unique maximal ideal.

    Conversely, let \( R \) be a local ring with unique maximal ideal \( \mathfrak m \).
    Then \( \mathfrak m \subseteq R \setminus R^\star \), and if \( x \in R \setminus R^\star \) we must have \( (x) \neq R \), so \( (x) \subseteq \mathfrak m \) by maximality.
    Hence \( \mathfrak m = R \setminus R^\star \).
\end{proof}
Note that this proves the previous proposition, as \( \frac{f}{g} \in \mathcal O_{V,P} \) is invertible if and only if \( \qty(\frac{f}{g})(P) \neq 0 \).
\begin{example}
    Let
    \[ R = \qty{\frac{f}{g} \in \mathbb C(t) \mid \text{ignoring factors, } g(0) \neq 0} = \mathcal O_{\mathbb A^1, 0} \]
    Here, the maximal ideal is \( (t) \), and \( \faktor{R}{(t)} = \mathbb C \).

    Let \( S = \mathbb C\Brackets{t} \) be the ring of formal power series in \( t \).
    This is a local ring by the lemma; its maximal ideal is \( (t) \).
    Note that in fact \( R \subseteq S \).
\end{example}
