\subsection{?}
Quantum information is encoded in the states of a quantum system.
Classical information is encoded in states chosen from an orthonormal set, since all distinct classical messages can be distinguished.
Given a quantum system \( S \) and a quantum state \( \ket{\psi} \), we can perform this sequence of operations.
\begin{itemize}
    \item (ancilla) Consider an auxiliary system \( A \) in a fixed state \( \ket{A} \in \mathcal V_A \).
    The composite system \( SA \) has vector space \( \mathcal V_S \otimes \mathcal V_A \).
    The initial joint state is \( \ket{\psi}\ket{A} \).
    This results in an embedding of quantum information in a higher dimensional space.
    \item (unitary) Consider the action of a unitary operator \( U \) on \( SA \) (or on \( S \)), modelling the time evolution of the quantum system.
    \item (measure) We can perform measurements on \( SA \) (or on \( S \)).
    The post-measurement state of \( S \) is retained, and the auxiliary system \( A \) is discarded.
\end{itemize}
This process is sometimes known as `going to the church of the higher Hilbert space'.
The presence of the ancilla allows for entanglement with other quantum systems.

\subsection{No-cloning theorem}
Classically, information can be easily copied by measuring all relevant information and reproducing it.
% We can model this by considering a state \( A \) containing information and a `blank' state \( B \), and performing an operation to yield the original state \( A \) together with a new copy of \( A \) in the place of \( B \).
Quantum copying involves three systems
\begin{itemize}
    \item a system \( A \) containing some quantum information to be copied;
    \item a system \( B \) with \( \mathcal V_B \simeq \mathcal V_A \) initially in some fixed state \( \ket{0} \) where the information is to be copied;
    \item a system \( M \) which represents any physical machinery in some `ready' state \( \ket{M_0} \) required for performing the copy.
\end{itemize}
The initial state of this composite system \( ABM \) is \( \ket{\psi} \ket{0} \ket{M_0} \).
Note that the \( \ket{\psi} \) and \( \ket{0}\ket{M_0} \) are \emph{uncorrelated} in this state, as we are using the tensor product to combine them.
Suppose that the cloning process is performed using some unitary operator \( U \), so \( U\ket{\psi_A}\ket{0}\ket{M_0} = \ket{\psi_A}\ket{\psi_B}\ket{M_\psi} \).
This cloning process may be required to work either for all states of \( A \), or for some subset of \( A \).
\begin{theorem}
    Let \( \mathcal S \) be any set of states of the system \( A \) that contains at least one pair of distinct non-orthogonal states.
    Then there does not exist any unitary operator \( U \) that clones all states in \( \mathcal S \).
\end{theorem}
\begin{proof}
    Let \( \ket{\xi}, \ket{\eta} \) be distinct non-orthogonal states in \( \mathcal S \), so \( \ip{\xi}{\eta} \neq 0 \).
    Suppose such a unitary operator \( U \) exists.
    Then, we must have
    \[ U \ket{\xi_A}\ket{0_B}\ket{M_0} = \ket{\xi_A}\ket{\xi_B}\ket{M_\xi};\quad U \ket{\eta_A}\ket{0_B}\ket{M_0} = \ket{\eta_A}\ket{\eta_B}\ket{M_\eta} \]
    Unitary operators preserve inner products.
    Hence,
    \[ \ip{\xi_A}{\eta_A} \ip{0_B}{0_B} \ip{M_0}{M_0} = \ip{\xi_A}{\eta_A} \ip{\xi_B}{\eta_B} \ip{M_\xi}{M_\eta} \]
    Hence, \( \ip{\xi}{\eta} = \qty(\ip{\xi}{\eta})^2 \ip{M_\xi}{M_\eta} \).
    By taking the absolute value, \( \abs{\ip{\xi}{\eta}} = \abs{\ip{\xi}{\eta}}^2 \abs{\ip{M_\xi}{M_\eta}} \).
    Since \( \xi \neq \eta \), we must have \( 0 < \abs{\ip{\xi}{\eta}} < 1 \), and \( 0 \leq \abs{\ip{M_\xi}{M_\eta}} \leq 1 \).
    Therefore, \( 1 = \abs{\ip{\xi}{\eta}} \abs{\ip{M_\xi}{M_\eta}} < 1 \), which is a contradiction.
\end{proof}
If quantum cloning were possible, superluminal (indeed, instantaneous) communication would also be possible.
Suppose we have a state \( \ket{\psi^+_{AB}} = \frac{1}{\sqrt{2}} \qty(\ket{00} + \ket{11}) \in \mathbb C^2 \otimes \mathbb C^2 \).
Let \( A, B \) be the entangled parts of this quantum state, and suppose that we send the qubits \( A \) and \( B \) far apart from each other.

If we want to send the bit `yes' from \( A \) to \( B \), we measure the qubit \( A \) in the basis \( \qty{\ket{0}, \ket{1}} \), which gives outcomes \( 0, 1 \) with probability \( \frac{1}{2} \).
If the outcome is 0, the final state of \( B \) is \( \ket{0} \), and if the outcome is 1, the final state of \( B \) is \( \ket{1} \).
If we want to send `no', we instead measure \( A \) in the basis \( \qty{\ket{+}, \ket{-}} \), which gives the outcomes \( +, - \) with probability \( \frac{1}{2} \).
Similarly, the final state of \( B \) is \( \ket{+} \) or \( \ket{-} \).

We claim that these `yes' \( (\ket{0}, \ket{1}) \) and `no' \( (\ket{+}, \ket{-}) \) \emph{preparations} of qubit \( B \) are indistinguishable by \( B \) with any local action on the qubit.
That is, they each give exactly the same probability distribution of outcomes of any measurement.
In fact, the distribution matches the prior distribution before qubit \( A \) was measured.

Let \( \Pi_i \) be the projection operator for outcome \( i \) on qubit \( B \).
Suppose that `yes' was sent.
Then,
\[ p_{\text{yes}}(i) = \frac{1}{2}\ev{\Pi_i}{0} + \frac{1}{2}\ev{\Pi_i}{1} = \frac{1}{2} \Tr \qty[\Pi_i \op{0}{0}] + \frac{1}{2} \Tr \qty[\Pi_i \op{1}{1}] = \frac{1}{2} \Tr \qty[\Pi_i \qty(\op{0}{0} + \op{1}{1})] = \frac{1}{2} \Tr \Pi_i \]
In the `no' case,
\[ p_{\text{no}}(i) = \frac{1}{2}\ev{\Pi_i}{+} + \frac{1}{2}\ev{\Pi_i}{-} = \frac{1}{2} \Tr \qty[Pi_i \qty(\op{+}{+} + \op{-}{-})] = \frac{1}{2} \Tr \Pi_i \]
These probability distributions match.

Suppose that cloning were possible.
We clone the qubit \( B \) multiple times after the message was sent, to produce one of the states \( \ket{0} \dots \ket{0}, \ket{1} \dots \ket{1}, \ket{+} \dots \ket{+}, \ket{-} \dots \ket{-} \).
We now measure each qubit in the basis \( \ket{0}, \ket{1} \) separately.
If the `yes' message was sent, all measurements will result in 0 or 1.
If `no' was sent, it is possible that two measurements would differ.
In expectation, half of the measurements would result in the outcome 0 and half would result in the outcome 1.
Therefore, the `yes' and `no' errors can be distinguished with probability of error \( 2^{-N+1} \) if we make \( N \) copies of \( B \).

\subsection{Distinguishing non-orthogonal states}
Suppose you know a state \( \ket{\psi} \) has state \( \ket{\alpha_0} \) or \( \ket{\alpha_1} \) with probability \( \frac{1}{2} \), where \( \ip{\alpha_0}{\alpha_1} \neq 0 \).
Since the states are non-orthogonal, we cannot perfectly distinguish the states, but must allow some error rate.
The simplest possibility is to not make a measurement and guess randomly; in which case, the guess is correct with probability \( \frac{1}{2} \).

Suppose we append an auxiliary system \( \ket{A} \) to \( \ket{\alpha_i} \).
Note that \( \bra{A}\bra{\alpha_i}\ket{\alpha_i}\ket{A} = \ip{\alpha_i}{\alpha_i} \) as \( \ket{A} \) is normalised.
Applying a unitary operator \( U \) to \( \ket{\alpha_i} \) then performing a projection \( \Pi_0 \) or \( \Pi_1 \), this corresponds to simply performing a measurement \( \Pi_0' = U^\dagger \Pi_0 U \) or \( \Pi_1' = U^\dagger \Pi_1 U \), which leads to the same probabilities of outcomes.
Indeed,
\[ p(i) = \ev{\Pi_i}{U\xi} = \ev{U^\dagger \Pi_i U}{\xi} = \ev{\Pi_i'}{\xi} \]
Therefore, in this particular problem, we gain no benefit from moving to a larger Hilbert space or applying unitary operators.

We now describe the \emph{state estimation} or \emph{state discrimination} process.
We will consider a two-outcome measurement \( \qty{\Pi_0, \Pi_1} \), where \( \Pi_0 + \Pi_1 = I \).
The average success probability is
\begin{align*}
    p_S(\Pi_0, \Pi_1) &= \frac{1}{2} \prob{0 \mid \ket{\psi} = \ket{\alpha_0}} + \frac{1}{2} \prob{1 \mid \ket{\psi} = \ket{\alpha_1}} \\
    &= \frac{1}{2}\ev{\Pi_0}{\alpha_0} + \frac{1}{2}\ev{\Pi_1}{\alpha_1} \\
    &= \frac{1}{2} + \frac{1}{2} \Tr \qty[\Pi_0 \qty(\op{\alpha_0}{\alpha_0} - \op{\alpha_1}{\alpha_1})]
\end{align*}
as \( \Tr(A\op{\psi}{\psi}) = \ev{A}{\alpha} \).
The optimal choice of measurement maximises the average success probability \( p_S \).
Note that \( \Delta = \op{\alpha_0}{\alpha_0} - \op{\alpha_1}{\alpha_1} \) is self-adjoint, and we can write \( p_S = \frac{1}{2} + \frac{1}{2} \Tr(\Pi_0 \Delta) \).
Therefore, the eigenvalues of \( \Delta \) are real, and the eigenvalues form an orthonormal basis.
For a state \( \ket{\beta} \) orthogonal to both \( \ket{\alpha_0} \) and \( \ket{\alpha_1} \), we have \( \Delta \ket{\beta} = 0 \).
Therefore, \( \Delta \) acts nontrivially only in the vector space spanned by \( \ket{\alpha_0} \) and \( \ket{\alpha_1} \), and hence has at most two nonzero eigenvalues, and its eigenvectors lie in \( \vecspan\qty{\ket{\alpha_0}, \ket{\alpha_1}} \).

Now, \( \Tr\Delta = 0 \) so the eigenvalues are \( \delta \) and \( -\delta \) for some \( \delta \in \mathbb R \).
Let \( \ket{p} \) be the eigenvector for \( \delta \), and \( \ket{m} \) be the eigenvector for \( -\delta \), so \( \ip{p}{m} = 0 \).
We can write \( \Delta \) in its spectral decomposition, giving \( \Delta = \delta \op{p}{p} - \delta \op{m}{m} \).

Let \( \ket{\alpha_0^\perp} \in \vecspan\qty{\ket{\alpha_0}, \ket{\alpha_1}} \) be a normalised vector such that \( \ip{\alpha_0^\perp}{\alpha_0} = 0 \).
Then, \( \vecspan\qty{\ket{\alpha_0}, \ket{\alpha_0^\perp}} \) forms an orthonormal basis.
Hence, we can write \( \ket{\alpha_1} = c_0 \ket{\alpha_0} + c_1 \ket{\alpha_0^\perp} \).
In this basis,
\[ \Delta = \begin{pmatrix}
    1 & 0 \\
    0 & 0
\end{pmatrix} + \begin{pmatrix}
    -\abs{c_0}^2 & -c_0 c_1^\star \\
    -c_0^\star c_1 & -\abs{c_1}^2
\end{pmatrix} = \begin{pmatrix}
    1 - \abs{c_0}^2 & -c_0 c_1^\star \\
    -c_0^\star c_1 & -\abs{c_1}^2
\end{pmatrix} = \begin{pmatrix}
    \abs{c_1}^2 & -c_0 c_1^\star \\
    -c_0^\star c_1 & -\abs{c_1}^2
\end{pmatrix} \]
which has eigenvalues \( \delta = \abs{c_1}, -\delta = -\abs{c_1} \).
Since \( \abs{c_0} = \abs{\ip{\alpha_0}{\alpha_1}} = \cos \theta \) where \( \theta \geq 0 \), we have \( \delta = \sin \theta \).
Then,
\begin{align*}
    p_S(\Pi_0, \Pi_1) &= \frac{1}{2} + \frac{1}{2} \Tr(\Pi_0 \Delta) \\
    &= \frac{1}{2} + \frac{1}{2} \Tr(\Pi_0 \qty[\sin\theta \op{p}{p} - \sin\theta \op{m}{m}]) \\
    &= \frac{1}{2} + \frac{\sin\theta}{2} \qty[\ev{\Pi_0}{p} - \ev{\Pi_0}{m}]
\end{align*}
Note that for any \( \ket{\varphi} \), we have \( 0 \leq \ev{\Pi}{\varphi} \leq 1 \), so the measurement is maximised when \( \ev{\Pi_0}{p} = 1 \) and \( \ev{\Pi_0}{m} = 0 \).
We therefore define \( \Pi_0 = \op{p}{p} \).
Then, the optimal average success probability is
\[ p_S^\star = \frac{1}{2} + \frac{\sin\theta}{2} \]
\begin{theorem}[Holevo--Helstrom; pure states]
    Let \( \ket{\alpha_0}, \ket{\alpha_1} \) be equally likely states, with \( \abs{\ip{\alpha_0}{\alpha_1}} = \cos\theta \), \( \theta \geq 0 \).
    Then, the probability \( p_S \) of correctly identifying the state by any quantum measurement satisfies
    \[ p_S \leq \frac{1}{2} + \frac{\sin\theta}{2} \]
    and this bound can be attained.
\end{theorem}
In the case of orthogonal states, the theorem implies that \( p_S \leq 1 \) and the bound can be attained, which was shown before.

\subsection{No-signalling principle}
Suppose we have a state \( \ket{\phi_{AB}} \in \mathcal V_A \otimes \mathcal V_B \) shared between two agents \( A, B \).
Suppose we perform a complete projective measurement on \( \ket{\phi_A} \).
By the extended Born rule, each measurement outcome will lead to an instantaneous change of \( \ket{\phi_B} \).
If this change in state could be detected by measuring \( \ket{\phi_B} \), instantaneous communication between \( A \) and \( B \) would be possible.

Consider \( \ket{\phi_{AB}^+} = \frac{1}{\sqrt{2}}\qty(\ket{00} + \ket{11}) \).
Suppose qubit \( A \) is measured in the standard basis \( \qty{\ket{0}, \ket{1}} \).
\begin{tabular}{c c c c}
    outcome & probability & post-measurement state & final state of \( B \) \\
    0 & \( \frac{1}{2} \) & \( \ket{00} \) & \( \ket{0} \) \\
    1 & \( \frac{1}{2} \) & \( \ket{11} \) & \( \ket{1} \)
\end{tabular}
Therefore, if the outcomes could be controlled, instantaneous communication would be possible.
