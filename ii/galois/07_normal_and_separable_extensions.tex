\subsection{Normal extensions}
\begin{definition}
	An extension \( L / K \) is a \emph{normal extension} if it is algebraic and for all \( x \in L \), the minimal polynomial splits in \( L \).
\end{definition}
\begin{remark}
	This condition is equivalent to the statement that for every \( x \in L \), \( L \) contains a splitting field for \( x \).
	In other words, if an irreducible polynomial \( f \in K[T] \) has a single root in \( L \), it splits and has all roots in \( L \).
\end{remark}

\subsection{Normal extensions and splitting fields}
\begin{theorem}
	Let \( L / K \) be a finite extension.
	Then \( L \) is normal over \( K \) if and only if \( L \) is a splitting field for some (not necessarily irreducible) polynomial \( f \in K[T] \).
\end{theorem}
\begin{proof}
	Suppose \( L \) is normal.
	Then \( L = K(x_1, \dots, x_n) \) since \( L \) is algebraic.
	Then the minimal polynomial \( m_{x_i,K} \) of each \( x_i \) over \( K \) splits in \( L \).
	\( L \) is generated by the roots of \( \prod_i m_{x_i,K} \), so \( L \) is a splitting field for \( f \).

	For the converse, suppose \( L \) is a splitting field for \( f \in K[T] \).
	Let \( x \in L \), and let \( g = m_{x,K} \) be its minimal polynomial.
	We want to show that \( g \) splits in \( L \).
	Let \( M \) be a splitting field for \( g \) over \( L \), and let \( y \in M \) be a root of \( g \).
	We want to show \( y \in L \).

	Since \( L \) is a splitting field for \( f \) over \( K \), \( L \) is a splitting field for \( f \) over \( K(x) \), and \( L(y) \) is a splitting field for \( f \) over \( K(y) \).
	Now, there exists a \( K \)-isomorphism between \( K(x) \) and \( K(y) \), because \( x, y \) are roots of the same irreducible polynomial \( g \).
	By the uniqueness of splitting fields, \( [L:K(x)] = [L(y):K(y)] \).
	Multiplying by \( [K(x):K] \), we find \( [L:K] = [L(y):K] \) because \( [K(y):K] = [K(x):K] \) as they are roots of the same irreducible polynomial.
	Hence \( [L(y):L] = 1 \), so \( y \in L \) as required.
\end{proof}
\begin{corollary}[normal closure]
	Let \( L / K \) be a finite extension.
	Then there exists a finite extension \( M / L \) such that \( M / K \) is normal, and if \( L \subseteq M' \subseteq M \) and \( M' / K \) is normal, \( M = M' \).
	Moreover, any two such extensions \( M \) are \( L \)-isomorphic.
\end{corollary}
Such an \( M \) is said to be a \emph{normal closure} of \( L / K \).
\begin{proof}
	Let \( L = K(x_1, \dots, x_k) \), and \( f = \prod_{i=1}^k m_{x_i,K} \in K[T] \).
	Then let \( M \) be a splitting field of \( f \) over \( L \).
	Then, since the \( x_i \) are roots of \( f \), \( M \) is also a splitting field for \( f \) over \( K \).
	So \( M / K \) is normal.

	Let \( M' \) be such that \( L \subseteq M' \subseteq M \) and \( M' / K \) be normal.
	Then as \( x_i \in M' \), the minimal polynomial \( m_{x_i,K} \) splits in \( M' \).
	So \( M' = M \).

	Any normal extension \( M / K \) must contain a splitting field for \( f \), and by the minimality condition, \( M \) must be a splitting field.
	By uniqueness of splitting fields, any two such extensions are \( L \)-isomorphic as required.
\end{proof}

\subsection{Separable polynomials}
Recall that over \( \mathbb C \), a polynomial has a multiple zero by considering its derivative.
Over arbitrary fields, the same is true, but the analytic concept of derivative must be replaced with an algebraic process.
\begin{definition}
	The \emph{formal derivative} of a polynomial \( f(T) = \sum_{i=0}^d a_i X^i \) is
	\[ f'(T) = \sum_{i=1}^d i a_i X^{i-1} \]
\end{definition}
\begin{remark}
	One can check from the definition that the familiar rules \( (f + g)' = f' + g' \), \( (fg)' = f'g + fg' \), and \( (f^n)' = nf'f^{n-1} \) hold.
\end{remark}
\begin{example}
	Consider a field \( K \) of characteristic \( p > 0 \), and let \( f = T^p + a_0 \).
	Then \( f' = 0 \), so a non-constant polynomial can have a zero derivative.
\end{example}
\begin{proposition}
	Let \( f \in K[T] \), \( L / K \) be a field extension, and \( x \in L \) a root of \( f \).
	Then \( x \) is a simple root if and only if \( f'(x) \neq 0 \).
\end{proposition}
\begin{proof}
	We can write \( f = (T-x)g \in L[T] \).
	Then \( f' = g + (T-x)g' \), so \( f'(x) = g(x) \).
	In particular, \( g(x) \neq 0 \) if and only if \( (T-x) \) does not divide \( g \), which is the criterion that \( x \) is a simple root of \( f \).
\end{proof}
\begin{definition}
	A polynomial \( f \in K[T] \) is \emph{separable} if it splits into distinct linear factors in a splitting field.
	Equivalently, it has \( \deg f \) distinct roots.
\end{definition}
\begin{corollary}
	\( f \) is separable if and only if the greatest common divisor of \( f \) and \( f' \) is \( 1 \).
\end{corollary}
For convenience, we will take \( \gcd(f, g) \) to be the unique monic polynomial \( h \) such that \( (h) = (f, g) \).
Then since \( K[T] \) is a Euclidean domain, we can compute a representation \( h = af + bg \) for polynomials \( a, b \).
Note that \( \gcd(f, g) \) is invariant under a field extension, because Euclid's algorithm does not depend on the ambient field structure.
\begin{proof}
	We can replace \( K \) by a splitting field of \( f \), so we can factorise \( f \) into a product of linear factors in \( K \).
	The two are separable if \( f, f' \) have no common root, which is true if and only if \( \gcd(f, f') = 1 \).
\end{proof}
\begin{example}
	Let \( K \) have characteristic \( p > 0 \), and let \( f = T^p - b \) for \( b \in K \).
	Then \( f' = 0 \), so \( \gcd(f, f') = f \neq 1 \).
	Hence \( f \) is inseparable.
	Let \( L \) be an extension of \( K \) containing a \( p \)th root \( a \in L \) of \( b \), so \( a^p = b \).
	Then \( f = (T - a)^p = T^p + (-a)^p = T^p - b \).
	In particular, \( f \) has only one root in a splitting field.

	If \( b \) is not a \( p \)th power in \( K \), then \( f \) is irreducible.
	This is seen on the example sheets.
\end{example}
\begin{theorem}
	Let \( f \in K[T] \) be an irreducible polynomial.
	Then \( f \) is separable if and only if \( f' \neq 0 \).

	In addition, if \( K \) has characteristic zero, every irreducible polynomial \( f \in K[T] \) is separable.
	If \( K \) has positive characteristic \( p > 0 \), an irreducible polynomial \( f \in K[T] \) is inseparable if and only if \( f(T) = g(T^p) \) for some \( g \in K[T] \).
\end{theorem}
\begin{proof}
	Without loss of generality, we can assume \( f \) is monic.
	Then, since \( f \) is irreducible, the greatest common divisor \( \gcd(f,f') \) is either \( f \) or \( 1 \).
	If \( \gcd(f,f') = f \), then \( f' = 0 \) by considering the degree.

	For a polynomial \( f \), we can write \( f = \sum_{i=0}^d a_i T^i \) and \( f' = \sum_{i=1}^d i a_i T^{i-1} \), so \( f' = 0 \) if and only if \( i a_i = 0 \) for all \( 1 \leq i \leq d \).
	In particular, if \( K \) has characteristic zero, this is true if and only if \( a_i = 0 \) for all \( 1 \leq i \leq d \), so \( f = a_0 \) is a constant so not irreducible.
	If \( K \) has characteristic \( p > 0 \), the requirement is that \( a_i = 0 \) for all \( i \) not divisible by \( p \), or equivalently, \( f(T) = g(T^p) \).
\end{proof}

\subsection{Separable extensions}
\begin{definition}
	Let \( L / K \) be a field extension.
	We say \( x \in L \) is \emph{separable} over \( K \) if \( x \) is algebraic and its minimal polynomial \( f \) is separable over \( K \).
	\( L \) is \emph{separable} over \( K \) if all elements \( x \) are separable over \( K \).
\end{definition}
\begin{theorem}
	Let \( x \) be algebraic over \( K \), and \( L / K \) be an extension in which the minimal polynomial \( m_{x,K} \) splits.
	Then \( x \) is separable over \( K \) if and only if there are exactly \( \deg x \) \( K \)-homomorphisms from \( K(x) \) to \( L \).
\end{theorem}
\begin{proof}
	The number of \( K \)-homomorphisms from \( K(x) \) to \( L \) is the number of roots of \( m_{x,K} \) in \( L \).
	This is equal to the degree of \( x \) if and only if \( x \) is separable.
\end{proof}
Let \( \mathrm{Hom}_K(L,M) \) be the set of \( K \)-homomorphisms from \( L \) to \( M \).
Note that not all \( K \)-linear maps from \( L \) to \( M \) are \( K \)-homomorphisms.
\begin{theorem}[counting embeddings]
	Let \( L = K(x_1, \dots, x_k) \) be a finite extension of \( K \), so the \( x_i \) are algebraic.
	Let \( M / K \) be any field extension.
	Then \( \abs{\mathrm{Hom}_K(L,M)} \leq [L : K] \), with equality if and only if
	\begin{enumerate}
		\item for all \( i \), the minimal polynomial \( m_{x_i,K} \) splits into linear factors in \( M \);
		\item all the \( x_i \) are separable over \( K \).
	\end{enumerate}
\end{theorem}
\begin{remark}
	The conditions (i) and (ii) are equivalent to the statement that \( m_{x_i,K} \) split into distinct linear factors over \( M \).
	There is a variant of this theorem: let \( \sigma : K \to M \) be a field homomorphism, then \( \abs{\mathrm{Hom}_\sigma(L,M)} \leq \abs{L:K} \), and equality holds if and only if the \( \sigma m_{x_i,K} \) split over \( M \).
\end{remark}
\begin{proof}
	We prove this by induction on \( k \).
	The case \( k = 0 \) is trivial.
	Let \( K_1 = K(x_1) \) and write \( d = \deg_K x_1 = [K_1 : K] \).
	Then the number of \( K \)-homomorphisms from \( K_1 \) to \( M \), denoted \( e = \abs{\mathrm{Hom}_K(K_1,M)} \), is the number of roots of \( m_{x_1,K} \) in \( M \).
	Let \( \sigma : K_1 \to M \) be a \( K \)-homomorphism.
	By the inductive hypothesis, there exist at most \( [L : K_1] \) extensions of \( \sigma \) to a \( K \)-homomorphism \( L \to M \).
	Hence the number of \( K \)-homomorphisms from \( L \) to \( M \) is at most \( e[L : K_1] \leq d[L : K_1] = [L : K] \).

	If equality holds, then \( e = d \), and so \( m_{x_1,K} \) splits into \( d \) distinct linear factors in \( M \), so (i) and (ii) hold for \( x_1 \).
	Replacing \( x_1 \) with an arbitrary \( x_i \), one implication follows.
	Conversely, suppose conditions (i) and (ii) hold.
	Then, by the previous theorem, there are \( d \) distinct homomorphisms from \( K_1 \) to \( M \).
	Conditions (i) and (ii) still hold over \( K_1 \), then by induction, each \( \sigma \colon K_1 \to M \) has \( [L : K_1] \) extensions to a homomorphism \( L \to M \).
	Hence \( \abs{\mathrm{Hom}_K(L,M)} = [L : K] \) as required.
\end{proof}
\begin{theorem}[separably generated implies separable]
	Let \( L = K(x_1, \dots, x_k) \) be a finite extension of \( K \).
	Then \( L / K \) is a separable extension if and only if each \( x_i \) is separable over \( K \).
\end{theorem}
\begin{proof}
	If \( L / K \) is separable, the \( x_i \) are separable by definition.
	Suppose the \( x_i \) are separable.
	Let \( M \) be a normal closure of \( L / K \), so the splitting field of the product of the \( m_{x_i,K} \) over \( L \).
	By the counting embeddings theorem, conditions (i) and (ii) are satisfied so \( \abs{\mathrm{Hom}_K(L,M)} = [L : K] \).
	But if \( x \in L \), \( L = K(x, x_1, \dots, x_k) \), so \( x \) is separable.
\end{proof}
\begin{corollary}
	Let \( x, y \in L \), and \( L / K \) a field extension.
	If \( x, y \) are separable over \( K \), so are \( x + y, xy, x^{-1} \) for \( x \neq 0 \).
\end{corollary}
\begin{proof}
	Consider the fields \( K(x,y) \) and \( K(x) \).
	These are separable extensions of \( K \).
	In particular, \( \qty{x \in L \mid x \text{ separable over } K} \) is a subfield of \( L \).
\end{proof}
\begin{theorem}[primitive element theorem for separable extensions]
	Let \( K \) be an infinite field and \( L = K(x_1, \dots, x_k) \) be a finite separable extension.
	Then there exists \( x \in L \) such that \( L = K(x) \).
	In particular, \( x \) is separable over \( K \).
\end{theorem}
\begin{proof}
	It suffices to consider the case when \( k = 2 \), because if we can turn \( K(x,y) \) into \( K(z) \) for \( z \in K(x,y) \), we can perform this inductively.
	Let \( L = K(x,y) \) with \( x, y \) separable over \( K \).
	Let \( n = [L : K] \), and let \( M \) be a normal closure for \( L / K \).
	Then there exist \( n \) distinct \( K \)-homomorphisms \( \sigma_i \colon L \to M \).
	Let \( a \in K \), and consider \( z = x + ay \).
	We will choose \( a \) such that \( L = K(z) \).

	Since \( L = K(x,y) \), we have \( \sigma_i(x) = \sigma_j(x) \) and \( \sigma_i(y) = \sigma_j(y) \) implies \( i = j \).
	Consider \( \sigma_i(z) = \sigma_i(x) + a \sigma_i(y) \).
	If \( \sigma_i(z) = \sigma_j(z) \), we must have \( \qty(\sigma_i(x) - \sigma_j(x)) + a \qty(\sigma_i(y) - \sigma_j(y)) = 0 \).
	If \( i \neq j \), at least one of the parenthesised terms is nonzero.
	Therefore there is at most one \( a \in K \) such that \( \sigma_i(z) = \sigma_j(z) \).
	Since \( K \) is infinite, there exists \( a \in K \) such that all of the \( \sigma_i(z) \) are distinct.
	But then \( \deg_K z = n \), so \( L = K(z) \).
\end{proof}
\begin{theorem}
	Let \( L / K \) be an extension of finite fields.
	Then \( L = K(x) \) for some \( x \in L \).
\end{theorem}
\begin{proof}
	The multiplicative group \( L^\times \) is cyclic.
	Let \( x \) be a generator of this group.
	Then \( L = K(x) \), since every nonzero element is a power of \( x \).
\end{proof}
