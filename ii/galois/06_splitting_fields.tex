\subsection{Definition and uniqueness}
Suppose \( K \) is a field and \( f \in K[T] \).
We wish to find an extension \( L / K \) of degree as small as possible such that \( f \) is expressible as a product of linear factors in \( L[T] \).
\begin{example}
	Let \( K = \mathbb Q \).
	Then by the fundamental theorem of algebra, a monic polynomial \( f \in \mathbb Q[T] \) is expressible as a product of \( n \) linear factors \( (T - x_i) \) in \( \mathbb C[T] \).
	One example of such a field extension is \( L = \mathbb Q(x_1, \dots, x_n) \), which is a finite extension of \( \mathbb Q \).
\end{example}
We will later give another proof of the fundamental theorem of algebra using techniques from Galois theory.
\begin{example}
	Let \( K = \mathbb F_p \), and \( f \) is irreducible and has degree \( d > 1 \).
	Since there is no ambient field structure, explicitly finding \( L \) is more challenging.
	We will first find an extension in which \( f \) has at least one root, and then use induction.
\end{example}
\begin{theorem}
	Let \( f \) be a monic irreducible polynomial.
	Let \( L_f = \faktor{K[T]}{(f)} \).
	Since \( f \) is irreducible, \( (f) \) is maximal, hence \( L_f \) is a field.
	Let \( t \in L_f \) be the residue class \( T \) modulo \( (f) \).
	Then \( L_f/K \) is a finite field extension of degree \( \deg f \), and \( f \) is the minimal polynomial for \( t \).
\end{theorem}

\subsection{Uniqueness}
We have thus constructed a field extension of \( K \) for which \( f \) has at least a single root.
Recall that if \( x \) is algebraic over \( K \), then \( K(x) \cong \faktor{K[T]}{(f)} \) where \( f \) is minimal for \( x \).
\begin{definition}
	Let \( K \) be a field, and \( L / K, M / K \) are field extensions.
	A \emph{\( K \)-homomorphism} or \emph{\( K \)-embedding} from \( L \) to \( M \) is a field homomorphism \( \sigma \colon L \to M \) such that \( \eval{\sigma}_K = \mathrm{id}_K \).
\end{definition}
The naming `\( K \)-embedding' is justified because any field homomorphism is injective.
\begin{theorem}
	Let \( f \in K[T] \) be irreducible, and \( L / K \) a field extension.
	Then:
	\begin{enumerate}
		\item If \( x \in L \) is a root of \( f \), there exists a unique \( K \)-homomorphism \( \sigma \colon L_f = \faktor{K[T]}{(f)} \to L \) such that \( t = T + (f) \mapsto x \).
		\item Every \( K \)-homomorphism \( \sigma \colon L_f \to L \) arises in this way.
	\end{enumerate}
	Hence, we have a bijection between \( K \)-homomorphisms \( \sigma \colon L_f \to L \) and the set of roots of \( f \) in \( L \).
	In particular, there are at most \( \deg f \)-many \( K \)-homomorphisms.
\end{theorem}
\begin{proof}
	Let \( x \in L \) be a root of \( f \).
	Then \( f(x) = 0 \), so \( \mathrm{ev}_x(f) = 0 \) where \( \mathrm{ev}_x \colon K[T] \to L \) is the evaluation homomorphism \( g \mapsto g(x) \).
	Equivalently, \( \ker \mathrm{ev}_x = (f) \).
	Hence, by the isomorphism theorem, \( \mathrm{ev}_x \) comes from a homomorphism \( \sigma \colon \faktor{K[T]}{(f)} \to L \).
	Since the evaluation map is an identity on \( K \), this is a \( K \)-homomorphism as required.
\end{proof}
\begin{corollary}
	Let \( L = K(x) \) for some \( x \) algebraic over \( K \).
	Then there exists a unique isomorphism \( \sigma \colon L_f \to K(x) \) such that \( \sigma(t) = x \), where \( f \) is minimal for \( x \) over \( K \).
\end{corollary}
\begin{definition}
	Let \( x, y \) be algebraic over \( K \).
	We say \( x, y \) are \emph{\( K \)-conjugate} if they have the same minimal polynomial over \( K \).
\end{definition}
By the corollary above, \( K(x) \) and \( K(y) \) are isomorphic to \( L_f \) where \( f \) is minimal for \( x \) and \( y \) over \( K \).
\begin{corollary}
	Algebraic elements \( x, y \) are \( K \)-conjugate if and only if there exists a \( K \)-isomorphism \( \sigma \colon K(x) \to K(y) \) such that \( \sigma(x) = y \).
\end{corollary}
\begin{proof}
	The above corollary shows the forward direction.
	Conversely, for all \( g \in K[T] \), we have \( \sigma(g(x)) = g(\sigma(x)) \) so they have the same minimal polynomial.
\end{proof}
Informally, the roots of an irreducible polynomial are algebraically indistinguishable.

\subsection{?}
It can be useful for inductive arguments to have a generalisation of the above theorem.
\begin{definition}
	Let \( L / K, L' / K' \) be field extensions, and let \( \sigma \colon K \to K' \) be a field homomorphism.
	Let \( \tau \colon L \to L' \) be a field homomorphism such that \( \tau(x) = \sigma(x) \) for all \( x \in K \).
	Then we say \( \tau \) is a \emph{\( \sigma \)-homomorphism} from \( L \) to \( L' \).
	We also say \( \tau \) \emph{extends} \( \sigma \), or that \( \sigma \) is the \emph{restriction} of \( \tau \) to \( K \).
\end{definition}
We can now define the following variant of the previous theorem.
\begin{theorem}
	Let \( f \in K[T] \) be irreducible, and \( \sigma \colon K \to L \) be a field homomorphism.
	Let \( \sigma f \) be the polynomial obtained by applying \( \sigma \) to the coefficients of \( f \).
	\begin{enumerate}
		\item If \( x \in L \) is a root of \( f \), there exists a unique \( \sigma \)-homomorphism \( \tau \colon L_f \to L \) such that \( \tau(t) = x \).
		\item Every \( \sigma \)-homomorphism \( L_f \to L \) is of this form.
	\end{enumerate}
	Therefore there is a bijection between the \( \sigma \)-homomorphisms \( L_f \to L \) and the roots of \( f \) in \( L \).
\end{theorem}
\begin{example}
	Let \( K = \mathbb Q(\sqrt 2) \subset \mathbb R \), and \( L = \mathbb C \).
	Let \( \sigma \colon K \to L \) be the homomorphism such that \( \sigma(x+y\sqrt 2) = x-y\sqrt 2 \).
	Then let \( f = T^2 - (1 + \sqrt 2) \).
	Then the map \( \tau \colon L_f \to \mathbb C \) must satisfy \( \tau(t) = \pm \sqrt{1 - \sqrt 2} = \pm i \sqrt{\sqrt 2 - 1} \in \mathbb C \).
	If instead we let \( \sigma(x+y\sqrt 2) = x+y\sqrt 2 \), we have \( \tau(t) = \pm\sqrt{\sqrt 2 + 1} \), which are both real.
\end{example}

\subsection{Splitting completely}
\begin{definition}
	Let \( f \in K[T] \) be a nonzero polynomial that is not necessarily irreducible.
	We say that an extension \( L / K \) is a \emph{splitting field} for \( f \) over \( K \) if
	\begin{enumerate}
		\item \( f \) splits into linear factors in \( L[T] \);
		\item \( L = K(x_1, \dots, x_n) \), where the \( x_i \) are the roots of \( f \) in \( L \).
	\end{enumerate}
\end{definition}
\begin{remark}
	The second criterion ensures that \( f \) does not split into linear factors in any proper subfield of \( L \).
	Note that any splitting field is finite, because the adjoined elements are algebraic.
\end{remark}
\begin{theorem}
	Every nonzero polynomial has a splitting field.
\end{theorem}
\begin{proof}
	Let \( f \in K[T] \).
	We prove this by induction on the degree of \( f \), but allow \( K \) to vary.
	If \( f \) is constant, there is nothing to prove, since \( K \) is already a splitting field.
	Suppose that for all fields \( K' \) and all polynomials in \( K'[T] \) of degree less than \( f \), there is a splitting field.
	Consider an irreducible factor \( g \) of \( f \), and consider \( K' = L_g = \faktor{K[T]}{(g)} \).
	Let \( x_1 = T + (g) \).
	Then \( g(x_1) = 0 \), so \( f(x_1) = 0 \), hence \( f = (T - x_1)f_1 \), where \( f_1 \in K'[T] \).
	By induction, there exists a splitting field \( L \) for \( f_1 \) over \( K' \) since \( \deg f_1 < \deg f \).
	Let \( x_2, \dots, x_n \in L \) be the roots of \( f_1 \) in \( L \).
	Then \( f \) splits into linear factors in \( L \) with roots \( \qty{x_1, x_2, \dots, x_n} \).
	Because \( L \) is a splitting field for \( f_1 \) over \( K' \), we have \( L = K'(x_2, \dots, x_n) = K(x_1)(x_2, \dots, x_n) = K(x_1, \dots, x_n) \), so \( L \) is a splitting field for \( f \).
\end{proof}
\begin{remark}
	If \( K \subseteq \mathbb C \), we already know by the fundamental theorem of algebra that any polynomial over \( K \) has a subfield of \( \mathbb C \) as its splitting field.
\end{remark}

\subsection{Uniqueness}
\begin{theorem}
	Let \( f \in K[T] \) be a polynomial and \( L / K \) be a splitting field for \( f \).
	Then let \( \sigma \colon K \to M \) be a field homomorphism such that \( \sigma f \) splits in \( M[T] \).
	Then
	\begin{enumerate}
		\item \( \sigma \) can be extended to a homomorphism \( \tau \colon L \to M \);
		\item if \( M \) is a splitting field for \( \sigma f \) over \( \sigma K \), then any \( \tau \colon L \to M \) is an isomorphism.
	\end{enumerate}
	In particular, any two splitting fields are \( K \)-isomorphic.
\end{theorem}
\begin{remark}
	When constructing the splitting field for a polynomial, we had choice in which irreducible factors to consider first.
	It is not clear, without this theorem, that two splitting fields have the same degree.

	Note that we can have different \( \tau_1, \tau_2 \colon L \to M \) for splitting fields \( L, M \) of \( f \) over \( K \).
\end{remark}
\begin{proof}
	We will prove (i) by induction on \( [L : K] \).
	If \( n = 1 \), we have \( L = K \) and there is nothing to prove.
	Suppose \( x \in L \setminus K \) is a root of an irreducible factor \( g \) of \( f \) in \( K \), so \( \deg g > 1 \).
	Let \( y \in M \) be a root of \( \sigma g \in M[T] \), which exists because \( \sigma f \) splits in \( M \).
	Then, there exists \( \sigma_1 \colon K(x) \to M \) such that \( \sigma_1(x) = y \), and \( \sigma_1 \) extends \( \sigma \).
	Then, \( [L : K(x)] < [L : K] \), so by induction, \( \sigma_1 \colon K(x) \to M \) can be extended to \( \tau \colon L \to M \), because \( L \) is a splitting field for \( f \) over \( K(x) \).
	This \( \tau \) therefore extends \( \sigma \) as required.

	To prove (ii), suppose \( M \) is a splitting field for \( \sigma f \) over \( \sigma K \).
	Let \( \tau \) be as in (i), and \( \qty{x_i} \) be the roots of \( f \) in \( L \).
	Then the roots of \( \sigma f \) in \( M \) are \( \qty{\tau(x_i)} \).
	Since \( M \) is a splitting field, \( M = \sigma K(\qty{\tau(x_i)}) = \tau L \) as \( L = K(\qty{x_i}) \).
	So \( \tau \) is an isomorphism.

	If \( K \subseteq M \) and \( \sigma \) is the inclusion homomorphism, \( \tau \) is a \( K \)-isomorphism.
\end{proof}
\begin{example}
	Let \( f = T^3 - 2 \in \mathbb Q[T] \).
	This has splitting field \( L = \mathbb Q(\sqrt[3]{2}, \omega) \subseteq \mathbb C \) where \( \omega = e^{\frac{2\pi i}{3}} \).
	We know \( [\mathbb Q(\sqrt[3]{2}) : \mathbb Q] = 3 \), but \( \omega \not\in \mathbb R \) and \( \omega^2 + \omega + 1 = 0 \), so \( [L : \mathbb Q(\sqrt[2]{3})] = 2 \) giving \( [L : \mathbb Q] = 6 \) by the tower law.
	In particular, adjoining a single root to \( \mathbb Q \) is not enough to generate \( L \).
\end{example}
\begin{example}
	Let \( f = \frac{T^5 - 1}{T - 1} = T^4 + \dots + T + 1 \in \mathbb Q[T] \).
	Let \( z = e^{\frac{2\pi i}{5}} \), then this is the minimal polynomial of \( z \).
	We find \( f = \prod_{1 \leq a \leq 4} (T - z^a) \), so \( \mathbb Q(z) \) is already a splitting field for \( f \) over \( \mathbb Q \), and \( [\mathbb Q(z) : \mathbb Q] = 4 \).
\end{example}
\begin{example}
	Let \( f = T^3 - 2 \in \mathbb \mathbb F_7[T] \).
	This is irreducible because 2 is not a cube in \( \mathbb F_7 \).
	Consider \( L = \faktor{\mathbb F_7[X]}{X^3 - 2} = \mathbb F_7(x) \), so \( x^3 = 2 \).
	Since \( 2^3 = 4^3 = 1 \) in \( \mathbb F_7 \), we have \( (2x)^3 = (4x)^3 = 2 \), so \( x, 2x, 4x \) are roots of \( f \) in \( L \).
	In particular, \( L \) is a splitting field for \( f \), since \( f = (T - x)(T - 2x)(T - 4x) \); here, adjoining one root is enough to make \( f \) split.
\end{example}
